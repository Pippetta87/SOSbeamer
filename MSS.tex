\begin{comment}
\section{Osservabili stellari/demo beamer}
\begin{frame}<1>[label=noinside]{Modello stellare}{Come indagare la fisica interna a una stella?}
\onslide<1->\begin{block}{Osservabili stellari:}
$L$, $M$, $R$, $T_e$, $(\frac{Z}{X})_{ph}$, $g_{ph}$.
\end{block}
\onslide<1->\begin{block}{Informazioni sulla struttura interna?} Condizione di equilibrio idrostatico
\end{block}
%Teorema Vogt-Russel: $X_i(r)$, $M$ \pause equilibrio (idrostatico/termico) determinano struttura stellare .
%\pause
\onslide<1->\begin{block}{Modello stellare: diagramma di \hr{}.}
\end{block}
\onslide<2->\begin{block}{Descrizione fisica interno stellare: parametri aggiuntivi}
Convezione, diffusione e sedimentazione elementi pesanti, equazione di stato, opacit\'a
\end{block}
\onslide<2->\begin{block}{Astrosismologia}
Restringo spazio parametri sistemi stellari lontani
\end{block}
\end{frame}
{ % all template changes are local to this group.
    \setbeamertemplate{navigation symbols}{}
    \begin{frame}[plain]{Diagramma di \hr{}}
        \begin{tikzpicture}[remember picture,overlay]
            \node[at=(current page.center)] {
                %\includegraphics[width=\paperwidth]{yourimage}
            };
        \end{tikzpicture}
     \end{frame}
}
\againframe<2>{noinside}
\begin{frame}{Pulsazioni stellari}{Modi Normali}
\begin{columns}
\begin{column}{0.5\textwidth}  %%<--- here
    \begin{center}
     %\includegraphics[width=0.5\textwidth]{image1}
     \end{center}
\end{column}
\begin{column}{0.5\textwidth}
\onslide<1-> \begin{block}{Stelle pulsanti}
Onde stazionarie: Pulsazione radiale/non radiale: .
\onslide<2-> meccanismo di eccitazione: solar-like pulsator, Cefeidi.
\onslide<3-> Modo fondamentale $\Pi\approx\tau_{dyn}=\sqrt{\frac{R^3}{GM}}\propto\overline{\rho}\expy{-\frac{1}{2}}$.
\onslide<4-> Modi di oscillazione\onslide<5-> - informazioni sull'interno stellare
\onslide<5-> Elio-sismologia: Modi $\Leftrightarrow$ Modelli solari
\onslide<5-> Astero-sismologia: Modi $\Leftrightarrow$ Spazio parametri modello stellare
\end{block}
\end{column}
\end{columns}
\end{frame}

\end{comment}

\section{Osservabili solari}

\begin{frame}{Dati osservativi}

\begin{block}{Et\'a, luminosit\'a, raggio solari}
\begin{tabular}{l|c}
\hline
$\agesun{}$&\SI[separate-uncertainty=true]{4.57\pm0.02e9}{\year}\\
\hline
$\rsun{}$&\SI{695658+-140}{\kilo\meter}\\
\hline
$G\msun$&\num{132712440018+-8}\SI{e9}{\cubic\meter\per\square\second}\\
\hline
$\lsun{}$&\SI{3.8275+-0.0014e33}{\erg\per\second}\\
\hline
\end{tabular}
%\caption[Osservabili solari principali.]{Osservabili solari principali. \cite{haberreiter2008solving}.}
\label{tab:sunO}
\end{block}

\begin{block}{Simmetria sferica}
Deviazioni da forma sferica trascurabili (campi magnetici, rotazione)
\end{block}

\end{frame}

\begin{frame}{Dati osservativi}

\begin{block}{Composizione chimica}
\begin{itemize}
\item Righe di assorbimento: attuale (non $Y_{ph}$)
\item Meteoriti CI: primordiale (refrattari)
\end{itemize}

\begin{table}[]

\pgfplotstabletypeset[
every head row/.style={
 before row={\toprule &\multicolumn{4}{c|}{Attuale}
 %&\multicolumn{4}{c|}{Primordiale}
 \\\midrule},
 every last row/.style={after row=\bottomrule},
 after row={\midrule}
},
every nth row={2}{before row=\midrule},every last row/.style={after row=\bottomrule},
every first column/.style={column type/.add={|}{}},
every last column/.style={column type/.add={}{|}},
columns/x/.style = {column type/.add={|}{}},
columns/xi/.style = {column type/.add={|}{}},
display columns/0/.style={column name={}},
display columns/1/.style={column name={$X$}},
display columns/2/.style={column name={$Y$}},
display columns/3/.style={column name={$Z$}},
display columns/4/.style={column name={$\frac{Z}{X}$}},
%display columns/5/.style={column name={$X$}},
%display columns/6/.style={column name={$Y$}},
%display columns/7/.style={column name={$Z$}},
%display columns/8/.style={column name={$\frac{Z}{X}$}},
create on use/authors/.style={create col/set list={
%Anders \& Grevesse (1989),Grevesse \& Noels (1993),
Grevesse et al. (1998),Lodders (2003),Asplund et al. (2005),Lodders et al. (2009),\cite{asplund2009chemical},\cite{caffau2011solar}}},
columns/authors/.style={string type},
columns={authors,x, y, z, zx
%,xi,yi,zi, zxi
},
/pgf/number format/precision=4
     ]{asplund.txt} %%%
\captionof{table}{Metallicit\'a attuale determinata da varii autori.}\label{tab:Zhistory}
\end{table}

\end{block}

\end{frame}


\section{Strutture autogravitanti in equilibrio}

\begin{frame}{Distribuzione di massa - Conservazione di massa e momento - tempo scala dinamico}

\begin{block}{Massa}

%\begin{align}
%&dm=4\pi r^2\rho \,dr-4\pi r^2\rho v\,dt\label{eq:massvar}\\
%\end{align}

\begin{equation}
\PDy{t}{\rho}+\nabla\cdot(\rho\vec{v})=0\label{eq:continuityeq}
\end{equation}

\begin{equation}
dm=4\pi r^2\rho \,dr\label{eq:massaguscio}
\end{equation}

\end{block}

\begin{block}{Momento}
\begin{align}
&\rho\TDy{t}{\vec{v}}=-\nabla P+\rho\vec{f}\label{eq:motion}\\
&\vec{g}=-\PDy{r}{\Phi}=-\frac{Gm(r)}{r^2}\hat{r}
\end{align}
\end{block}

\end{frame}

\begin{frame}{Equilibrio idrostatico: $\ddvec{r}=0$.}

\begin{align*}
\nabla P=\rho \vec{f}\Label{eq:idrosta} \TDy{r}{P}=-\frac{Gm(r)\rho(r)}{r^2}\Label{eq:fidroequilibrio}
\end{align*}


Per giustificare l'ipotesi di equilibrio idrostatico stimo i tempi caratteristici di evoluzione della struttura solare nel caso la forza dovuta alla pressione o la forza di gravit\'a non fossero bilanciate, approssimando il valore caratteristico della derivata di due variabili con il rapporto del loro valore caratteristico.

\begin{equation}
\tau_{ff}\approx\tau_{esp}\approx\tau_{idro}^{\odot}= \sqrt{\frac{R^3}{GM}}\approx\frac{1}{2}(G\overline{\rho})\expy{-\frac{1}{2}}\approx\SI{27}{\minute}
\end{equation}

\end{frame}

\subsection{Equazione di stato $P(\rho,T)$}

\begin{frame}{Gas perfetto ioni-elettroni}


\begin{equation}
P_G=P_I+P_e=\frac{\rho}{\mu}\gasconstant{}T
\end{equation}

\begin{block}{Peso molecolare medio}
massa media in amu per particella libera
\begin{align}
&\mu=\frac{1}{\bar{n}_HX+\bar{n}_{He}Y+\bar{n}_{Z}Z}\label{eq:meanmw}\\
&\bar{n}_i=\frac{1+f_i}{A_i}
\end{align}

\end{block}


\end{frame}

\subsection{Energia interna per unit\'a di massa}

\begin{frame}{Energia interna: traslazioni}

\begin{align}
&u=\frac{1}{\rho}\sum_i\int f^{(0)}(\vec{p}_i)\frac{p^2_i}{2m_i}=\frac{3}{2}\frac{P}{\rho}=\frac{3}{2}\frac{\gasconstant T}{\mu}\\
&E_i=\int_0^Mu\,dm=\frac{3}{2}\int_M\frac{P}{\rho}\,dm\label{eq:traslintenergy}
\end{align}

 $f^{(0)}(\vec{p}_i)$ \'e il numero di particelle della specie i per unit\'a di volume con impulso in $[\vec{p},\vec{p}+d\vec{p}]$

\end{frame}


\subsection{Correzioni alla legge dei gas perfetti}

\begin{frame}{Correzioni alla legge dei gas perfetti}

\begin{itemize}
\item Degenerazione elettronica ($\Delta P\leq2\%$).

\begin{align}
&P_{FD}=[\exp{\psi(\rho,T)+\midfrac{p^2}{2mKT}}+1]\expy{-1}\\
&n_e=\frac{\rho N_A}{\mu_e}=\frac{8\pi}{3h^3m_e}(2m_eKT)\expy{\midfrac{3}{2}}F_{\midfrac{3}{2}}(\psi(\rho,T))\\
%n_e=\intzi{}\frac{8\pi p^2\,dp}{h^3(\exp{\frac{u_k}{KT}-\psi}+1)}\\
&P_e=\frac{1}{3}\intzi{}pn_e\TDy{p}{u_k}\,dp
\end{align}

\item Pressione di radiazione: $P_r=\frac{1}{3}aT^4$.

\item Ionizzazione.

\end{itemize}

\end{frame}

\begin{frame}{Correzioni alla legge dei gas perfetti: Interazioni coulombiane}

\begin{align}
&\frac{1}{r_D^2}=\frac{4\pi e^2}{kT}\sum Z^2\overline{n}_Z=\frac{4\pi e^2}{kT}N_A\zeta\\
&\zeta=\sum_{i}(Z_i^2+Z_i)\frac{\rho X_i}{A_i}\xrightarrow{FD}\sum_{i}(Z_i^2+\frac{F_{\midfrac{3}{2}}'(\psi)}{F_{\midfrac{3}{2}}(\psi)}_i)\frac{\rho X_i}{A_i}
\end{align}

\begin{equation}
u_c=\frac{1}{2}\int\phi(\vec{r})\rho_c(\vec{r})\,d^3r,\ P_c=\frac{1}{3}u_c
\end{equation}

Regioni di ionizzazione parziale di idrogeno ed elio

\end{frame}

\begin{frame}{EOS}

\begin{itemize}
\item Schema chimico (MHD): atomi e molecole, stati eccitati e diversi gradi di ionizzazione

\item Schema fisico (OPAL): nuclei ed elettroni, potenziale Coulombiano, Schr\"oedinger per un problema a molti corpi.
\end{itemize}

\begin{figure}[!ht]
\subfigure[Popolazione dei diversi gradi di ionizzazione per $\cel{He}{4}{}{}$, CNO, $\cel{Ne}{20}{}{}$, $\cel{Fe}{56}{}{}$. Da \cite{basu2008helioseismology}.]{\includegraphics[width=0.4\textwidth,keepaspectratio]{ionfraction}}\label{ionfraction}
%\subcaption{Andamento di $\Gamma_1$ calcolato tramite equazione di stato MHD/OPAL. Da \cite{trampedach2006synoptic}.]{\includegraphics[width=0.4\textwidth,keepaspectratio]{ionfraction}}
~
\subfigure[Confronto $\Gamma_1$ MHD/OPAL. Da \cite{trampedach2006synoptic}.]{\includegraphics[width=0.4\textwidth,keepaspectratio]{gamma1eos}}
%\subcaption{Profilo radiale della popolazione dei diversi gradi di ionizzazione per $\cel{He}{4}{}{}$, CNO, $\cel{Ne}{20}{}{}$, $\cel{Fe}{56}{}{}$. Stati di ionizzazione maggiore sono pi\'u interni. Da \cite{basu2008helioseismology}.}\label{ionfraction}
%\subcaption{Andamento di $\Gamma_1$ calcolato tramite equazione di stato MHD/OPAL. Da \cite{trampedach2006synoptic}.}\label{fig:gamma1eos}
\end{figure}


\end{frame}


\subsection{Trasporto dell'energia}


\begin{frame}{T Viriale}

Il teorema del viriale esprime una propriet\'a statistica di particelle interagenti: 

L'energia potenziale gravitazionale della stella \'e
\begin{equation}
\Omega=-\int_0^M\frac{Gm(r)}{r}\,dm\label{eq:energiapg}
\end{equation}

\begin{equation}
\frac{1}{2}\TtwoDy{t}{I}=2E_i+\Omega
\end{equation}
con $E_i$ data da \eqref{eq:traslintenergy} e $I=\int r^2\,dm$. In condizioni stazionarie $\frac{1}{2}\TtwoDy{t}{I}\approx0$:
\begin{equation}
0=\int_M\frac{3P}{\rho}\,dm(r)+\Omega
\end{equation}

Detta $W=E_i+\Omega$ l'energia totale della stella, si ha:
\begin{equation}
\Omega=-2E_i\label{eq:virialegpm}
\end{equation}
e dalla conservazione dell'energia $\TDy{t}{W}+L=0$ segue che durante la fase di collasso prima dell'inizio della sequenza principale met\'a dell'energia gravitazionale viene spesa per aumentare l'energia interna e met\'a in luminosit\'a:
\begin{equation}
L=-\frac{1}{2}\dot{\Omega}=\dot{E}_i
\end{equation}

\end{frame}

\begin{frame}{Struttura in equilibrio idrostatico e termico in assenza di reazioni nucleari}

Nel caso in cui la contrazione gravitazionale sia l'unica fonte di energia per una massa gassosa in equilibrio idrostatico, il suo tempo di evoluzione caratteristico \'e il tempo di \kh{}:
\begin{equation}
\tkh{}=\frac{\Omega}{L}\approx\frac{GM^2}{2RL}\approx\SI{1.6e7}{\year}
\end{equation}

cammino libero medio degli atomi e dei fotoni sia breve, si raggiunge rapidamente l'equilibrio idrostatico e termico locale. Il processo di contrazione gravitazionale continua, su tempi-scala termodinamici, fino a che l'energia prodotta dalle reazioni nucleari bilancia l'energia irradiata.


\end{frame}


\subsection{Conservazione dell'energia interna}

\begin{frame}{Prima legge TD}

$dq$ per unit\'a di massa per $dt$:
\begin{align}
&\TDy{t}{q}=\TDy{t}{u}+P\TDof{t}(\frac{1}{\rho})\label{eq:prima}\\
%\TDy{t}{u}+P\TDy{t}{V}
&\TDy{t}{\ln{T}}=\frac{\Gamma_2-1}{\Gamma_2}\TDy{t}{\ln{P}}+\frac{\TDy{t}{q}}{c_PT}\label{eq:primatemp}\\
&\TDy{t}{\ln{P}}=\Gamma_1\TDy{t}{\ln{\rho}}+\frac{\rho(\Gamma_3-1)}{P}\TDy{t}{q}\label{eq:primapres}
\end{align}
Esponenti adiabatici $\Gamma_i$:
\begin{equation}\label{eq:adibatexp}
\Gamma_1=\Dcvar{\TDly{\rho}{P}}{Ad}, \ \Gamma_3-1=\Dcvar{\TDly{\rho}{T}}{Ad},\ \frac{\Gamma_2-1}{\Gamma_2}=\Dcvar{\TDly{P}{T}}{Ad}
\end{equation}

\end{frame}

\subsection{Equilibrio termico}

\begin{frame}{Equilibrio termico}

Scrivo il bilancio di calore per un elemento di massa unitaria di gas:
\begin{equation}
\TDy{t}{q}=\epsilon-\frac{1}{\rho}\nabla\cdot\vec{F}\label{eq:heatgl}
\end{equation}
dove $\epsilon$ \'e l'energia prodotta per unit\'a di tempo e massa e $\vec{F}$ \'e il flusso di energia verso l'esterno generalmente dovuto alla diffusione di fotoni dalla zona pi\'u calda verso la superficie; sostituendo in \eqref{eq:prima} si ha
\begin{equation}
\TDy{r}{L}=4\pi r^2[\rho\epsilon-\rho\TDof{t}u+\frac{P}{\rho}\TDy{t}{\rho}]\label{eq:fenergyconservation}
\end{equation}

Nel caso stazionario:
\begin{equation}
\TDy{t}{q}=0\ \Rightarrow\ dL=4\pi r^2\rho\epsilon\,dr
\end{equation}
e i processi nucleari che avvengono nella parte centrale forniscono il calore per bilanciare il flusso di energia irradiata.

\end{frame}

\subsection{Diffusione}

\begin{frame}{Diffusione elementi}

Velocit\'a di diffusione relativa (\citetitle{aller1960diffusion}):
\begin{equation}
v_{12}=\frac{1}{n_1}\int\,d^3v_1f_1\vec{v}_1-\frac{1}{n_{2}}\int\,d^3v_{2}f_{2}\vec{v}_{2}
\end{equation}

\begin{itemize}\label{itm:diffusionaller}
\item disomogeneit\'a di composizione
\begin{equation}
\propto\frac{1}{c_1c_{2}}\PDy{r}{c_1}
\end{equation}
\item Disomogeneit\'a pressione e forza per unit\'a di massa $F_i$:
\begin{equation}
\frac{m_{2}-m_1}{c_1m_1+c_{2}m_{2}}\frac{1}{P}\PDy{r}{P}-\frac{m_1m_{2}(\vec{F}_1-\vec{F}_{2})}{KT(c_1m_1+c_{2}m_{2})}
\end{equation}
\item disomogeneit\'a temperatura:
\begin{equation}
\frac{K_T}{n_1n_{2}}\frac{1}{T}\PDy{r}{T}
\end{equation}

\end{itemize}

$(-D_{12})/D_{12}=\frac{1}{3}lv_{th}$ con $l\approx(n\sigma)\expy{-1}$ cammino libero medio


\end{frame}

\begin{frame}{Urti}


\begin{equation}
\TDy{t}{f_i}=\PDy{t}{f_i}+\vec{v}_i\cdot\PDy{\vec{r}}{f_i}+\vec{F}_i\cdot\PDy{\vec{v}}{f_i}=-\Div_{\vec{p}}(\vec{s})=C(f_j)\label{eq:Btransport}
\end{equation}
$\tau_{diff}\approx\SI{6e13}{\year}$: soluzione dell'equazione del trasporto di Boltzmann approssimabile con la distribuzione di equilibrio traslata della velocit\'a di diffusione .

Sezione d'urto collisionale di Rutherford:
\begin{align*}
&d\sigma=\frac{4\pi(Z_iZ_j)^2}{\mu^2(\vec{v}-\vec{v}')^4}\frac{d\chi}{\chi^3}\Label{eq:dsruther}L=\int\frac{d\chi}{\chi}=\log{(\frac{1}{\chi_{min}})}\Label{eq:cL}\\
&\lambda=\max{(r_D,a_0)}:\Label{eq:maxb}
\end{align*}

\begin{align*}
\sigma_{ij}\propto \frac{e^4Z_i^2Z_j^2}{(KT)^2}\ln{\Lambda_{st}}\Label{eq:sruther}\ln{\Lambda_{ij}}\propto\ln{[1+0.18769(\frac{4KT\lambda}{Z_iZ_je^2})]}\Label{eq:clog}
\end{align*}

\end{frame}

\begin{frame}{Trasferimento impilso nelle colliioni}

Termine collisionale $C(f)$ (urti con la specie j con distribuzione $f_j$):
\begin{equation}
C(f_i,f_j)=\int\,d^3p_j\,d\sigma|\vec{v}_i-\vec{v}_j|(f_i'f_j'-f_if_j)
\end{equation}

La forza netta dovuta agli urti \'e:
\begin{align}
&\vec{R}_{ij}=\int m\vec{v}C(f_i,f_j)\,d^3v_i%\label{eq:friction}
\vec{R}_{ij}=n_in_j\mu_{ij}\alpha_{ij}\vec{V}_{ij}\label{eq:resistance}\\
&\alpha_{ij}=\frac{\mu_{ij}}{KT}\int v_r^3\sigma^Tf^{(0)}(\vec{v_r})\,d^3v_r%\label{eq:collisionintegral}
\sigma^T=\int(1-\cos{\chi})\,d\sigma\label{eq:sigmatransport}
\end{align}

\begin{block}{Diffusione termica}

La diffusione termica concentra le particelle pi\'u pesanti (e pi\'u cariche) nelle zone pi\'u calde: in presenza di un gradiente termico si ha un trasferimento netto di momento negli urti in direzione del gradiente, in \eqref{eq:friction}, dovuto al maggior numero di particelle energetiche provenienti dalle regioni pi\'u calde, ci\'o \'e dovuto alla dipendenz della sezione d'urto coulombiana dalla velocit\'a relativa $\propto v\expy{-4}$ (probabilit\'a di collisione $\propto v\expy{-3}$).

\end{block}

\end{frame}

\begin{comment}

Considero il problema in cui le due specie hanno velocit\'a relativa media diversa da zero ma piccola rispetto alla velocit\'a termica: nel sistema in cui la prima specie ha velocit\'a media nulla la seconda ha velocit\'a $V_{ij}$ quindi la distribuzione di velocit\'a della prima \'e la distribuzione di equilibrio a temperatura T quella della seconda \'e  la distribuzione di equilibrio a temperatura T traslata di $V_{ij}$, velocit\'a di diffusione:
\begin{equation}
f_j=f_j^{(0)}+\frac{m_j}{KT}(\vec{V}_{ij}\cdot\vec{v}_j)f_j^{(0)}
\end{equation}
da cui si ottiene:
\begin{align}
&\vec{R}_{ij}=n_in_j\mu_{ij}\alpha_{ij}\vec{V}_{ij}\label{eq:resistance}\\
&\alpha_{ij}=\frac{\mu_{ij}}{KT}\int v_r^3\sigma^Tf^{(0)}(\vec{v_r})\,d^3v_r\label{eq:collisionintegral}\\ &\sigma^T=\int(1-\cos{\chi})\,d\sigma\label{eq:sigmatransport}
\end{align}

\end{comment}

\begin{frame}{Sedimentazione gravitazionale}

Nella sedimentazione gravitazionale la forza per unit\'a di volume agente sulle particelle di specie i \'e
\begin{equation}
\vec{F}_i=-\nabla P_i+n_i(q_i\vec{E}+m_i\vec{g})
\end{equation}
e in condizioni di equilibrio il momento trasferito tramite urti con le altre specie \'e uguale alla forza per unit\'a di volume:
\begin{equation}
\vec{F}_i=\sum_{i\neq j}\vec{R}_{ij}%\vec{F}_{ij}=m_{ij}n_in_j\alpha_{ij}\vec{v}_{ij}
\end{equation}


In un plasma costituito da idrogeno, elio ed elettroni per mantenere la neutralit\'a la velocit\'a di diffusione degli elettroni \'e dello stesso ordine di grandezza di quella degli ioni, quindi l'impulso trasferito \'e trascurabile, da cui:
\begin{align}
&F_H\approx F_{HHe}=-\PDy{r}{P_H}+n_H(eE+m_Hg)\\
&F_{He}\approx -F_{HHe}=-\PDy{r}{P_{He}}+n_{He}(2eE+4m_Hg)\\
&E=-\frac{1}{en_e}\PDy{r}{P_e}\\
&v_{HHe}=-\frac{m_{He}T}{m_{HHe}Y\rho \alpha_{HHe}}[\PDy{r}{\ln{(P_eP_H)}}-m_Hg],\ 
n_Hv_H=\frac{(m_{He}n_Hn_{He})}{\rho}v_{HHe}\\
\end{align}

La velocit\'a di diffusione degli elementi pesanti \'e determinata prevalentemente dagli urti con H e He.
Indico con $\eta(A,r)$ la forza per unit\'a di volume sull'elemento di numero atomico Z e di massa A:
\begin{equation}\label{eq:forceperVheavy}
\begin{split}
&\eta(A,r)=-\PDy{r}{P_A}+n_A(Z_Ae\vec{E}+Am_H\vec{g})\\
&=-n_Ak_BT(\PDy{r}{\ln{P_A}}+Z_A\PDy{r}{\ln{P_e}}+\frac{AGm_HM}{r^2k_BT})
\end{split}
\end{equation}

In condizioni stazionarie si ha:
\begin{equation}\label{eq:diffheavystatinary}
\vec{F}_A\approx\vec{R}_{A,H}+\vec{R}_{A,He}=\eta(A,r)
\end{equation}
e poich\'e
\begin{equation}
\frac{\eta(A,r)}{n_Av_H(m_{A,H}n_H\alpha_{A,H})}\propto\frac{1}{XZ_A}
\end{equation}
posso trascurare $\eta(A,r)$ in \eqref{eq:diffheavystatinary} ed esplicitando i contributi $R_{A,H}$ e $R_{A,He}$
ottengo
\begin{equation}\label{eq:diffvelocityA}
v_A(1+2\frac{Y}{X})\approx-v_H
\end{equation}
%&n_Av_A(m_{AH}n_Hw_{AH})(1+2\frac{Y}{X})\approx-n_Av_H(m_{AH}n_Hw_{AH})

\end{frame}

\begin{comment}%%Contributi alla velocit\'a di diffusione di H-He in modello solare. Da \cite{wam88hydrogen}%
\begin{minipage}{\linewidth}
\begin{tikzpicture}
\node[inner sep=0pt] (image) at (0,0)
  {\includegraphics[keepaspectratio=true,width=0.5\textwidth]{Hdiffusion}};
  \draw [thick,dotted] (-3.4,1.5) -- (-3,1.5) node[right] {$\propto\nabla c$};
 \draw [thick,dashed] (-3.4,1.9) -- (-3,1.9) node[right] {$\propto\nabla T$};
    \draw [thick,dash dot] (-3.4,2.3) -- (-3,2.3) node[right] {$\propto\nabla P$};
    \node (caption) at (8.5,-2.5) { \begin{minipage}[c]{0.48\textwidth}
\captionof{figure}{Contributi alla velocit\'a di diffusione di H-He in modello solare. Da \cite{wam88hydrogen}.}%   
    \end{minipage}};
\node[] (massconsdiff) at (8.5,1) {\begin{minipage}[c]{0.48\textwidth}
\end{minipage}
};
\end{tikzpicture}
\end{minipage}
\end{comment}


\section{Trasporto di energia}

\subsection{Trasporto radiativo.}

Nell'interno stellare il cammino libero medio dei fotoni \'e molto corto $\frac{1}{\kappa\rho}\approx\SI{1}{\cm}\ll \rsun{}$, con $\kappa$ opacit\'a radiativa per unit\'a di massa, quindi considero la radiazione localmente in equilibrio con la materia. Il flusso di energia verso la superficie \'e generato da una piccola anisotropia nell'intensit\'a descritta al prim'ordine tramite:
\begin{equation}
I_{\nu}=B(\nu,T)-\frac{1}{\kappa_{\nu}'\rho}\nabla_s B(\nu,T)
\end{equation}
integrando sull'angolo solido, il flusso di energia risulta
\begin{align}
&\vec{F}_{\nu}=-\frac{4\pi}{3\kappa_{\nu}\rho}\nabla B(\nu,T)\shortintertext{ed esplicitando il gradiente termico e integrando sulle frequenze}
&\vec{F}=-[\frac{4\pi}{3\rho}\intzi{}\frac{1}{\kappa_{\nu}}\PDy{T}{B(\nu,T)}\,d\nu]\nabla T\label{eq:radiativeflux}
\end{align}

Definisco l'opacit\'a media di Rosseland:
\begin{equation}
\frac{1}{\kappa}=(\intzi{}\PDy{T}{B(\nu,T)})\expy{-1}\intzi{}\,d\nu\frac{1}{\kappa_{\nu}}\PDy{T}{B(\nu,T)}=(\frac{acT^3}{\pi})\expy{-1}\intzi{}\,d\nu\frac{1}{\kappa_{\nu}}\PDy{T}{B(\nu,T)}\label{eq:rosselandopacity}
\end{equation}
quindi riscrivo \eqref{eq:radiativeflux} utilizzando la pressione di radiazione $P_{rad}=\int\,d\nu\frac{4\pi}{3c}B_{\nu}=\frac{1}{3}aT^4$:
\begin{align}
&\vec{F}=-\frac{4\pi}{3\kappa\rho}\nabla B=-\frac{4\pi}{3\kappa\rho}\nabla B=-\frac{c}{\kappa\rho}\nabla P_{rad}\shortintertext{che per una distribuzione sferica di materia diventa}
&F_r=-\frac{c}{\kappa\rho}\TDof{r}(\frac{1}{3}aT^4)=-\frac{4acT^3}{3\kappa\rho}\TDy{r}{T}\label{eq:radfluxTgradrelation}
\end{align}

Definisco il gradiente radiativo a partire da \eqref{eq:radfluxTgradrelation}
\begin{equation}
\nrad{}=\Dcvar{\PDly{P}{T}}{rad}=\frac{3}{16\pi acG}\frac{\kappa l(r)P}{m(r)T^4}\label{eq:radiativegradient}
\end{equation}
con $l(r)=4\pi r^2F$ luminosit\'a totale in funzione di r.

\subsection{Opacit\'a.}

I fenomeni che contribuiscono all'opacit\'a nel Sole sono:

\begin{itemize}

\item \parbox[t]{\dimexpr\textwidth-\leftmargin}{%
\vspace{-2.5mm}
\begin{wrapfigure}{r}{0.5\textwidth}
\centering
\vspace{-\baselineskip}
\includegraphics[keepaspectratio,width=0.9\linewidth]{opacitylld}
\caption{Profilo radiale di $\kappa$ e $\PDly{T}{\kappa}$. Da \cite{stix91sun}.}
\end{wrapfigure}


Scattering fotone-elettrone (sc). Classicamente \'e descritto come lo scattering di un'onda elettromagnetica piana da parte di un dipolo oscillante, scattering Thomson per $h\nu\ll m_ec^2$

\begin{align}
&\kappa_{\nu}\propto\frac{r_e^2}{\mu_em_u}, \kappa_{sc}=0.20(1+X)\si{\squared\cm\per\gram}\\
&r_e=\SI{2.82e-13}{\cm}
\end{align}

Per $T\geq\SI{e8}{\kelvin}$ il momento trasferito all'elettrone non \'e trascurabile, scattering Compton.

}

\item Brehmstrahlung inverso (ff). L'assorbimento di un fotone da parte di un elettrone libero \'e energeticamente possibile quando l'elettrone \'e vicino ad uno ione, l'opacit\'a in questo caso ha l'andamento

\begin{equation}
\kappa_{ff}\propto\rho T\expy{-\frac{7}{2}}
\end{equation}

si tiene conto degli effetti quantistici tramite un opportuno coefficiente, il fattore di Gaunt.

\item Reazioni di ionizzazione (bf).

\item Transizione elettronica a livelli eccitati (bb).

\item Scattering atomici e ione $H^-$. La presenza di metalli con potenziale di ionizzazione minore di H ed He rendono disponibile elettroni per la formazione di $H^-$, sistema debolmente legato per cui un fotone con $h\nu>\SI{0.75}{\ev}$ o $\lambda<\SI{1655}{\nano\meter}$ pu\'o essere assorbito.

\end{itemize}

La conduzione pu\'o essere trascurata in quanto nel Sole $l_{\Pphoton}\gg l_{\Pelectron}$, cio\'e i tempi caratteristici per il trasporto di calore per conduzione sono molto maggiori di quelli radiativi.

%\begin{equation}
%\delta\kappa(r)=\frac{\kappa_{OPAL}(\bar{T}(r),\bar{\rho}(r),\bar{Y}(r),\bar{Z}(r))}{\kappa_{OP}(\bar{T}(r),\bar{\rho}(r),\bar{Y}(r),\bar{Z}(r))}-1\leq0.025
%\end{equation}

\begin{tikzpicture}[]
%([shift={(1.5,0)}]0,0)

\node[anchor=north west] (opint) at (0,0) {\includegraphics[height=0.35\textheight,keepaspectratio]{opcontrib-int-g}};
\node[anchor=west,right=2.5cm of opint.east] (opout) {\includegraphics[height=0.35\textheight,keepaspectratio]{opcontrib-out-g}};
\begin{scope}[scale=0.6]
\node[draw,anchor=west,label={[label distance=2mm]-90:Scattering \Pphoton\Pelectron},minimum size=5mm,below right=1cm and 9mm of opint.east] (sc) {};
\node[draw,label={[label distance=2mm]-90:ff},fill=black,minimum size=5mm,above=10mm of sc] (ff) {};
\node[draw,label={[label distance=2mm]-90:bb},fill=bb,minimum size=5mm,above=10mm of ff] (bb) {};
\node[draw,label={[label distance=2mm]-90:bf},fill=bf,minimum size=5mm,above=10mm of bb] (bf) {};
\end{scope}

\begin{scope}[node distance=3.62mm]
\node[minimum size=2mm,name=hydrogen, right=6.5mm of opint.south west] {\tiny H};
\node[minimum size=2mm,name=helium, right=of hydrogen.west] {\tiny He};
\node[minimum size=2mm,name=carbonium, right=of helium.west] {\tiny C};
\node[minimum size=2mm,name=nitrum, right=of carbonium.west] {\tiny N};
\node[minimum size=2mm,name=oxygen, right=of nitrum.west] {\tiny O};
\node[minimum size=2mm,name=neon, right=of oxygen.west] {\tiny Ne};
\node[minimum size=2mm,name=sodium, right=of neon.west] {\tiny Na};
\node[minimum size=2mm,name=magnesium, right=of sodium.west] {\tiny Mg};
\node[minimum size=2mm,name=alluminium, right=of magnesium.west] {\tiny Al};
\node[minimum size=2mm,name=silicium, right=of alluminium.west] {\tiny Si};
\node[minimum size=2mm,name=sulfur, right=of silicium.west] {\tiny S};
\node[minimum size=2mm,name=argon, right=of sulfur.west] {\tiny Ar};
\node[minimum size=2mm,name=calcium, right=of argon.west] {\tiny Ca};
\node[minimum size=2mm,name=cromum, right=of calcium.west] {\tiny Cr};
\node[minimum size=2mm,name=manganese, right=of cromum.west] {\tiny Mn};
\node[minimum size=2mm,name=ferrum, right=of manganese.west] {\tiny Fe};
\node[minimum size=2mm,name=nikel, right=of ferrum.west] {\tiny Ni};

\end{scope}
 
\node[anchor=north west, below right=1mm and 0.5cm of opint.south west] {\parbox{\textwidth}{\captionof{figure}{Importanza dei varii contributi all'opacit\'a nell'interno solare; composizione GS98. Da \cite{bla11opacity}.}\label{fig:opacitycontrib} }};
 
\end{tikzpicture}


\subsection{Condizione di in-stabilit\'a dinamica: trasporto convettivo.}

Una regione stellare \'e convettivamente stabile se una perturbazione di densit\'a infinitesima non cresce ad ampiezza finita. Considero l'equazione del moto per blob di materia che subisce spostamento $\Delta r$ dalla  posizione di equilibrio:
\begin{equation}
\rho\PtwoDy{t}{(\Delta r)}=-g\Delta\rho=-g[\Dcvar{\TDy{r}{\rho}}{e}-\Dcvar{\TDy{r}{\rho}}{amb}]\Delta r
\end{equation}

La forza di archimede ha verso opposta alla perturbazione se $[\Dcvar{\TDy{r}{\rho}}{e}-\Dcvar{\TDy{r}{\rho}}{amb}]>0$ .

Considero un'equazione di stato generica $\rho(P,T,\mu)$ e definita tramite:
\begin{align}
&\frac{d\rho}{\rho}=\alpha\frac{dP}{P}-\delta\frac{dT}{T}+\phi\frac{d\mu}{\mu}\label{eq:deltatherm}\\
&P=\frac{\rho\gasconstant{}T}{\mu}\quad\Rightarrow\quad\alpha=\delta=\phi=1
\end{align}

Definisco le lunghezze caratteristiche per variazione di densit\'a e pressione:
\begin{equation}
\densityscale{}=-\frac{dr}{d\ln{\rho}},\ H_P=-\frac{dr}{d\ln{P}}
\end{equation}
e i gradienti termici per il blob, l'ambiente e il gradiente di composizione chimica ambientale
\begin{equation}
\nabla=\Dcvar{\TDly{P}{T}}{amb},\ \nabla_e=\Dcvar{\TDly{P}{T}}{blob},\ \nmu{}=\Dcvar{\TDly{P}{\mu}}{amb}\label{eq:nablavitense}
\end{equation}

Riscrivo l'equazione del moto utilizzando l'equazione di stato, supponendo il moto dell'elemento in equilibrio di pressione con l'ambiente, essendo $\nmu{}_{blob}\approx0$:
\begin{equation}
\PtwoDy{t}{(\Delta r)}=-g\frac{\delta}{H_P}[\nabla_e-\nabla-\frac{\phi}{\delta}\nmu{}]\Delta r\label{eq:galleggiamento}
\end{equation}
suppongo inoltre un moto del blob adiabatico $\nabla_e=\nabla_{ad}$ con
\begin{equation}
\nabla_{ad}=\frac{P\delta}{T\rho c_P}
\end{equation}

Introduco la frequenza di \bv{}:
\begin{equation}
N^2=g(\frac{1}{\Gamma_1P}\TDy{r}{P}-\frac{1}{\rho}\TDy{r}{\rho})=g(\frac{1}{\densityscale{}}-\frac{g}{c_s^2})\label{eq:bvfs}
\end{equation}
$N^2$ rappresenta la massima frequenza sotto cui pu\'o oscillare una particella di fluido sottoposta a onde di gravit\'a mantenendo l'equilibrio di pressione con l'ambiente.

Riscrivo l'equazione \eqref{eq:galleggiamento}
\begin{equation}
\PtwoDy{t}{(\Delta r)}=-N^2\Delta r
\end{equation}
che descrive un comportamento oscillatorio per $N^2>0$, cio\'e una regione solare \'e stabile per convezione se

\begin{equation}
\nrad{}<\nad+\frac{\phi}{\delta}\nmu{}\label{eq:ledoux}
\end{equation}

dove ho usato $\nabla_{amb}=\nrad{}$ definito in \eqref{eq:radiativegradient}, cio\'e il gradiente che si ha nel caso la luminosit\'a si trasportata dai fotoni; nelle zone in cui il criterio di Ledoux (o \sch{} se trascuro il gradiente di $\mu$) non \'e verificato si ha trasporto convettivo.

Le stelle con massa $M\leq1.1\msun{}$ hanno una regione radiativa interna mentre la parte esterna \'e convettiva: la regione in cui vale l'uguaglianza in \eqref{eq:ledoux} \'e la base della zona convettiva, la cui posizione \'e determinata dall'aumentare dell'opacit\'a col diminuire della temperatura.
% e dal gradiente adiabatico, il cui valore \'e diminuito dal calore latente di idrogeno ed elio nelle regioni di ionizzazione parziale.

Una maggiore efficienza del trasporto convettivo di energia si riflette in una minore differenza tra il gradiente di temperature adiabatico ed effettivo.

\begin{figure}[!h]
\begin{subfigure}[b]{0.55\textwidth}
    \includegraphics[ width=0.99\textwidth,keepaspectratio]{proportionflux}
    \subcaption{Profilo radiale (profondit\'a in \si{\kilo\meter}) del flusso convettivo $F_c$ rispetto al flusso totale $F$, della super-adiabaticit\'a $\nabla-\nad{}$ e regioni di ionizzazione idrogeno e $\cel{He}{4}{}{}$. Da \cite{gou76convection}.}
    \label{fluxproportion}
\end{subfigure}
~
\begin{subfigure}[b]{0.4\textwidth}
\centering
\includegraphics[keepaspectratio,width=0.9\textwidth]{specificheatnablaa}
\subcaption{Profilo radiale di $c_P$ e $\nabla_a$: si ha cambiamento di comportamento nelle regioni di ionizzazione parziale di idrogeno ed elio. Da \cite{stix91sun}.}
\end{subfigure}
\end{figure}


\subsection{Teoria della mixing-length.}

In presenza di convezione il flusso di energia verso l'esterno ha una componente radiativa, determinata dal gradiente di temperatura, e una componente dominante convettiva 
\begin{equation}
F=F_{con}+F_{rad}=\frac{\lsun{}}{4\pi r^2}
\end{equation}
Definisco il gradiente radiativo fittizio:
\begin{equation}
F=\frac{4acG}{3}\frac{T^4m}{\kappa Pr^2}\nrad{}\label{eq:fictionrad}
\end{equation}

Per determinare il gradiente di temperatura effettivo $\nabla$ uso la teoria della mixing-length (\cite{prandtl25tur} e \cite{vitense53kon}):
si considera l'eccesso di calore trasportato dai blob di gas nel moto convettivo $c_P\Delta T$ rispetto all'ambiente, il cui cammino libero medio \'e la mixing-length $l_m=\alpha H_P$, che da luogo al flusso di energia

\begin{equation}
F_{con}=\exv{\rho vc_P\Delta T}\label{eq:convectiveflux}
\end{equation}

dove $\exv{}$ indica una media opportuna sulla sfera di raggio r. Determino il valor medio della differenza di temperatura prendendo come valore caratteristico dello spostamento del blob, considerando moti in entrambi i versi, $\Delta r\approx\frac{l_m}{2}$:
\begin{equation}
\frac{\Delta T}{T}\approx\frac{1}{T}\PDy{r}{(\Delta T)}\frac{l_m}{2}=(\nabla-\nabla_e)\frac{l_m}{2}\frac{1}{H_P}
\end{equation}

Assumo il lavoro medio fatto dalla forza di galleggiamento per unit\'a di massa $-g\frac{\Delta\rho}{\rho}$ uguale al valore medio della forza, cio\'e la met\'a di quello alla superficie sferica data, moltiplicato lo spostamento medio $\frac{l_m}{2}$ quindi, assumendo in oltre che in media met\'a del lavoro fatto dalla forza di galleggiamento sia trasformato in energia cinetica del blob si ottiene

\begin{equation}
v^2=g\delta(\nabla-\nabla_e)\frac{l_m^2}{8H_P}\label{eq:blobvelocity}
\end{equation}

Infine determino gli scambi radiative del blob: il modulo del flusso radiativo \'e proporzionale al gradiente termico in direzione normale alla superficie del blob
\begin{equation}
f=\frac{4acT^3}{3\kappa\rho}|\PDy{n}{T}|
\end{equation}
quindi l'energia scambiata dall'intera superficie S del blob \'e $\lambda=Sf$ che determina, per la prima legge della termodinamica, una variazione di temperatura per unit\'a di tempo:
\begin{equation}
\PDy{t}{T_e}=-\frac{\lambda}{\rho Vc_P}
\end{equation}
indicato con $V$ il volume del blob.

La variazione della temperatura del blob per unit\'a distanza percorsa \'e quindi
\begin{equation}
\Dcvar{\TDy{r}{T}}{e}=\Dcvar{\TDy{r}{T}}{ad}-\frac{\lambda}{\rho Vc_Pv}\label{eq:Tchangelength}
\end{equation}
e approssimando il gradiente normale alla superficie con $\exv{\Delta T}$ ed usando le definizioni \eqref{eq:nablavitense} si ottiene:
\begin{equation}
\frac{\nabla_e-\nad{}}{\nabla-\nabla_e}=\frac{6acT^3}{\kappa\rho^2c_Pl_mv}
\end{equation}

Le 5 equazioni \eqref{eq:fictionrad},\eqref{eq:radiativegradient}, \eqref{eq:convectiveflux}, \eqref{eq:blobvelocity}, \eqref{eq:nablavitense} determinano completamente le variabili $F_{rad}, F_{con}, v, \nabla_e, \nabla$ in funzione di $P,T,l(r),m(r),c_P,\nad{},\nrad{},g$ .



\section{Produzione di energia - reazioni di fusione}


\section{Modello solare standard e osservabili sismologiche}


