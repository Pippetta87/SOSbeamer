\begin{comment}
\section{Osservabili stellari/demo beamer}
\begin{frame}<1>[label=noinside]{Modello stellare}{Come indagare la fisica interna a una stella?}
\onslide<1->\begin{block}{Osservabili stellari:}
$L$, $M$, $R$, $T_e$, $(\frac{Z}{X})_{ph}$, $g_{ph}$.
\end{block}
\onslide<1->\begin{block}{Informazioni sulla struttura interna?} Condizione di equilibrio idrostatico
\end{block}
%Teorema Vogt-Russel: $X_i(r)$, $M$ \pause equilibrio (idrostatico/termico) determinano struttura stellare .
%\pause
\onslide<1->\begin{block}{Modello stellare: diagramma di \hr{}.}
\end{block}
\onslide<2->\begin{block}{Descrizione fisica interno stellare: parametri aggiuntivi}
Convezione, diffusione e sedimentazione elementi pesanti, equazione di stato, opacit\'a
\end{block}
\onslide<2->\begin{block}{Astrosismologia}
Restringo spazio parametri sistemi stellari lontani
\end{block}
\end{frame}
{ % all template changes are local to this group.
    \setbeamertemplate{navigation symbols}{}
    \begin{frame}[plain]{Diagramma di \hr{}}
        \begin{tikzpicture}[remember picture,overlay]
            \node[at=(current page.center)] {
                %\includegraphics[width=\paperwidth]{yourimage}
            };
        \end{tikzpicture}
     \end{frame}
}
\againframe<2>{noinside}
\begin{frame}{Pulsazioni stellari}{Modi Normali}
\begin{columns}
\begin{column}{0.5\textwidth}  %%<--- here
    \begin{center}
     %\includegraphics[width=0.5\textwidth]{image1}
     \end{center}
\end{column}
\begin{column}{0.5\textwidth}
\onslide<1-> \begin{block}{Stelle pulsanti}
Onde stazionarie: Pulsazione radiale/non radiale: .
\onslide<2-> meccanismo di eccitazione: solar-like pulsator, Cefeidi.
\onslide<3-> Modo fondamentale $\Pi\approx\tau_{dyn}=\sqrt{\frac{R^3}{GM}}\propto\overline{\rho}\expy{-\frac{1}{2}}$.
\onslide<4-> Modi di oscillazione\onslide<5-> - informazioni sull'interno stellare
\onslide<5-> Elio-sismologia: Modi $\Leftrightarrow$ Modelli solari
\onslide<5-> Astero-sismologia: Modi $\Leftrightarrow$ Spazio parametri modello stellare
\end{block}
\end{column}
\end{columns}
\end{frame}

\end{comment}

\section{Osservabili solari}

\begin{frame}{Dati osservativi}

\begin{block}{Et\'a, luminosit\'a, raggio solari}
\begin{tabular}{l|c}
\hline
$\agesun{}$&\SI[separate-uncertainty=true]{4.57\pm0.02e9}{\year}\\
\hline
$\rsun{}$&\SI{695658+-140}{\kilo\meter}\\
\hline
$G\msun$&\num{132712440018+-8}\SI{e9}{\cubic\meter\per\square\second}\\
\hline
$\lsun{}$&\SI{3.8275+-0.0014e33}{\erg\per\second}\\
\hline
\end{tabular}
%\caption[Osservabili solari principali.]{Osservabili solari principali. \cite{haberreiter2008solving}.}
\label{tab:sunO}
\end{block}

\begin{block}{Simmetria sferica}
Deviazioni da forma sferica trascurabili (campi magnetici, rotazione)
\end{block}

\end{frame}

\begin{frame}{Dati osservativi}

\begin{block}{Composizione chimica}
\begin{itemize}
\item Righe di assorbimento: attuale (non $Y_{ph}$)
\item Meteoriti CI: primordiale (refrattari)
\end{itemize}

\begin{table}[]

\pgfplotstabletypeset[
every head row/.style={
 before row={\toprule &\multicolumn{4}{c|}{Attuale}
 %&\multicolumn{4}{c|}{Primordiale}
 \\\midrule},
 every last row/.style={after row=\bottomrule},
 after row={\midrule}
},
every nth row={2}{before row=\midrule},every last row/.style={after row=\bottomrule},
every first column/.style={column type/.add={|}{}},
every last column/.style={column type/.add={}{|}},
columns/x/.style = {column type/.add={|}{}},
columns/xi/.style = {column type/.add={|}{}},
display columns/0/.style={column name={}},
display columns/1/.style={column name={$X$}},
display columns/2/.style={column name={$Y$}},
display columns/3/.style={column name={$Z$}},
display columns/4/.style={column name={$\frac{Z}{X}$}},
%display columns/5/.style={column name={$X$}},
%display columns/6/.style={column name={$Y$}},
%display columns/7/.style={column name={$Z$}},
%display columns/8/.style={column name={$\frac{Z}{X}$}},
create on use/authors/.style={create col/set list={
%Anders \& Grevesse (1989),Grevesse \& Noels (1993),
Grevesse et al. (1998),Lodders (2003),Asplund et al. (2005),Lodders et al. (2009),\cite{asplund2009chemical},\cite{caffau2011solar}}},
columns/authors/.style={string type},
columns={authors,x, y, z, zx
%,xi,yi,zi, zxi
},
/pgf/number format/precision=4
     ]{asplund.txt} %%%
\captionof{table}{Metallicit\'a attuale determinata da varii autori.}\label{tab:Zhistory}
\end{table}

\end{block}

\end{frame}


\section{Strutture autogravitanti in equilibrio}

\begin{frame}{Distribuzione di massa - Conservazione di massa e momento - tempo scala dinamico}

\begin{block}{Massa}

%\begin{align}
%&dm=4\pi r^2\rho \,dr-4\pi r^2\rho v\,dt\label{eq:massvar}\\
%\end{align}

\begin{equation}
\PDy{t}{\rho}+\nabla\cdot(\rho\vec{v})=0\label{eq:continuityeq}
\end{equation}

\begin{equation}
dm=4\pi r^2\rho \,dr\label{eq:massaguscio}
\end{equation}

\end{block}

\begin{block}{Momento}
\begin{align}
&\rho\TDy{t}{\vec{v}}=-\nabla P+\rho\vec{f}\label{eq:motion}\\
&\vec{g}=-\PDy{r}{\Phi}=-\frac{Gm(r)}{r^2}\hat{r}
\end{align}
\end{block}

\end{frame}

\begin{frame}{Equilibrio idrostatico: $\ddvec{r}=0$.}

\begin{align*}
\nabla P=\rho \vec{f}\Label{eq:idrosta} \TDy{r}{P}=-\frac{Gm(r)\rho(r)}{r^2}\Label{eq:fidroequilibrio}
\end{align*}


Per giustificare l'ipotesi di equilibrio idrostatico stimo i tempi caratteristici di evoluzione della struttura solare nel caso la forza dovuta alla pressione o la forza di gravit\'a non fossero bilanciate, approssimando il valore caratteristico della derivata di due variabili con il rapporto del loro valore caratteristico.

\begin{equation}
\tau_{ff}\approx\tau_{esp}\approx\tau_{idro}^{\odot}= \sqrt{\frac{R^3}{GM}}\approx\frac{1}{2}(G\overline{\rho})\expy{-\frac{1}{2}}\approx\SI{27}{\minute}
\end{equation}

\end{frame}

\subsection{Equazione di stato $P(\rho,T)$}

\begin{frame}{Gas perfetto ioni-elettroni}


\begin{equation}
P_G=P_I+P_e=\frac{\rho}{\mu}\gasconstant{}T
\end{equation}

\begin{block}{Peso molecolare medio}
massa media in amu per particella libera
\begin{align}
&\mu=\frac{1}{\bar{n}_HX+\bar{n}_{He}Y+\bar{n}_{Z}Z}\label{eq:meanmw}\\
&\bar{n}_i=\frac{1+f_i}{A_i}
\end{align}

\end{block}


\end{frame}

\subsection{Energia interna per unit\'a di massa}

\begin{frame}{Energia interna: traslazioni}

\begin{align}
&u=\frac{1}{\rho}\sum_i\int f^{(0)}(\vec{p}_i)\frac{p^2_i}{2m_i}=\frac{3}{2}\frac{P}{\rho}=\frac{3}{2}\frac{\gasconstant T}{\mu}\\
&E_i=\int_0^Mu\,dm=\frac{3}{2}\int_M\frac{P}{\rho}\,dm\label{eq:traslintenergy}
\end{align}

 $f^{(0)}(\vec{p}_i)$ \'e il numero di particelle della specie i per unit\'a di volume con impulso in $[\vec{p},\vec{p}+d\vec{p}]$

\end{frame}


\subsection{Correzioni alla legge dei gas perfetti}

\begin{frame}{Correzioni alla legge dei gas perfetti}

\begin{itemize}
\item Degenerazione elettronica ($\Delta P\leq2\%$).

\begin{align}
&P_{FD}=[\exp{\psi(\rho,T)+\midfrac{p^2}{2mKT}}+1]\expy{-1}\\
&n_e=\frac{\rho N_A}{\mu_e}=\frac{8\pi}{3h^3m_e}(2m_eKT)\expy{\midfrac{3}{2}}F_{\midfrac{3}{2}}(\psi(\rho,T))\\
%n_e=\intzi{}\frac{8\pi p^2\,dp}{h^3(\exp{\frac{u_k}{KT}-\psi}+1)}\\
&P_e=\frac{1}{3}\intzi{}pn_e\TDy{p}{u_k}\,dp
\end{align}

\item Pressione di radiazione: $P_r=\frac{1}{3}aT^4$.

\item Ionizzazione.

\end{itemize}

\end{frame}

\begin{frame}{Correzioni alla legge dei gas perfetti: Interazioni coulombiane}

\begin{align}
&\frac{1}{r_D^2}=\frac{4\pi e^2}{kT}\sum Z^2\overline{n}_Z=\frac{4\pi e^2}{kT}N_A\zeta\\
&\zeta=\sum_{i}(Z_i^2+Z_i)\frac{\rho X_i}{A_i}\xrightarrow{FD}\sum_{i}(Z_i^2+\frac{F_{\midfrac{3}{2}}'(\psi)}{F_{\midfrac{3}{2}}(\psi)}_i)\frac{\rho X_i}{A_i}
\end{align}

\begin{equation}
u_c=\frac{1}{2}\int\phi(\vec{r})\rho_c(\vec{r})\,d^3r,\ P_c=\frac{1}{3}u_c
\end{equation}

Regioni di ionizzazione parziale di idrogeno ed elio

\end{frame}

\begin{frame}{EOS}

\begin{itemize}
\item Schema chimico (MHD): atomi e molecole, stati eccitati e diversi gradi di ionizzazione

\item Schema fisico (OPAL): nuclei ed elettroni, potenziale Coulombiano, Schr\"oedinger per un problema a molti corpi.
\end{itemize}

\begin{figure}[!ht]
\subfigure[Popolazione dei diversi gradi di ionizzazione per $\cel{He}{4}{}{}$, CNO, $\cel{Ne}{20}{}{}$, $\cel{Fe}{56}{}{}$. Da \cite{basu2008helioseismology}.]{\includegraphics[width=0.4\textwidth,keepaspectratio]{ionfraction}}\label{ionfraction}
%\subcaption{Andamento di $\Gamma_1$ calcolato tramite equazione di stato MHD/OPAL. Da \cite{trampedach2006synoptic}.]{\includegraphics[width=0.4\textwidth,keepaspectratio]{ionfraction}}
~
\subfigure[Confronto $\Gamma_1$ MHD/OPAL. Da \cite{trampedach2006synoptic}.]{\includegraphics[width=0.4\textwidth,keepaspectratio]{gamma1eos}}
%\subcaption{Profilo radiale della popolazione dei diversi gradi di ionizzazione per $\cel{He}{4}{}{}$, CNO, $\cel{Ne}{20}{}{}$, $\cel{Fe}{56}{}{}$. Stati di ionizzazione maggiore sono pi\'u interni. Da \cite{basu2008helioseismology}.}\label{ionfraction}
%\subcaption{Andamento di $\Gamma_1$ calcolato tramite equazione di stato MHD/OPAL. Da \cite{trampedach2006synoptic}.}\label{fig:gamma1eos}
\end{figure}


\end{frame}


\subsection{Trasporto dell'energia}


\begin{frame}{T Viriale}

Il teorema del viriale esprime una propriet\'a statistica di particelle interagenti: 

L'energia potenziale gravitazionale della stella \'e
\begin{equation}
\Omega=-\int_0^M\frac{Gm(r)}{r}\,dm\label{eq:energiapg}
\end{equation}

\begin{equation}
\frac{1}{2}\TtwoDy{t}{I}=2E_i+\Omega
\end{equation}
con $E_i$ data da \eqref{eq:traslintenergy} e $I=\int r^2\,dm$. In condizioni stazionarie $\frac{1}{2}\TtwoDy{t}{I}\approx0$:
\begin{equation}
0=\int_M\frac{3P}{\rho}\,dm(r)+\Omega
\end{equation}

Detta $W=E_i+\Omega$ l'energia totale della stella, si ha:
\begin{equation}
\Omega=-2E_i\label{eq:virialegpm}
\end{equation}
e dalla conservazione dell'energia $\TDy{t}{W}+L=0$ segue che durante la fase di collasso prima dell'inizio della sequenza principale met\'a dell'energia gravitazionale viene spesa per aumentare l'energia interna e met\'a in luminosit\'a:
\begin{equation}
L=-\frac{1}{2}\dot{\Omega}=\dot{E}_i
\end{equation}

\end{frame}

\begin{frame}{Struttura in equilibrio idrostatico e termico in assenza di reazioni nucleari}

Nel caso in cui la contrazione gravitazionale sia l'unica fonte di energia per una massa gassosa in equilibrio idrostatico, il suo tempo di evoluzione caratteristico \'e il tempo di \kh{}:
\begin{equation}
\tkh{}=\frac{\Omega}{L}\approx\frac{GM^2}{2RL}\approx\SI{1.6e7}{\year}
\end{equation}

cammino libero medio degli atomi e dei fotoni sia breve, si raggiunge rapidamente l'equilibrio idrostatico e termico locale. Il processo di contrazione gravitazionale continua, su tempi-scala termodinamici, fino a che l'energia prodotta dalle reazioni nucleari bilancia l'energia irradiata.


\end{frame}


\subsection{Conservazione dell'energia interna}

\begin{frame}{Prima legge TD}

$dq$ per unit\'a di massa per $dt$:
\begin{align}
&\TDy{t}{q}=\TDy{t}{u}+P\TDof{t}(\frac{1}{\rho})\label{eq:prima}\\
%\TDy{t}{u}+P\TDy{t}{V}
&\TDy{t}{\ln{T}}=\frac{\Gamma_2-1}{\Gamma_2}\TDy{t}{\ln{P}}+\frac{\TDy{t}{q}}{c_PT}\label{eq:primatemp}\\
&\TDy{t}{\ln{P}}=\Gamma_1\TDy{t}{\ln{\rho}}+\frac{\rho(\Gamma_3-1)}{P}\TDy{t}{q}\label{eq:primapres}
\end{align}
Esponenti adiabatici $\Gamma_i$:
\begin{equation}\label{eq:adibatexp}
\Gamma_1=\Dcvar{\TDly{\rho}{P}}{Ad}, \ \Gamma_3-1=\Dcvar{\TDly{\rho}{T}}{Ad},\ \frac{\Gamma_2-1}{\Gamma_2}=\Dcvar{\TDly{P}{T}}{Ad}
\end{equation}

\end{frame}

\subsection{Equilibrio termico}

\begin{frame}{Equilibrio termico}

Scrivo il bilancio di calore per un elemento di massa unitaria di gas:
\begin{equation}
\TDy{t}{q}=\epsilon-\frac{1}{\rho}\nabla\cdot\vec{F}\label{eq:heatgl}
\end{equation}
dove $\epsilon$ \'e l'energia prodotta per unit\'a di tempo e massa e $\vec{F}$ \'e il flusso di energia verso l'esterno generalmente dovuto alla diffusione di fotoni dalla zona pi\'u calda verso la superficie; sostituendo in \eqref{eq:prima} si ha
\begin{equation}
\TDy{r}{L}=4\pi r^2[\rho\epsilon-\rho\TDof{t}u+\frac{P}{\rho}\TDy{t}{\rho}]\label{eq:fenergyconservation}
\end{equation}

Nel caso stazionario:
\begin{equation}
\TDy{t}{q}=0\ \Rightarrow\ dL=4\pi r^2\rho\epsilon\,dr
\end{equation}
e i processi nucleari che avvengono nella parte centrale forniscono il calore per bilanciare il flusso di energia irradiata.

\end{frame}

\subsection{Diffusione}

\begin{frame}{Diffusione elementi}

Velocit\'a di diffusione relativa (\citetitle{aller1960diffusion}):
\begin{equation}
v_{12}=\frac{1}{n_1}\int\,d^3v_1f_1\vec{v}_1-\frac{1}{n_{2}}\int\,d^3v_{2}f_{2}\vec{v}_{2}
\end{equation}

\begin{itemize}\label{itm:diffusionaller}
\item disomogeneit\'a di composizione
\begin{equation}
\propto\frac{1}{c_1c_{2}}\PDy{r}{c_1}
\end{equation}
\item Disomogeneit\'a pressione e forza per unit\'a di massa $F_i$:
\begin{equation}
\frac{m_{2}-m_1}{c_1m_1+c_{2}m_{2}}\frac{1}{P}\PDy{r}{P}-\frac{m_1m_{2}(\vec{F}_1-\vec{F}_{2})}{KT(c_1m_1+c_{2}m_{2})}
\end{equation}
\item disomogeneit\'a temperatura:
\begin{equation}
\frac{K_T}{n_1n_{2}}\frac{1}{T}\PDy{r}{T}
\end{equation}

\end{itemize}

$(-D_{12})/D_{12}=\frac{1}{3}lv_{th}$ con $l\approx(n\sigma)\expy{-1}$ cammino libero medio


\end{frame}

\begin{frame}{Urti}


\begin{equation}
\TDy{t}{f_i}=\PDy{t}{f_i}+\vec{v}_i\cdot\PDy{\vec{r}}{f_i}+\vec{F}_i\cdot\PDy{\vec{v}}{f_i}=-\Div_{\vec{p}}(\vec{s})=C(f_j)\label{eq:Btransport}
\end{equation}
$\tau_{diff}\approx\SI{6e13}{\year}$: soluzione dell'equazione del trasporto di Boltzmann approssimabile con la distribuzione di equilibrio traslata della velocit\'a di diffusione .

Introduco la sezione d'urto collisionale di Rutherford:
\begin{equation}
d\sigma=\frac{4\pi(Z_iZ_j)^2}{\mu^2(\vec{v}-\vec{v}')^4}\frac{d\chi}{\chi^3}
\end{equation}

Parametro d'impatto massimo: $\lambda=\max{(r_D,a_0)}$: $L=\int\frac{d\chi}{\chi}=\log{(\frac{1}{\chi_{min}})}$

\begin{align}
&\sigma_{ij}\propto \frac{e^4Z_i^2Z_j^2}{(KT)^2}\ln{\Lambda_{st}}\\
&\ln{\Lambda_{ij}}\propto\ln{[1+0.18769(\frac{4KT\lambda}{Z_iZ_je^2})]}
\end{align}

Termine collisionale $C(f)$ (urti con la specie j con distribuzione $f_j$):
\begin{equation}
C(f_i,f_j)=\int\,d^3p_j\,d\sigma|\vec{v}_i-\vec{v}_j|(f_i'f_j'-f_if_j)
\end{equation}

La forza netta dovuta agli urti \'e:
\begin{align}
&\vec{R}_{ij}=\int m\vec{v}C(f_i,f_j)\,d^3v_i\label{eq:friction}\\
&\vec{R}_{ij}=n_in_j\mu_{ij}\alpha_{ij}\vec{V}_{ij}\label{eq:resistance}\\
&\alpha_{ij}=\frac{\mu_{ij}}{KT}\int v_r^3\sigma^Tf^{(0)}(\vec{v_r})\,d^3v_r\label{eq:collisionintegral}\\
&\sigma^T=\int(1-\cos{\chi})\,d\sigma\label{eq:sigmatransport}
\end{align}

\begin{block}{Diffusione termica}

La diffusione termica concentra le particelle pi\'u pesanti (e pi\'u cariche) nelle zone pi\'u calde: in presenza di un gradiente termico si ha un trasferimento netto di momento negli urti in direzione del gradiente, in \eqref{eq:friction}, dovuto al maggior numero di particelle energetiche provenienti dalle regioni pi\'u calde, ci\'o \'e dovuto alla dipendenz della sezione d'urto coulombiana dalla velocit\'a relativa $\propto v\expy{-4}$ (probabilit\'a di collisione $\propto v\expy{-3}$).

\end{block}

\end{frame}

\begin{comment}

Considero il problema in cui le due specie hanno velocit\'a relativa media diversa da zero ma piccola rispetto alla velocit\'a termica: nel sistema in cui la prima specie ha velocit\'a media nulla la seconda ha velocit\'a $V_{ij}$ quindi la distribuzione di velocit\'a della prima \'e la distribuzione di equilibrio a temperatura T quella della seconda \'e  la distribuzione di equilibrio a temperatura T traslata di $V_{ij}$, velocit\'a di diffusione:
\begin{equation}
f_j=f_j^{(0)}+\frac{m_j}{KT}(\vec{V}_{ij}\cdot\vec{v}_j)f_j^{(0)}
\end{equation}
da cui si ottiene:
\begin{align}
&\vec{R}_{ij}=n_in_j\mu_{ij}\alpha_{ij}\vec{V}_{ij}\label{eq:resistance}\\
&\alpha_{ij}=\frac{\mu_{ij}}{KT}\int v_r^3\sigma^Tf^{(0)}(\vec{v_r})\,d^3v_r\label{eq:collisionintegral}\\ &\sigma^T=\int(1-\cos{\chi})\,d\sigma\label{eq:sigmatransport}
\end{align}

\end{comment}

\begin{frame}{Sedimentazione gravitazionale}

Nella sedimentazione gravitazionale la forza per unit\'a di volume agente sulle particelle di specie i \'e
\begin{equation}
\vec{F}_i=-\nabla P_i+n_i(q_i\vec{E}+m_i\vec{g})
\end{equation}
e in condizioni di equilibrio il momento trasferito tramite urti con le altre specie \'e uguale alla forza per unit\'a di volume:
\begin{equation}
\vec{F}_i=\sum_{i\neq j}\vec{R}_{ij}%\vec{F}_{ij}=m_{ij}n_in_j\alpha_{ij}\vec{v}_{ij}
\end{equation}


In un plasma costituito da idrogeno, elio ed elettroni per mantenere la neutralit\'a la velocit\'a di diffusione degli elettroni \'e dello stesso ordine di grandezza di quella degli ioni, quindi l'impulso trasferito \'e trascurabile, da cui:
\begin{align}
&F_H\approx F_{HHe}=-\PDy{r}{P_H}+n_H(eE+m_Hg)\\
&F_{He}\approx -F_{HHe}=-\PDy{r}{P_{He}}+n_{He}(2eE+4m_Hg)\\
&E=-\frac{1}{en_e}\PDy{r}{P_e}\\
&v_{HHe}=-\frac{m_{He}T}{m_{HHe}Y\rho \alpha_{HHe}}[\PDy{r}{\ln{(P_eP_H)}}-m_Hg],\ 
n_Hv_H=\frac{(m_{He}n_Hn_{He})}{\rho}v_{HHe}\\
\end{align}

La velocit\'a di diffusione degli elementi pesanti \'e determinata prevalentemente dagli urti con H e He.
Indico con $\eta(A,r)$ la forza per unit\'a di volume sull'elemento di numero atomico Z e di massa A:
\begin{equation}\label{eq:forceperVheavy}
\begin{split}
&\eta(A,r)=-\PDy{r}{P_A}+n_A(Z_Ae\vec{E}+Am_H\vec{g})\\
&=-n_Ak_BT(\PDy{r}{\ln{P_A}}+Z_A\PDy{r}{\ln{P_e}}+\frac{AGm_HM}{r^2k_BT})
\end{split}
\end{equation}

In condizioni stazionarie si ha:
\begin{equation}\label{eq:diffheavystatinary}
\vec{F}_A\approx\vec{R}_{A,H}+\vec{R}_{A,He}=\eta(A,r)
\end{equation}
e poich\'e
\begin{equation}
\frac{\eta(A,r)}{n_Av_H(m_{A,H}n_H\alpha_{A,H})}\propto\frac{1}{XZ_A}
\end{equation}
posso trascurare $\eta(A,r)$ in \eqref{eq:diffheavystatinary} ed esplicitando i contributi $R_{A,H}$ e $R_{A,He}$
ottengo
\begin{equation}\label{eq:diffvelocityA}
v_A(1+2\frac{Y}{X})\approx-v_H
\end{equation}
%&n_Av_A(m_{AH}n_Hw_{AH})(1+2\frac{Y}{X})\approx-n_Av_H(m_{AH}n_Hw_{AH})

\end{frame}

\begin{comment}%%Contributi alla velocit\'a di diffusione di H-He in modello solare. Da \cite{wam88hydrogen}%
\begin{minipage}{\linewidth}
\begin{tikzpicture}
\node[inner sep=0pt] (image) at (0,0)
  {\includegraphics[keepaspectratio=true,width=0.5\textwidth]{Hdiffusion}};
  \draw [thick,dotted] (-3.4,1.5) -- (-3,1.5) node[right] {$\propto\nabla c$};
 \draw [thick,dashed] (-3.4,1.9) -- (-3,1.9) node[right] {$\propto\nabla T$};
    \draw [thick,dash dot] (-3.4,2.3) -- (-3,2.3) node[right] {$\propto\nabla P$};
    \node (caption) at (8.5,-2.5) { \begin{minipage}[c]{0.48\textwidth}
\captionof{figure}{Contributi alla velocit\'a di diffusione di H-He in modello solare. Da \cite{wam88hydrogen}.}%   
    \end{minipage}};
\node[] (massconsdiff) at (8.5,1) {\begin{minipage}[c]{0.48\textwidth}
\end{minipage}
};
\end{tikzpicture}
\end{minipage}
\end{comment}


\section{Produzione di energia - reazioni di fusione}


\section{Modello solare standard e osservabili sismologiche}


