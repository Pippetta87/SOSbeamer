\begin{comment}
\section{Modi normali della struttura solare}

\begin{frame}[label=noinside]{Modi di oscillazione adiabatici}{Perturbazione dello stato di equilibrio.}

\begin{block}{campi di velocit\'a/effetti non lineari}
Descrivo le oscillazioni come piccole perturbazioni attorno allo stato di equilibrio stazionario (gli effetti non lineari, fra cui lo scambio di energia tra i modi, sono dell'ordine di $\frac{v}{c_s}$ dove v \'e l'ampiezza della velocit\'a dell'oscillazione). 
In generale pu\'o essere presente un campo di velocit\'a $\vec{v}_0$:
\begin{align}
&\vec{v}=\vec{v}_0+\vec{v}'\\
&\TDof{t}=\PDof{t}+(\vec{v}_0\cdot\nabla)
\end{align}
in prima approssimazione prendo $\vec{v}_0=0$ per poi considerare come perturbazioni gli effetti dovuti a campi di velocit\'a in specie rotazione.

\end{block}

\begin{block}{Perturbazione pressione densit\'a}

Indico con $P'(\vec{r},t)$ e $\delta P$ la perturbazione euleriana e lagrangiana della pressione e con $\rho'$, $\Phi'$ e $\vec{g}'$ la perturbazione euleriana della densit\'a , e le perturbazioni euleriane del potenziale gravitazionale e dell'accelerazione di gravit\'a conseguenti  con $\delta\vec{r}=\vec{\xi}$ il vettore spostamento perturbato:
\begin{align}
&P(\vec{r},t)=P_0(\vec{r})+P'(\vec{r},t)\label{eq:pressureperturbation}\\
&\Lvar{P(\vec{r})}=P(\vec{r}+\Lvar{\vec{r}})-P_0(\vec{r})=P'(\vec{r})+\Lvar{\vec{r}}\cdot\nabla P_0\\
&\vec{g}'=-\nabla\Phi',\ \nabla^2\Phi'=4\pi G\rho'\label{eq:gapert}
\end{align}

\end{block}


\end{frame}

\begin{frame}[label=noinside]{Modi di oscillazione adiabatici}{Modi di oscillazione lineari adiabatici.}

\begin{block}{Equazione del moto perturbata}

l'equazione del moto perturbato sostituendo \eqref{eq:pressureperturbation} nell'equazione del moto \eqref{eq:motion} considerando solo i termini lineari nella perturbazione:
\begin{equation}
\rho_0\TDof{t}\vec{v}=\rho_0\PtwoDy{t}{\Lvar{\vec{r}}}=-\nabla P'+\rho_0\vec{g}'+\rho'\vec{g}_0\label{eq:emper}
\end{equation}

\end{block}

\end{frame}

\begin{frame}[label=noinside]{Modi di oscillazione adiabatici}{Equazione di continuit\'a perturbata}

\begin{block}{Equazione di continuit\'a perturbata}

Analogamente per l'equazione di continuit\'a ottengo
\begin{equation}
\rho'+\div{(\rho_0\Lvar{\vec{r}})}=0\label{eq:contper}
\end{equation}

\end{block}

\end{frame}

\begin{frame}[label=noinside]{Modi di oscillazione adiabatici}{Condizione di adiabaticit\'a}


  \begin{overlayarea}{\textwidth}{1cm}
   \only<1>{
   energia interna per unit\'a di massa
\begin{equation}
\TDy{t}{q}=\TDy{t}{u}+P\TDof{t}(\frac{1}{\rho})\label{eq:prima}
\end{equation}

\begin{equation}
\TDy{t}{T}-\frac{\Gamma_2-1}{\Gamma_2}\frac{T}{P}\TDy{t}{P}=\frac{1}{c_P}(\epsilon-\frac{1}{\rho}\scap{\nabla}{F})
\end{equation}
il termine a destra \'e trascurabile:
\begin{equation}
\TDy{t}{q}=0
\end{equation}
   }
   \only<2>{
   Il moto di una elemento di fluido \'e descritto dalla relazione adiabatica
\begin{equation}
\TDy{t}{P}=\frac{\Gamma_1P}{\rho}\TDy{t}{\rho}
\end{equation}
}
   \only<3>{
  
  La condizione di perturbazione adiabatica linearizzata \'e
\begin{align}
&\PDy{t}{\Lvar{P}}-\frac{\Gamma_{1,0}P_0}{\rho_0}\PDy{t}{\Lvar{\rho}}=0\\
&P'+\Lvar{\vec{\xi}}\cdot\nabla P_0=\frac{\Gamma_{1,0}P_0}{\rho}(\rho'+\Lvar{\vec{\xi}}\cdot\nabla\rho_0)\label{eq:adper}
\end{align}

   }
  \end{overlayarea}


\end{frame}


\begin{figure}[!ht]

\subfigure[Distribuzione dei modi con $l\leq300$ nel diagramma $\nu-l$ determinata usando i primi 144 giorni di osservazione di MDI. Da \cite{chr02helioseismology}.]{
\includegraphics[keepaspectratio,width=0.45\textwidth]{midlmodes}}
\label{fig:midlmodes}
~
\subfigure[Modi adiabatici calcolati sulla base di un modello solare. Da \cite{chr02helioseismology}.]{
\includegraphics[keepaspectratio,width=0.6\textwidth]{nrmodesLAWE}\label{fig:nrmodesLAWE}}

\end{figure}
\end{comment}

%%% SOS

\begin{frame}{Oscillazioni dei 5 minuti}

\begin{block}{Comportamento periodico dell'atmosfera solare con grande coerenza spaziale e temporale}

In \citet{lei62velocity} si osserva che la superficie solare ha scale spazio-temporali privilegiate: in particolare \'e presente un comportamento periodico nell'atmosfera a tutte le altezze rilevato tramite effetto doppler. Il periodo \'e di circa 300 secondi e la lunghezza caratteristica di qualche \si{\mega\meter}.

\end{block}

\begin{block}{Modi normali - onde gravo-acustiche in cavit\'a risonanti di varia profondit\'a}

Le oscillazioni solari sono in prevalenza dovute ad onde acustiche, in cui la forza di richiamo \'e prodotta dal gradiente di pressione, modi p, e onde in cui la forza di richiamo \'e la forza di gravit\'a, i modi g: i modi osservabili pi\'u facilmente hanno periodo attorno a \SI{5}{\minute} ci\'o \'e dovuto all'aumento del rumore solare a basse frequenze e ai processi di eccitazione dei modi che trasferiscono la massima energia ai modi di tale periodo.

Il modello proposto da \citet{ulrich70five} e \citet*{stein71five} considera le propriet\'a dei modi normali non radiali di oscillazione del Sole, in particolare usando la relazione di dispersione per onde acustiche, si ha la definizione di cavit\'a risonanti all'interno del Sole: la variazione delle propriet\'a del gas delimita le regioni di propagazione a diverse profondit\'a a seconda delle caratteristiche trasversali del moto.

\end{block}

\end{frame}

\subsection{Perturbazione adiabatiche.}

\begin{frame}{Perturbazioni adiabatiche}


con $\vec{v}'$ perturbazione euleriana della velocit\'a. La variazione di una grandezza euleriana nel riferimento solidale con l'elemento di fluido si esprime tramite
\begin{align}
&\vec{v}=\vec{v}_0+\vec{v}'
&\TDof{t}=\PDof{t}+(\vec{v}_0\cdot\nabla)
\end{align}

\begin{align}
&P(\vec{r},t)=P_0(\vec{r})+P'(\vec{r},t)\label{eq:pressureperturbation}\\
&\Lvar{P(\vec{r})}=P(\vec{r}+\Lvar{\vec{r}})-P_0(\vec{r})=P'(\vec{r})+\Lvar{\vec{r}}\cdot\nabla P_0\\
&\vec{g}'=-\nabla\Phi',\ \nabla^2\Phi'=4\pi G\rho'\label{eq:gapert}
\end{align}
Ricavo l'equazione del moto perturbato sostituendo \eqref{eq:pressureperturbation} nell'equazione del moto \eqref{eq:motion} considerando solo i termini lineari nella perturbazione:
\begin{equation}
\rho_0\TDof{t}\vec{v}=\rho_0\PtwoDy{t}{\Lvar{\vec{r}}}=-\nabla P'+\rho_0\vec{g}'+\rho'\vec{g}_0\label{eq:emper}
\end{equation}

Analogamente per l'equazione di continuit\'a ottengo
\begin{equation}
\rho'+\div{(\rho_0\Lvar{\vec{r}})}=0\label{eq:contper}
\end{equation}

L'equazione di conservazione dell'energia interna \eqref{eq:primatemp}, esplicitando il bilancio di calore \eqref{eq:heatgl}, \'e

\begin{equation}
\TDy{t}{T}-\frac{\Gamma_2-1}{\Gamma_2}\frac{T}{P}\TDy{t}{P}=\frac{1}{c_P}(\epsilon-\frac{1}{\rho}\scap{\nabla}{F})
\end{equation}

Mostro schematicamente che il termine a destra \'e trascurabile su tempi del periodo delle oscillazioni solari.

Entrambi i tempi scala per scambio di calore sono molto maggiori del periodo delle oscillazioni quindi su un periodo i termini dovuti allo scambio di calore sono trascurabili; l'approssimazione adiabatica non \'e pi\'u valida vicino alla superficie solare dove i tempi per lo scambio di calore sono pi\'u brevi.

Il moto di una elemento di fluido \'e descritto dalla relazione adiabatica
\begin{equation}
\TDy{t}{P}=\frac{\Gamma_1P}{\rho}\TDy{t}{\rho}
\end{equation}

La condizione di perturbazione adiabatica linearizzata \'e:
\begin{align}
&\PDy{t}{\Lvar{P}}-\frac{\Gamma_{1,0}P_0}{\rho_0}\PDy{t}{\Lvar{\rho}}=0\\
&P'+\vec{\xi}\cdot\nabla P_0=\frac{\Gamma_{1,0}P_0}{\rho}(\rho'+\vec{\xi}\cdot\nabla\rho_0)\label{eq:adper}
\end{align}

Cerco una soluzione della forma di un'onda stazionaria: esprimo la dipendenza temporale tramite $\exp{i\omega t}$ e angolare tramite le funzioni armoniche sferiche $Y_{lm}(\theta,\phi)$ con:
\begin{align}
&Y_{lm}(\theta,\phi)=(-)^mc_{lm}P_l^m(\cos{\theta})\exp{im\phi}\\
&L^2Y_l^m=-\frac{1}{\sin{\theta}}\PDof{\theta}(\sin{\theta}\PDy{\theta}{Y_l^m})+\frac{1}{\sin^2{\theta}}\PtwoDy{\phi}{Y_l^m}=-r^2\nabla_h^2Y_l^m=l(l+1)Y_l^m\label{eq:SHperturb}
\end{align}
e scrivo la variazione euleriana di densit\'a, pressione e potenziale gravitazionale nella forma
\begin{equation}
(\rho',P',\Phi')=\exp{i\omega t}[\rho'(r),P'(r),\Phi'(r)]Y_l^m
\end{equation}

\end{frame}

\begin{frame}{Frequenze e autofunzioni del vettore perturbato}


Dall'equazione del moto \eqref{eq:emper}, poich\'e le deviazioni dalla simmetria sferica sono in prima approssimazione trascurabili, \'e evidente che:
\begin{equation}
%&\omega^2\vec{\xi}=\nabla a+\frac{N^2c^2}{g}b\frac{\vec{r}}{r}\\
\hat{r}\cdot(\rot{\vec{\xi}})=0\ \Rightarrow\ \PDof{\theta}(\sin{\theta}\xi_{\phi})-\PDy{\phi}{\xi_{\theta}}=0
\end{equation}
ed \'e quindi possibile ricavare la componente tangenziale della perturbazione da una funzione scalare:
\begin{equation}
\vec{\xi}=\exp{i\omega t}(\xi_r(r),\xi_h(r)\PDof{\theta},\frac{\xi_h(r)}{\sin{\theta}}\PDof{\phi})Y_l^m(\theta,\phi)
\end{equation}
dove ho scomposto il vettore spostamento perturbato in componente radiale $\xi_r(r)$ e tangenziale $\xi_h(r)$.

Ricavo la componente $\xi_h(r)$ applicando la parte tangenziale della divergenza all'equazione del moto:
\begin{equation}
\xi_h(r)=\frac{L}{r\omega^2}(\frac{P'(r)}{\rho_0}+\Phi'(r))
\end{equation}
infine $\xi_r(r)$, $P'(r)$ e $\Phi'(r)$ sono determinati da
\begin{subequations}\label{eigenomega}
\begin{align}
&\frac{1}{r^2}\TDof{r}(r^2\xi_r)-\frac{\xi_rg}{c^2}+\frac{1}{\rho_0}(\frac{1}{c^2}-\frac{l(l+1)}{r^2\omega^2})P'-\frac{l(l+1)}{r^2\omega^2}\Phi'=0\\
&\frac{1}{\rho_0}(\TDof{r}+\frac{g}{c^2})P'-(\omega^2-N^2)\xi_r+\TDy{r}{\Phi'}=0\\
&\frac{1}{r^2}\TDof{r}(r^2\TDy{r}{\Phi'})-\frac{l(l+1)}{r^2}\Phi'-\frac{4\pi G\rho_0}{g}N^2\xi_r-\frac{4\pi G}{c^2}P'=0
\end{align}
\end{subequations}
con $g=-\frac{1}{\rho_0}\TDy{r}{P_0}$.

Il sistema di equazione \eqref{eigenomega} ha soluzione con le opportune condizioni al contorno per un insieme discreto di valori delle frequenze, $\omega_{nlm}$. L'ordine angolare m non compare nelle equazioni quindi gli autovalori $\omega_{nlm}$ sono $2l+1$ degeneri: la degenerazione \'e rimossa nel caso si tenga conto della rotazione ($\frac{\Omega}{\omega}\approx\num{e-4}$).
%o di effetti gravitazionali di altri corpi.

Condizioni al contorno: sono necessarie 4 condizioni.
\begin{itemize}
\item Due condizioni per $r=0$ selezionano le soluzioni regolari:
\begin{equation}
P'=0,\ \Phi'=0
\end{equation}
Vicino a zero risulta un andamento asintotico
\begin{equation}
(l\neq0):\ \xi_r\propto r\expy{l-1};\ (l=0):\ \xi_r\propto r;P',\ \Phi'\propto r^l
\end{equation}

\item Alla superficie solare richiediamo la continuit\'a di $\Lvar{\nabla\Phi}$ e che non si abbia propagazione verso l'esterno.
All'esterno della stella ho $\rho'=0$ quindi scelgo la soluzione nulla a infinito dell'equazione di Poisson $\Phi'=Ar\expy{-l-1}$:
\begin{equation}
\TDy{r}{\Phi'}+\frac{l+1}{r}\Phi'=0,\ r=\rsun{}    
\end{equation}
La condizione di non propagazione oltre la fotosfera dipende dalla descrizione dell'atmosfera solare. Nella versione pi\'u semplice impongo che la variazione di pressione sia zero alla superficie perturbata della stella
\begin{align}
&\Lvar{P}=P'+\xi_r\TDy{r}{P}=0
\end{align}
\end{itemize}

\end{frame}

\begin{frame}{Diagramma $\nu-l$ per i modi adiabatici}

\begin{figure}[!ht]
\subfigure[I picchi della densit\'a spettrale si dispongono su creste in cui \'e concentrata la potenza in accordo al modello. Determinata usando i primi 144 giorni di osservazione di MDI con $l\leq300$. Da \cite{chr02helioseismology}.]{\includegraphics[keepaspectratio,width=0.45\textwidth]{midlmodes}}
\label{fig:midlmodes}
~
\subfigure[Modi adiabatici calcolati sulla base di un modello solare. Da \cite{chr02helioseismology}.]{\includegraphics[keepaspectratio,width=0.95\textwidth]{nrmodesLAWE}}
\label{fig:nrmodesLAWE}

\end{figure}

\end{frame}

\subsection{Energia dei modi-effetti di superficie}

\begin{frame}{Energia dei modi}

Indico con $\exv{E_{kin}^{nl}}$ la media temporale dell'energia cinetica del modo
\begin{equation}
E_{kin}^{nl}=\frac{1}{2}\int_V|\vec{v}|^2\rho_0\,dV=\frac{\omega^2}{2}\int_V|\vec{\xi}|^2\rho_0\,dV
\end{equation}
cio\'e
\begin{equation}
\exv{E_{kin}^{nl}}=\frac{1}{\Pi}\int_0^{\Pi}E_{kin}^{nl}=\frac{1}{2}I_{nl}\exv{\dvec{\xi}_{nl}(R)}^2
\end{equation}
dove $\exv{\dvec{\xi}_{nl}(R)}$ indica la velocit\'a quadratica media superficiale e ho definito l'inerzia normalizzata:
\begin{equation}\label{eq:normalizedinertia}
I_{nl}=\frac{1}{M\exv{\vec{\xi}(R)\vec{\xi}^*(R)}}\int_V\,d^3x\rho_0\vec{\xi}\vec{\xi}^*
\end{equation}


\begin{figure}[!ht]

\subfigure[Differenza di fase per il segnale Doppler delle righe $(5930)$ Fe I, pi\'u profonda, e $(5896)$ Na I, pi\'u alta nell'atmosfera. Da \cite{staiger1987observations}.]{\includegraphics[keepaspectratio,width=0.9\textwidth]{phasepropagation}}
\label{fig:phasedifference}
~
\subfigure[In alto: differenze tra frequenze dei modi teoriche e osservate. In basso: differenze moltiplicata per $Q_{nl}$: viene rimossa la dipendenza da l dovuta alla maggiore ampiezza delle autofunzioni nelle regioni superficiali in cui l'approssimazione adiabatica non \'e corretta. Da \cite{rhodesmeasurements}.]{\includegraphics[keepaspectratio,width=0.9\textwidth]{domega}}
\label{fig:nFreqdiff}

\end{figure}

Nella figura (\subref{fig:nrmodesLAWE}) sono mostrate le frequenze dei modi soluzioni del sistema \eqref{eigenomega} calcolati usando le grandezze di equilibrio di un modello solare, in (\subref{fig:midlmodes}) le frequenze misurate e in \subref{fig:nFreqdiff} le differenze tra frequenze predette e osservate.

Gli effetti dovuti alla non corretta descrizione fisica dei modi vicino alla superficie, $r>0.95\rsun{}$ sono maggiori per i modi di l elevato perch\'e confinati in gusci meno profondi al crescere di l. Introduco il rapporto d'inerzia:
\begin{align}
%&\frac{V_{nl}}{V_0(\nu_{nl})}=(Q_{nl})\expy{-\midfrac{1}{2}}\intxt{dove ho introdotto l'inerzia e il rapporto d'inerzia $Q_{nl}$}
&Q_{nl}=\frac{I_{nl}}{I^0_{nl}}\label{eq:surfaceeffects}
\end{align}
$Q_{nl}=1$ per $l=0$ e diminuisce al crescere di $l$; l'ampiezza superficiale relativa al modo radiale estrapolato a frequenza $\nu_{nl}$, $\frac{A_{nl}(\nu_{nl})}{A_{nl}^0(\nu_{nl})}\propto(Q_{nl})\expy{-\midfrac{1}{2}}$, cresce con l.

Il comportamento all'aumentare della frequenza mostrato in \subref{fig:nFreqdiff} \'e dovuto al fatto che per frequenze maggiori la regione in cui si ha riflessione interna \'e pi\'u in alto dove l'approssimazione adiabatica \'e meno valida; la figura \subref{fig:phasedifference} mostra la propagazione di fase ad alte frequenze dovuta ad effetti dissipativi.

\end{frame}

\subsection{Principio variazionale}    %Vedi pg 110 dalsnote.

\begin{frame}{Variazioni nel modello provocano variazioni nelle frequenze dei modi}

EOM linearizzata

\begin{equation}\label{eq:EOMrhoc}%Unno89
-\omega^2\rho_0\xi=\nabla(c_s^2\rho\scap{\nabla}{\xi}+\nabla P\cdot\vec{\xi})-\vec{g}_0\nabla\cdot(\rho_0\vec{\xi})-G\rho_0\nabla(\int_V\frac{\nabla\cdot(\rho_0\vec{\xi})\,d^3r'}{|\vec{r}-\vec{r}'|})
\end{equation}

\citet{Cha64Variational} ha dimostrato che questo costituisce un problema agli autovalori hermitiano per condizioni ai bordi di pressione e densit\'a nulle quindi $\omega^2$ \'e reale e autofunzioni di frequenza caratteristica diversa sono ortogonali:
\begin{equation}
\int\vec{\xi}_i*\vec{\xi}_j\rho\,d^3x=0
\end{equation}

\'E utile scrivere la relazione tra variazioni nell'operatore lineare $L$ e differenze nelle frequenze caratteristiche:
\begin{equation}
(L+\Lvar{L})(\xi+\Lvar{\xi})=-(\omega+\Lvar{\omega})^2(\xi+\Lvar{\xi})\label{eq:EMvar}
\end{equation}
quindi considerando i termini lineari nelle variazioni e dato che le frequenze caratteristiche sono stazionarie per variazioni di $\xi$ si ha la relazione:
\begin{equation}\label{eq:variational}
\frac{\Lvar{\omega}}{\omega}=-\frac{\int_V\rho\vec{\xi}\Lvar{L}\vec{\xi}\,d^3x}{2\omega^2\int_V\rho\scap{\xi}{\xi}d^3x}
\end{equation}

\end{frame}

\subsection{Variazioni struttura idrostatica e variazioni nell'equazione di stato e nella  composizione}

\begin{frame}{Variazioni struttura idrostatica}

Esplicito la forma della perturbazione $\delta L$ in \eqref{eq:EOMrhoc} introducendo piccole variazioni nel profilo di $\rho$ e $c_s$:
\begin{equation}
\Lvar{L}\vec{\xi}=\nabla(\Lvar{c_s^2}\nabla\cdot\vec{\xi}+\Lvar{\vec{g}}\cdot\vec{
\xi})+\nabla(\frac{\Lvar{\rho}}{\rho_0})c^2\nabla\cdot\vec{\xi}+\frac{1}
{\rho_0}\nabla\rho_0\Lvar{c_s^2}\nabla\vec{\xi}+\Lvar{\vec{g}}\nabla\cdot\vec{\xi}-
G\nabla\int_V\frac{\nabla\cdot(\Lvar{\rho}\xi)}{|\vec{x}-
\vec{x}'|}\,d^3x'\label{eq:eqmotvar}
\end{equation}
con
\begin{equation}
\Lvar{g}(r)=\frac{4\pi G}{r^2}\int_0^r\Lvar{\rho}(s)s^2\,ds
\end{equation}

\'E quindi possibile esprimere le differenze nelle frequenze dei modi nella forma
\begin{equation}\label{eq:hydroefdiff}
\frac{\delta\omega_{nl}}{\omega_{nl}}=\int_0^R[K^{nl}_{c^2,\rho}(r)\frac{\delta_rc^2}{c^2}(r)+K^{nl}_{\rho,c^2}(r)\frac{\delta_r\rho}{\rho}(r)]\,dr+I_{nl}\expy{-1}F_{Surf}
\end{equation}

I kernel $K_Q^j$ dipendono dalle autofunzioni del modello solare, il termine $I_{nl}\expy{-1}F_{Surf}(\omega_{nl})$ tiene conto degli effetti superficiali.

\begin{equation}
\Lvar{P}=\int_r^R(g\Lvar{\rho}+\rho\Lvar{g})\,dr\label{eq:pressurecorrho}
\end{equation}

\end{frame}

\begin{frame}{Variazioni equazione di stato e composizione}

Dipendenza di $\Gamma_1(P,\rho,X_i)$ dalla composizioneriscrivendo \eqref{eq:hydroefdiff} nella forma:
\begin{align}
&\frac{\delta\omega_{nl}}{\omega_{nl}}=\int_0^R[K^{nl}_{\Gamma_1,\rho}(r)\frac{\delta_r\Gamma_1}{\Gamma_1}(r)+K^{nl}_{\rho,\Gamma_1}(r)\frac{\delta_r\rho}{\rho}(r)]\,dr+I_{nl}\expy{-1}F_{Surf}(\omega_{nl})\label{eq:invdGammadrho}\\
&\frac{\delta\omega_{nl}}{\omega_{nl}}=\int_0^RK^{nl}_{u,Y}(r)\frac{\delta_ru}{u}(r)\,dr+\int K^{nl}_{Y,u}(r)\delta_rY\,dr+\int_0^RK^{nl}_{c^2,\rho}(r)(\frac{\delta\Gamma_1}{\Gamma_1})_{int}\,dr+I_{nl}\expy{-1}F_{Surf}(\omega_{nl})\label{eq:diffthermo}
\end{align}
dove $(\delta\Gamma_1)_{int}$ \'e la variazione all'equazione di stato a $(P,\rho,Y)$ fissati.

La determinazione dell'abbondanza di elio nella regione convettiva \'e dovuta principalmente alla deviazione da $\Gamma_1=\frac{5}{3}$ nella regione di seconda ionizzazione dell'elio.

\end{frame}


\section{Campo di velocit\'a solare}

\subsection{Analisi fenomeni stazionari stocastici}

\begin{frame}{Misura velocit\'a Doppler}

I fenomeni periodici sulla superficie solare sono osservabili tramite tecniche fotometriche o spettroscopiche, tramite effetto doppler: una parte basilare dell'informazione contenuta nei modi di oscillazione \'e ricavata analizzando  lo spostamento doppler delle righe di assorbimento dei metalli presenti negli strati visibili pi\'u esterni del sole.

L'ampiezza della velocit\'a di oscillazione per il singolo modo \'e  al pi\'u \SI{15}{\cm\per\second} quindi l'effetto doppler causa uno shift, al massimo, di $\frac{\Delta\lambda}{\lambda}\approx\num{e-10}$. Le misure Doppler su luce integrata sull'intero disco solare sono effettuate tramite uno spettroscopio a scattering risonante, in cui lo splitting  Zeeman dovuto a un campo magnetico esterno applicato a vapori di $Na/K$ permette la trasmissione  in due bande molto strette simmetriche rispetto alla linea spettrale di cui si vuole misurare lo shift: la velocit\'a Doppler \'e proporzionale alla differenza di intesit\'a osservata nelle due bande. Il tacometro di Fourier \'e uno degli strumenti pi\'u utilizzati per misure Doppler  con risoluzione spaziale.

Il segnale Doppler osservato \'e proporzionale a:
\begin{equation}
    V_D(\theta,\phi,t)=\sin{\theta}\cos{\phi}\sum_{n,l,m}A_{nlm}c_{lm}P_l^m(\cos{\theta})\cos{(m\phi-\omega_{nlm}t-\beta_{nlm})}
\end{equation}

Il fattore $\sin{\theta}\cos{\phi}$ deriva dalla proiezione della velocit\'a radiale sulla linea di vista: i modi con periodo attorno ai 5 minuti di grado l non elevato causano uno spostamento quasi totalmente radiale:
\begin{align}
&\frac{\delta h|_{rms}}{\delta r|_{rms}}=\frac{\sqrt{l(l+1)}}{\sigma^2}\\
&\sigma^2=\frac{\rsun{}}{G\msun{}}\omega^2\approx1000
\end{align}

Per isolare il contributo di una singola $Y_{l_0m_0}$ considero
\begin{equation}\label{eq:dopplerTS}
V_{l_0m_0}(t)=\int_AV_D(\theta,\phi,t)W_{l_0m_0}(\theta,\phi)\,dA=\sum_{n,l,m}S_{l_0m_0,lm}A_{nlm}(t)\cos{(\omega_{nlm}t+\beta_{nlm,L_0m_0})}
\end{equation}
e ho integrato sul disco solare con $W_{l_0m_0}\approx Y_{l_0m_0}$. $S_{l_0m_0,lm}$ \'e la funzione di risposta che non \'e esattamente $\propto\delta_{ll_0}\delta_{mm_0}$ poich\'e le armoniche sferiche sono ortogonali sull'intera sfera ma contiene contributi da valori di $(l,m)$ vicini.

Considero la trasformata di Fourier di \eqref{eq:dopplerTS} per tempo di osservazione $T$
\begin{equation}
V_{l_0m_0}(\nu)=\int_{-\midfrac{T}{2}}^{\midfrac{T}{2}}V_{l_0m_0}(t)\exp{i2\pi\nu t}\,dt
\end{equation}

\end{frame}

\begin{comment}
\begin{align}
&\delta r|_{rms}=\exv{\vec{\xi}\cdot\hat{r}}=\frac{1}{2}|\tilde{xi}_r|^2\\
&=\frac{1}{\Pi}\int_0^{\pi}\,dt\frac{1}{4\pi}\oint\Re{\tilde{\xi}_rY_{l,m}(\theta,\phi)\exp{-i\omega t}}^2\,d\omega\\
&\delta h_{rms}^2=\exv{|\vec{\xi}_h|^2}=\frac{1}{2}l(l+1)|\tilde{\xi}_h(r)|^2\\
&\frac{\delta h|_{rms}}{\delta r|_{rms}}=\frac{\sqrt{l(l+1)}}{\sigma}
\end{align}
\end{comment}


\begin{frame}{Fenomeni stazionari stocastici}

\begin{figure}[!ht]
\centering
\includegraphics[keepaspectratio,width=0.3\textwidth]{Powerspectraldensity}
\caption{Densit\'a spettrale di oscillatore smorzato di frequenza naturale $\midfrac{\omega_{nl}}{2\pi}$ con forzante stocastica. $P_f$ \'e la DSP della forzante. $P=P_fP_L$. Da \cite{houdek2006stochastic}.}\label{fig:Powerspectraldensity}
\end{figure}

Un segnale di durata T permette una risoluzione $\Delta\omega=\frac{2\pi}{T}$ mentre il limite superiore delle frequenze osservate \'e dato dalla frequenza di Nyquist $\omega_{Ny}=\frac{\pi}{\Delta t}$ con $\Delta t$ risoluzione temporale; nel caso le dimensioni lineari $L_i$ della regione osservata siano piccole da poter trascurare la curvatura si hanno analoghe relazioni per le variabili spaziali cartesiane e vettore d'onda associato:
\begin{equation}
\Delta\omega=\frac{2\pi}{T}\leq\omega\leq\frac{\pi}{\Delta t},\ \Delta k_i=\frac{2\pi}{L_i}\leq k_i\leq\frac{\pi}{\Delta x_i}
\end{equation}

I modi normali sono identificati tramite la distribuzione spaziale delle oscillazioni in superficie, cio\'e tramite $(l,m)$ o, approssimando localmente l'oscillazione con un onda piana, tramite $k_h$, con
\begin{equation}
\vec{k}=k_r\hat{r}+\vec{k}_h,\ k_h^2=\frac{l(l+1)}{r^2}
\end{equation}
e dalla distribuzione delle frequenze, che identifica l'ordine n ovvero il numero di zeri radiali della velocit\'a perturbata.

Considero la trasformata di Fourier di un processo stocastico $x(t)$ campionato per un tempo T:
\begin{align}
&X_T(\nu)=\int_{-\frac{T}{2}}^{\frac{T}{2}}x(t)\exp{i2\pi\nu t}\\
%\intxt{la cui densit\'a spettrale di potenza (DSP) \'e}
&P_T(\nu)=\frac{1}{T}|X_T(\nu)|^2\label{eq:powerspectraldensity}
%&S_T(\nu)=\frac{1}{T}\int_{-T/2}^{T/2}\int_{-T/2}^{T/2}x(t)x(t')\exp{i2\pi\nu t}\exp{i2\pi\nu t}\,dt\,dt'\\
=\int_{-T/2}^{T/2}\left[\frac{1}{T}\int_{-T/2}^{T/2}x(t')x(t'+\tau)\,dt'\right]\exp{i2\pi\nu \tau}\,d\tau\\
&P_T(\nu)=\int_{-T/2}^{T/2}C_T(\tau)\exp{i2\pi\nu\tau}\,d\tau\\
&C(\tau)=E(x(t)x(t+\tau))=\lim_{T\to\infty}{C_T(\tau)}
\end{align}

\end{frame}

\begin{frame}{Densit\'a spettrale per osservazioni sull'intero disco}

\begin{figure}[!ht]
\centering
\includegraphics[keepaspectratio,width=0.48\textwidth]{lowlmodes}
\caption{Densit\'a spettrale modi p di basso grado angolare. Da \cite{chr02helioseismology}.}\label{fig:lowlmodes}
\end{figure}

\end{frame}

\begin{frame}{Forma del picco della densit\'a spettrale di potenza}

\begin{columns}

\begin{column}{0.45\textwidth}

\begin{figure}[!ht]
\centering
\includegraphics[keepaspectratio,width=0.4\textwidth]{modespheomenology}
\caption{Risultati osservativi per $\Gamma$, $\exv{V^2}$ e $E_{nl}$ in funzione della frequenza per modi con $l\approx20$. Da \cite{libbrecht1988solar}.}\label{fig:Powerspectraldensity}
\end{figure}

\end{column}

\begin{column}{0.45\textwidth}

Il valore di aspettazione della densit\'a spettrale per un tempo di osservazione tendente a infinito \'e:
\begin{align}
&P_x(\nu)=\lim_{T\to\infty}{E[P_T(\nu)]}=\int_{-\infty}^{-\infty}C(\tau)\exp{2\pi i\nu\tau}\,d\tau\\
%\intxt{da cui si ha il teorema di Wiener-Khinchin}
&C(\tau)=\int_{-\infty}^{-\infty}P_x(\nu)\exp{-2\pi i\nu\tau}\,d\nu\\
%\intxt{che per $\tau=0$ da l'identit\'a di Parseval:}
&C(0)=\int_{-\infty}^{-\infty}P_x(\nu)\,d\nu=E(x^2)
\end{align}

Il segnale solare osservato \'e sovrapposizione di un segnale armonico e del rumore: la densit\'a spettrale si scrive
\begin{equation}
P_x(\nu)=\sum_i\frac{AB_i\midfrac{\Gamma^2}{4}}{(\nu-\nu_0-\Delta\nu_i)^2+\midfrac{\Gamma^2}{4}}+P_n(\nu)
\end{equation}
dove $A$ \'e l'altezza del picco $\Gamma$ la larghezza a met\'a altezza, i termini $\Delta\nu_i$ e $B_i$ descrivono i lobi parassiti; il secondo termine descrive il rumore solare e strumentale.

Il profilo lorentziano della densit\'a spettrale del segnale periodico deriva dall'assunzione che le autofunzioni dei modi obbediscano all'equazione del moto di un oscillatore armonico smorzato con forzante stocastica. Determino la densit\'a spettrale dell'oscillatore
\begin{equation}
I_{nl}[\TtwoDy{t}{\xi_{nl}}+\Gamma_{nl}\TDy{t}{\xi_{nl}}+\omega_0^2\xi_{nl}]=f(t)
\end{equation}

con $\midfrac{\Gamma}{2}=\Im{\omega}$ costante di smorzamento, inverso del tempo di vita.

Assumendo $\omega_{nl}\gg\Gamma_{nl}$, cio\'e gli scambi di energia tra i modi e la turbolenza hanno tempo caratteristico $\tau_{nad}\gg\Pi_{osc}$, lo spettro in frequenza dell'oscillatore $P(\omega)=\exv{|\xi_{nl}(\omega)|^2}$,vicino a $\omega_0$ \'e della forma:
\begin{equation}
P(\omega)\propto P_LP_f=\frac{\midfrac{\Gamma_{nl}}{2\pi}}{(\omega-\omega_0)^2+\midfrac{\Gamma_{nl}}{4}}P_f
\end{equation}


L'energia dell'oscillatore varia su tempi scala proporzionali a $\invers{\Gamma}_{nl}$ attorno al valor medio $\bar{E}_{nl}$:
\begin{align}
&\TDy{t}{E_{nl}}+\Gamma_{nl}E_{nl}=\exv{\dvec{\xi}_{nl}\cdot\vec{F}}\\
&\exv{{E}_{nl}}=\frac{\exv{\dvec{\xi}_{nl}\cdot\vec{F}}}{\Gamma_{nl}}
\end{align}
dove $\exv{\dvec{\xi}_{nl}\cdot\vec{F}}$ \'e il lavoro della forzante mediato su un periodo.

Il valore di aspettazione dell'energia totale di un modo si ottiene dalle osservazioni tramite:
\begin{equation}
E_{nl}=I_{nl}\exv{V_{nl}^2}=I_{nl}\frac{1}{T_{obs}}\int_{-\infty}^{\infty}|V(\nu)|^2\,d\nu\propto I_{nl}\Gamma_{nl}A_{nl}
\end{equation}
dove $A_{nl}$ \'e l'altezza di picco della densit\'a spettrale di potenza per $V_{l_0m_0}(\nu)$.

\end{column}

\end{columns}

\end{frame}

\subsection{Rotazione}

\begin{frame}{Perturbazioni alle frequenze dei modi}

\begin{equation}
\rho_0(\PDof{t}+\scap{v_0}{\nabla})^2\vec{\xi}\label{eq:inertialtermvf}
\end{equation}

Il Sole \'e un rotatore lento: le osservazioni della superficie mostrano una dipendenza dalla co-latitudine 
\begin{equation*}
\frac{\Omega(\theta)}{2\pi}=\SI{451.5}{\nano\hertz}-\SI{65.3}{\nano\hertz}\cos^2{\theta}-\SI{66.7}{\nano\hertz}\cos^4{\theta}
\end{equation*}
questa legge \'e soggetta a discrepanze e variazioni temporali.

Il campo di velocit\'a rotazionale \'e 
\begin{equation}
\vec{v_0}=\vecp{\Omega}{r}
\end{equation}

Considero il termine dovuto alla rotazione come una piccola correzione alle frequenze dei modi
\begin{equation}
\omega_{n,l,m}=\omega_{(n,l)}+\Delta\omega_{(l,m)}
\end{equation}
Nel caso di rotazione uniforme \'e facile trovare le correzioni dovute alla rotazione: passando al SR corotante $(r',\theta',\phi')=(r,\theta,\phi-\Omega t)$ e data la dipendenza delle oscillazioni dall'angolo azimutale e dal tempo deve essere $\cos{(m\phi'-\omega_{n,l}t)}=\cos{(m\phi-\omega_{n,l,m}t)}$ cio\'e $\Delta\omega_{(l,m)}=m\Omega$.

I modi di oscillazioni sono onde stazionarie nel rifermento corotante mentre nel riferimento inerziale si ha uno spostamento delle frequenze per le onde che hanno una componente parallela all'equatore; le onde prograde/retrograde ($m=\pm l$) hanno shift massimo. 

\end{frame}

\begin{frame}{Effetto Doppler della rotazione sulle frequenze dei modi}

\begin{columns}

\begin{column}{0.5\textwidth}

\'E possibile identificare l'effetto della rotazione effettuando una misura sensibile a onde con direzione di propagazione parallela all'equatore
\begin{equation}
\frac{\Delta\omega}{\omega}=\pm\frac{V_{adv}}{V_{ph}}
%&V_{adv}=\pm\frac{\Delta\omega}{k_h}
\end{equation}
che da una indicazione della velocit\'a media dovuta alla rotazione $V_{adv}$ nella regione in cui \'e confinato il modo con $V_{ph}=\frac{\omega}{k_h}$.

\end{column}

\begin{column}{0.5\textwidth}

\begin{minipage}{0.9\textwidth}
\captionof{figure}{Diagramma $\omega-k_h$ per osservazione di una regione solare $\ang{;;2}\times\ang{;;2}$ mediata in direzione nord-sud: la misura \'e maggirmente sensibile alle onde che si propagano in direzione est-ovest lungo l'equatore. Si osserva lo spostamento dovuto alla rotazione dalla predizione per modello non-rotante (linee continue). Da \cite{rhodes1979new}.}\label{fig:rotationshiftridge}
\end{minipage}

\begin{minipage}{0.9\textwidth}
\includegraphics[keepaspectratio,angle=0,width=0.9\textwidth]{rotationshiftridge}
\end{minipage}

\end{column}

\end{columns}

\end{frame}

\begin{frame}{Principio variazionale per splitting da rotazione}

L'equazione del moto al primo ordine nella perturbazione, con $\alpha=(l,m)$, \'e:
\begin{equation}\label{eq:eomrotation}
\begin{split}
&\rho_0(\omega_{\alpha}^2+2\omega_{\alpha}\Delta\omega_{\alpha})\vec{\xi}=\\
&\nabla P_1-\frac{\rho_1}{\rho_0}\nabla P_0+\rho_0\nabla\Phi_1+2i\omega_{\alpha}\rho_0(\scap{v_0}{\nabla})\vec{\xi}
\end{split}
\end{equation}
dove ho esplicitato il termine inerziale \eqref{eq:inertialtermvf} e usando $(\scap{v_0}{\nabla})\vec{\xi}=im\vec{\Omega}\xi+\vecp{\Omega}{\xi}$, lo splitting dovuto alla rotazione \'e:
\begin{equation}\label{eq:splitfreqrotation}
\begin{split}
&\Delta\omega_{\alpha}=\frac{i\int\rho_0\xi_{\alpha}^*(\scap{v_0}{\nabla})\xi_{\alpha}\,d^3x}{\int\rho_0\xi_{\alpha}^*\xi_{\alpha}\,d^3x}\\
&=\frac{-m\int\rho_0\Omega\xi_{\alpha}^*\xi_{\alpha}\,d^3x+i\int\rho_0\xi_{\alpha}^*(\vecp{\Omega}{\xi_{\alpha}})\,d^3x}{\int\rho_0\xi_{\alpha}^*\xi_{\alpha}\,d^3x}
\end{split}
\end{equation}

La velocit\'a angolare contribuisce a $\Delta\omega_{\alpha}$ negli strati in cui $\xi_{\alpha}$ \'e apprezzabile. Nel caso di rotazione dipendente solo da r si ha che $\Delta\omega_{\alpha}$ \'e lineare in m: ho $2l+1$ frequenze equispaziate e per modi p di alto n approssimo lo splitting delle frequenze con:
\begin{align}
&\delta\omega_{nlm}\approx m\frac{\int_{r_t}^R\Omega(r)\,\frac{dr}{c}}{\int_{r_t}^R\frac{dr}{c}}\\
%\intxt{per $\Omega(r)$ puramente radiale, cio\'e  per una media sulla latitudine; per velocit\'a angolare generale $\Omega(r,\theta)$, usando le propriet\'a dei polinomi di Legendre $P_l^m(x)$:}
&\delta\omega_{nlm}\approx m\frac{\int_{\cos{\Theta}}^{-\cos{\Theta}}(\cos^2{\Theta}-\cos^2{\theta})\expy{-\frac{1}{2}}\int_{r_t}^R(1-\frac{L^2c^2}{r^2\omega^2})\expy{\frac{1}{2}}\Omega(r,\theta)\frac{dr}{c}\,d(\cos{\theta})}{\int_{r_t}^R(1-\frac{L^2c^2}{r^2\omega^2})\expy{\frac{1}{2}}\,\frac{dr}{c}}\\
&\Theta=\sin^{-1}{(\frac{m}{L}).}
\end{align}


Le autofunzioni hanno ampiezza apprezzabile nella fascia compresa tra le latitudini $\pm\Theta$: questo permette la risoluzione anche della dipendenza da $\theta$ della rotazione.

Il problema di trovare $\Omega(r,\theta)$ dalla differenza $\Delta\omega_{\alpha}$ \'e lineare in $\Omega$ quindi $\Delta\omega_{\alpha}\propto\Omega$. Per determinare quindi la rotazione si utilizzano tecniche di inverzione basate sul principio variazionale.


Parametrizzano le frequenze nel multipletto $(n,l)$ tramite:
\begin{equation}
\nu_{nlm}=\nu_{nl0}+\sum_{j=1}^{j_{\max}}a_j(n,l)\pol_j^{(l)}(m)\label{eq:freqmulti}
\end{equation}
e ricavo i coefficienti tramite fit con le frequenze osservate; i coefficienti dispari contengono il contributo della rotazione al termine lineare, i coefficienti pari effetti delle asfericit\'a nella struttura solare e effetti quadratici della rotazione.

\end{frame}

\section{Caratteristiche asintotiche delle oscillazioni adiabatiche}


\subsection{Comportamento asintotico dei modi.}

\begin{frame}{Approssimazione di Cowling}

(\cite{cow41oscillations})
\begin{equation}
\Phi'(r)=-\frac{4\pi G}{2l+1}\left[\frac{1}{r^{l+1}}\int_0^r\rho'(r)r'^{l+2}\,dr'+r^l\int_r^R\frac{\rho'(r')}{r'^{r'^{l-1}}}\,dr'\right]\label{eq:perturbedgravitational}
\end{equation}

da cui si vede che $|\Phi'|$ \'e trascurabile rispetto a $\rho'$ sia nel caso in cui $l\gg1$, per l'esponente crescente con l al denominatore dei due addendi in \eqref{eq:perturbedgravitational}, sia nel caso $|n|\gg1$, per il comportamento rapidamente oscillante di $\rho'$.

Il sistema \eqref{eigenomega} si riduce a:
\begin{subequations}\label{cowosc:main}
\begin{align}
&\frac{1}{r^2}\TDof{r}(r^2\xi_r)-\frac{\xi_rg_0}{c^2}+\frac{1}{\rho_0}(\frac{1}{c^2}-\frac{l(l+1)}{r^2\omega^2})P'=0\label{cowosc:a}\\
&\frac{1}{\rho_0}(\TDof{r}+\frac{g}{c^2})P'-(\omega^2-N^2)\xi_r=0\label{cowosc:b}
\end{align}
\end{subequations}

\end{frame}

\subsection{Relazione di dispersione per i modi gravo-acustici.}

\begin{frame}{Relazione di dispersione}

\begin{block}{Lontano da superficie}

Approssimo il comportamento spaziale delle oscillazioni localmente con un'onda piana $\vec{\xi}\propto\exp{i\scap{k}{x}}$ con $\vec{k}=k_r\hat{r}+\vec{k}_h$.

\'E utile fare l'ulteriore approssimazione, valida per grandi n, di trascurare la variabile perturbata rispetto alla sua derivata radiale, e cos\'i si ottiene
\begin{equation}\label{eq:secondorder}
\TtwoDy{r}{\xi_r}=\frac{\omega^2}{c_s^2}(1-\frac{N^2}{\omega^2})(\frac{S_l^2}{\omega^2}-1)\xi_r=-K(r)\xi_r
\end{equation}
$K(r)>0$: comportamento oscillante.

\begin{equation}
k_r^2=\frac{\omega^2}{c_s^2}(\frac{N^2}{\omega^2}-1)(\frac{S_l^2}{\omega^2}-1)\label{eq:approximatedispersion}
\end{equation}

Le approssimazioni fatte non sono pi\'u valide vicino alla superficie dove il termine in $\invers{H_P}$ non \'e trascurabile per $H_P$ piccolo e vicino al centro solare dove il termine in $\frac{2}{r}$ non \'e trascurabile.

\begin{block}{Frequenza acustica critica}

Considero \eqref{cowosc:main}  per lunghezza d'onda delle perturbazioni molto minore della scala caratteristica di variazione di $N, c$.


Sostituisco
\begin{align}
&\xi_r\propto\rho_0\expy{-\frac{1}{2}}\exp{ik_rr},\ P_1\propto\rho_0\expy{\frac{1}{2}}\exp{ik_rr}\\
%\intxt{dove la dipendenza dalla densit\'a di equilibrio \'e dovuta alla conservazione dell'energia. Definisco la frequenza critica acustica:}
&\omega_A=\frac{c_s}{2\densityscale{}}\sqrt{1-2\TDy{r}{\densityscale{}}}\propto T\expy{-\frac{1}{2}}\label{eq:acusticcutoff}\\
%\intxt{quindi ottengo la relazione di dispersione:}
&k_r^2=\frac{\omega^2-\omega_A^2}{c^2}+S_l\frac{N^2-\omega^2}{c^2\omega^2}=\frac{\omega^2}{c^2}(1-\frac{\omega_{l,+}^2}{\omega^2})(1-\frac{\omega_{l,-}^2}{\omega^2})\label{eq:localdispersion}\\
%\intxt{dove ho definito le frequenze critiche per i modi gravo-acustici}
&\omega_{\pm}=\frac{1}{2}(S_l^2+\omega_c^2)\pm\sqrt{\frac{1}{2}(S_l^2+\omega_c^2)^2-N^2S_l^2}
\end{align}

\end{block}

\end{frame}

\subsection{Cavit\'a risonanti.} \label{sec:resonantcavity} %Definizione frequenze critiche

\begin{frame}{Frequenze critiche}

\begin{minipage}{\linewidth}
\begin{tikzpicture}
\node[inner sep=0pt] (image) at (-0.5,0)
  {\includegraphics[keepaspectratio=true,scale=0.35]{cutoff}};
  \node (legenda) at (5.8,3) { Legenda };
  \draw [draw=black!50] (legenda.north west) rectangle +(0.35\textwidth, -4cm);
  \draw [dotted] (6,2) -- (5.5,2) node[at start, anchor=west] {$\frac{\omega_c}{2\pi}$;};
  \draw [dashed] (7.5,2) -- (7,2) node[at start, anchor=west] {$\frac{S_l}{2\pi}$;};
  \draw [] (9,2) -- (8.5,2) node[at start, anchor=west] {$\frac{N}{2\pi}$;};
  \node at (7.9,0.5) {\parbox{0.32\textwidth}{Le linee orizzontali a \SI{100}{\micro\hertz} e \SI{3000}{\micro\hertz} demarcano le regione in cui sono confinati risp. un modo g e p.}};
  \node (caption) at (7.6,-2.3) { \begin{minipage}[l]{0.3\textwidth}
\captionof{figure}{Frequenze caratteristiche calcolate tramite il modello S. Da \cite{chr02helioseismology}.\label{cutoff}}%   
    \end{minipage}};
\end{tikzpicture}
\end{minipage}

\end{frame}

\begin{frame}{Raggio di inversione del moto}

\begin{figure}[17]{r}{0.4\textwidth}
\centering
\includegraphics[keepaspectratio,angle=0,width=0.4\textwidth]{plowertp}
\caption{Andamento del raggio di inversione del moto in funzione del grado l. Da \cite{dal03notes}.}\label{fig:plowertp}
\end{figure}

Le onde acustiche sono confinate in una regione che \'e limitata superiormente dall'aumento della frequenza critica acustica
\begin{equation}
\omega_A=\frac{c_s}{2\densityscale{}}\sqrt{1-2\TDy{r}{\densityscale{}}}\propto T\expy{-\frac{1}{2}}\label{eq:acusticcutoff}
\end{equation}
 causato dalla diminuzione della temperatura che provoca la riflessione delle onde con periodo attorno ai 5-min, mentre l'aumento della velocit\'a del suono con la profondit\'a e la conseguente rifrazione dell'onda porta a propagazione del moto puramente tangenziale $k_r=0$ nel guscio sferico per cui $c_s=\frac{\omega}{k_h}\approx\omega \frac{r}{L}$ ovvero per $\omega=S_l$ frequenza di Lamb definita da
\begin{equation}
S_l^2=\frac{l(l+1)c_s^2}{r^2}\label{eq:Lambf}
\end{equation}
che riscivo in termini del raggio di inversione del moto:

\begin{equation}
\frac{c(r_t)}{r_t}=\frac{\omega}{L}
\end{equation}

Le regione di propagazione dei modi g sono definite da $\omega<N$: i modi g sono confinati nelle regioni pi\'u interne del Sole.
Le onde di gravit\'a sono presenti nelle regioni in cui il gas \'e neutro o completamente ionizzato ($N^2$ grande) mentre sono riflesse dalle regioni dove $N$ \'e piccolo o immaginario: ionizzazione parziale, instabilit\'a convettiva, centro del Sole.

I modi g sono confinati tra la la parte centrale dove $g\to0$ e il fondo della zona convettiva dove $N^2<0$.

\end{frame}

\begin{frame}{Traiettoria delle onde}

\begin{figure}[!ht]
\includegraphics[keepaspectratio,width=0.8\textwidth]{raypath-gp}
\caption{(a): Percorso di due onda acustiche $p_8(l=2), p_8(l=100)$; (b): Percorso di un'onda di gravit\'a: $g_{10}(l=5)$. Da\cite{gou91seismic}.}
\end{figure}

\end{frame}

