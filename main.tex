\documentclass[10pt,xcolor={usenames},fleqn,mathserif,serif]{beamer}

%%%Usefull link
%tikz-equations:
%http://www.wekaleamstudios.co.uk/posts/creating-a-presentation-with-latex-beamer-equations-and-tikz/

% There are many different themes available for Beamer. A comprehensive
% list with examples is given here:
% http://deic.uab.es/~iblanes/beamer_gallery/index_by_theme.html
% You can uncomment the themes below if you would like to use a different
% one:
%\usetheme{AnnArbor}
%\usetheme{Antibes}
%\usetheme{Bergen}
%\usetheme{Berkeley}
%\usetheme{Berlin}
%\usetheme{Boadilla}
%\usetheme{boxes}
%\usetheme{CambridgeUS}
%\usetheme{Copenhagen}
%\usetheme{Darmstadt}
\usetheme{default}
%\usetheme{Frankfurt}
%\usetheme{Goettingen}
%\usetheme{Hannover}
%\usetheme{Ilmenau}
%\usetheme{JuanLesPins}
%\usetheme{Luebeck}
%\usetheme{Madrid}
%\usetheme{Malmoe}
%\usetheme{Marburg}
%\usetheme{Montpellier}
%\usetheme{PaloAlto}
%\usetheme{Pittsburgh}
%\usetheme{Rochester}
%\usetheme{Singapore}
%\usetheme{Szeged}
%\usetheme{Warsaw}

\hypersetup{pdfpagemode=FullScreen}

\addtobeamertemplate{block begin}{%
  \setlength{\textwidth}{0.95\textwidth}%
  \setlength\abovedisplayskip{0pt}%
}{}


\setbeamertemplate{caption}{\insertcaption}

%% colors
\definecolor{bittersweet}{rgb}{1.0, 0.44, 0.37}
\definecolor{brilliantlavender}{rgb}{0.96, 0.73, 1.0}
\definecolor{antiquefuchsia}{rgb}{0.57, 0.36, 0.51}
\definecolor{violetw}{rgb}{0.93, 0.51, 0.93}
\definecolor{Veronica}{rgb}{0.63, 0.36, 0.94}
\definecolor{atomictangerine}{rgb}{1.0, 0.6, 0.4}
\definecolor{darkgray}{rgb}{0.66, 0.66, 0.66}
\definecolor{brightcerulean}{rgb}{0.11, 0.67, 0.84}
\definecolor{cadmiumorange}{rgb}{0.93, 0.53, 0.18}
\definecolor{ochre}{rgb}{0.8, 0.47, 0.13}
\definecolor{midnightblue}{rgb}{0.1, 0.1, 0.44}
\definecolor{lemon}{rgb}{1.0, 0.97, 0.0}
\definecolor{grey}{rgb}{0.7, 0.75, 0.71}
\definecolor{amber}{rgb}{1.0, 0.75, 0.0}
\definecolor{almond}{rgb}{0.94, 0.87, 0.8}
\definecolor{bf}{RGB}{88, 86, 88}
\definecolor{bb}{RGB}{177, 177, 177}


%%%%%%%%%%%%%%%%%%%%%%%%%%%%%%%%%%% importa pacchetti
\usepackage{usepkg}
%%%%%%%%%%%%%%%%%%%%%%%%%%%%%%%%%%% Funzioni generali
\usepackage{functions}
%http://tex.stackexchange.com/questions/246/when-should-i-use-input-vs-include
\newcommand{\setmuskip}[2]{#1=#2\relax} %%problem usinig mu with calc (req by mathtools) loaded
\usepackage{sources}
%\usepackage{length}
%%%%%%%%%%%%%%%%%%%%%%%%%%%%%%%%%%% Funzioni per questo file main
\usepackage{mathOp}

\def\status{keeptrying}
\def\keeptrying{keeptrying}
\usepackage{LocalF}
%%%%%%%%%%%%%%%%%%%%%%%%%%%%%%%%%

\title{Modi normali di oscillazione del Sole (Presentazione)}

% A subtitle is optional and this may be deleted
\subtitle{Struttura interna e modi di oscillazione}

%\author{F.~Author\inst{1} \and S.~Another\inst{2}}
% - Give the names in the same order as the appear in the paper.
% - Use the \inst{?} command only if the authors have different
%   affiliation.

%\institute[Universities of Somewhere and Elsewhere] % (optional, but mostly needed)
%{
% \inst{1}
% Department of Computer Science\\
%  University of Somewhere
%  \and
%  \inst{2}%
%  Department of Theoretical Philosophy\\
%  University of Elsewhere}
% - Use the \inst command only if there are several affiliations.
% - Keep it simple, no one is interested in your street address.

\date{Appello Luglio, \today}
% - Either use conference name or its abbreviation.
% - Not really informative to the audience, more for people (including
%   yourself) who are reading the slides online

\subject{Eliosismologia}
% This is only inserted into the PDF information catalog. Can be left
% out. 

% If you have a file called "university-logo-filename.xxx", where xxx
% is a graphic format that can be processed by latex or pdflatex,
% resp., then you can add a logo as follows:

% \pgfdeclareimage[height=0.5cm]{university-logo}{university-logo-filename}
% \logo{\pgfuseimage{university-logo}}

% Delete this, if you do not want the table of contents to pop up at
% the beginning of each subsection:
%\AtBeginPart[]
%{
%  \begin{frame}<beamer>{Outline}    %\tableofcontents[currentsection]
%  \end{frame}
%}

\makeatletter
\AtBeginPart{%
  \addtocontents{toc}{\protect\beamer@partintoc{\the\c@part}{\beamer@partnameshort}{\the\c@page}}%
}
%% number, shortname, page.
\providecommand\beamer@partintoc[3]{%
  \ifnum\c@tocdepth=-1\relax
    % requesting onlyparts.
    \makebox[6em]{PART #1:} #2
    \par
  \fi
}
\define@key{beamertoc}{onlyparts}[]{%
  \c@tocdepth=-1\relax
}
\makeatother%

\setbeamertemplate{navigation symbols}{}

\makeatletter
\setbeamertemplate{headline}
{
    \leavevmode%
    \hbox{%Refintro
        \begin{beamercolorbox}[wd=.1\paperwidth,ht=2.25ex,dp=1ex,center]{author in head/foot}%
            \hyperlink{intro}{Intro}
        \end{beamercolorbox}%

 \begin{beamercolorbox}[wd=.1\paperwidth,ht=2.25ex,dp=1ex,center]{author in head/foot}%refs Part 1
            \hyperlink{part:MSS}{MSS}
        \end{beamercolorbox}%

 \begin{beamercolorbox}[wd=.2\paperwidth,ht=2.25ex,dp=1ex,center]{author in head/foot}%refs Part 2
            \hyperlink{part:oscillations}{Oscillazioni lineari adiabatiche}
        \end{beamercolorbox}%
        
         \begin{beamercolorbox}[wd=.2\paperwidth,ht=2.25ex,dp=1ex,center]{author in head/foot}%refs Part 3
            \hyperlink{part:inverseproblem}{Problema inverso}
        \end{beamercolorbox}%inverseproblem
        
        \begin{beamercolorbox}[wd=.35\paperwidth,ht=2.25ex,dp=1ex,right]{date in head/foot}%
            %   \usebeamerfont{date in head/foot}\insertshortdate{}\hspace*{2em}
            \insertframenumber{} \hspace*{2ex}  / \hspace*{2ex} \inserttotalframenumber
            \hspace*{2ex} 
        \end{beamercolorbox}}%
        \vskip0pt%
    }
    \makeatother

\AtBeginSection{\frame{\sectionpage}}

% Let's get started
\begin{document}

\begin{frame}
  \titlepage
\end{frame}


% Section and subsections will appear in the presentation overview
% and table of contents.
%\frame{\tableofcontents[onlyparts]}

\begin{frame}[label={intro}]{Studio delle oscillazioni solari}{Argomenti trattati nella tesina}

Lo studio dei modi adiabatici permette di valutare l'accuratezza del MSS.

\begin{equation*}
c_s^2=\Gamma_1\frac{P}{\rho}
\end{equation*}

%\begin{columns}
%\begin{column}{0.6\textwidth}
\begin{figure}[!ht]
\includegraphics[height=0.45\textheight,keepaspectratio]{dsoundspeedduvall} 

%\end{column}
%\begin{column}{0.4\textwidth}
\captionof{figure}{%Differenza relativa del profilo di $c_s$ (determinata invertendo \eqref{eq:analinversionc}) per frequenze dei modi calcolate con MSS e osservate. La differenza relativa \'e minore del $5\%$.
Da \cite{christensen1985speed}.}
\label{dsoundduvall}
\end{figure}
%\end{column}
%\end{columns}

\end{frame}

\begin{wordonframe}{Intro modi gravo-acustici-risonanza-MSS}

L'eliosismologia nasce all'inizio degli anni '60 quando Leighton e collaboratori osservarono un comportamento periodico nell'atmosfera solare dovuto a modi gravo-acustici, cio\'e onde gravo-acustiche confinate in gusci sferici di diversa profondit\'a che interferiscono costruttuvamente. %La condizione di risonanza radiale, cio\'e che la cavi\'a contenga un numero intero di lunghezze d'onda e tenendo conto di uno sfasamento ai bordi della cavit\'a, permette
Le frequenze osservate compongono la parte ad alte frequenze dello spettro dei modi solari, cio\'e onde acustiche la cui forza di richiamo \'e la perturbazione della pressione, e usando la condizione di risonanza radiale (la cavit\'a contiene un numero intero di lunghezze d'onda e un fattore di fase che tiene conto della riflessione in superficie) \'e possibile ricavare il profilo radiale della velocit\'a del suono all'interno del Sole.

Le piccol differenz nel profilo radiale della velocit\'a del suono ricavato dalle frequenze osservate e quello calcolato tramite un modello solare supporta il modello proposto per spiegare il fenomeno oscillatoria e supporta inoltre il modello solare standard. Descrivo le caratteristiche fondamentali del Sole.


\end{wordonframe}

\part{Modello solare e osservabili sismologiche}\label{part:MSS}
\frame{\partpage}

%\begin{frame}{Argomenti}
%  \tableofcontents[part=1,hideallsubsections%,pausesections
%  ]
%  % You might wish to add the option [pausesections]
%\end{frame}


%\section{Equilibrio idrostatico}

\begin{frame}{Equilibrio idrostatico e termico}

%\begin{equation*}
%\tau_{idro}^{\odot}= \sqrt{\frac{R^3}{GM}}\approx\frac{1}{2}(G\overline{\rho})\expy{-\frac{1}{2}}\approx\SI{27}{\minute}
%\end{equation*}

\begin{block}{Simmetria sferica}
Deviazioni da forma sferica trascurabili (campi magnetici, rotazione)
\end{block}


\begin{block}{Conservazione momento (Macroscopico)}

\begin{equation*}
\TDy{r}{P}=-\frac{Gm(r)\rho(r)}{r^2}%\label{eq:fidroequilibrio}
\end{equation*}

\end{block}



\end{frame}

\begin{wordonframe}{Distribuzione di materia}

A livello macroscopico la distribuzione a simmetria sferica del gas \'e determinata dall'equilibrio tra forza di gravit\'a e gradiente di pressione.

A livello microsopico la composizione chimica \'e modificata da processi di diffusione: in condizioni stazionarie la velocit\'a di diffusione di una specie rispetto alle altre \'e determinata dall'equilibrio tra le forze per unit\'a di volume agenti sulla specie e il trasferimento di momento negli urti con le altre specie; inoltre la presenza di un gradiente termico modifica la probabilit\'a di collisione fra particelle provenienti da zone pi\'u calde e pi\'u fredde.

\end{wordonframe}

\begin{frame}{Conservazione energia interna}

\begin{align*}
&\TDy{t}{q}=\epsilon-\frac{1}{\rho}\nabla\cdot\vec{F}\\%\label{eq:heatgl}
&\TDy{r}{L}=4\pi r^2[\rho\epsilon-\rho\TDof{t}u+\frac{P}{\rho}\TDy{t}{\rho}]%\label{eq:fenergyconservation}
\end{align*}


\end{frame}


%\section{Meccanismi di trasporto dell'energia}

\begin{frame}{Trasporto radiativo}

\begin{block}{Equilibrio termodinamico locale}
%Il flusso di energia verso la superficie \'e generato da una piccola anisotropia nell'intensit\'a descritta al prim'ordine tramite:
\begin{align*}
%&I_{\nu}=B(\nu,T)-\frac{1}{\kappa_{\nu}'\rho}\nabla_s B(\nu,T)\\
%P_{rad}=\int\,d\nu\frac{4\pi}{3c}B_{\nu}=\frac{1}{3}aT^4\\
%\vec{F}=-\frac{4\pi}{3\kappa\rho}\nabla B=-\frac{4\pi}{3\kappa\rho}\nabla B=-\frac{c}{\kappa\rho}\nabla P_{rad}\\
&\frac{dI_{\nu}}{\rho\,ds}=\kappa_{\nu}B_{\nu}(T)-\kappa_{\nu}I_{\nu}\\
%\frac{1}{\kappa}=(\frac{acT^3}{\pi})\expy{-1}\intzi{}\,d\nu\frac{1}{\kappa_{\nu}}\PDy{T}{B(\nu,T)}\label{eq:rosselandopacity}
&\vec{F}=-\frac{4\pi}{3\kappa\rho}\nabla B(T)=-\frac{c}{\kappa\rho}\nabla P_{rad}
\end{align*}
\end{block}

\begin{block}{Gradiente radiativo}

\begin{equation*}
\nrad{}=\Dcvar{\PDly{P}{T}}{rad}=\frac{3}{16\pi acG}\frac{\kappa l(r)P}{m(r)T^4}%\label{eq:radiativegradient}
\end{equation*}

\end{block}

\end{frame}

\begin{wordonframe}{Trasporto radiativo}

Il trasporto radiativo nell'interno solare \'e descritto considerando il flusso di fotoni  dalle regioni interne pi\'u calde alla superficie come un processo diffusivo e poich\'e il cammino libero medio di un fotone $\invers{\kappa\rho}$ \'e molto piccolo. Posso quindi considerare la radiazione localmente in equilibrio con la materia cio\'e emissione di corpo nero descritta dalla funzione di Plank $B_{\nu}(T)$ (emissivit\'a per unit\'a di frequenza:$B(T)=\frac{ac}{4\pi}T^4$). La soluzione dell'equazione del trasporto radiativo \'e la somma di tutti i contributi $B_{\nu}(-\tau_{\nu})$ pesati dall'esponenziale che descrive l'assorbimento del plasma solare e si pu\'o dimostrare che in condizioni solari il flusso verso la superficie \'e proporzionale a $\nabla B_{\nu}(T)$ (gradiente locale: Espansione $B(\tau_{\nu})$) e al flusso totale integrando sulle frequenze. Infine ricordando che la pressione di radiazione \'e $P_{rad}=\int\,d\nu\frac{4\pi}{3c}B_{\nu}=\frac{1}{3}aT^4$ trovo la relazione tra flusso e gradiente termico.

\end{wordonframe}

\begin{frame}{Instabilit\'a convettiva - Gradiente termico nelle regioni convettive}

\begin{align*}
&\rho\PtwoDy{t}{(\Delta r)}=-g\Delta\rho=-g[\Dcvar{\TDy{r}{\rho}}{e}-\Dcvar{\TDy{r}{\rho}}{amb}]\Delta r=-N^2\Delta r\\
&N^2=g(\frac{1}{\Gamma_1P}\TDy{r}{P}-\frac{1}{\rho}\TDy{r}{\rho})=g(\frac{1}{\densityscale{}}-\frac{g}{c_s^2})%\label{eq:bvfs}
\end{align*}

\begin{columns}[t]

\begin{column}{0.6\textwidth}

\begin{figure}[!h]
    \includegraphics[width=0.9\textwidth,keepaspectratio]{proportionflux}
    \caption{Da \cite{christensen1997effects}.}
    \label{fluxproportion}
\end{figure}

\end{column}

\begin{column}{0.4\textwidth}
Instabilit\'a convettiva:
\begin{align*}
&\nrad{}>\nad{}\\
&F_{con}=\exv{\rho vc_P\Delta T}\\%\label{eq:convectiveflux}
&\nabla-\nad=\nabla-\nabla_e+(\nabla_e-\nad{})
\end{align*}


\end{column}

\end{columns}

\end{frame}

\begin{wordonframe}{Instabilit\'a convettiva - Gradiente termico nelle regioni convettive}

La convezione caratterizza le regioni in cui uno spostamento infinitesimo di un blob di gas cresce esponenzialmente a causa della forza di Archimede. Considero il moto del blob in equilibrio di pressione con l'ambiente e uso l'equazione di stato per esprimere la differenza di densit\'a tra blob e ambiente. Considero infine il moto del blob adiabatico ($dq=c_P\,dT-\frac{\delta}{\rho}\,dP$) (e $\nmu{}\approx0$): la condizione di stabilit\'a diventa (criterio di \sch{}) $\nrad{}<\nad{}$

\end{wordonframe}


%\section{Costruzione modello solare}

\begin{frame}{Modello plasma solare}

\begin{block}{Popolazione di stati atomici. EOS e grandezze termodinamiche.}

Interazioni coulombiane:
\begin{align*}
&\frac{1}{r_D^2}=\frac{4\pi e^2}{kT}\sum Z^2\overline{n}_Z=\frac{4\pi e^2}{kT}N_A\sum_{i}(Z_i^2+Z_i)\frac{\rho X_i}{A_i}\\
%&\zeta=\sum_{i}(Z_i^2+Z_i)\frac{\rho X_i}{A_i}\xrightarrow{FD}\sum_{i}(Z_i^2+\frac{F_{\midfrac{1}{2}}'(\psi)}{F_{\midfrac{1}{2}}(\psi)}Z_i)\frac{\rho X_i}{A_i}
\end{align*}

\end{block}

\begin{block}{Interazioni radiazione-materia}

Processi di assorbimento atomico e scattering di elettroni

\end{block}

\begin{block}{Conservazione momento (microscopico)}
\begin{align*}
&\vec{F}_i=-\nabla P_i+n_i(q_i\vec{E}+m_i\vec{g})\\
&=\sum_{i\neq j}\vec{R}_{ij}
\end{align*}

\end{block}

\end{frame}

\begin{wordonframe}{Modellizzazione plasma solare}

L'opacit\'a \'e determinata dai processi di assorbimento e scattering dei fotoni quindi dal numero dalla densit\'a di particelle che danno luogo a un determinato fenomeno di assorbimento o scattering e dalla sezione d'urto per singola particella.

\end{wordonframe}


\begin{frame}{Reazioni nucleari}

\begin{block}{probabilit\'a reazioni di fusione}
\noindent\begin{align*}
&\epsilon_{ij}=Q_{ij}\frac{n_in_j}{\rho(1+\delta_{ij})}\lambda_{ij}=\frac{1}{1+\delta_{ij}}Q_{ij}\frac{\rho N_A^2X_jX_k}{{A_jA_k}}\exv{\sigma v}\\%\label{eq:energyrate}
&E_G=\SI{5.665}{\kilo\ev} A\expy{\frac{1}{3}}T_7\expy{\frac{2}{3}},\ \Delta E=\SI{4.249}{\kilo\ev}W\expy{\frac{1}{6}}T_7\expy{\frac{5}{6}}
\end{align*}

\end{block}

\begin{block}{Fusione idrogeno in elio: catena PP}

\setmuskip{\thinmuskip}{0mu}\setmuskip{\medmuskip}{0mu}
\tikzset{->-/.style={decoration={
  markings,
  mark=at position .5 with {\arrow{>}}},postaction={decorate}},
-->/.style={decoration={
  markings,
  mark=at position .8 with {\arrow{>}}},postaction={decorate}},
box/.style={%
%draw,
minimum width=25mm,%
    minimum height=6mm,%
    align=center}
}

{\centering
\begin{tikzpicture}

\begin{scope}[scale=0.6,transform shape]
\node[box] (pp) at (0,0) {$\Pproton{+}\Pproton{\to}\cel{H}{2}{}{}{+}\Pnue{+}\APelectron$};%%pp
\node[box,right=2cm of pp]  (pep) {$\Pproton{+}\Pproton{+}\Pelectron{\to}\cel{H}{2}{}{}+\Pnue$};%%pep
\coordinate[below=0.3cm of pp] (bpp);
\node[left] at (bpp) {$99.76\%$};
\coordinate[below=0.3cm of pep] (bpep);
\node[right] at (bpep) {$0.24\%$};

\coordinate[] (ttriton) at ($(bpp)!0.5!(bpep)$);
\draw[->-] (pp)--(bpp)--(ttriton);
\draw[->-] (pep)--(bpep)--(ttriton);
\node[box,below=0.3cm of ttriton] (triton) {$\Pproton+\cel{H}{2}{}{}\to\cel{He}{3}{}{}+\Pphoton$};%%triton
\coordinate[below=0.3cm of triton] (btriton);
\draw[-->] (ttriton)--(triton.north);
\draw[->-] (triton.south)--(btriton.north);
\coordinate[left=2.5cm of btriton] (tpp1);
\node[left] at (tpp1) {$83.3\%$};
\coordinate[right=2.0cm of btriton] (tberillium7);
\node[above] at (tberillium7) {$16.7\%$};
\coordinate[right=6.5cm of btriton] (thep);
\node[right] at (thep) {$\num{2e-5}\%$};

\draw[] (btriton)--(tpp1);
\draw[] (btriton)--(tberillium7);
\draw[] (tberillium7)--(thep);
\node[box,below=0.5cm of tpp1,label={[xshift=0.1cm, yshift=-1.5cm]PPI}]  (pp1) {$\cel{He}{3}{}{}+\cel{He}{3}{}{}\to\cel{He}{4}{}{}+2\Pproton$};%%pp1
\node[box,below=0.5cm of tberillium7]  (berillium7) {$\cel{He}{3}{}{}+\cel{He}{4}{}{}\to\cel{Be}{7}{}{}+\Pphoton$};%%berillium7
\node[box,below=0.5cm of thep,label={[xshift=-0.1cm, yshift=-1.5cm]HEP}]  (hep) {$\cel{He}{3}{}{}+\Pproton\to\cel{He}{4}{}{}+\APelectron+\Pnue$};%%hep

\draw[->-] (tpp1)--(pp1.north);
\draw[->-] (tberillium7)--(berillium7.north);
\draw[-->] (thep)--(hep.north);

\coordinate[below=0.3cm of berillium7] (bberillium7);
\coordinate[left=1.5cm of bberillium7] (tlithium7);
\node[left] at (tlithium7) {$99.88\%$};
\coordinate[right=2.0cm of bberillium7] (tboron8);
\node[right] at (tboron8) {$0.12\%$};

\node[box,below=0.5cm of tlithium7]  (li7) {$\cel{Be}{7}{}{}+\Pelectron\to\cel{Li}{7}{}{}+\Pnue$};%%Li7
\node[box,below=0.5cm of li7,label={[xshift=0.1cm, yshift=-1.5cm]PPII}] (pp2) {$\cel{Li}{7}{}{}+\Pproton\to2\cel{He}{4}{}{}$};%% PP2

\node[box,below=0.5cm of tboron8]  (b8) {$\cel{Be}{7}{}{}+\Pproton\to\cel{B}{8}{}{}+\Pphoton$};%%B8
\node[box,below=0.25cm of b8]  (be7) {$\cel{B}{8}{}{}\to\cel{Be}{8}{}{}^*+\APelectron+\Pnue$};%%Be8*
\node[box,below=0.25cm of be7,label={[xshift=0.1cm, yshift=-1.5cm]PPIII}]  (pp3) {$\cel{Be}{8}{}{}^*\to2\cel{He}{4}{}{}$};%%pp3

\draw[->-] (berillium7.south)--(bberillium7);
\draw[] (bberillium7)--(tlithium7);
\draw[] (bberillium7)--(tboron8);

\draw[->-] (tlithium7)--(li7.north);
\draw[->-] (li7.south)--(pp2.north);

\draw[->-] (tboron8.south)--(b8.north);
\draw[->-] (b8.south)--(be7.north);
\draw[->-] (be7.south)--(pp3.north);
\end{scope}

\node[anchor=north ]  at (2,-3.5) {\parbox{0.4\textwidth}{\captionof{figure}{Da \cite{adelberger2011solar}.}}};

\end{tikzpicture}
}

\end{block}

\end{frame}

\begin{wordonframe}{Importanza reazioni catena PP}

La funzione $\epsilon(T,\rho,X_i)$ descrive la produzione di energia. Nella fase attuale il Sole produce circa il $99\%$ della luminosit\'a tramite la catena PP e circa $1\%$ tramite ciclo CNO

\end{wordonframe}

%\section{Equazioni struttura stellare}

\begin{frame}{Costruzione modello solare}%<handout:0 beamer:0>

\begin{block}{Et\'a, luminosit\'a, massa, raggio solari}
\begin{tabular}{l|c}
\hline
$\agesun{}$&\SI[separate-uncertainty=true]{4.57\pm0.02e9}{\year}\\
\hline
$\rsun{}$&\SI{695658+-140}{\kilo\meter}\\
\hline
$G\msun$&\num{132712440018+-8}\SI{e9}{\cubic\meter\per\square\second}\\
\hline
$\lsun{}$&\SI{3.8275+-0.0014e33}{\erg\per\second}\\
\hline
\end{tabular}
%\caption[Osservabili solari principali.]{Osservabili solari principali. \cite{haberreiter2008solving}.}
\label{tab:sunO}
\end{block}


\end{frame}

\begin{wordonframe}{Ipotesi e giustificazione struttura equilibrio idrostatico}

La costruzione di un modello solare \'e molto meno incerta rispetto a una stella qualsiasi poich\'e \'e nota la distanza tramite lo studio delle orbite dei corpi celesti attorno al Sole, l'et\'a di formazione del sistema solare tramite studio dei meteoriti, la luminosit\'a, e il raggio tramite la misura delle dimensioni apparenti del disco solare.

Il modello solare s

\end{wordonframe}

\begin{frame}{Dati osservativi}

\begin{block}{Composizione chimica}

%\begin{itemize}[noitemsep,topsep=0pt,parsep=0pt,partopsep=0pt]
Righe di assorbimento della fotosfera.

Meteoriti CI: primordiale (refrattari).

\end{block}


\begin{table}[]

\pgfplotstabletypeset[
every head row/.style={
 before row={\toprule &\parbox{0.3\textwidth}{Attuale}
 %&\multicolumn{4}{c|}{Primordiale}
 \\\midrule},
 every last row/.style={after row=\bottomrule},
 after row={\midrule}
},
every nth row={2}{before row=\midrule},every last row/.style={after row=\bottomrule},
every first column/.style={column type/.add={|}{}},
every last column/.style={column type/.add={}{|}},
%columns/x/.style = {column type/.add={|}{}},
%columns/xi/.style = {column type/.add={|}{}},
display columns/0/.style={column name={}},
%display columns/1/.style={column name={$X$}},
%display columns/2/.style={column name={$Y$}},
%display columns/3/.style={column name={$Z$}},
display columns/1/.style={column name={$\frac{Z}{X}$}},
%display columns/5/.style={column name={$X$}},
%display columns/6/.style={column name={$Y$}},
%display columns/7/.style={column name={$Z$}},
%display columns/8/.style={column name={$\frac{Z}{X}$}},
create on use/authors/.style={create col/set list={
%Anders \& Grevesse (1989),Grevesse \& Noels (1993),
Grevesse et al. (1998),Lodders (2003),Asplund et al. (2005),Lodders et al. (2009),\cite{asplund2009chemical},\cite{caffau2011solar}}},
columns/authors/.style={string type},
columns={authors,zx
%,xi,yi,zi, zxi
},
/pgf/number format/precision=4
     ]{asplund.txt} %%%
\captionof{table}{Metallicit\'a attuale determinata da varii autori.}\label{tab:Zhistory}
\end{table}

\end{frame}

\begin{wordonframe}{Composizione solare: righe di assorbimento}

Il metodo pi\'u diretto per determinare la composizione solare \'e l'analisi delle righe di assorbimento nell'atmosfera; inoltre \'e possibile la composizione primordiale del sistema solare analizzando la composizione dei meteoriti CI.

L'incertezza sul contenuto di metalli \'e dovuta alle difficolt\'a insite nella descrizione dell'atmosfera.

Le incertezze nella determinazione dei metalli si ripercuotono soprattutto nell'opacit\'a.

\end{wordonframe}

\begin{frame}{Equazioni della struttura solare}

%Determino la struttura solare integrando numericamente le equazioni fondamentali della struttura stellare
\begin{subequations}\label{subeqn:stellarstructure}
\begin{align}
&\TDy{r}{m}=4\pi r^2\rho\\
&\TDy{r}{P}=-\frac{Gm(r)\rho(r)}{r^2}\\
&\TDy{r}{T}=\nabla\frac{T}{p}\TDy{r}{p}\\
&\TDy{r}{L}=4\pi r^2[\rho(\epsilon-\epsilon_{\nu})-\rho\TDof{t}u+\frac{P}{\rho}\TDy{t}{\rho}]
\end{align}
\begin{equation}
\PDy{t}{n_i}+\frac{1}{r^2}\PDof{r}(r^2n_iv_i)=\Dcvar{\PDy{t}{n_i}}{Nucl}\label{eq:difffusionchange}
\end{equation}
\end{subequations}
%con $v_i$ velocit\'a di diffusione specie i. Ottengo il profilo radiale delle grandezze $\{P,m,T,L,X_i\}$, note la metallicit\'a iniziale Z, l'equazione di stato $P(\rho,T,X_i)$, l'opacit\'a $\kappa(P,T,X_i)$, il rate di produzione di energia nucleare per grammo $\epsilon(P,T,X_i)$.

\begin{block}{Condizionial bordo}
\begin{itemize}
    \item Superficie: $T(r=\rsun{})$, $P(r=\rsun{})$
    \item Centro: $l(r=0)$, $m(r=0)$.
\end{itemize}
\end{block}

\end{frame}

\begin{frame}{Calibrazione modello solare}

\begin{block}{Abbondanze elio iniziale}
Un maggiore peso molecolare medio comporta un maggiore temperatura nel core di fusione
\end{block}

\begin{block}{efficienza convezione}
Mixing length $l_m=H_p\alpha$
\end{block}

\begin{block}{diffusione}

\end{block}

\end{frame}

\begin{frame}{Struttura del MSS}

\begin{figure}[!h]
\includegraphics[width=0.75\textwidth,trim=4 4 4 4,clip]{BP00-SSM-R}
\caption{Profilo radiale della densit\'a, abbondanza di idrogeno, luminosit\'a e temperatura; sono indicati i valori centrali e alla base della zona convettiva. Dati da \cite{BP2000}.}
\end{figure}

\end{frame}

\begin{frame}{Caratteristiche zona convettiva}

\begin{figure}[!ht]
        \includegraphics[width=0.5\textwidth,keepaspectratio]{WrbczHHeioniz}
        \caption{Profilo di W calcolato da un modello solare. Da \cite{basu2008helioseismology}.}\label{fig:dlessc}
    
\end{figure}

\begin{equation}
W=\frac{1}{g}\TDy{r}{c^2}%=\Gamma_1(\invers{\PDly{\rho}{P}}-1)
\end{equation}

\end{frame}

\part{Oscillazioni della fotosfera con grande coerenza spaziale e temporale - Modi normali di cavit\'a risonanti dell'interno solare}\label{part:oscillations}

\frame{\partpage}

\begin{frame}{Argomenti}
  \tableofcontents[part=2,hideallsubsections%,pausesections
  ]
  % You might wish to add the option [pausesections]
\end{frame}

\section{Modi normali della struttura solare}

\begin{frame}{Modi normali: perturbazioni di una struttura a simmetria sferica}
\begin{block}{Equazione del moto perturbato linearizzata}
\begin{equation*}
\rho_0\TDof{t}\vec{v}=\rho_0\PtwoDy{t}{\vec{\xi}}=-\nabla P'+\rho_0\vec{g}'+\rho'\vec{g}_0%\label{eq:emper}
\end{equation*}
\end{block}
\begin{block}{Equazione di continuit\'a e del moto perturbate}
\begin{equation*}
\rho'+\div{(\rho_0\Lvar{\vec{r}})}=0%\label{eq:contper}
\end{equation*}
\end{block}
\begin{block}{Condizione di moto adiabatico}
\begin{equation*}
P'+\vec{\xi}\cdot\nabla P_0=\frac{\Gamma_{1,0}P_0}{\rho}(\rho'+\vec{\xi}\cdot\nabla\rho_0)%\label{eq:adper}
\end{equation*}
\end{block}
\end{frame}

\begin{frame}{Modi normali struttura sferica}

\begin{block}{Onde stazionarie}
\begin{align*}
&Y_{lm}(\theta,\phi)=(-)^mc_{lm}P_l^m(\cos{\theta})\exp{im\phi}\\
&(\rho',P',\Phi')=\exp{i\omega t}[\rho'(r),P'(r),\Phi'(r)]Y_l^m
\end{align*}
\end{block}

\begin{block}{Componente tangenziale dello spostamento perturbato}

\begin{align*}
&\vec{\xi}=\exp{i\omega t}(\xi_r(r),\xi_h(r)\PDof{\theta},\frac{\xi_h(r)}{\sin{\theta}}\PDof{\phi})Y_l^m(\theta,\phi)\\
&\xi_h(r)=\frac{L}{r\omega^2}(\frac{P'(r)}{\rho_0}+\Phi'(r))
\end{align*}

\end{block}

\end{frame}


\begin{frame}{Modi normali: equazione fondamentale}

\begin{block}{Modi normali}

\begin{subequations}%\label{eigenomega}
\begin{align*}
&\frac{1}{r^2}\TDof{r}(r^2\xi_r)-\frac{\xi_rg}{c^2}+\frac{1}{\rho_0}(\frac{1}{c^2}-\frac{l(l+1)}{r^2\omega^2})P'-\frac{l(l+1)}{r^2\omega^2}\Phi'=0\\
&\frac{1}{\rho_0}(\TDof{r}+\frac{g}{c^2})P'-(\omega^2-N^2)\xi_r+\TDy{r}{\Phi'}=0\\
&\frac{1}{r^2}\TDof{r}(r^2\TDy{r}{\Phi'})-\frac{l(l+1)}{r^2}\Phi'-\frac{4\pi G\rho_0}{g}N^2\xi_r-\frac{4\pi G}{c^2}P'=0
\end{align*}
\end{subequations}

Condizioni al contorno:

Soluzioni regolari per $r=0$: $P'=0$, $\Phi'=0$.

La condizione di non propagazione oltre la fotosfera: $\Lvar{P}=P'+\xi_r\TDy{r}{P}=0$.

\end{block}

\begin{block}{Approssimazione di Cowling}

\end{block}

\end{frame}

\begin{wordonframe}{Spettro dei modi: frequenze discrete}

Questo \'e il sistema che determina i modi normali: ha soluzioni per $\omega_{nlm}$ discrete. Le condizioni al bordo richiedono regolarit\'a delle perturbazioni in 0 e riflessione totale delle perturbazioni alla superficie solare. $(l,m)$ caratterizzano la dipendenza angolare dei modi mentre n \'e l'ordine radiale che identifica il numero di zeri radiali del vettore spostamento.

Le frequenze sono $2l+1$ degeneri perch\'e il sistema non dipende da m.

L'approssimazione di Cowling, valida per grandi l o n, consiste nel trascurare la perturbazione al potenziale gravitazionale, restringendosi quindi alle prime 2 equazioni.

\end{wordonframe}

\begin{frame}{Spettro dei modi solari}


\begin{columns}

\begin{column}{0.45\textwidth}

\begin{figure}[!ht]

%\includegraphics[keepaspectratio,width=0.95\textwidth]{midlmodes}
%\caption{I picchi della densit\'a spettrale si dispongono su creste in cui \'e concentrata la potenza in accordo al modello. Determinata usando i primi 144 giorni di osservazione di MDI con $l\leq300$. Da \cite{chr02helioseismology}.}\label{fig:midlmodes}

\includegraphics[keepaspectratio,height=0.85\textheight]{nrmodesLAWE}
\caption{Modi adiabatici calcolati sulla base di un modello solare. Da \cite{chr02helioseismology}.}\label{fig:nrmodesLAWE}

\end{figure}

\end{column}

\begin{column}{0.45\textwidth}
\captionof{figure}{In alto: differenze tra frequenze dei modi teoriche e osservate. In basso: differenze moltiplicata per $Q_{nl}$.}

\end{column}

\end{columns}


\end{frame}

\begin{wordonframe}{Spettro dei modi: modi p, modi g, modi f}

La figura mostra lo spettro delle frequenze discrete per cui il sistema dei modi ha soluzione in funzione del grado angolare l. I modi con stesso ordine radiale sono uniti da una linea: nella parte ad alte frequenze abbiamo i modi p, di natura acustica, nella parte a basse frequenze modi g, che sono onde di gravit\'a, e modi f, confinati in uno strato superficiale e dovuti alla rapida variazione di densit\'a in superficie ($\div{\vec{\xi}}\approx0$, $\omega^2=gk_h$). Il comportamento di $\xi_r$ \'e determinato dalle propriet\'a del gas, in particolare come vedremo pi\'u avanti, dall'andamento della velocit\'a del suono e delle frequenze critiche

\end{wordonframe}

\begin{frame}{Cavit\'a risonanti}

\begin{figure}[!ht]
\centering
\includegraphics[keepaspectratio,width=0.6\textwidth]{raypath-gp}
\caption{Da\cite{gou91seismic}.}
\end{figure}
%$\omega_c=\frac{c_s}{2\densityscale{}}\sqrt{1-2\TDy{r}{\densityscale{}}}\propto T\expy{-\frac{1}{2}}$
%$S_l^2=\frac{l(l+1)c_s^2}{r^2}$.
%$N^2=g(\frac{1}{\Gamma_1P}\TDy{r}{P}-\frac{1}{\rho}\TDy{r}{\rho})=g(\frac{1}{\densityscale{}}-\frac{g}{c_s^2})$
\begin{align*}
&\omega_A=\frac{c_s}{2\densityscale{}}\sqrt{1-2\TDy{r}{\densityscale{}}}\propto T\expy{-\frac{1}{2}}\\%\label{eq:acusticcutoff} \intxt{quindi ottengo la relazione di dispersione:}
&k_r^2=\frac{\omega^2-\omega_A^2}{c^2}+S_l\frac{N^2-\omega^2}{c^2\omega^2}\\%\label{eq:localdispersion}\intxt{dove ho definito le frequenze critiche per i modi gravo-acustici}
&\omega\int_{r_1}^{r_2}\sqrt{1-\frac{\omega_A^2}{\omega^2}-\frac{S_l^2}{\omega^2}(1-\frac{N^2}{\omega^2})}\,\frac{dr}{c}\approx\pi(n-\frac{1}{2})%\label{eq:JWKBmode}
\end{align*}

\end{frame}

\begin{frame}{Modello dell'ampiezza superficiale delle oscillazioni}

\begin{columns}

\begin{column}{0.4\textwidth}

\begin{figure}[!ht]
\centering
\includegraphics[keepaspectratio,width=0.85\textwidth]{modespheomenology}
\caption{Da \cite{libbrecht1988solar}.}\label{fig:Powerspectraldensity}
\end{figure}

\end{column}

\begin{column}{0.6\textwidth}

\begin{equation*}
I_{nl}[\TtwoDy{t}{A_{nl}}+\Gamma_{nl}\TDy{t}{A_{nl}}+\omega_0^2A_{nl}]=f(t)
\end{equation*}

Per $\omega_{nl}\gg\Gamma_{nl}$%, cio\'e gli scambi di energia tra i modi e la turbolenza hanno tempo caratteristico $\tau_{nad}\gg\Pi_{osc}$, lo spettro in frequenza dell'oscillatore $P(\omega)=\exv{|\xi_{nl}(\omega)|^2}$,vicino a $\omega_0$ \'e della forma:
\begin{equation*}
P(\omega)\propto P_LP_f=\frac{\midfrac{\Gamma_{nl}}{2\pi}}{(\omega-\omega_0)^2+\midfrac{\Gamma_{nl}}{4}}P_f
\end{equation*}

L'energia dell'oscillatore varia su tempi scala proporzionali a $\invers{\Gamma}_{nl}$ attorno al valor medio $\bar{E}_{nl}$:
\begin{align*}
&\TDy{t}{E_{nl}}+\Gamma_{nl}E_{nl}=\exv{\dvec{\xi}_{nl}\cdot\vec{F}}\\
&\bar{E}_{nl}=\frac{\exv{\dvec{\xi}_{nl}\cdot\vec{F}}}{\Gamma_{nl}}\\
&\bar{E}_{nl}=I_{nl}\exv{V_{nl}^2}=I_{nl}\frac{1}{T_{obs}}\int_{-\infty}^{\infty}|V(\nu)|^2\,d\nu\propto I_{nl}\Gamma_{nl}A_{nl}
\end{align*}
%dove $\exv{\dvec{\xi}_{nl}\cdot\vec{F}}$ \'e il lavoro della forzante mediato su un periodo.



\end{column}

\end{columns}

\end{frame}

\begin{frame}{Spettro delle oscillazioni}

\begin{figure}[!ht]

%\includegraphics[keepaspectratio,width=0.95\textwidth]{midlmodes}
%\caption{I picchi della densit\'a spettrale si dispongono su creste in cui \'e concentrata la potenza in accordo al modello. Determinata usando i primi 144 giorni di osservazione di MDI con $l\leq300$. Da \cite{chr02helioseismology}.}\label{fig:midlmodes}

\includegraphics[keepaspectratio,width=0.95\textwidth]{PSD}
\caption{Da \cite{houdek2006stochastic}.}\label{fig:PSD}

\end{figure}

\begin{equation}
P_x(\nu)=\sum_i\frac{AB_i\midfrac{\Gamma^2}{4}}{(\nu-\nu_0-\Delta\nu_i)^2+\midfrac{\Gamma^2}{4}}+P_n(\nu)
\end{equation}

\end{frame}

\begin{wordonframe}{Forma dei picchi nella PSD}

Le frequenze dei modi sono determinate principalmente misurando lo spostamento doppler di opportune righe di assorbimento in atmosfera ($Ca \lambda=6439 A 129 Km$, $K: 7699 A, 250Km$, $Ni I: 6708 A,300 Km$, $Na D1/D2: 5690 A, 500 Km$). Per identificare modi con $l>2$ \'e necessaria risolouzione spaziale e la proiezione di un'opportuna maschera $W_{l_0m_0}$ isoler\'a il modo con dipendenza spaziale descritta da $Y_{l_0m_0}$. La serie temporale del segnale doppler viene analizzata in termini di componenti armoniche. Le frequenze sono quindi determinate fittando i picchi dello spettro secondo una forma lorentziana  giustificata assumendo che l'ampiezza superficiale dei modi obbedisca all'equazione del moto di un'oscillatore armonico smorzato di frequenza naturale data dalla frequenza del modo e forzante stocastica.

La PSD $P_T(\nu)=\frac{1}{T_{obs}}|X_T(\nu)|^2$ contiene oltre al picco principale (lorentziano) picchi parassiti dovuti a gap nella serie temporale e il contributo del rumore solare.

\end{wordonframe}


\part{Tecniche e risultati di inversione}\label{part:inverseproblem}

\frame{\partpage}

\begin{frame}{Argomenti}
  \tableofcontents[part=3,hideallsubsections%,pausesections
  ]
  % You might wish to add the option [pausesections]
\end{frame}

\section{Principio variazionale}

\begin{frame}{Effetto di perturbazioni nell'equazione del moto sulle frequenze}


\begin{equation}
=\frac{1}{\rho_0}\nabla p'-\vec{g}'-\frac{\rho'}{\rho_0}\vec{g}_0=L(\vec{\xi})\label{eq:eigenhermitian}
\end{equation}

\begin{equation}
(L+\Lvar{L})(\xi+\Lvar{\xi})=-(\omega+\Lvar{\omega})^2(\xi+\Lvar{\xi})\label{eq:EMvar}
\end{equation}
quindi considerando i termini lineari nelle variazioni e dato che le frequenze caratteristiche sono stazionarie per variazioni di $\xi$ si ha la relazione:
\begin{equation}\label{eq:variational}
\frac{\Lvar{\omega}}{\omega}=-\frac{\int_V\rho\vec{\xi}\Lvar{L}\vec{\xi}\,d^3x}{2\omega^2\int_V\rho\scap{\xi}{\xi}d^3x}
\end{equation}

\end{frame}

\begin{frame}{Correzioni struttura idrostatica e composizione}

\begin{equation}\label{eq:EOMrhoc}%Unno89
-\omega^2\rho_0\xi=\nabla(c_s^2\rho\scap{\nabla}{\xi}+\nabla P\cdot\vec{\xi})-\vec{g}_0\nabla\cdot(\rho_0\vec{\xi})-G\rho_0\nabla(\int_V\frac{\nabla\cdot(\rho_0\vec{\xi})\,d^3r'}{|\vec{r}-\vec{r}'|})
\end{equation}

Per investigare la composizione \'e necessario usare l'equazione di stato per esplicitare $\Gamma_1(P,\rho,X_i)$ riscrivendo \eqref{eq:hydroefdiff} nella forma:
\begin{align*}
&\frac{\delta\omega_{nl}}{\omega_{nl}}=\int_0^R[K^{nl}_{\Gamma_1,\rho}(r)\frac{\delta_r\Gamma_1}{\Gamma_1}(r)+K^{nl}_{\rho,\Gamma_1}(r)\frac{\delta_r\rho}{\rho}(r)]\,dr+I_{nl}\expy{-1}F_{Surf}(\omega_{nl})\\
&\frac{\delta\omega_{nl}}{\omega_{nl}}=\int_0^RK^{nl}_{u,Y}(r)\frac{\delta_ru}{u}(r)\,dr+\int K^{nl}_{Y,u}(r)\delta_rY\,dr+\int_0^RK^{nl}_{c^2,\rho}(r)(\frac{\delta\Gamma_1}{\Gamma_1})_{int}\,dr+I_{nl}\expy{-1}F_{Surf}(\omega_{nl})
\end{align*}
dove $(\delta\Gamma_1)_{int}$ \'e la variazione all'equazione di stato a $(P,\rho,Y)$ fissati.
\end{frame}

% SOS
%\part{SSM (sos)}
%\begin{comment}
\section{Osservabili stellari/demo beamer}
\begin{frame}<1>[label=noinside]{Modello stellare}{Come indagare la fisica interna a una stella?}
\onslide<1->\begin{block}{Osservabili stellari:}
$L$, $M$, $R$, $T_e$, $(\frac{Z}{X})_{ph}$, $g_{ph}$.
\end{block}
\onslide<1->\begin{block}{Informazioni sulla struttura interna?} Condizione di equilibrio idrostatico
\end{block}
%Teorema Vogt-Russel: $X_i(r)$, $M$ \pause equilibrio (idrostatico/termico) determinano struttura stellare .
%\pause
\onslide<1->\begin{block}{Modello stellare: diagramma di \hr{}.}
\end{block}
\onslide<2->\begin{block}{Descrizione fisica interno stellare: parametri aggiuntivi}
Convezione, diffusione e sedimentazione elementi pesanti, equazione di stato, opacit\'a
\end{block}
\onslide<2->\begin{block}{Astrosismologia}
Restringo spazio parametri sistemi stellari lontani
\end{block}
\end{frame}
{ % all template changes are local to this group.
    \setbeamertemplate{navigation symbols}{}
    \begin{frame}[plain]{Diagramma di \hr{}}
        \begin{tikzpicture}[remember picture,overlay]
            \node[at=(current page.center)] {
                %\includegraphics[width=\paperwidth]{yourimage}
            };
        \end{tikzpicture}
     \end{frame}
}
\againframe<2>{noinside}
\begin{frame}{Pulsazioni stellari}{Modi Normali}
\begin{columns}
\begin{column}{0.5\textwidth}  %%<--- here
    \begin{center}
     %\includegraphics[width=0.5\textwidth]{image1}
     \end{center}
\end{column}
\begin{column}{0.5\textwidth}
\onslide<1-> \begin{block}{Stelle pulsanti}
Onde stazionarie: Pulsazione radiale/non radiale: .
\onslide<2-> meccanismo di eccitazione: solar-like pulsator, Cefeidi.
\onslide<3-> Modo fondamentale $\Pi\approx\tau_{dyn}=\sqrt{\frac{R^3}{GM}}\propto\overline{\rho}\expy{-\frac{1}{2}}$.
\onslide<4-> Modi di oscillazione\onslide<5-> - informazioni sull'interno stellare
\onslide<5-> Elio-sismologia: Modi $\Leftrightarrow$ Modelli solari
\onslide<5-> Astero-sismologia: Modi $\Leftrightarrow$ Spazio parametri modello stellare
\end{block}
\end{column}
\end{columns}
\end{frame}

\end{comment}

\section{Osservabili solari}

\begin{frame}{Dati osservativi}

\begin{block}{Et\'a, luminosit\'a, raggio solari}
\begin{tabular}{l|c}
\hline
$\agesun{}$&\SI[separate-uncertainty=true]{4.57\pm0.02e9}{\year}\\
\hline
$\rsun{}$&\SI{695658+-140}{\kilo\meter}\\
\hline
$G\msun$&\num{132712440018+-8}\SI{e9}{\cubic\meter\per\square\second}\\
\hline
$\lsun{}$&\SI{3.8275+-0.0014e33}{\erg\per\second}\\
\hline
\end{tabular}
%\caption[Osservabili solari principali.]{Osservabili solari principali. \cite{haberreiter2008solving}.}
\label{tab:sunO}
\end{block}

\begin{block}{Simmetria sferica}
Deviazioni da forma sferica trascurabili (campi magnetici, rotazione)
\end{block}

\end{frame}

\begin{frame}{Dati osservativi}

\begin{block}{Composizione chimica}
\begin{itemize}
\item Righe di assorbimento: attuale (non $Y_{ph}$)
\item Meteoriti CI: primordiale (refrattari)
\end{itemize}

\begin{table}[]

\pgfplotstabletypeset[
every head row/.style={
 before row={\toprule &\multicolumn{4}{c|}{Attuale}
 %&\multicolumn{4}{c|}{Primordiale}
 \\\midrule},
 every last row/.style={after row=\bottomrule},
 after row={\midrule}
},
every nth row={2}{before row=\midrule},every last row/.style={after row=\bottomrule},
every first column/.style={column type/.add={|}{}},
every last column/.style={column type/.add={}{|}},
columns/x/.style = {column type/.add={|}{}},
columns/xi/.style = {column type/.add={|}{}},
display columns/0/.style={column name={}},
display columns/1/.style={column name={$X$}},
display columns/2/.style={column name={$Y$}},
display columns/3/.style={column name={$Z$}},
display columns/4/.style={column name={$\frac{Z}{X}$}},
%display columns/5/.style={column name={$X$}},
%display columns/6/.style={column name={$Y$}},
%display columns/7/.style={column name={$Z$}},
%display columns/8/.style={column name={$\frac{Z}{X}$}},
create on use/authors/.style={create col/set list={
%Anders \& Grevesse (1989),Grevesse \& Noels (1993),
Grevesse et al. (1998),Lodders (2003),Asplund et al. (2005),Lodders et al. (2009),\cite{asplund2009chemical},\cite{caffau2011solar}}},
columns/authors/.style={string type},
columns={authors,x, y, z, zx
%,xi,yi,zi, zxi
},
/pgf/number format/precision=4
     ]{asplund.txt} %%%
\captionof{table}{Metallicit\'a attuale determinata da varii autori.}\label{tab:Zhistory}
\end{table}

\end{block}

\end{frame}


\section{Strutture autogravitanti in equilibrio}

\begin{frame}{Distribuzione di massa - Conservazione di massa e momento - tempo scala dinamico}

\begin{block}{Massa}

%\begin{align}
%&dm=4\pi r^2\rho \,dr-4\pi r^2\rho v\,dt\label{eq:massvar}\\
%\end{align}

\begin{equation}
\PDy{t}{\rho}+\nabla\cdot(\rho\vec{v})=0\label{eq:continuityeq}
\end{equation}

\begin{equation}
dm=4\pi r^2\rho \,dr\label{eq:massaguscio}
\end{equation}

\end{block}

\begin{block}{Momento}
\begin{align}
&\rho\TDy{t}{\vec{v}}=-\nabla P+\rho\vec{f}\label{eq:motion}\\
&\vec{g}=-\PDy{r}{\Phi}=-\frac{Gm(r)}{r^2}\hat{r}
\end{align}
\end{block}

\end{frame}

\begin{frame}{Equilibrio idrostatico: $\ddvec{r}=0$.}


\begin{align}
\nabla P=\rho \vec{f}\Label{eq:idrosta} \TDy{r}{P}=-\frac{Gm(r)\rho(r)}{r^2}\Label{eq:fidroequilibrio}
\end{align}


Per giustificare l'ipotesi di equilibrio idrostatico stimo i tempi caratteristici di evoluzione della struttura solare nel caso la forza dovuta alla pressione o la forza di gravit\'a non fossero bilanciate, approssimando il valore caratteristico della derivata di due variabili con il rapporto del loro valore caratteristico:
\begin{align}
&\tau_{ff}\approx\sqrt{\frac{\rsun{}}{g}}\\
&\tau_{esp}\approx \rsun{}\sqrt{\frac{\rho}{P}}
\end{align}

Per i valori solari \ref{wrap-tab:sunO} $\tau_{ff}\approx\tau_{esp}\approx\SI{27}{\minute}$.

\begin{equation}
\tau_{idro}^{\odot}= \sqrt{\frac{R^3}{GM}}\approx\frac{1}{2}(G\overline{\rho})\expy{-\frac{1}{2}}
\end{equation}

\end{frame}


\subsection{Equazione di stato $P(\rho,T)$}

deviazioni dalla legge dei gas perfetti per tenere conto dei fenomeni di ionizzazione parziale e stati atomici eccitati, della radiazione, della statistica di Fermi-Dirac per gli elettroni, \'e necessario considerare l'interazione Coulombiana.

\begin{frame}{Gas perfetto ioni-elettroni}


\begin{equation}
P_G=P_I+P_e=\frac{\rho}{\mu}\gasconstant{}T
\end{equation}

\begin{block}{Peso molecolare medio}
massa media in amu per particella libera
\begin{align}
&\mu=\frac{1}{\bar{n}_HX+\bar{n}_{He}Y+\bar{n}_{Z}Z}\label{eq:meanmw}\\
&\bar{n}_i=\frac{1+f_i}{A_i}
\end{align}

\end{block}


\end{frame}

\subsection{Energia interna per unit\'a di massa}

\begin{frame}{Energia interna: traslazioni}

\begin{align}
&u=\frac{1}{\rho}\sum_i\int f^{(0)}(\vec{p}_i)\frac{p^2_i}{2m_i}=\frac{3}{2}\frac{P}{\rho}=\frac{3}{2}\frac{\gasconstant T}{\mu}\\
&E_i=\int_0^Mu\,dm=\frac{3}{2}\int_M\frac{P}{\rho}\,dm\label{eq:traslintenergy}
\end{align}

 $f^{(0)}(\vec{p}_i)$ \'e il numero di particelle della specie i per unit\'a di volume con impulso in $[\vec{p},\vec{p}+d\vec{p}]$

\end{frame}


\subsection{Correzioni alla legge dei gas perfetti}

\begin{frame}{Correzioni alla legge dei gas perfetti}

\begin{itemize}
\item Degenerazione elettronica: $\Delta P\leq2\%$.

\item Pressione di radiazione: $P_r=\frac{1}{3}aT^4$.

\item Ionizzazione.

\item Interazioni coulombiane.

\begin{align}
&\frac{1}{r_D^2}=\frac{4\pi e^2}{kT}\sum Z^2\overline{n}_Z=\frac{4\pi e^2}{kT}N_A\zeta\label{eq:debyeradius}\\
&\zeta=\sum_{i}(Z_i^2+Z_i)\frac{\rho X_i}{A_i}
\end{align}

\begin{equation}
u_c=\frac{1}{2}\int\phi(\vec{r})\rho(\vec{r})\,d^3r,\ P_c=\frac{1}{3}u_c
\end{equation}

Regioni di ionizzazione parziale di idrogeno ed elio

\end{itemize}

\end{frame}

\begin{frame}{EOS}


Due approcci usati per determinare l'equazione di stato e quindi le grandezze termodinamiche del plasma solare sono lo schema chimico e lo schema fisico: il primo considera atomi e molecole, la cui popolazione per stati eccitati e diversi gradi di ionizzazione \'e ottenuto minimizzando l'energia libera da cui sono ricavate le altre grandezze termodinamiche; utilizzando questo approccio \'e stata ricavata l'equazione di stato MHD. Il secondo considera nuclei ed elettroni come costituenti fondamentali interagenti tramite potenziale Coulombiano e trova le soluzione dell'equazione di Schr\"oedinger per un problema a molti corpi, questo approccio, usato per ricavare l'equazione di stato OPAL, \'e pi\'u adatto per trattare le regioni interne del Sole.


\begin{figure}[!ht]
        \includegraphics[height=0.4\textwidth,keepaspectratio]{ionfraction}\label{fig:ionfraction}
        \caption{Profilo radiale della popolazione dei diversi gradi di ionizzazione per $\cel{He}{4}{}{}$, CNO, $\cel{Ne}{20}{}{}$, $\cel{Fe}{56}{}{}$. Stati di ionizzazione maggiore sono pi\'u interni. Da \cite{basu2008helioseismology}.}
\end{figure}

\begin{figure}[!ht]
        \includegraphics[height=0.4\textwidth,keepaspectratio]{gamma1eos}\label{fig:gamma1eos}
        \subcaption{Andamento di $\Gamma_1$ calcolato tramite equazione di stato MHD/OPAL. Da \cite{trampedach2006synoptic}.}
\end{figure}


\end{frame}


\section{Trasporto dell'energia}

\section{Produzione di energia - reazioni di fusione}

\begin{block}{Schermaggio debole: $e\phi\ll KT$.}




Formula di Boltzmann per la densit\'a delle particelle con carica Z:
\begin{align}
&n_Z=\overline{n}_Z\exp{-\frac{Ze\phi_i}{kT}}\\
&\nabla^2\phi=-4\pi e\sum Zn_Z-4\pi\sum Z_i\delta(\vec{r}-\vec{r}_i)\label{eq:poissonscreened}
\end{align}
In \eqref{eq:poissonscreened} rimane il termine lineare in $\phi$.





\begin{align}
&\frac{r_D^2}{r_i}\TtwoDy{r}{(r_i\phi_i)}\tag{\ref{eq:poissonscreened} $Z_i$}\\
&\phi=\sum_i\phi_i
\end{align}

La soluzione di \eqref{eq:poissonscreened} \'e
\begin{equation}\label{eq:screenedpotential}
\phi_i=\frac{Z_ie}{r_i}\exp{-\midfrac{r}{r_D}}
\end{equation}


\end{block}

\section{Modello solare standard e osservabili sismologiche}




%%% SOS
%\part{OSCILLAZIONI SOS}
%\begin{comment}
\section{Osservabili stellari/demo beamer}
\begin{frame}<1>[label=noinside]{Modello stellare}{Come indagare la fisica interna a una stella?}
\onslide<1->\begin{block}{Osservabili stellari:}
$L$, $M$, $R$, $T_e$, $(\frac{Z}{X})_{ph}$, $g_{ph}$.
\end{block}
\onslide<1->\begin{block}{Informazioni sulla struttura interna?} Condizione di equilibrio idrostatico
\end{block}
%Teorema Vogt-Russel: $X_i(r)$, $M$ \pause equilibrio (idrostatico/termico) determinano struttura stellare .
%\pause
\onslide<1->\begin{block}{Modello stellare: diagramma di \hr{}.}
\end{block}
\onslide<2->\begin{block}{Descrizione fisica interno stellare: parametri aggiuntivi}
Convezione, diffusione e sedimentazione elementi pesanti, equazione di stato, opacit\'a
\end{block}
\onslide<2->\begin{block}{Astrosismologia}
Restringo spazio parametri sistemi stellari lontani
\end{block}
\end{frame}
{ % all template changes are local to this group.
\setbeamertemplate{navigation symbols}{}
    \begin{frame}[plain]{Diagramma di \hr{}}
        \begin{tikzpicture}[remember picture,overlay]
            \node[at=(current page.center)] {
                %\includegraphics[width=\paperwidth]{yourimage}
            };
        \end{tikzpicture}
     \end{frame}
}
\againframe<2>{noinside}
\section{Osservazioni}
\subsection{Fitting polinomiale: inversione ''1.5-D''.}
\begin{frame}<1>[label=noinside]{Modello stellare}{Come indagare la fisica interna a una stella?}
\onslide<1->\begin{block}{Rotazione superficiale}
\begin{equation*}
\frac{\Omega(\theta)}{2\pi}=\SI{451.5}{\nano\hertz}-\SI{65.3}{\nano\hertz}\cos^2{\theta}-\SI{66.7}{\nano\hertz}\cos^4{\theta}
\end{equation*}
\end{block}
\onslide<1->\begin{block}{Informazioni sulla struttura interna?} Condizione di equilibrio idrostatico
\end{block}
%Teorema Vogt-Russel: $X_i(r)$, $M$ \pause equilibrio (idrostatico/termico) determinano struttura stellare .
%\pause
\onslide<1->\begin{block}{Modello stellare: diagramma di \hr{}.}
\end{block}
\onslide<2->\begin{block}{Descrizione fisica interno stellare: parametri aggiuntivi}
Convezione, diffusione e sedimentazione elementi pesanti, equazione di stato, opacit\'a
\end{block}
\onslide<2->\begin{block}{Astrosismologia}
Restringo spazio parametri sistemi stellari lontani
\end{block}
\end{frame}
\subsection{Osservazione dello splitting in m: inversione ''2D''.}
\begin{figure}[!ht]
\centering
\includegraphics[keepaspectratio,width=0.8\textwidth]{invertedrotation}
\caption{Inversione della velocit\'a di rotazione a diverse latitudini. La linea verticale tratteggiata indica la base della zona convettiva. Da \cite{chr02helioseismology}.}
\end{figure}
Considero la correzione al primo ordine in $\Omega$. Il campo di velocit\'a rotazionale in coordinate sferiche \'e 
\begin{align}
&\vec{v_0}=(0,0,r\Omega\sin{\theta})=\vecp{\Omega}{r}\\
&\vec{\Omega(r,\theta)}=(\Omega(r,\theta)\cos{\theta},-\Omega(r,\theta)\sin{\theta},0)
\end{align}
In assenza di moti macroscopici il termine d'inerzia \'e $\rho_0\TDy{t}{\vec{v}}=\rho_0\PtwoDy{t}{\vec{\xi}}$, mentre in caso di rotazione si ha
\begin{equation}
\rho_0(\PDof{t}+\scap{v_0}{\nabla})^2\vec{\xi}
\end{equation}
Considero il termine dovuto alla rotazione come una piccola correzione alle frequenze dei modi
\begin{align}
&\omega_{(l,m)}+\Delta\omega_{(l,m)}&\intertext{quindi l'equazione del moto al primo ordine nella perturbazione, con $\alpha=(l,m)$, \'e}\nonumber\\
&\rho_0(\omega_{\alpha}^2+2\omega_{\alpha}\Delta\omega_{\alpha})\vec{\xi}=\nabla P_1-\frac{\rho_1}{\rho_0}\nabla P_0+\rho_0\nabla\Phi_1+2i\omega_{\alpha}\rho_0(\scap{v_0}{\nabla})\vec{\xi}\\
&\intertext{da cui si deduce}\nonumber\\
&\Delta\omega_{\alpha}=\frac{i\int\rho_0\xi_{\alpha}^*(\scap{v_0}{\nabla})\xi_{\alpha}}{\int\rho_0\xi_{\alpha}^*\xi_{\alpha}}=\frac{-m\int\rho_0\Omega\xi_{\alpha}^*\xi_{\alpha}\,dV+i\int\rho_0\xi_{\alpha}^*(\vecp{\Omega}{\xi_{\alpha}})\,dV}{\int\rho_0\xi_{\alpha}^*\xi_{\alpha}}
\end{align}
Il problema di trovare $\Omega(r,\theta)$ dalla differenza $\Delta\omega_{\alpha}$ \'e lineare in $\Omega$ quindi $\Delta\omega_{\alpha}\propto\Omega$. Per determinare quindi la rotazione dobbiamo conoscere l'autovalore $\xi_{\alpha}$ dello stato imperturbato.
%Per rotazione puramente radiale $\Omega(r)$ la relazione tra lo splitting delle frequenze e la rotazione \'e
%\begin{equation}
%\Delta\omega_{\alpha}=-m\frac{\int_0^{\rsun{}}\rho_0\Omega\{|\xi_r-\xi_h|^2+[l(l+1)-2]|\xi_h|^2\}r^2\,dr}{\int_0^{\rsun{}}\rho_0\{|\xi_r|^2+l(l+1)|\xi_h|^2\}r^2\,dr}=\int_0^{\rsun{}}K_{\alpha}(r)\Omega(r)\,dr
%\end{equation}
%Any given $\Delta\omega_{\alpha}$ samples angular velocity in the depth range corresponding to $\xi_{\alpha}$.
La velocit\'a angolare contribuisce a $\Delta\omega_{\alpha}$ negli strati in cui $\xi_{\alpha}$ \'e apprezzabile. Nel caso di rotazione dipendente solo da r si ha che $\Delta\omega_{\alpha}$ \'e lineare in m: ho $2l+1$ frequenze equispaziate.

\end{comment}

\section{Modi normali della struttura solare}

\begin{frame}[label=noinside]{Modi di oscillazione adiabatici}{Perturbazione dello stato di equilibrio.}

\begin{block}{campi di velocit\'a/effetti non lineari}
Descrivo le oscillazioni come piccole perturbazioni attorno allo stato di equilibrio stazionario (gli effetti non lineari, fra cui lo scambio di energia tra i modi, sono dell'ordine di $\frac{v}{c_s}$ dove v \'e l'ampiezza della velocit\'a dell'oscillazione). 
In generale pu\'o essere presente un campo di velocit\'a $\vec{v}_0$:
\begin{align}
&\vec{v}=\vec{v}_0+\vec{v}'\\
&\TDof{t}=\PDof{t}+(\vec{v}_0\cdot\nabla)
\end{align}
in prima approssimazione prendo $\vec{v}_0=0$ per poi considerare come perturbazioni gli effetti dovuti a campi di velocit\'a in specie rotazione.

\end{block}

\begin{block}{Perturbazione pressione densit\'a}

Indico con $P'(\vec{r},t)$ e $\delta P$ la perturbazione euleriana e lagrangiana della pressione e con $\rho'$, $\Phi'$ e $\vec{g}'$ la perturbazione euleriana della densit\'a , e le perturbazioni euleriane del potenziale gravitazionale e dell'accelerazione di gravit\'a conseguenti  con $\delta\vec{r}=\vec{\xi}$ il vettore spostamento perturbato:
\begin{align}
&P(\vec{r},t)=P_0(\vec{r})+P'(\vec{r},t)\label{eq:pressureperturbation}\\
&\Lvar{P(\vec{r})}=P(\vec{r}+\Lvar{\vec{r}})-P_0(\vec{r})=P'(\vec{r})+\Lvar{\vec{r}}\cdot\nabla P_0\\
&\vec{g}'=-\nabla\Phi',\ \nabla^2\Phi'=4\pi G\rho'\label{eq:gapert}
\end{align}

\end{block}


\end{frame}

\begin{frame}[label=noinside]{Modi di oscillazione adiabatici}{Modi di oscillazione lineari adiabatici.}

\begin{block}{Equazione del moto perturbata}

l'equazione del moto perturbato sostituendo \eqref{eq:pressureperturbation} nell'equazione del moto \eqref{eq:motion} considerando solo i termini lineari nella perturbazione:
\begin{equation}
\rho_0\TDof{t}\vec{v}=\rho_0\PtwoDy{t}{\Lvar{\vec{r}}}=-\nabla P'+\rho_0\vec{g}'+\rho'\vec{g}_0\label{eq:emper}
\end{equation}

\end{block}

\end{frame}

\begin{frame}[label=noinside]{Modi di oscillazione adiabatici}{Equazione di continuit\'a perturbata}

\begin{block}{Equazione di continuit\'a perturbata}

Analogamente per l'equazione di continuit\'a ottengo
\begin{equation}
\rho'+\div{(\rho_0\Lvar{\vec{r}})}=0\label{eq:contper}
\end{equation}

\end{block}

\end{frame}

\begin{frame}[label=noinside]{Modi di oscillazione adiabatici}{Condizione di adiabaticit\'a}


  \begin{overlayarea}{\textwidth}{1cm}
   \only<1>{
   energia interna per unit\'a di massa
\begin{equation}
\TDy{t}{q}=\TDy{t}{u}+P\TDof{t}(\frac{1}{\rho})\label{eq:prima}
\end{equation}

\begin{equation}
\TDy{t}{T}-\frac{\Gamma_2-1}{\Gamma_2}\frac{T}{P}\TDy{t}{P}=\frac{1}{c_P}(\epsilon-\frac{1}{\rho}\scap{\nabla}{F})
\end{equation}
il termine a destra \'e trascurabile:
\begin{equation}
\TDy{t}{q}=0
\end{equation}
   }
   \only<2>{
   Il moto di una elemento di fluido \'e descritto dalla relazione adiabatica
\begin{equation}
\TDy{t}{P}=\frac{\Gamma_1P}{\rho}\TDy{t}{\rho}
\end{equation}
}
   \only<3>{
  
  La condizione di perturbazione adiabatica linearizzata \'e
\begin{align}
&\PDy{t}{\Lvar{P}}-\frac{\Gamma_{1,0}P_0}{\rho_0}\PDy{t}{\Lvar{\rho}}=0\\
&P'+\Lvar{\vec{\xi}}\cdot\nabla P_0=\frac{\Gamma_{1,0}P_0}{\rho}(\rho'+\Lvar{\vec{\xi}}\cdot\nabla\rho_0)\label{eq:adper}
\end{align}

   }
  \end{overlayarea}


\end{frame}


\begin{figure}[!ht]

\subfigure[Distribuzione dei modi con $l\leq300$ nel diagramma $\nu-l$ determinata usando i primi 144 giorni di osservazione di MDI. Da \cite{chr02helioseismology}.]{
\includegraphics[keepaspectratio,width=0.45\textwidth]{midlmodes}}
\label{fig:midlmodes}
~
\subfigure[Modi adiabatici calcolati sulla base di un modello solare. Da \cite{chr02helioseismology}.]{
\includegraphics[keepaspectratio,width=0.6\textwidth]{nrmodesLAWE}\label{fig:nrmodesLAWE}}

\end{figure}

\section{Campo di velocit\'a solare}

\section{Caratteristiche asintotiche delle oscillazioni adiabatiche}



%%% SOS
%\part{INVERSIONE SOS}
%\begin{comment}
\section{Osservabili stellari/demo beamer}

\begin{frame}<1>[label=noinside]{Modello stellare}{Come indagare la fisica interna a una stella?}

\onslide<1->\begin{block}{Osservabili stellari:}
$L$, $M$, $R$, $T_e$, $(\frac{Z}{X})_{ph}$, $g_{ph}$.
\end{block}

\onslide<1->\begin{block}{Informazioni sulla struttura interna?} Condizione di equilibrio idrostatico
\end{block}

%Teorema Vogt-Russel: $X_i(r)$, $M$ \pause equilibrio (idrostatico/termico) determinano struttura stellare .
%\pause

\onslide<1->\begin{block}{Modello stellare: diagramma di \hr{}.}
\end{block}

\onslide<2->\begin{block}{Descrizione fisica interno stellare: parametri aggiuntivi}
Convezione, diffusione e sedimentazione elementi pesanti, equazione di stato, opacit\'a
\end{block}

\onslide<2->\begin{block}{Astrosismologia}
Restringo spazio parametri sistemi stellari lontani
\end{block}

\end{frame}

{ % all template changes are local to this group.
    \setbeamertemplate{navigation symbols}{}
    \begin{frame}[plain]{Diagramma di \hr{}}
        \begin{tikzpicture}[remember picture,overlay]
            \node[at=(current page.center)] {
                %\includegraphics[width=\paperwidth]{yourimage}
            };
        \end{tikzpicture}
     \end{frame}
}
\againframe<2>{noinside}

\section{Inversione della legge di Duvall}

\section{Inversione non asintotica}

\section{Vincoli al modello solare dalle osservazioni sismologiche}

\end{comment}


In questa parte considero come estrarre informazioni sulla struttura di equilibrio dalle frequenze dei modi osservati: modi distinti sono confinati in cavit\'a di profondit\'a diversa e le ampiezze di oscillazione hanno differente comportamento spaziale, \'e quindi possibile invertire il problema date le frequenze osservate per ricavare il profilo radiale di $(P,\rho,\Gamma_1)$.

Un'inversione indipendente dal modello \'e possibile utilizzando l'approssimazione asintotica, valida nelle regioni in cui le autofunzioni variano molto pi\'u rapidamente delle grandezze di equilibrio e che trascura la perturbazione del potenziale gravitazionale.
 
L'inversione del sistema completo di equazioni dei modi si effettua considerando le perturbazioni al MSS che danno un miglior accordo tra frequenze osservate e misurate: considero solo i termini lineari nelle perturbazioni e quindi le correzioni agli autovettori sono trascurate.

I risultati dell'inversione sismologica permettono di valutare l'accuratezza della struttura del \mss{} , in particolare del profilo radiale della velocit\'a del suono, della densit\'a e di $\Gamma_1$; inoltre, usando l'equazione di stato per esplicitare la dipendenza di $\Gamma_1$ da $Y$, \'e possibile ricavare l'abbondanza di elio nella zona convettiva $Y_{CZ}$.

\'E possibile ricavare le caratteristiche della base della zona convettiva, profondit\'a della zona convettiva $d_{CZ}=\rsun{}-R_{CZ}$, $\rho_b=\rho(R_{CZ})$, $c_s=c_s(R_{CZ})$, oltre a $Y_{CZ}$, con grande accuratezza.

\section{Inversione della legge di Duvall. % Inizio chapter "Inversione asintotica." senza nuava pagina

%Si ricava, usando le espressioni asintotiche \eqref{cowosc:main}, il profilo radiale della velocit\'a del suono indipendente dal modello solare e si determina l'effetto dell'evoluzione stellare sui modi p di basso ordine radiale.
%Considerando la differenza tra risultati relativi a diversi set di frequenze \'e possibile attenuare gli effetti degli errori sistematici dovuti alla descrizione asintotica.


L'inversione della legge di Duvall \eqref{eq:duvallexpli} permette di ricavare il profilo di $c_s(r)$ sulla base dei modi osservati, tuttavia le approssimazioni fatte introducono errori sistematici: per i modi pi\'u penetranti nell'interno solare la perturbazione del potenziale gravitazionale influenza sensibilmente $F(\frac{\omega}{L})$ mentre per modi confinati vicino alla superficie $\alpha$ dipende da l.


\subsection{Inversione analitica della velocit\'a del suono}

\begin{figure}[!ht]
        \includegraphics[width=0.44\textwidth,keepaspectratio]{soundspeed}
        \caption{Profilo radiale di $c_s^2$ determinato invertendo \eqref{eq:analinversionc} dalle frequenze dei modi osservate. Da \cite{christensen1985speed}.}\label{fig:soundspeedccm}
\end{figure}

L'equazione \eqref{eq:duvallf} pu\'o essere invertita analiticamente:
\begin{equation}
r=R\Exp{-\frac{2}{\pi}\int_{a_s}^a(w\expy{-2}-a\expy{-2})\expy{-\frac{1}{2}}\TDy{w}{F}\,dw}\label{eq:analinversionc}
\end{equation}

dove $a=\frac{c_s}{r}$.

Dall'equazione precedente, nota la funzione $F(w)$ dalle osservazioni, di ricavare $c_s(r)$ (vedi figura \ref{fig:soundspeedccm}). Il confronto tra $c_{sm}(r)$ calcolato tramite un modello solare e $c_{s0}(r)$ invertito usando l'equazione precedente per lo stesso modello mostra che l'errore sistematico dovuto alla tecnica di inversione nel range $0.4\leq x \leq 0.9$ \'e minore del $2.5\%$.


\subsection{Struttura dei modi penetranti nel core stellare}

La deviazione dalla \eqref{eq:freqequi} fornisce informazioni sull'evoluzione chimica del core di fusione: infatti estendendo ancora l'espansione di \eqref{eq:duvallf} si ha una misura della variazione di $c_s$ nel core della stella
\begin{equation}\label{eq:tassoul}
    d_{nl}=\nu_{nl}-\nu_{n-1,l+2}\approx-(4l+6)\frac{\Delta\nu}{4\pi^2\nu_{nl}}\int_0^R\frac{dc_s}{dr}\frac{dr}{r}
\end{equation}
La velocit\'a del suono \'e ridotta a causa dell'aumentare di $\mu$ durante la fusione di H in He durante l'evoluzione stellare: il centro solare \'e un minimo locale per la velocit\'a del suono e quindi, essendo il gradiente della velocit\'a del suono positivo, la parte centrale da un contributo sempre pi\'u negativo in \eqref{eq:tassoul} con l'evolversi della stella.

\subsection{Forma differenziale della legge di Duvall}

Considero l'effetto di perturbazioni del modello sulle frequenze dei modi introducendo nella legge di Duvall \eqref{eq:duvallexpli} perturbazioni nel profilo della velocit\'a del suono e in $\alpha$:
\begin{equation}
S_{nl}\frac{\delta\omega_{nl}}{\omega_{nl}}\approx H_1(\frac{\omega_{nl}}{L})+H_2(\omega_{nl})\label{eq:Dlinear}
\end{equation}

\begin{align}
&S_{nl}=\int_{r_t}^R(1-\frac{L^2c^2}{r^2\omega_{nl}^2})\expy{-\frac{1}{2}}\frac{dr}{c}-\pi\TDy{\omega}{\alpha}\\
&H_1(w)=\int_{r_t}^R(1-\frac{c^2}{r^2w^2})\expy{-\frac{1}{2}}\frac{\delta_rc}{c}\frac{dr}{c},\ H_2(\omega)=\frac{\pi}{\omega}\delta\alpha(\omega)
\end{align}

La funzione $S_{nl}$ \'e approssimabile con un temine proporzionale a $Q_{nl}$ (\cite{christensen1991solar}):
\begin{align}
&\frac{S_{nl}}{\tau_0}\approx Q_{nl}\intxt{con}
&\tau_0=\int_{0}^R\frac{dr}{c_s}
\end{align}

Le funzioni $H_1(\frac{\omega_{nl}}{L})$ e $H_2(\omega_{nl})$ possono essere ottenute separatamente attraverso fitting dei dati sperimentali: la prima caratterizza il contributo alle differenze nelle frequenze dei modi dovuto alle differenze del profilo radiale della velocit\'a del suono, la seconda alle diffenze nella regione vicino alla superficie.

\begin{figure}[!ht]
        \includegraphics[width=0.44\textwidth,keepaspectratio]{H2dnd}
        \caption{a) Residuo della differenza di frequenze fra il sole e un modello senza diffusione a cui \'e stato sottratto $H_1$. b) Fit di $H_2$ linea continua e per contrasto fit di $H_2$ per differenze di frequenze tra Sole e modello con diffusione. Da \cite{dal03notes}.}\label{fig:H2dnd}
\end{figure}

La relazione \eqref{eq:Dlinear}, considerando che $1-\midfrac{L^2c^2}{r^2\omega^2}\approx1$ ad eccezione delle regioni vicino al punto d'inversione $r_t$, pu\'o essere approssimata da
\begin{equation}
\frac{\delta\omega}{\omega}\approx\frac{\int_{r_t}^{R}\frac{\delta_rc_s}{c_s}\frac{dr}{c_s}}{\int_{r_t}^R\frac{dr}{c_s}}
\end{equation}
Le differenze nella velocit\'a del suono nelle varie regioni influiscono sulle differenze nelle frequenze con un peso dato dal tempo impiegato da un'onda sonora ad attraversare la regione: le differenze nella regione vicino alla superficie dove $c_s$ \'e minore hanno un effetto relativamente grande sulle differenze di frequenza.

Una volta determinato $H_1$ le differenze nel profilo radiale di $c_s$ sono determinate tramite
\begin{equation}
\frac{\delta_rc_s}{c_s}=-\frac{2a}{\pi}\TDof{\ln{r}}\int_{a_s}^a(a^2-w^2)\expy{-\frac{1}{2}}H_1(w)\,dw
\end{equation}


La funzione $H_2$ \'e determinata dalla regione sotto la fotosfera. \cite{chr92phase}, analizzando la relazione tra $H_2(\omega)$ e le differenze in $c_s(r)$ e $\Gamma_1$ nelle regioni esterne, hanno visto che discrepanze pi\'u vicino alla superficie generano una componente lentamente oscillante in $H_2(\omega)$ e la ''frequenza'' aumenta con l'aumentare della profondit\'a. \'E inoltre possibile indagare l'andamento di $\Gamma_1$ nella regione di seconda ionizzazione di He e pi\'u in generale il comportamento di $H_2(\omega)$ nelle zone di ionizzazione di H e He consente un'analisi dell'equazione di stato e determinazione dell'abbondanza di elio nella zona convettiva. In figura \ref{fig:H2dnd} si mostra le differenze nelle frequenze tra il Sole ed un modello che non considera la diffusione degli elementi: l'andamento oscillatorio della linea continua nel pannello b \'e dovuto alla differenza nell'abbondanza di idrogeno negli strati superficiali.


\section{Inversione non asintotica.} % Inizio chapter "Inversione non asintotica." senza nuava pagina

La soluzione del problema inverso per il sistema completo di equazioni si basa sulla linearizzazione delle variazioni attorno ad un modello di cui siano calcolabili le autofunzioni dell'operatore $L$ definito in \eqref{eq:variational}.

Utilizzo la formula \eqref{eq:variational} specializzata al problema dell'inversione delle differenze $\delta\omega_{nl}=\omega_{\odot}-\omega_{Mod}$ fra frequenze osservate e predette da un modello. Per l'inversione della struttura idrostatica si ha:
\begin{align}
&\frac{\delta\omega_{nl}}{\omega_{nl}}=\int_0^R[K^{nl}_{c^2,\rho}(r)\frac{\delta_rc^2}{c^2}(r)+K^{nl}_{\rho,c^2}(r)\frac{\delta_r\rho}{\rho}(r)]\,dr+I_{nl}\expy{-1}F_{Surf}(\omega_{nl})+\sigma_i\label{eq:invstructure}\intxt{dove $\sigma_i$ \'e l'incertezza sulle frequenze osservate e}
&\frac{\delta_rc^2}{c^2}(r)=\frac{[c_{\odot}^2(r)-c_{mod}^2(r)]}{c^2(r)},\ \frac{\delta_r\rho}{\rho}(r)=\frac{[\rho_{\odot}(r)-\rho_{mod}(r)]}{\rho(r)}
\end{align}

E quindi si determinano attraverso procedure numeriche le correzioni alla struttura del modello sulla base delle differenze tra frequenze dei modi. Il peso che una perturbazione ha sulla differenze in frequenza \'e determinato dalle autofunzioni dei modi calcolate tramite un modello solare.
%Asymptotic approximation for radial eigenfunction (integral equation connectin sound speed $c(r)$ to $\Omega_{nl}$) is inadequate (especially in deep interior)

In \ref{fig:deltacwu} la banda chiara attorno allo zero, che indica il modello solare in perfetto accordo con le osservazioni sismologiche, mostra l'incertezza nell'inversione della velocit\'a del suono; essa \'e dovuta a

\begin{itemize}

\item Incertezze nelle frequenze dei modi osservate. Oltre all'incertezza statistica propria del determinato strumento \'e possibile valutare gli effette di eventuali errori sistematici confrontando i risultati dell'inversione per set di frequenze ottenute con strumenti diversi: la differenza nella velocit\'a del suono \'e minore di $0.02\%$.

\item Incertezze inerenti la tecnica di inversione.

\item Incertezze legate alla dipendenza da un modello solare per il calcolo dei kernel in \eqref{eq:invstructure}, \eqref{eq:invdGammadrho}, \eqref{eq:splitfreqrotation}.

\end{itemize}


\subsection{Tecniche di inversione numeriche.}
%vedi JCD 2002 pg 25-32

Elenco alcune tecniche numeriche usate. Considero per sempicit\'a una sola funzione da invertire $\frac{\delta f}{f}$, legata a $\delta\omega$ da \eqref{eq:variational}.


\subsubsection{RLS}

Usando la tecnica del minimo $\chi^2$ regolarizzato si parametrizza la funzione incognita $\frac{\delta f}{f}$ tramite funzioni di base opportune.

La funzione da minimizzare \'e
\begin{equation}
Y=(N-N_p)\chi^2+\alpha N\int_0^1(x\TDof{x}\frac{\Delta f}{f})^2\,dx\label{eq:minimizerls}
\end{equation}
con
\begin{equation}
\chi^2=\frac{1}{N-N_p}\sum_{\alpha=1}^N(\frac{\delta\omega_{obs}-\delta\omega_{fit}}{\sigma})^2_{\alpha}
\end{equation}
N indica il numero totale di modi $\alpha$, $N_p$ il numero di parametri da determinare, $\Delta\nu_{fit}$, ricavato tramite \eqref{eq:variational}, contiene la funzione incognita opportunamente parametrizzata; il secondo addendo del lato destro di \eqref{eq:minimizerls} \'e introdotto per ridurre oscillazioni indesiderate nel risultato dell'inverisone con $\alpha$, parametro di regolarizzazione, scelto opportunamente.

\subsubsection{Subtractive Optimally Localized Averaging}

Scelgo dei coefficienti $c_i(r_0)$ tali che $\sum c_i(r_0)\frac{\delta\omega_i}{\omega_i}$ fornisca una media del valore di $\frac{\delta f(r)}{f(r)}$ in $r=r_0$:

\begin{equation}\label{eq:SOLAfmean}
\sum_ic_i(r_0)\frac{\delta\omega_i}{\omega_i}=\int_0^R\sum_ic_i(r_0)K^i(r)\frac{\delta f(r)}{f(r)}\,dr=\exv{\frac{\delta f(r_0)}{f(r_0)}}
\end{equation}

e i coefficienti $c_i(r_0)$ sono determina minimizzando la funzione

\begin{equation}\label{eq:SOLAcir0min}
\int_0^R[\mathcal{K}(r_0,r)-\mathcal{T}(r_0,r)]^2\,dr+\mu\sum_i\sigma_ic_i(r_0)c_j(r_0)\\
\end{equation}
con $\mathcal{K}(r_0,r)=\sum_ic_i(r_0)K^i(r)$ e la funzione target $\mathcal{T}(r_0,r)$, la cui larghezza \'e anch'essa parametro del fit, determina la natura precisa della localizzazione.

La larghezza finita di $\mathcal{K}(r_0,r)$ determina il valor medio di $\frac{\delta f}{f}$ in intorno di $r_0$ e ci\'o causa una differenza sistematica  dal valore effettivo in $r_0$: per $c_s$ l'errore \'e minore del $0.03\%$ e le regioni pi\'u afflitte sono la base della zona convettiva, per la rapida variazione di $\frac{\delta c_s}{c_s}$ e la regione centrale, per il ridotto numero di modi che penetrano in questar zona.


Illustro la tecnica SOLA per determinare $\frac{\delta_rc^2}{c^2}$: si formano delle combinazioni lineari di $\frac{\Lvar{\omega_i}}{\omega_i}$ pesate da coefficienti $c_i(r_0)$ tali che $\frac{\Lvar{c^2}}{c^2}$ sia centrato attorno $r_0$ e che gli altri termini in \eqref{eq:invstructure} siano soppressi, queste compongo un averaging kernel $\mathcal{K}_{c^2,\rho}(r_0,r)=\sum_ic_i(r_0)K_{c^2,\rho}^i(r)$, con $\int_0^R\mathcal{K}(r_0,r)\,dr=1$.

Determino i coefficienti minimizzando l'espressione
\begin{equation}
\int_0^R[\mathcal{K}_{c^2,\rho}(r_0,r)-\mathcal{T}(r_0,r)]^2\,dr+\beta\int_0^R\mathcal{L}_{\rho,c^2}(r_0,r)\,dr+\mu\sum_i\sigma_ic_i(r_0)c_j(r_0)
\end{equation}
dove il kernel cross-term
\begin{equation}
\mathcal{L}_{\rho,c^2}(r_0,r)=\sum_ic_i(r_0)K_{\rho,c^2}^i(r)
\end{equation}
contralla i contributi indesiderati di $\frac{\delta_r\rho}{\rho}$.


Il primo termine approssima il valore di $\frac{\delta c^2}{c^2}$ pesato dal kernel $\mathcal{K}(r,r_0)=\sum_ic_i(r_0)K_{c^2,\rho}^i(r)$, il secondo tiene conto dell'influenza che hanno le discrepanze della seconda funzione su quelle della funzione che abbiamo scelto di invertire pesate da $\mathcal{L}_{\rho,c^2}(r_0,r)=\sum_ic_i(r_0)K_{\rho,c^2}^i(r)$, il terzo \'e il termine di superficie: i coefficienti $c_i(r_0)$ sono scelti in maniera da riprodurre la funzione target, minimizzare la contaminazione delle $\frac{\delta \rho}{\rho}$ via $\mathcal{L}_{\rho,c^2}$ e il rumore.


\subsection{Inversione della rotazione.}

\begin{figure}[!ht]
\centering
\includegraphics[keepaspectratio,width=0.8\textwidth]{invertedrotation}
\caption{Inversione della velocit\'a di rotazione a diverse latitudini. La linea verticale tratteggiata indica la base della zona convettiva. Da \cite{chr02helioseismology}.}
\end{figure}

Per inversione 2D, cio\'e che considera la dipendenza generica $\Omega(r,\theta)$, si esprimono direttamente le differenze in frequenze:
\begin{equation}
\omega_{nlm}-\omega_{nl0}=m\int_0^R\int_0^{\pi}K_{nlm}(r,\theta)\Omega(r,\theta)r\,dr\,d\theta\label{eq:invrot2D}
\end{equation}

mentre nel caso si abbiano i coefficienti $a_{2s+1}$, scrivo la velocit\'a angolare nella forma
\begin{equation}
\Omega(r,\theta)=\sum_{s=0}^{s_m}\Omega_{s}(r)\psi_{2s}(\cos{\theta})\label{eq:angularv15}
\end{equation}
dove $\psi_{2s}$ sono polinomi opportuni.

Esiste una funzione opportuna $K_{nls}^{s}(r)$ tale che
\begin{equation}
2\pi a_{2j+1}(n,l)=\int_0^R\int_0^{\pi}K_{nls}^{s}(r)\Omega_s(r)\,dr
\end{equation}
e quindi \'e possibile determinare $\Omega_s(r)$.

\section{Vincoli al modello solare dalle osservazioni sismologiche.} %%chapter: vincoli al modello solare: HCSM.


\begin{figure}[!ht]%{r}{0.5\textwidth}
        \includegraphics[width=0.9\textwidth,keepaspectratio]{deltacwu}
        \caption{Differenza relativa nel profilo di $c_s$ risultanei dall'inversione delle differenze in frequenza tra Sole e modello: la linea chiara si riferisce alle frequenze di un modello con composzione GS98, la linea scura a composizione AGSS09. La banda chiara mostra l'errore inerente l'inversione eliosismologica, la banda scura l'incertezza a $1\sigma$ sul profilo di $c_s$ predetto dal modello. Da \cite{villante2014chemical}.}\label{fig:deltacwu}
\end{figure}

\begin{table}[!ht]%{r}{0.7\textwidth}

\pgfplotstabletypeset[
math/.style={%
        preproc cell content/.append style={/pgfplots/table/@cell content/.add={$}{$}},
    },
every head row/.style={
 before row={\toprule
 %&\multicolumn{4}{c|}{Primordiale}
 },
 every last row/.style={after row=\bottomrule},
 after row={\midrule}
},
every last row/.style={after row=\bottomrule},
every first column/.style={column type/.add={|}{}},
every last column/.style={column type/.add={}{|}},
%columns/0/.style = {column type/.add={|}{}},
display columns/0/.style={column name={Composizione}},
display columns/1/.style={column name={$Z/X$}},
display columns/2/.style={column name={$R_{CZ}$}},
display columns/3/.style={column name={$Y_{CZ}$}},
display columns/4/.style={column name={$Y_0$}},
create on use/comp/.style={create col/set list={
inversione,GS98,AGS05,AGSS09,C+11}},
columns/comp/.style = {column type/.add={|}{}},
columns/comp/.style={string type},
columns/ZX/.style={string type},
columns/ZX/.append style={math},
columns/RCZ/.style={string type},
columns/RCZ/.append style={math},
columns/YCZ/.style={string type},
columns/YCZ/.append style={math},
columns/Y0/.style={string type},
columns/Y0/.append style={math},
columns={comp,ZX,RCZ,YCZ,Y0},
%/pgf/number format/precision=4
     ]{CZvsZ.txt} %%%
     \caption{Caratteristiche della zona convettiva: confronto tra valore eliosismologico e valore ricavato da modello solare con diverse metallicit\'a del raggio della base della zona convettiva $R_{CZ}$, dell'abbondanza di elio superficiale $Y_{CZ}$ e dell'abbondanza di elio primordiale. Da \cite{basu2016global}.}
\label{tab:CZZvar}
\end{table}

L'inversione di $c_s$ o $\rho$ mostra se un modello solare riproduce accuratamente la posizione della base della zona convettiva in quanto nella zona convettiva si ha gradiente adiabtico maggiore del gradiente radiativo; diminuzione di opacit\'a, nel caso determinata da una minore metallicit\'a, sposta la base della zona convettiva pi\'u in alto come da tabella \ref{tab:CZZvar}.

La figura \ref{fig:deltacwu} mostra che un modello solare con composizione GS98, meno accurata di AGSS09, riproduce il profilo di $c_s$ in maniera pi\'u accurata: ci\'o pu\'o indicare un'opacit\'a da incrementare nel modello. Analogamente la diminuzione dell'opacit\'a diminuisce il gradiente termico nella regione radiativa quindi a pari luminosit\'a si ha un contenuto di idrogeno maggiore.




\end{document}