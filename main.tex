\documentclass[10pt,xcolor={usenames},fleqn,mathserif,serif]{beamer}

%%%Usefull link
%tikz-equations:
%http://www.wekaleamstudios.co.uk/posts/creating-a-presentation-with-latex-beamer-equations-and-tikz/

% There are many different themes available for Beamer. A comprehensive
% list with examples is given here:
% http://deic.uab.es/~iblanes/beamer_gallery/index_by_theme.html
% You can uncomment the themes below if you would like to use a different
% one:
%\usetheme{AnnArbor}
%\usetheme{Antibes}
%\usetheme{Bergen}
%\usetheme{Berkeley}
%\usetheme{Berlin}
%\usetheme{Boadilla}
%\usetheme{boxes}
%\usetheme{CambridgeUS}
%\usetheme{Copenhagen}
%\usetheme{Darmstadt}
\usetheme{default}
%\usetheme{Frankfurt}
%\usetheme{Goettingen}
%\usetheme{Hannover}
%\usetheme{Ilmenau}
%\usetheme{JuanLesPins}
%\usetheme{Luebeck}
%\usetheme{Madrid}
%\usetheme{Malmoe}
%\usetheme{Marburg}
%\usetheme{Montpellier}
%\usetheme{PaloAlto}
%\usetheme{Pittsburgh}
%\usetheme{Rochester}
%\usetheme{Singapore}
%\usetheme{Szeged}
%\usetheme{Warsaw}

\hypersetup{pdfpagemode=FullScreen}

\addtobeamertemplate{block begin}{%
  \setlength{\textwidth}{0.95\textwidth}%
  \setlength\abovedisplayskip{0pt}%
}{}


\setbeamertemplate{caption}{\insertcaption}

%% colors
\definecolor{bittersweet}{rgb}{1.0, 0.44, 0.37}
\definecolor{brilliantlavender}{rgb}{0.96, 0.73, 1.0}
\definecolor{antiquefuchsia}{rgb}{0.57, 0.36, 0.51}
\definecolor{violetw}{rgb}{0.93, 0.51, 0.93}
\definecolor{Veronica}{rgb}{0.63, 0.36, 0.94}
\definecolor{atomictangerine}{rgb}{1.0, 0.6, 0.4}
\definecolor{darkgray}{rgb}{0.66, 0.66, 0.66}
\definecolor{brightcerulean}{rgb}{0.11, 0.67, 0.84}
\definecolor{cadmiumorange}{rgb}{0.93, 0.53, 0.18}
\definecolor{ochre}{rgb}{0.8, 0.47, 0.13}
\definecolor{midnightblue}{rgb}{0.1, 0.1, 0.44}
\definecolor{lemon}{rgb}{1.0, 0.97, 0.0}
\definecolor{grey}{rgb}{0.7, 0.75, 0.71}
\definecolor{amber}{rgb}{1.0, 0.75, 0.0}
\definecolor{almond}{rgb}{0.94, 0.87, 0.8}
\definecolor{bf}{RGB}{88, 86, 88}
\definecolor{bb}{RGB}{177, 177, 177}


%%%%%%%%%%%%%%%%%%%%%%%%%%%%%%%%%%% importa pacchetti
\usepackage{usepkg}
%%%%%%%%%%%%%%%%%%%%%%%%%%%%%%%%%%% Funzioni generali
\usepackage{functions}
%http://tex.stackexchange.com/questions/246/when-should-i-use-input-vs-include
\newcommand{\setmuskip}[2]{#1=#2\relax} %%problem usinig mu with calc (req by mathtools) loaded
\usepackage{sources}
%\usepackage{length}
%%%%%%%%%%%%%%%%%%%%%%%%%%%%%%%%%%% Funzioni per questo file main
\usepackage{mathOp}

\def\status{keeptrying}
\def\keeptrying{keeptrying}
\usepackage{LocalF}
%%%%%%%%%%%%%%%%%%%%%%%%%%%%%%%%%


\title{Modi normali di oscillazione del Sole}

% A subtitle is optional and this may be deleted
\subtitle{Struttura interna e modi di oscillazione}

%\author{F.~Author\inst{1} \and S.~Another\inst{2}}
% - Give the names in the same order as the appear in the paper.
% - Use the \inst{?} command only if the authors have different
%   affiliation.

%\institute[Universities of Somewhere and Elsewhere] % (optional, but mostly needed)
%{
% \inst{1}
% Department of Computer Science\\
%  University of Somewhere
%  \and
%  \inst{2}%
%  Department of Theoretical Philosophy\\
%  University of Elsewhere}
% - Use the \inst command only if there are several affiliations.
% - Keep it simple, no one is interested in your street address.

\date{Appello Luglio, \today}
% - Either use conference name or its abbreviation.
% - Not really informative to the audience, more for people (including
%   yourself) who are reading the slides online

\subject{Eliosismologia}
% This is only inserted into the PDF information catalog. Can be left
% out. 

% If you have a file called "university-logo-filename.xxx", where xxx
% is a graphic format that can be processed by latex or pdflatex,
% resp., then you can add a logo as follows:

% \pgfdeclareimage[height=0.5cm]{university-logo}{university-logo-filename}
% \logo{\pgfuseimage{university-logo}}

% Delete this, if you do not want the table of contents to pop up at
% the beginning of each subsection:
%\AtBeginPart[]
%{
%  \begin{frame}<beamer>{Outline}    %\tableofcontents[currentsection]
%  \end{frame}
%}

%\AtBeginDocument{%
%\addtolength\abovedisplayskip{-0.5\baselineskip}%
%\addtolength\belowdisplayskip{-1\baselineskip}%
%\addtolength\abovedisplayshortskip{-0.5\baselineskip}%
%\addtolength\belowdisplayshortskip{-1\baselineskip}%
%}

\makeatletter
\AtBeginPart{%
  \addtocontents{toc}{\protect\beamer@partintoc{\the\c@part}{\beamer@partnameshort}{\the\c@page}}%
}
%% number, shortname, page.
\providecommand\beamer@partintoc[3]{%
  \ifnum\c@tocdepth=-1\relax
    % requesting onlyparts.
    \makebox[6em]{PART #1:} #2
    \par
  \fi
}
\define@key{beamertoc}{onlyparts}[]{%
  \c@tocdepth=-1\relax
}
\makeatother%

\setbeamertemplate{navigation symbols}{}

\makeatletter
\setbeamertemplate{headline}
{
    \leavevmode%
    \hbox{%Refintro
        \begin{beamercolorbox}[wd=.1\paperwidth,ht=2.25ex,dp=1ex,center]{author in head/foot}%
            \hyperlink{intro}{Intro}
        \end{beamercolorbox}%

 \begin{beamercolorbox}[wd=.1\paperwidth,ht=2.25ex,dp=1ex,center]{author in head/foot}%refs Part 1
            \hyperlink{part:MSS}{MSS}
        \end{beamercolorbox}%

 \begin{beamercolorbox}[wd=.2\paperwidth,ht=2.25ex,dp=1ex,center]{author in head/foot}%refs Part 2
            \hyperlink{part:oscillations}{Oscillazioni lineari adiabatiche}
        \end{beamercolorbox}%
        
         \begin{beamercolorbox}[wd=.2\paperwidth,ht=2.25ex,dp=1ex,center]{author in head/foot}%refs Part 3
            \hyperlink{part:inverseproblem}{Problema inverso}
        \end{beamercolorbox}%inverseproblem
        
        \begin{beamercolorbox}[wd=.35\paperwidth,ht=2.25ex,dp=1ex,right]{date in head/foot}%
            %   \usebeamerfont{date in head/foot}\insertshortdate{}\hspace*{2em}
            \insertframenumber{} \hspace*{2ex}  / \hspace*{2ex} \inserttotalframenumber
            \hspace*{2ex} 
        \end{beamercolorbox}}%
        \vskip0pt%
    }
    \makeatother

\AtBeginSection{\frame{\sectionpage}}

% Let's get started
\begin{document}

\addtobeamertemplate{block begin}{\setlength\abovedisplayskip{2pt}\setlength\belowdisplayskip{2pt}\setlength\abovedisplayshortskip{2pt}\setlength\belowdisplayshortskip{2pt}}

\addtobeamertemplate{block begin}{\vspace*{-3pt}}{}
\addtobeamertemplate{block end}{}{\vspace*{-3pt}}

\begin{frame}
  \titlepage
\end{frame}

% Section and subsections will appear in the presentation overview
% and table of contents.
%\frame{\tableofcontents[onlyparts]}

\begin{frame}[label={intro}]{Modi di oscillazione della struttura solare}

Inserire figura sole senza spicchio modi

\end{frame}

\begin{wordonframe}{Intro MSS e i suoi modi}

In questa tesina mi sono occupato dei modi normali gravo-acustici solari cio\'e oscillazioni attorno alla struttura di equilibrio le cui caratteristiche sono determinate dal profilo radiale della densit\'a e velocit\'a del suono. Il gran numero di frequenze osservate impone vincoli stringenti alle caratteristiche del modello solare standard.

Nella prima parte ho considerato analizzato il modello solare standard, nella seconda ho descritto i modi normali adiabatici, in particolare la relazione tra le frequenze dei modi e la struttura solare; infine ho considerato le tecniche che permettono di invertire la dipendenza delle frequenze dalla struttura interna del Sole.

Il modello MSS si basa su leggi di conservazione, valide per qualsiasi stella: le prime 3 equazioni usate per determinare la struttura stellare ... 

\end{wordonframe}



\part{Modello solare e osservabili sismologiche}\label{part:MSS}
\frame{\partpage}

%\begin{frame}{Argomenti}
%  \tableofcontents[part=1,hideallsubsections%,pausesections
%  ]
%  % You might wish to add the option [pausesections]
%\end{frame}


%\section{Equilibrio idrostatico}

\begin{frame}{Leggi di conservazione applicate alla struttura stellare}

%\begin{equation*}
%\tau_{idro}^{\odot}= \sqrt{\frac{R^3}{GM}}\approx\frac{1}{2}(G\overline{\rho})\expy{-\frac{1}{2}}\approx\SI{27}{\minute}
%\end{equation*}

\begin{block}{Simmetria sferica}
Deviazioni da forma sferica trascurabili (campi magnetici, rotazione)
\end{block}

\begin{block}{Conservazione della massa}

\begin{equation*}
%\TDy{t}{\rho}+\nabla\cdot(\rho\vec{v})=0
\TDy{r}{m}=4\pi r^2\rho
\end{equation*}

\end{block}

\begin{block}{Equilibrio idrostatico}

\begin{equation*}
0=-\nabla P(\rho,T)+\rho\vec{g}
%\TDy{r}{P}=-\frac{Gm(r)\rho(r)}{r^2}%\label{eq:fidroequilibrio}
\end{equation*}

\end{block}

\begin{block}{Conservazione energia interna}

\begin{equation*}
\TDy{t}{q}=\epsilon(\rho,T,X_i)-\frac{1}{\rho}\nabla\cdot\vec{F}=\TDy{t}{u}+P\TDof{t}(\frac{1}{\rho})%\label{eq:heatgl}
%&\TDy{r}{L}=4\pi r^2[\rho\epsilon-\rho\TDof{t}u+\frac{P}{\rho}\TDy{t}{\rho}]%\label{eq:fenergyconservation}
\end{equation*}

\end{block}

\begin{block}{Gradiente termico}
\begin{equation*}
\TDy{r}{T}=\nabla\frac{T}{p}\TDy{r}{p}
\end{equation*}
\end{block}

\end{frame}

\begin{wordonframe}{Simmetria sferica, conservazione massa, equilibrio idrostatico, conservazione energia interna (PRIMA LEGGE)}

Le equazioni che descrivono la struttura di una stella ... sono un caso particolare della conservazione della massa, dell'impulso, dell'energia interna.

Considero le deviazioni dalla simmetria sferica (rotazione campo magnetico) come perturbazioni, quindi le trascuro nella descrizione della struttura di equilibrio. 

L'evoluzione di una stella in sequenza principale come il Sole attuale avviene su tempi nucleari quindi assumo equilibrio idrostatico.

Quant'\'e il calore aggiunto per unit\'a di massa?? $\approx\midfrac{E_i}{\tau M}$ In sequenza principale il calore aggiunto per unit\'a di massa \'e trascurabile in qunato i tempi evolutivi sono molto maggiori dei tempi di rilassamento radiativi (tipicamente viene indicato il tempo impiegato ad irradiare l'energia interna).

Per determinare il gradiente termico considero il flusso di energia dalle regioni pi\'u calde alla superficie: oltre al trasporto radiativo devo considerare la convezione presente nella regione esterna per stelle con $M<1.1\msun{}$.

\end{wordonframe}


%\section{Meccanismi di trasporto dell'energia}

\begin{frame}{Trasporto radiativo}

\begin{columns}

\begin{column}{0.5\textwidth}

\begin{block}{Equazione del Trasporto radiativo}
\begin{align*}
%P_{rad}=\int\,d\nu\frac{4\pi}{3c}B_{\nu}=\frac{1}{3}aT^4\\
%\vec{F}=-\frac{4\pi}{3\kappa\rho}\nabla B=-\frac{4\pi}{3\kappa\rho}\nabla B=-\frac{c}{\kappa\rho}\nabla P_{rad}\\
\frac{dI_{\nu}}{\rho\,ds}=j_{\nu}-\kappa_{\nu}I_{\nu}\\
\end{align*}
\end{block}

\end{column}

\begin{column}{0.5\textwidth}

\begin{block}{Equilibrio termodinamico locale}

\begin{align*}
&4\pi j_{\nu}=\kappa_{\nu a}cu(\nu)\\
&B(\nu,T)=\frac{2h\nu^3}{c^2}\frac{1}{\exp{\midfrac{h\nu}{KT}}-1}
\end{align*}

\end{block}

\end{column}

\end{columns}

\begin{equation*}
I_{\nu}=\int_0^{\infty}B_{\nu}(-\tau_{\nu})\exp{-\tau_{\nu}}\,d\tau_{\nu}\approx B(\nu,T)-\frac{1}{\kappa_{\nu}\rho}\nabla_s B(\nu,T)
\end{equation*}

\begin{block}{Relazione tra flusso radiativo e gradiente termico negli interni stellari}

%Il flusso di energia verso la superficie \'e generato da una piccola anisotropia nell'intensit\'a descritta al prim'ordine tramite:
\begin{align*}
%&I_{\nu}=B(\nu,T)-\frac{1}{\kappa_{\nu}'\rho}\nabla_s B(\nu,T)\\
%P_{rad}=\int\,d\nu\frac{4\pi}{3c}B_{\nu}=\frac{1}{3}aT^4\\
%\vec{F}=-\frac{4\pi}{3\kappa\rho}\nabla B=-\frac{4\pi}{3\kappa\rho}\nabla B=-\frac{c}{\kappa\rho}\nabla P_{rad}\\
%&\frac{dI_{\nu}}{\rho\,ds}=\kappa_{a\nu}B_{\nu}(T)-\kappa_{\nu}I_{\nu}\\
%\frac{1}{\kappa}=(\frac{acT^3}{\pi})\expy{-1}\intzi{}\,d\nu\frac{1}{\kappa_{\nu}}\PDy{T}{B(\nu,T)}\label{eq:rosselandopacity}
&\vec{F}=-[\frac{4\pi}{3\rho}\intzi{}\frac{1}{\kappa_{\nu}}\PDy{T}{B(\nu,T)}\,d\nu]\nabla T%=-\frac{c}{\kappa\rho}\nabla P_{rad}
\end{align*}

\end{block}

\begin{block}{Gradiente radiativo}

\begin{equation*}
\nrad{}=\Dcvar{\PDly{P}{T}}{rad}=\frac{3}{16\pi acG}\frac{\kappa l(r)P}{m(r)T^4}%\label{eq:radiativegradient}
\end{equation*}

\end{block}

\end{frame}

\begin{wordonframe}{Relation between energy density u, energy flus in polar direction F and radiation pressure $P_r$ are related to 3 moments of radiation field $I(\theta)$}

\begin{align*}
&I(\theta)=I_0+I_1\cos{\theta}+\ldots\\
&u=\frac{1}{c}\int I(\theta)\,d\Omega=\frac{2\pi}{c}\int_0^{\pi} I(\theta)\sin{\theta}\,d\theta=\frac{4\pi}{c}I_0\\
&F=\int I(\theta)\cos{\theta}\,d\Omega=2\pi\int_0^{\pi} I(\theta)\cos{\theta}\sin{\theta}\,d\theta=\frac{4\pi}{3}I_1\\
&P_r=\frac{1}{c}\int I(\theta)\cos{\theta}^2\,d\Omega=\frac{2\pi}{c}\int_0^{\pi} I(\theta)\cos{\theta}^2\sin{\theta}\,d\theta=\frac{4\pi}{3c}I_0
\end{align*}

\end{wordonframe}

\begin{wordonframe}{Densit\'a di energia fotoni per unit\'a di volume e equilibrio termodinamico radiazione-materia}

Per determinare il gradiente termico considero il flusso di energia dalle regioni pi\'u calde alla superficie: oltre al trasporto radiativo devo considerare la convezione presente nella regione esterna per stelle con $M<1.1\msun{}$.

Il cammino libero medio nell'interno solare \'e molto breve quindi considero la radiazione in equilibro con la materia. La densit\'a di energia dei fotoni per unit\'a di frequenza \'e
\begin{align*}
&u(\nu)=\frac{8\pi h\nu^3}{c^3}\frac{1}{\exp{\midfrac{h\nu}{KT}}-1}\\
&u=aT^4\\
&I(\nu)=\frac{u(\nu)c}{4\pi}
\end{align*}

Equilibrio termodinamico locale (Emissione=assorbimento): opacit\'a dovuta a processi di assorbimento.
\begin{align*}
&4\pi j_{\nu}=\kappa_{\nu a}cu(\nu)\\
&B(\nu,T)=\frac{2h\nu^3}{c^2}\frac{1}{\exp{\midfrac{h\nu}{KT}}-1}
\end{align*}

\end{wordonframe}

\begin{wordonframe}{Emissione spontanea/Indotta}

In realt\'a devo tener conto del\[\frac{\text{Spontaneous emission}}{\text{Total emission}}=1-\exp{-\midfrac{h\nu}{KT}}\]

\[j_{\nu}(\theta)=\kappa_{a,\nu}(1-\exp{-\midfrac{h\nu}{KT}})B_{\nu}(T)+\exp{-\midfrac{h\nu}{KT}})I_{\nu}(T)\]

In LTE si riduce alla legge di Kirkoff.

\end{wordonframe}

\begin{wordonframe}{Trasporto radiativo negli interni stellari: equilibrio termodinamico locale ed espansione $B_{\nu}(-\tau_{\nu})$ al prim'ordine in $\tau_{\nu}$.}

{\small

Infine determino la relazione tra flusso verso la superficie e gradiente termico: oltre al trasporto radiativo devo considerare la convezione presente nella regione esterna per stelle con $M<1.1\msun{}$...

Negli interni stellari il cammino libero medio di un fotone $\invers{(\kappa_{\nu}\rho)}$ \'e molto breve quindi considero localmente equilibrio radiazione-materia.

L'intensit\'a di radiazione attraverso il punto $\vec{P}$ nella direzione $\hat{s}$ \'e determinata dalla somma dei contributi degli strati precedenti 


%Per quanto rigurda il trasporto radiativo la soluzione per l'intensit\'a al raggio r in direzione $\hat{s}$ in condizioni di equilibrio termodinamico locale pu\'o essere approssimata come somma (integrale) dei contributi all'intensit\'a  degli strati sottostanti diminuiti a causa dell'opacit\'a con il termine $\exp{\tau_{\nu}}$.

%Poich\'e negli interni stellari il cammino libero medio di un fotone \'e piccolo (per condizioni solari $\invers{(\kappa\rho)}$) considero l'espansione di $B_{\nu}(-\tau_{\nu})$ al prim'ordine

%(emissivit\'a per unit\'a di frequenza:$B(T)=\frac{ac}{4\pi}T^4$). La soluzione dell'equazione del trasporto radiativo \'e la somma di tutti i contributi $B_{\nu}(-\tau_{\nu})$ pesati dall'esponenziale che descrive l'assorbimento del plasma solare.
%In condizioni tipiche degli interni stellari si pu\'o esprimere l'intensit\'a specifica di radiazione nella direzione $\hat{s}$ come somma del termine isotropo $B_{\nu}(T)$ e del termine lineare nel cammino libero medio del fotone di frequenza $\nu$ che tiene conto della piccola anisotropia nell'intensit\'a specifica tramite $\nabla B_{\nu}(T)$.

%Integrando sulle direzioni uscenti e su tutte le frequenze si ottiene la relazione tra flusso radiativo e gradiente termico.

Integrando su tutte le direzioni e su tutte le frequenze si ottiene la relazione tra flusso e gradiente terico cercata.

La quantit\'a $\frac{1}{\kappa_{\nu}}\PDy{T}{B_{\nu}(T)}$ \'e l'opacit\'a media di Rosseland. Nel calcolo della struttura stellare si usa il gradiente termico espresso in termini della derivata logaritmica della pressione: ho usato la definizione di luminosit\'a e la condizione di equilibrio idrostatico.

}

\end{wordonframe}


\begin{frame}{Instabilit\'a convettiva - Gradiente termico nelle regioni convettive}

\begin{align*}
&\rho\PtwoDy{t}{(\Delta r)}=-g\Delta\rho=-g[\Dcvar{\TDy{r}{\rho}}{e}-\Dcvar{\TDy{r}{\rho}}{amb}]\Delta r=-N^2\Delta r\\
&N^2=g(\frac{1}{\Gamma_1P}\TDy{r}{P}-\frac{1}{\rho}\TDy{r}{\rho})=g(\frac{1}{\densityscale{}}-\frac{g}{c_s^2})%\label{eq:bvfs}
\end{align*}

\begin{columns}

\begin{column}{0.45\textwidth}

\begin{block}{Instabilit\'a convettiva}

\begin{align*}
&N^2<0\\
&\nrad{}>\nad{}
\end{align*}

\end{block}

\begin{block}{Trasporto convettivo}

\begin{align*}
&F=F_c+F_r\\
&F_{con}=\exv{\rho vc_P\Delta T}\\
&l_m=\alpha H_P
\end{align*}

\end{block}

\end{column}

\begin{column}{0.55\textwidth}

\begin{figure}[!h]
    \includegraphics[width=0.99\textwidth,keepaspectratio]{proportionflux}
    \caption{Da \cite{christensen1997effects}.}
\end{figure}


\end{column}

\end{columns}

\end{frame}

\begin{wordonframe}{Instabilit\'a convettiva - Gradiente termico nelle regioni convettive}

La convezione caratterizza le regioni in cui uno spostamento infinitesimo di un blob di gas cresce esponenzialmente a causa della forza di Archimede.

{\small Considero il moto del blob in equilibrio di pressione con l'ambiente: esprimo l'equazione del moto dell'elemento in funzione della frequenza di \bv{} $N^2$. Le regioni stabili, in cui l'equazione del moto descrive un comportamento oscillatorio, sono per $N^2>0$  oppure usando l'equazione di stato per esprimere i gradienti di densit\'a in funzione dei gradienti di temperatura $\nrad{}<\nad{}$.)
}

% e uso l'equazione di stato per esprimere la differenza di densit\'a tra blob e ambiente. Considero in prima approssimazione il moto del blob adiabatico ($dq=c_P\,dT-\frac{\delta}{\rho}\,dP$) (e $\nmu{}\approx0$).

Per determinare il gradiente ambientale nelle regioni convettive esterne, che caratterizzano stelle $M\leq1.1\msun{}$ si usa la teoria della ML. Questa considera l'eccesso di calore trasportato dal moto dell'elemento di gas in equilibrio di pressione con l'ambiente che rilasci il calore dopo aver percorso una lunghezza caratteristica $l_m=\alpha H_P$.

L'efficienza delle regioni convettive esterne \'e quindi un parametro libero determinato tramite calibrazione del modello solare per riprodurre $T_e$.

\end{wordonframe}


%\section{Costruzione modello solare}

\begin{frame}{Modello plasma solare}

\begin{block}{EOS e grandezze termodinamiche. Popolazione di stati atomici.}

\begin{columns}

\begin{column}{0.5\textwidth}

\begin{figure}[!ht]
        \includegraphics[width=0.9\textwidth,keepaspectratio]{ionfraction}
        \caption{Da \cite{basu2008helioseismology}.}
\end{figure}%


\end{column}

\begin{column}{0.5\textwidth}

%Interazioni coulombiane:
%\begin{equation*}
%\frac{1}{r_D^2}=\frac{4\pi e^2}{kT}\sum_Z Z^2\overline{n}_Z=\frac{4\pi e^2}{kT}N_A\sum_{i}(Z_i^2+Z_i)\frac{\rho X_i}{A_i}
%\end{equation*}
\[\Gamma_1=\left(\PDly{\rho}{P}\right)_s\]

\begin{figure}[!ht]
        \includegraphics[width=0.9\textwidth,keepaspectratio]{gamma1eos}
        \caption{Da \cite{trampedach2006synoptic}.}
\end{figure}

\end{column}

\end{columns}

\end{block}

\end{frame}

\begin{wordonframe}{Modellizzazione plasma solare}

{\small Per risolvere le equazioni della struttra stellare \'e necessario conoscere le propriet\'a termodinamiche del gas, in particolare l'equazione di stato $P(\rho,T)$.

{\small Nel caso solare l'equazione dei gas perfetti \'e una prima approssimazione poich\'e gli effetti della degenerazione elettronica e delle interazioni coulombiane sono piccoli.}

%Gli approcci pi\'u usati per ricavare un'equazione di stato accurata per il plasma stellare considerano le propriet\'a statistiche di un gas di nuclei ed elettroni oppure di un gas di atomi: nel secondo caso sono necessarie ipotesi ad hoc per tener conto delle perturbazioni ai livelli atomici causati dalle particelle vicine.
\begin{align*}
&P=P_r+P_G=\frac{1}{3}aT^4+\frac{\rho}{\mu}\gasconstant{}T\\
&u=\frac{1}{\rho}\sum_i\int f^{(0)}(\vec{p}_i)\frac{p^2_i}{2m_i}\,d^3p_i=\frac{3}{2}\frac{P}{\rho}%=\frac{3}{2}\frac{\gasconstant{}T}{\mu}
\end{align*}

{\small La principale correzione che tiene conto della natura non puntiforme delle interazioni \'e la correzione coulombiana: le interazioni elettriche modificano la distribuzione delle cariche nel plasma neutro su distanze maggiori di $R_D$ (???cariche su sfera di raggio $R_D$ attorno a ione hanno energia coulombiana e termica paragonabili????).}

Una determinazione accurata delle caratteristiche del plasma solare, in particolare dell'esponente adiabatico $\Gamma_1$, permette di investigare la composizione solare poich\'e nelle regioni di ionizzazione degli elementi pi\'u abbondanti si hanno variazioni dal valore $\Gamma_1=\frac{5}{3}$.

A destra \'e mostrato l'andamento dell'esponente adiabatico calcolato tramite due EOS 

}

\end{wordonframe}


\begin{frame}{Evoluzione chimica: reazioni nucleari e diffusione.}

\begin{block}{Rate di reazione per coppia nuclei}

\begin{align*}
%&\sigma=\pi\lambdabar^2\Exp{-\dfrac{2\pi Z_1Z_2e^2}{\hbar v}}S(E)\\
%&\exv{\sigma v}=\num{1.3005e-15}[\frac{Z_1Z_2}{AT_6^2}]\expy{\frac{1}{3}}fS_{eff}\exp{-\tau}\si{\cubic\cm\per\second}\\
&\exv{\sigma v}=\num{1.3005e-15}[\frac{Z_1Z_2}{AT_6^2}]\expy{\frac{1}{3}}fS_{eff}\exp{-\num{42.487}(Z_1^2Z_2^2AT_6\expy{-1})\expy{\frac{1}{3}}}\si{\cubic\cm\per\second}\\
%&\tau=\frac{3E_G}{kT}\approx\num{42.487}(Z_1^2Z_2^2AT_6\expy{-1})\expy{\frac{1}{3}}\\
%&E_G=A\expy{\frac{1}{3}}T_7\expy{\frac{2}{3}}\SI{5.665}{\kilo\ev}
\end{align*}

\end{block}

\begin{block}{Sequenza principale}
\[4\cel{H}{1}{}{}\to\cel{He}{4}{}{}+2\APelectron+2\Pnue\]
\[Q_e=\SI{28.6}{\mega\ev}-\exv{\nu}\]
\end{block}


\begin{block}{Diffusione}

\begin{equation*}
\vec{F}_i=-\nabla P_i+n_i(q_i\vec{E}+m_i\vec{g})=\sum_{i\neq j}\int m_{ij}\vec{v}_{ij}C(f_i,f_j)\,d^3v_i%
\end{equation*}

\end{block}

\end{frame}

\begin{comment}
\begin{columns}
\begin{column}{0.5\textwidth}
\end{column}
\begin{column}{0.5\textwidth}
\end{column}
\end{columns}
\end{comment}

\begin{wordonframe}{Evoluzione chimica: reazioni di fusione e diffusione}

Le reazioni nucleari generano la luminosit\'a stellare nella maggior parte della vita di una stella e ne determinano l'evoluzione chimica.

Il rate di fusione per coppia di nuclei \'e la media della seziode d'urto di fusione per la velocit\'a rispetto alla funzione di MB.
La sezione d'urto di fusione contiene il termine geometrico $\pi\lambdabar^2\propto\frac{1}{E}$ (contributo alla sezione d'urto in onda S: $b=l\lambdabar$),  la probabilit\'a di attraversamento della barriera coulombiana $P_0=\exp{-2\pi\eta}$ e il contributo delle funzoni d'onda nucleari.

Lo schermaggio elettronico aumenta la probabilit\'a di penetrare la barriera coulombiana.

%La funzione $\epsilon(T,\rho,X_i)$ descrive la produzione di energia.

Per stelle di sequenza principale la reazione efficace converte 4 protoni n un elio: $Q=28.6 meV$ meno l'energia dei neutrini per nucleo di elio prodotto.


Inoltre un piccolo contributo ma non trascurabile \'e dato 
dalla diffusione: la presenza di un gradiente di concentrazione determina diffusione in senso opposto; a livello microscopico la risultante delle forze per unit\'a di volume agenti su una specie di particelle pu\'o essere non nulla, quindi in condizioni stazionarie il momento trasferito tramite urti (proporzionale alla velocit\'a di diffusione relativa) con le altre specie \'e uguale alla forza per unit\'a di volume.

\end{wordonframe}



%\section{Equazioni struttura stellare}


\begin{frame}{Modello solare standard}

%Determino la struttura solare integrando numericamente le equazioni fondamentali della struttura stellare
\begin{subequations}%\label{subeqn:stellarstructure}
\begin{align*}
&\TDy{r}{m}=4\pi r^2\rho\\
&\TDy{r}{P}=-\frac{Gm(r)\rho(r)}{r^2}\\
&\TDy{r}{T}=\nabla\frac{T}{p}\TDy{r}{p}\\
&\TDy{r}{L}=4\pi r^2[\rho(\epsilon-\epsilon_{\nu})-\rho\TDof{t}u+\frac{P}{\rho}\TDy{t}{\rho}]
\end{align*}
\end{subequations}
%con $v_i$ velocit\'a di diffusione specie i. Ottengo il profilo radiale delle grandezze $\{P,m,T,L,X_i\}$, note la metallicit\'a iniziale Z, l'equazione di stato $P(\rho,T,X_i)$, l'opacit\'a $\kappa(P,T,X_i)$, il rate di produzione di energia nucleare per grammo $\epsilon(P,T,X_i)$.

\begin{columns}

\begin{column}{0.3\textwidth}
Condizioni al bordo:
\end{column}

\begin{column}{0.5\textwidth}
\begin{itemize}
    \item Superficie: $T_e$, $P_{atm}$
    \item Centro: $l(r=0)$, $m(r=0)$.
\end{itemize}

\end{column}

\end{columns}

\begin{block}{Evoluzione del modello solare}

\begin{equation*}
\PDy{t}{n_i}+\frac{1}{r^2}\PDof{r}(r^2n_iv_i)=\Dcvar{\PDy{t}{n_i}}{Nucl}%\label{eq:difffusionchange}
\end{equation*}

\end{block}

\end{frame}


\begin{wordonframe}{Soluzione equazioni modello solare}

Integro numericamente per la struttura solare con 2 condizioni in superficie per $T_e$ e $P$ e due condizioni al centro per $l$ e $m$ con metallicit\'a e abbondanza di elio da determinare.

L'evoluzione chimica del core di fusione comporta un aumento di temperatura che avviene su tempi-scala molto maggiori dei tempi caratteristici per trasporto radiativo quindi la struttura del modello solare \'e stazionaria.

La presenza di gradienti di concentrazione, di pressione e termici causano fenomeni di diffusione. La velocit\'a di diffusione \'e determinata supponendo il processo stazionario cio\'e la forza per unit\'a di volume agente sulla specie i \'e bilanciata dal trasperimento di impulso negli urti con le altre specie.

\end{wordonframe}

\begin{frame}{Calibrazione modello solare}


\begin{columns}

\begin{column}{0.5\textwidth}

\begin{block}{Et\'a, luminosit\'a, massa, raggio solari}

\begin{tabular}{l|c}
\hline
$\agesun{}$&\SI[separate-uncertainty=true]{4.57\pm0.02e9}{\year}\\
\hline
$\rsun{}$&\SI{695658+-140}{\kilo\meter}\\
\hline
$G\msun$&\num{132712440018+-8}\SI{e9}{\cubic\meter\per\square\second}\\
\hline
$\lsun{}$&\SI{3.8275+-0.0014e33}{\erg\per\second}\\
\hline
\end{tabular}

\end{block}

\begin{block}{Composizione chimica}

Righe di assorbimento della fotosfera - Abbondanza elementi pesant $\left(\frac{Z}{X}\right)_{odot}$

\end{block}

\end{column}

\begin{column}{0.5\textwidth}


\begin{block}{Abbondanze elio iniziale}
Un maggiore peso molecolare medio comporta un maggiore temperatura nel core di fusione
\end{block}

\begin{block}{efficienza convezione}
Mixing length $l_m=H_p\alpha$
\end{block}

\begin{block}{diffusione}
\[\left(\frac{Z}{X}\right)_{\odot}\]
\end{block}


\end{column}

\end{columns}


\end{frame}

\begin{wordonframe}{Costruzione del MSS e vincoli}


Determino i parametri del modello solare in maniera che riproduca la luminosit\'a, la temperatura efficace e la composizione attuale: il valore di Y determina principalmente la temperatura, l'efficienza della convezione la temperatura superficiale.

%La costruzione di un modello solare \'e molto meno incerta rispetto a un modello di una stella generica data la mole di informazioni disponibili. Conosciamo con grande accuratezza la distanza tramite lo studio delle orbite dei corpi celesti attorno al Sole, l'et\'a di formazione del sistema solare tramite studio dei meteoriti, la luminosit\'a, e il raggio tramite la misura delle dimensioni apparenti del disco solare.

Inoltre \'e possibile determinare la composizione solare della fotosfera supposta identica a quella della zona convettiva tramite l'analisi delle righe di assorbimento nell'atmosfera.

Le righe di assorbimento dell'elio si formano nelle regioni pi\'u esterne dell'atmosfera dove non \'e ovvia la relazione con la composizione della fotosfera quindi il metodo pi\'u accurato per determinare l'abbondanza di elio \'e tramite calibrazione del modello solare.

L'incertezza sul contenuto di metalli \'e dovuta alle difficolt\'a insite nella descrizione dell'atmosfera: come si vede dalla tabella a destra la disponibilit\'a di modelli pi\'u accurati nelle assunzioni fisiche ha portato a un'abbassamento della metallicit\'a. 

\end{wordonframe}


\begin{frame}{Struttura del MSS}

\begin{figure}[!h]
\includegraphics[width=0.75\textwidth,trim=4 4 4 4,clip]{BP00-SSM-R}
\caption{Dati da \cite{BP2000}.}
\end{figure}

\end{frame}

\begin{wordonframe}{Struttura del modello solare - Grandezze sismologiche}

I grafici mostrano il profilo radiale della densit\'a, abbondanza di idrogeno, temperatura e luminosit\'a.

Il profilo dell'abbondanza di idrogeno \'e determinato dalle reazioni di fusione e dalla diffusione, si assume che i moti convettivi rendano la regione convettiva chimicamente omogenea.

La discontinuit\'a nel gradiente termico alla base della zona convettiva \'e evidente nel grafico del profilo termico.

\end{wordonframe}

\begin{frame}{Velocit\'a del suono calcolato tramite MSS}

\begin{figure}[!ht]
\includegraphics[width=0.9\textwidth]{BP00-cs}
\end{figure}

\end{frame}

\begin{wordonframe}{Profilo radiale della velocit\'a del suono determinato tramite MSS}

In figura abbiamo il profilo della velocit\'a del suono del modello solare precedente.

\end{wordonframe}


\part{Oscillazioni della fotosfera con grande coerenza spaziale e temporale - Modi normali di cavit\'a risonanti dell'interno solare}\label{part:oscillations}

\frame{\partpage}

%\begin{frame}{Argomenti}
%  \tableofcontents[part=2,hideallsubsections%,pausesections
%  ]
  % You might wish to add the option [pausesections]
%\end{frame}

%\section{Modi normali della struttura solare}

\begin{frame}{Modi normali: perturbazioni di una struttura a simmetria sferica}
\begin{block}{Equazione del moto perturbato linearizzata}
\begin{equation*}
\rho_0\TDof{t}\vec{v}=\rho_0\PtwoDy{t}{\vec{\xi}}=-\nabla P'+\rho_0\vec{g}'+\rho'\vec{g}_0%\label{eq:emper}
\end{equation*}
\end{block}
\begin{block}{Equazione di continuit\'a e del moto perturbate}
\begin{equation*}
\rho'+\div{(\rho_0\vec{\xi})}=0%\label{eq:contper}
\end{equation*}
\end{block}
\begin{block}{Condizione di moto adiabatico}
\begin{equation*}
P'+\vec{\xi}\cdot\nabla P_0=\frac{\Gamma_{1,0}P_0}{\rho_0}(\rho'+\vec{\xi}\cdot\nabla\rho_0)%\label{eq:adper}
\end{equation*}
\end{block}
\end{frame}

\begin{wordonframe}{Determinazione del moto perturbato}

L'equazione del moto nel SR solidale con l'elemento di gas linearizzata nella perturbazione insieme alla condizione di adiabaticit\'a del moto e la continuit\'a della massa, sempre al termine lineare nella perturbazione, sono un sistema di 5 equazioni in 5 incognite che con le opportune condizioni al bordo determina i modi di oscillazione adiabatici della struttura solare.
Considero uno spostamento infinitesimo $\vec{\xi}$ dalla posizione di equilibrio di un elemento di gas: la forza agente sull'elemento \'e determinata da pressione densit\'a e accelerazione di gravit\'a a $\vec{r}+\vec{\xi}$. Ottengo l'equazione del moto perturbata al prim'ordine nelle perturbazioni.

Le quantit\'a primate sono perturbazioni euleriane e $\vec{g}'=-\nabla\Phi'$ e $\nabla^2\Phi'=4\pi G\rho'$.

Completo il sistema con l'equazione di continuit\'a e la condizione di moto adiabatico. La condizione di adiabaticit\'a \'e giustificata perch\'e il periodo dei modi \'e molto minore del tempo di rilassamento radiativo e dei tempi scala di generazione di energia.

\end{wordonframe}

\begin{frame}{Modi normali struttura sferica}

\begin{block}{Onde stazionarie}
\begin{align*}
&Y_{lm}(\theta,\phi)=(-)^mc_{lm}P_l^m(\cos{\theta})\exp{im\phi}\\
&(\rho',P',\Phi')=\exp{i\omega t}[\rho'(r),P'(r),\Phi'(r)]Y_l^m
\end{align*}
\end{block}

\begin{block}{Componente tangenziale dello spostamento perturbato}

\begin{align*}
&\vec{\xi}=\exp{i\omega t}(\xi_r(r),\frac{\xi_h(r)}{L}\PDof{\theta},\frac{\xi_h(r)}{L\sin{\theta}}\PDof{\phi})Y_l^m(\theta,\phi)\\
&\xi_h(r)=\frac{L}{r\omega^2}(\frac{P'(r)}{\rho_0}+\Phi'(r))
\end{align*}

\end{block}

\end{frame}

\begin{wordonframe}{Onde stazionarie simmetria sferica}

Cerco soluzioni della forma di onde stazionarie per le perturbazioni scalari e data la simmetria sferica con dipendenza spaziale descritta tramite armoniche sferiche. Utilizzo l'equazione del moto perturbato per mostrare che $\rot{\vec{\xi}}$ non ha componenti radiali: la dipendenza spaziale \'e determinata da un'ampiezza per lo spostamento radiale e un'ampiezza per lo spostamento tangenziale. Applicando la componente tangenziale dell'operatore divergenza all'equazione del moto si trova l'ampiezza tangenziale. 

Questo \'e il sistema che determina i modi normali con le opportune condizioni al bordo: ha soluzioni per $\omega_{nlm}$ discrete.

\end{wordonframe}

\begin{frame}{Modi normali: sistema equazioni fondamentale}

\begin{block}{Modi normali: frequenze discrete dei modi}

\begin{subequations}%\label{eigenomega}
\begin{align*}
&\frac{1}{r^2}\TDof{r}(r^2\xi_r)-\frac{\xi_rg}{c^2}+\frac{1}{\rho_0}(\frac{1}{c^2}-\frac{l(l+1)}{r^2\omega^2})P'-\frac{l(l+1)}{r^2\omega^2}\Phi'=0\\
&\frac{1}{\rho_0}(\TDof{r}+\frac{g}{c^2})P'-(\omega^2-N^2)\xi_r+\TDy{r}{\Phi'}=0\\
&\frac{1}{r^2}\TDof{r}(r^2\TDy{r}{\Phi'})-\frac{l(l+1)}{r^2}\Phi'-\frac{4\pi G\rho_0}{g}N^2\xi_r-\frac{4\pi G}{c^2}P'=0
\end{align*}
\end{subequations}

Condizioni al contorno:

Soluzioni regolari per $r=0$: $P'=0$, $\Phi'=0$.

La condizione di non propagazione oltre la fotosfera: $\Lvar{P}=P'+\xi_r\TDy{r}{P}=0$.

\end{block}

\begin{block}{Approssimazione di Cowling.}
\[\Phi'=0\]
\end{block}

\end{frame}

\begin{wordonframe}{Spettro dei modi: frequenze discrete}

Questo \'e il sistema che determina i modi normali: ha soluzioni per $\omega_{nlm}$ discrete. Le condizioni al bordo richiedono regolarit\'a delle perturbazioni in 0 e riflessione totale delle perturbazioni alla superficie solare. $(l,m)$ caratterizzano la dipendenza angolare dei modi mentre n \'e l'ordine radiale che identifica il numero di zeri radiali del vettore spostamento.

Le frequenze sono $2l+1$ degeneri perch\'e il sistema non dipende da m.

L'approssimazione di Cowling, valida per grandi l o n (uso la soluzione dell'equazione di Poisson per $\Phi'$), consiste nel trascurare la perturbazione al potenziale gravitazionale, restringendosi quindi alle prime 2 equazioni.

\end{wordonframe}

\begin{frame}{Spettro delle oscillazioni adiabatiche}

\begin{columns}

\begin{column}{0.5\textwidth}
\begin{tikzpicture}
\node (inertia) at (0,0) {\includegraphics[keepaspectratio,width=0.9\textwidth]{Inertianl}};
\node [below=2mm of inertia.south] {\parbox{0.85\textwidth}{\captionof{figure}{Da \cite{dal03notes}.}}};
\node [left=2mm of inertia.west,rotate=90] {$I_{nl}$};
\end{tikzpicture}
\begin{align*}
%&E_k^{nl}=\frac{1}{2}\int_V|\vec{v}|^2\rho_0\,dV=\frac{\omega_{nl}^2}{2}\int_V|\vec{\xi}_{nl}|^2\rho_0\,dV\\
&\bar{E_k^{nl}}=\frac{1}{\Pi}\int_0^{\Pi}\int\frac{1}{2}\rho_0|\dvec{\xi}_{nl}|^2\,d^3x\,dt=\frac{1}{2}M_{\odot}I_{nl}V_{nl}^2\\
&I_{nl}=\frac{1}{\msun{}A_{nl}^2}\int_V\,d^3x\rho_0\vec{\xi}_{nl}\vec{\xi}_{nl}^*%\label{eq:normalizedinertia}\\
%&\exv{\vec{\xi}(R)\vec{\xi}^*(R)}
%&M_{nl}=I_{nl}\msun{}\intxt{inoltre ho introdotto lo spostamento quadratico medio superficiale}
%&A_{nl}^2=\delta r|_{rms}^2+\delta h_{rms}^2\intxt{con}
%&\delta r|_{rms}^2=\frac{1}{\Pi}\int_0^{\pi}\,dt\frac{1}{4\pi}\oint\Re{[\xi_rY_{l,m}(\theta,\phi)\exp{-i\omega t}]}^2\,d\Omega=\frac{1}{2}|\xi_r|^2\\
%&\delta h_{rms}^2=\frac{1}{2}l(l+1)|\xi_h(r)|^2%=\frac{\delta h|_{rms}}{\delta r|_{rms}}=\frac{\sqrt{l(l+1)}}{\sigma}
\end{align*}
\end{column}

\begin{column}{0.5\textwidth}
\begin{figure}[!ht]
%\includegraphics[keepaspectratio,width=0.95\textwidth]{midlmodes}
%\caption{I picchi della densit\'a spettrale si dispongono su creste in cui \'e concentrata la potenza in accordo al modello. Determinata usando i primi 144 giorni di osservazione di MDI con $l\leq300$. Da \cite{chr02helioseismology}.}\label{fig:midlmodes}
\includegraphics[keepaspectratio,width=0.9\textwidth]{nrmodesLAWE}
\caption{Da \cite{chr02helioseismology}.}%\label{fig:nrmodesLAWE}
\end{figure}

\end{column}

\end{columns}

\end{frame}

\begin{wordonframe}{Spettro dei modi: modi p, modi g, modi f}

La figura mostra lo spettro delle frequenze discrete per cui il sistema dei modi ha soluzione in funzione del grado angolare l. I modi con stesso ordine radiale sono uniti da una linea: nella parte ad alte frequenze abbiamo i modi p, di natura acustica, nella parte a basse frequenze modi g, che sono onde di gravit\'a, e modi f, modi di superficie ($\div{\vec{\xi}}\approx0$, $\omega^2=gk_h$). 

L'energia cinetica del modo $(n,l)$ mediata su un periodo \'e la media temporale dell'integrale della densit\'a di energia sul volume solare: \'e utile introdurre l'inerzia normalizzata all'ampiezza superficiale.

\end{wordonframe}

\begin{frame}{Osservazione delle oscillazioni della fotosfera}

\begin{figure}[!ht]

%\includegraphics[keepaspectratio,width=0.95\textwidth]{midlmodes}
%\caption{I picchi della densit\'a spettrale si dispongono su creste in cui \'e concentrata la potenza in accordo al modello. Determinata usando i primi 144 giorni di osservazione di MDI con $l\leq300$. Da \cite{chr02helioseismology}.}\label{fig:midlmodes}

\includegraphics[keepaspectratio,width=0.95\textwidth]{PSD}
\caption{Da \cite{houdek2006stochastic}.}\label{fig:PSD}

\end{figure}

\begin{align*}
&I_{nl}[\TtwoDy{t}{A_{nl}}+\Gamma_{nl}\TDy{t}{A_{nl}}+\omega_{nl}^2A_{nl}]=f(t)\\
&P(\omega)\propto P_LP_f=\frac{\midfrac{\Gamma_{nl}}{2\pi}}{(\omega-\omega_{nl})^2+\midfrac{\Gamma_{nl}}{4}}P_f\\
&P_x(\nu)=\sum_i\frac{AB_i\midfrac{\Gamma^2}{4}}{(\nu-\nu_0-\Delta\nu_i)^2+\midfrac{\Gamma^2}{4}}+P_n(\nu)\\
&\exv{V_{nl}^2}=I_{nl}\frac{1}{T_{obs}}\int_{-\infty}^{\infty}|V_{nl}(\nu)|^2\,d\nu\propto I_{nl}\Gamma_{nl}A_{nl}
\end{align*}


\end{frame}

\begin{wordonframe}{Forma dei picchi nella PSD}

Lo studio delle oscillazioni solari come disciplina \'e l'eliosismologia. L'eliosismologia nasce all'inizio degli anni ''60 quando Leighton e collaboratori osservarono un comportamento oscillatorio nell'atmosfera solare con periodo attorno ai 5 minuti.

Questo comportamento \'e dovuto alla sovrapposizione di onde acustiche instappolate in cavit\'a risonanti a diversa profondit\'a a seconda delle grandezze caratteristiche orizzontale. Le frequenze dei modi sono determinate dalla velocit\'a del suono in queste regioni imponendo che  la cavit\'a contenga radialmente un numero semi-intero di lunghezze d'onda. \'E possibile quindi determinare il profilo radiale della velocit\'a del suono dalle frequenze dei modi osservate.


(modi basso l: N=13, n=21)

Le frequenze dei modi sono determinate principalmente misurando lo spostamento doppler di opportune righe di assorbimento in atmosfera ($Ca \lambda=6439 A 129 Km$, $K: 7699 A, 250Km$, $Ni I: 6708 A,300 Km$, $Na D1/D2: 5690 A, 500 Km$). Per identificare modi con $l>2$ \'e necessaria risolouzione spaziale e la proiezione di un'opportuna maschera $W_{l_0m_0}$ isoler\'a il modo con dipendenza spaziale descritta da $Y_{l_0m_0}$. La serie temporale del segnale doppler viene analizzata in termini di componenti armoniche. Le frequenze sono quindi determinate fittando i picchi dello spettro secondo una forma lorentziana  giustificata assumendo che l'ampiezza superficiale dei modi obbedisca all'equazione del moto di un'oscillatore armonico smorzato di frequenza naturale data dalla frequenza del modo e forzante stocastica.

La PSD $P_T(\nu)=\frac{1}{T_{obs}}|X_T(\nu)|^2$ contiene oltre al picco principale (lorentziano) picchi parassiti dovuti a gap nella serie temporale e il contributo del rumore solare.

\end{wordonframe}

\begin{frame}{Modello dell'ampiezza superficiale delle oscillazioni}

\begin{columns}

\begin{column}{0.4\textwidth}

\begin{figure}[!ht]
\centering
\includegraphics[keepaspectratio,width=0.85\textwidth]{modespheomenology}
\caption{Da \cite{libbrecht1988solar}.}\label{fig:Powerspectraldensity}
\end{figure}

\end{column}

\begin{column}{0.6\textwidth}

\begin{align*}
%&I_{nl}[\TtwoDy{t}{A_{nl}}+\Gamma_{nl}\TDy{t}{A_{nl}}+\omega_0^2A_{nl}]=f(t)\\
%&P(\omega)\propto P_LP_f=\frac{\midfrac{\Gamma_{nl}}{2\pi}}{(\omega-\omega_{nl})^2+\midfrac{\Gamma_{nl}}{4}}P_f\\
&\TDy{t}{E_{nl}}+\Gamma_{nl}E_{nl}=\exv{\dvec{\xi}_{nl}\cdot\vec{F}}\\
&\bar{E}_{nl}=I_{nl}\exv{V_{nl}^2}=I_{nl}\frac{1}{T_{obs}}\int_{-\infty}^{\infty}|V_{nl}(\nu)|^2\,d\nu\propto I_{nl}\Gamma_{nl}A_{nl}\\
&\bar{E}_{nl}=\frac{\exv{\dvec{\xi}_{nl}\cdot\vec{F}}}{\Gamma_{nl}}
\end{align*}
%dove $\exv{\dvec{\xi}_{nl}\cdot\vec{F}}$ \'e il lavoro della forzante mediato su un periodo.

\end{column}

\end{columns}

\end{frame}

\begin{wordonframe}{Ampiezza delle oscillazioni}

L'energia dell'oscillatore varia su tempi scala proporzionali a $\invers{\Gamma}_{nl}$ attorno al valor medio $\bar{E}_{nl}$:

\end{wordonframe}


\begin{frame}{Relazione di dispersione per onde gravo-acustiche: cavit\'a risonanti.}


\begin{align*}
%&\omega_A=\frac{c_s}{2\densityscale{}}\sqrt{1-2\TDy{r}{\densityscale{}}}\propto T\expy{-\frac{1}{2}}\\%\label{eq:acusticcutoff} \intxt{quindi ottengo la relazione di dispersione:}
%&\vec{\xi}\propto\exp{i\scap{k}{r}}\\
&\TtwoDy{r}{\xi_r}=\frac{\omega^2}{c_s^2}(1-\frac{N^2}{\omega^2})(\frac{S_l^2}{\omega^2}-1)\xi_r=-k_r^2\xi_r
\end{align*}

\begin{columns}

\begin{column}{0.4\textwidth}
Onde acustiche:\[k_r^2=\frac{\omega^2}{c_s^2}(\frac{S_l^2}{\omega^2}-1)\]
Raggio di inversione del moto:\[\frac{k_h^2}{\omega^2}=c_s^2\]
Onde di gravit\'a:\[k_r^2=\frac{S_l^2}{c_s^2}(\frac{N^2}{\omega^2}-1)\]

\end{column}

\begin{column}{0.6\textwidth}

\begin{figure}[!ht]
\centering
\includegraphics[keepaspectratio,angle=0,width=0.9\textwidth]{plowertp}
\caption{Da \cite{dal03notes}.}\label{fig:plowertp}
\end{figure}

\end{column}

\end{columns}

\end{frame}

\begin{wordonframe}{Relazione di dispersione e frequenze critiche}

\end{wordonframe}

\begin{frame}{Condizione di risonanza radiale}


\begin{figure}[!ht]
\centering
\includegraphics[keepaspectratio,width=0.6\textwidth]{raypath-gp}
\caption{Da\cite{gou91seismic}.}
\end{figure}

\begin{align*}
%&(n+\alpha)\pi\approx\int_{r_t}^Rk_r\,dr\\
&(n+\alpha)\pi\approx\int_{r_t}^R\frac{\omega}{c_s}\sqrt{1-\frac{S_l^2}{\omega^2}}\,dr
\quad
\frac{(n+\alpha_g)\pi}{L}\approx\int_{r_1}^{r_2}(\frac{N^2}{\omega^2}-1)\expy{\frac{1}{2}}\,d\ln{r}
\end{align*}

\end{frame}

\begin{wordonframe}{Condizione di risonanza radiale}

La condizione di risonanza radiale \'e nota come legge di Duvall per modi p

\end{wordonframe}

\begin{frame}{Effetti delle regioni superficiali sulle frequenze dei modi}

\begin{columns}

\begin{column}{0.5\textwidth}

\begin{figure}
\includegraphics[keepaspectratio,width=0.7\textwidth]{domega}
\captionof{figure}{Da \cite{rhodesmeasurements}.}
\end{figure}

\end{column}

\begin{column}{0.5\textwidth}
%\begin{figure}[!ht]
%\includegraphics[keepaspectratio,width=0.95\textwidth]{midlmodes}
%\caption{I picchi della densit\'a spettrale si dispongono su creste in cui \'e concentrata la potenza in accordo al modello. Determinata usando i primi 144 giorni di osservazione di MDI con $l\leq300$. Da \cite{chr02helioseismology}.}\label{fig:midlmodes}
%
\begin{align*}
&\omega_A=\frac{c_s}{2\densityscale{}}\sqrt{1-2\TDy{r}{\densityscale{}}}\propto T\expy{-\frac{1}{2}}
\end{align*}


%\end{figure}
\end{column}

\end{columns}

\end{frame}

\begin{wordonframe}{Effetti delle regioni superficiali sulle frequenze dei modi}

Contributo non-adiabatico ($\alpha(\omega)$): differenze dovute a effetti non adiabatici in superficie maggiori per l maggiori perche confinati in guscio sottile e maggiori ad alte frequenze perch\'e punto di inversione pi\'u in alto.

\end{wordonframe}
%$\omega_c=\frac{c_s}{2\densityscale{}}\sqrt{1-2\TDy{r}{\densityscale{}}}\propto T\expy{-\frac{1}{2}}$
%$S_l^2=\frac{l(l+1)c_s^2}{r^2}$.
%$N^2=g(\frac{1}{\Gamma_1P}\TDy{r}{P}-\frac{1}{\rho}\TDy{r}{\rho})=g(\frac{1}{\densityscale{}}-\frac{g}{c_s^2})$

\begin{comment}

\begin{frame}{Espressioni asintotiche per le frequenze}

\begin{columns}

\begin{column}{0.5\textwidth}

\begin{block}{Modi p con l grande}
\begin{equation*}
\omega_{nl}^2=\frac{2}{\mu_p}\frac{g}{\rsun{}}(n+\alpha)L
\end{equation*}
\end{block}

\end{column}

\begin{column}{0.5\textwidth}

\begin{block}{Modi p con l piccolo}

\begin{equation*}
\nu_{nl}=\frac{\omega_{nl}}{2\pi}\approx(n+\frac{l}{2}+\frac{1}{4}+\alpha)\Delta\nu*\label{eq:freqequi}
\end{equation*}
\[\Delta\nu=[2\int_0^R\frac{dr}{c}]\expy{-1}\approx\SI{136}{\micro\hertz}\]

\end{block}

\end{column}

\end{columns}

\begin{block}{Modi g di alto n e basso l}

\begin{equation*}
\omega=\frac{L\int_{r_1}^{r_2}N\frac{dr}{r}}{\pi(n+\frac{l}{2}+\alpha_g)}
\end{equation*}
 $\lim_{n\to\infty}{\alpha_g}\approx-\midfrac{1}{6}$

\end{block}

\end{frame}

\begin{wordonframe}{Espressioni asintotiche per le frequenze}

\end{wordonframe}

\end{comment}



\part{Tecniche e risultati di inversione}\label{part:inverseproblem}

\frame{\partpage}


\begin{frame}{Inversione assoluta}


\begin{equation*}
c_s^2=\Gamma_1\frac{P}{\rho}
\end{equation*}

%\begin{columns}
%\begin{column}{0.6\textwidth}
\begin{figure}[!ht]
\includegraphics[height=0.45\textheight,keepaspectratio]{dsoundspeedduvall} 

%\end{column}
%\begin{column}{0.4\textwidth}
\captionof{figure}{%Differenza relativa del profilo di $c_s$ (determinata invertendo \eqref{eq:analinversionc}) per frequenze dei modi calcolate con MSS e osservate. La differenza relativa \'e minore del $5\%$.
Da \cite{christensen1985speed}.}
\label{dsoundduvall}
\end{figure}
%\end{column}
%\end{columns}

\end{frame}

\begin{wordonframe}{Inversione legge di DUvall}

Il profilo radiale della velocit\'a del suono risultante  mostra la validit\'a del modello solare (detto standard per contrasto con i modelli pensati per risolvere il problema dei neutrini solari.)

(Errore sistematico dovuto alle approssimazione di Cowling intorno al $2.5\%$)

Il modello MSS si basa su leggi di conservazione, valide per qualsiasi stella: le prime 3 equazioni usate per determinare la struttura stellare ...

\end{wordonframe}

%\section{Principio variazionale}

\begin{frame}{Effetto di perturbazioni nell'equazione del moto sulle frequenze}

\begin{block}{Problema agli autovalori hermitiano}

\begin{align*}%\label{eq:EOMrhoc}%Unno89
&L(\vec{\xi})=\frac{1}{\rho_0}\nabla p'-\vec{g}'-\frac{\rho'}{\rho_0}\vec{g}_0\\
&\rho_0L(\xi)=\nabla(c_s^2\rho\scap{\nabla}{\xi}+\nabla P\cdot\vec{\xi})-\vec{g}_0\nabla\cdot(\rho_0\vec{\xi})-G\rho_0\nabla(\int_V\frac{\nabla\cdot(\rho_0\vec{\xi})\,d^3r'}{|\vec{r}-\vec{r}'|})\\
&L(\xi)=-\omega^2\xi
\end{align*}

\end{block}

\begin{block}{Variazioni nell'equazione del moto perturbato}

\begin{equation*}
(L+\Lvar{L})(\xi+\Lvar{\xi})=-(\omega+\Lvar{\omega})^2(\xi+\Lvar{\xi})%&\label{eq:EMvar}
\end{equation*}

\end{block}

\begin{block}{Variazioni nelle frequenze}

\[\frac{\Lvar{\omega}}{\omega}=-\frac{\int_V\rho\vec{\xi}\Lvar{L}\vec{\xi}\,d^3x}{2\omega^2\int_V\rho\scap{\xi}{\xi}d^3x}\]

\end{block}

\end{frame}

\begin{wordonframe}{Principio variazionale}

Definisco l'operatore lineare L riscrivendo l'equazione del moto perturbato in termini di $\vec{\xi}$ e delle grandezze di equilibrio. 

Chandrasekhar ha dimostrato che L definisce un problema agli autovalori hermitiano per condizioni alla superfice stellare di pressione e densit\'a nulle.

Considero gli effetti di una variazione piccola di L, cio\'e una variazione dell'equazione del moto perturbato, $\delta L$ sulle frequenze dei modi. Tengo i termini lineari nelle perturbazioni e data la stazionariet\'a di $\omega$ rispetto a $\xi$ il termine $L\delta \xi$ \'e nullo. Moltiplicando per $\xi*$  e integrando sul volume stellare ottengo l'espressione per la variazione nelle frequenze dei modi in funzione della variazione dell'equazione del moto perturbato.

\end{wordonframe}

\begin{frame}{Variazione nei modi per variazione in struttura, equazione di stato e composizione}

\begin{block}{Inversione per $(c^2,\rho)$}

\begin{equation*}
\frac{\delta\omega_{nl}}{\omega_{nl}}=\int_0^R[K^{nl}_{c^2,\rho}(r)\frac{\delta_rc^2}{c^2}(r)+K^{nl}_{\rho,c^2}(r)\frac{\delta_r\rho}{\rho}(r)]\,dr+I_{nl}\expy{-1}F_{Surf}
\end{equation*}

\end{block}

\begin{block}{Inversione per $(u,Y)$}

\begin{equation*}
\frac{\delta\omega_{nl}}{\omega_{nl}}=\int_0^RK^{nl}_{u,Y}(r)\frac{\delta_ru}{u}(r)\,dr+\int K^{nl}_{Y,u}(r)\delta_rY\,dr+I_{nl}\expy{-1}F_{Surf}
\end{equation*}

\end{block}

\end{frame}

\begin{wordonframe}{Variazioni delle grandezze d'equilibrio $(c^2,\rho)$ e $(u,Y)$}

Considero variazioni nel profilo radiale di $(c^2,\rho)$: posso esprimere la variazione nelle frequenze dei modi usando il principio variazionale in termini delle variazioni in densit\'a e pressione tramite le funzioni kernel dipendenti dalle autofunzioni dei modi.

Analogamente  esprimo le differenze nelle frequenze dei modi in termini di variazioni al profilo radiale di $(u,Y)$ ma devo esplicitare la dipendenza di $\Gamma_1$ da Y usando l'equazione di stato.

\end{wordonframe}

\begin{frame}{Tecniche d'inversione}

\begin{block}{RLS}

I parametri sono determinati minimizzando:
\begin{equation*}
Y=\sum_{\alpha=1}^N(\frac{\delta\omega_{obs}-\delta\omega_{fit}}{\sigma})^2_{\alpha}+\alpha N\int_0^1(x\TDof{x}\frac{\Delta f}{f})^2\,dx
\end{equation*}

\end{block}

\begin{block}{Subtractive Optimally Localized Averaging.}

\begin{equation*}
\sum_ic_i(r_0)\frac{\delta\omega_i}{\omega_i}=\int_0^R\sum_ic_i(r_0)K^i(r)\frac{\delta f(r)}{f(r)}\,dr=\exv{\frac{\delta f(r_0)}{f(r_0)}}
\end{equation*}

I coefficienti $c_i(r_0)$ sono determina minimizzando la funzione
\begin{equation*}
\int_0^R[\mathcal{K}(r_0,r)-\mathcal{T}(r_0,r)]^2\,dr+\mu\sum_{ij}E_{ij}c_i(r_0)c_j(r_0)
\end{equation*}

\end{block}

\end{frame}

\begin{wordonframe}{Tecniche d'inversione}

Usando la tecnica del minimo $\chi^2$ regolarizzato si parametrizza la funzione incognita $\frac{\delta f}{f}$, tramite funzioni di base opportune, determinata minimizzando:

N indica il numero totale di modi $\alpha$, $N_p$ il numero di parametri da determinare, $\Delta\omega_{fit}$, ricavato tramite eq:variational, contiene la funzione incognita; il secondo addendo del lato destro di eq:minimizerls \'e introdotto per ridurre oscillazioni indesiderate nel risultato dell'inverisone con $\alpha$ parametro di regolarizzazione.

Determino dei coefficienti $c_i(r_0)$ tali che \[\sum c_i(r_0)\frac{\delta\omega_i}{\omega_i}\] fornisca una media del valore di $\frac{\delta f(r)}{f(r)}$ in $r=r_0$:

La larghezza finita di $\mathcal{K}(r_0,r)$ determina il valor medio di $\frac{\delta f}{f}$ in intorno di $r_0$ e ci\'o causa una differenza sistematica  dal valore effettivo in $r_0$: per $c_s$ l'errore \'e minore del $0.03\%$ e le regioni pi\'u afflitte sono la base della zona convettiva, per la rapida variazione di $\frac{\delta c_s}{c_s}$ e la regione centrale, per il ridotto numero di modi che penetrano in questar zona.

Illustro la tecnica SOLA per determinare $\frac{\delta_rc^2}{c^2}$: si formano delle combinazioni lineari di $\frac{\Lvar{\omega_i}}{\omega_i}$ pesate da coefficienti $c_i(r_0)$ tali che $\frac{\Lvar{c^2}}{c^2}$ sia centrato attorno $r_0$ e che gli altri termini in \eqref{eq:invstructure} siano soppressi, queste compongo un averaging kernel $\mathcal{K}_{c^2,\rho}(r_0,r)=\sum_ic_i(r_0)K_{c^2,\rho}^i(r)$, con $\int_0^R\mathcal{K}(r_0,r)\,dr=1$.


Determino i coefficienti minimizzando l'espressione
\begin{equation}
\int_0^R[\mathcal{K}_{c^2,\rho}(r_0,r)-\mathcal{T}(r_0,r)]^2\,dr+\beta\int_0^R\mathcal{L}_{\rho,c^2}(r_0,r)\,dr+\mu\sum_i\sigma_ic_i(r_0)c_j(r_0)
\end{equation}
dove il kernel cross-term
\begin{equation}
\mathcal{L}_{\rho,c^2}(r_0,r)=\sum_ic_i(r_0)K_{\rho,c^2}^i(r)
\end{equation}
controlla i contributi indesiderati di $\frac{\delta_r\rho}{\rho}$.

\end{wordonframe}

\begin{frame}{Accuratezza posizione base zona convettiva e abbondanza elio}

\begin{table}[!ht]%{r}{0.7\textwidth}

\pgfplotstabletypeset[
math/.style={%
        preproc cell content/.append style={/pgfplots/table/@cell content/.add={$}{$}},
    },
every head row/.style={
 before row={\toprule
 %&\multicolumn{4}{c|}{Primordiale}
 },
 every last row/.style={after row=\bottomrule},
 after row={\midrule}
},
every last row/.style={after row=\bottomrule},
every first column/.style={column type/.add={|}{}},
every last column/.style={column type/.add={}{|}},
%columns/0/.style = {column type/.add={|}{}},
display columns/0/.style={column name={Composizione}},
display columns/1/.style={column name={$Z/X$}},
display columns/2/.style={column name={$R_{CZ}$}},
display columns/3/.style={column name={$Y_{CZ}$}},
display columns/4/.style={column name={$Y_0$}},
create on use/comp/.style={create col/set list={
inversione,GS98,AGS05,AGSS09,C+11}},
columns/comp/.style = {column type/.add={|}{}},
columns/comp/.style={string type},
columns/ZX/.style={string type},
columns/ZX/.append style={math},
columns/RCZ/.style={string type},
columns/RCZ/.append style={math},
columns/YCZ/.style={string type},
columns/YCZ/.append style={math},
columns/Y0/.style={string type},
columns/Y0/.append style={math},
columns={comp,ZX,RCZ,YCZ,Y0},
%/pgf/number format/precision=4
     ]{CZvsZ.txt} %%%
     \caption{Da \cite{basu2016global}.}
\label{tab:CZZvar}
\end{table}

\end{frame}

\begin{wordonframe}{Accuratezza posizione base zona convettiva e abbondanza elio}

Confronto le caratteristiche della zona convettiva ricavati tramite inversione e  ricavate tramite modelli con diversa composizione.

I modelli con una minore metallicit\'a hanno una opacit\'a minore nella zona radiativa e quindi un minore gradiente termico a parit\'a di flusso (temperatura centrale minore).

Le differenze quadratiche medie nel profilo della velocit\'a del suono sono  correlate con corretta predizione della posizione della zona convettiva: nel caso non si verifiche il modello non riproduce correttamente il gradiente termico ...

\end{wordonframe}

\begin{frame}{Accuratezza profilo radiale della velocit\'a del suono}

\begin{figure}[!ht]%{r}{0.5\textwidth}
        \includegraphics[width=0.8\textwidth,keepaspectratio]{deltacwu}
        \caption{Da \cite{villante2014chemical}.}\label{fig:deltacwu}
\end{figure}

\end{frame}

\begin{wordonframe}{Accuratezza profilo radiale della velocit\'a del suono}

Le differenze quadratiche medie nel profilo della velocit\'a del suono sono  correlate con corretta predizione della posizione della zona convettiva: nel caso non si verifiche il modello non riproduce correttamente il gradiente termico ...

Questa figura rappresenta le differenze nel profilo radiale della velocit\'a del suono risultanti dall'inversione delle frequenze osservate per un modello con composizione $\midfrac{Z}{X}=0.023$ (GS98), linea rossa, e uno con $\midfrac{Z}{X}=0.018$ (AGSS09), linea nera.

La banda rossa intorno allo zero che rappresenta il profilo effettivo della velocit\'a del suono nel Sole indica l'errore dovuto alle tecniche di inversione, alle incertezze sulle frequenze e alla residua dipendenza dell'inversione dal modello solare.

\end{wordonframe}

\section{Appendix}


\begin{frame}{Modello plasma solare}

\begin{block}{Processi che contribuiscono all'opacit\'a}

Transizioni tra stati elettronici (BB).

Ionizzazione (BF).

Brehmstrahlung inverso (FF).

Scattering di elettroni (S).

\end{block}

\vfill

\begin{block}{Opacit\'a media di Rosseland}

\vfill

\begin{equation*}
\frac{1}{\kappa}=\invers{(4aT^3)}\int_0^{+\infty}\PDy{T}{B_{\nu}(T)}\frac{1}{\kappa_{a}(\nu)(1-\exp{-\midfrac{h\nu}{KT}})+\kappa_s(\nu)}\,d\nu
\end{equation*}

\end{block}

\vfill

\end{frame}

\begin{wordonframe}{Interazioni radiazione-materia}

L'interazione radiazione-materia determina il gradiente termico della regione radiativa e nel caso solare la posizione della base della zona convettiva. L'opacit\'a \'e dovuta ai processi di assorbimento e scattering dei fotoni e dipende dal numero dalla densit\'a di particelle che danno luogo a un determinato fenomeno di assorbimento o scattering e dalla sezione d'urto per singola particella.

L'opacit\'a media per grammo () \'e determinata mediando l'opacit\'a a data lunghezza d'onda rispetto a $\PDy{T}{B_{\nu}(T)}$. L'opacit\'a a data frequenza \'e somma dei contributi da parte dei processi di assorbimento (bb, bf, ff) e scattering.

C'\'e un ulteriore correzione che tiene conto della frazione di emissione spontanea su totale.

\end{wordonframe}

\begin{frame}{Catena PP}

\begin{block}{Contributo della fusione di nuclei $(i,j)$ alla produzione di energia per grammo}

\begin{align*}
&\epsilon_{ij}=\frac{1}{1+\delta_{ij}}Q_{ij}\frac{\rho N_A^2X_iX_j}{{A_iA_j}}\exv{\sigma v}_{ij}%\label{eq:energyrate}
\end{align*}
\end{block}


\begin{columns}

\begin{column}{0.65\textwidth}

\begin{block}{Fusione idrogeno in elio: catena PP}

\setmuskip{\thinmuskip}{0mu}\setmuskip{\medmuskip}{0mu}
\tikzset{->-/.style={decoration={
  markings,
  mark=at position .5 with {\arrow{>}}},postaction={decorate}},
-->/.style={decoration={
  markings,
  mark=at position .8 with {\arrow{>}}},postaction={decorate}},
box/.style={%
%draw,
minimum width=25mm,%
    minimum height=6mm,%
    align=center}
}

\centering
\begin{tikzpicture}

\begin{scope}[scale=0.6,transform shape]
\node[box] (pp) at (0,0) {$\Pproton{+}\Pproton{\to}\cel{H}{2}{}{}{+}\Pnue{+}\APelectron$};%%pp
\node[box,right=2cm of pp]  (pep) {$\Pproton{+}\Pproton{+}\Pelectron{\to}\cel{H}{2}{}{}+\Pnue$};%%pep
\coordinate[below=0.3cm of pp] (bpp);
\node[left] at (bpp) {$99.76\%$};
\coordinate[below=0.3cm of pep] (bpep);
\node[right] at (bpep) {$0.24\%$};

\coordinate[] (ttriton) at ($(bpp)!0.5!(bpep)$);
\draw[->-] (pp)--(bpp)--(ttriton);
\draw[->-] (pep)--(bpep)--(ttriton);
\node[box,below=0.3cm of ttriton] (triton) {$\Pproton+\cel{H}{2}{}{}\to\cel{He}{3}{}{}+\Pphoton$};%%triton
\coordinate[below=0.3cm of triton] (btriton);
\draw[-->] (ttriton)--(triton.north);
\draw[->-] (triton.south)--(btriton.north);
\coordinate[left=2.5cm of btriton] (tpp1);
\node[left] at (tpp1) {$83.3\%$};
\coordinate[right=2.0cm of btriton] (tberillium7);
\node[above] at (tberillium7) {$16.7\%$};
\coordinate[right=6.5cm of btriton] (thep);
\node[right] at (thep) {$\num{2e-5}\%$};

\draw[] (btriton)--(tpp1);
\draw[] (btriton)--(tberillium7);
\draw[] (tberillium7)--(thep);
\node[box,below=0.5cm of tpp1,label={[xshift=0.1cm, yshift=-1.5cm]PPI}]  (pp1) {$\cel{He}{3}{}{}+\cel{He}{3}{}{}\to\cel{He}{4}{}{}+2\Pproton$};%%pp1
\node[box,below=0.5cm of tberillium7]  (berillium7) {$\cel{He}{3}{}{}+\cel{He}{4}{}{}\to\cel{Be}{7}{}{}+\Pphoton$};%%berillium7
\node[box,below=0.5cm of thep,label={[xshift=-0.1cm, yshift=-1.5cm]HEP}]  (hep) {$\cel{He}{3}{}{}+\Pproton\to\cel{He}{4}{}{}+\APelectron+\Pnue$};%%hep

\draw[->-] (tpp1)--(pp1.north);
\draw[->-] (tberillium7)--(berillium7.north);
\draw[-->] (thep)--(hep.north);

\coordinate[below=0.3cm of berillium7] (bberillium7);
\coordinate[left=1.5cm of bberillium7] (tlithium7);
\node[left] at (tlithium7) {$99.88\%$};
\coordinate[right=2.0cm of bberillium7] (tboron8);
\node[right] at (tboron8) {$0.12\%$};

\node[box,below=0.5cm of tlithium7]  (li7) {$\cel{Be}{7}{}{}+\Pelectron\to\cel{Li}{7}{}{}+\Pnue$};%%Li7
\node[box,below=0.5cm of li7,label={[xshift=0.1cm, yshift=-1.5cm]PPII}] (pp2) {$\cel{Li}{7}{}{}+\Pproton\to2\cel{He}{4}{}{}$};%% PP2

\node[box,below=0.5cm of tboron8]  (b8) {$\cel{Be}{7}{}{}+\Pproton\to\cel{B}{8}{}{}+\Pphoton$};%%B8
\node[box,below=0.25cm of b8]  (be7) {$\cel{B}{8}{}{}\to\cel{Be}{8}{}{}^*+\APelectron+\Pnue$};%%Be8*
\node[box,below=0.25cm of be7,label={[xshift=0.1cm, yshift=-1.5cm]PPIII}]  (pp3) {$\cel{Be}{8}{}{}^*\to2\cel{He}{4}{}{}$};%%pp3

\draw[->-] (berillium7.south)--(bberillium7);
\draw[] (bberillium7)--(tlithium7);
\draw[] (bberillium7)--(tboron8);

\draw[->-] (tlithium7)--(li7.north);
\draw[->-] (li7.south)--(pp2.north);

\draw[->-] (tboron8.south)--(b8.north);
\draw[->-] (b8.south)--(be7.north);
\draw[->-] (be7.south)--(pp3.north);
\end{scope}

\node[] at (2,-4.5) {\parbox{0.8\textwidth}{\captionof{figure}{Da \cite{adelberger2011solar}.}}};

\end{tikzpicture}

\end{block}

\end{column}

\begin{column}{0.35\textwidth}

\begin{block}{Perdite in neutrini (Sole)}
\begin{tabular}{cc}
Ramo&$Q_{eff} (MeV)$\\
PP1&$26.2 (2\%)$\\
PP2&$25.66  (4\%)$\\
PP3 &$19.17 (28\%)$\\
\end{tabular}

\end{block}

\begin{block}{Tempo di vita}

{\small\begin{tabular}{cc}
&$\tau(\si{\year})$\\
$\cel{H}{1}{}{}(p,e^+\Pnue)\cel{D}{2}{}{}$&\num{8e9}\\
$\cel{D}{2}{}{}(p,\gamma)\cel{He}{3}{}{}$&\num{4.4e-8}\\
$\cel{He}{3}{}{}(\cel{He}{3}{}{},2p)\cel{He}{4}{}{}$&\num{2.4e5}\\
%$\cel{B}{8}{}{}\cel{Be}{8}{}{}^*\cel{He}{4}{}{}$&\num{3e-8}\\

\end{tabular}
}

\end{block}

\end{column}

\end{columns}

%\begin{block}{Tempi reazione e energia efficace prodotta}
%\begin{tabular}{c|c|}
%\hline
%Reazione & $t_r$ \\
%\hline
%$^1H+^1H\to^2H+\APelectron+\Pnue$ & $7*10^9\,yr$\\
%$^2H+^1H\to ^3He+\gamma$ & $4 s$\\
%$^3He+^3He\to^4He+2^1H$ & $4*10^5\,yr$\\
%\hline
%\end{tabular}
%\end{block}

\end{frame}

\begin{wordonframe}{Catena PP}

nella fase di sequenza principale la reazione efficace \'e la conversione di 4 nuclei di idrogeno in elio + 2 positroni e 2 neutrini.

Nel centro solare circa il $99\%$ della di $\epsilon$ \'e prodotto dalla catena PP.
Questa figura descrive le reazioni nucleari per condizioni del centro solare. 

\end{wordonframe}


% SOS
%\part{SSM (sos)}
%\begin{comment}
\section{Osservabili stellari/demo beamer}
\begin{frame}<1>[label=noinside]{Modello stellare}{Come indagare la fisica interna a una stella?}
\onslide<1->\begin{block}{Osservabili stellari:}
$L$, $M$, $R$, $T_e$, $(\frac{Z}{X})_{ph}$, $g_{ph}$.
\end{block}
\onslide<1->\begin{block}{Informazioni sulla struttura interna?} Condizione di equilibrio idrostatico
\end{block}
%Teorema Vogt-Russel: $X_i(r)$, $M$ \pause equilibrio (idrostatico/termico) determinano struttura stellare .
%\pause
\onslide<1->\begin{block}{Modello stellare: diagramma di \hr{}.}
\end{block}
\onslide<2->\begin{block}{Descrizione fisica interno stellare: parametri aggiuntivi}
Convezione, diffusione e sedimentazione elementi pesanti, equazione di stato, opacit\'a
\end{block}
\onslide<2->\begin{block}{Astrosismologia}
Restringo spazio parametri sistemi stellari lontani
\end{block}
\end{frame}
{ % all template changes are local to this group.
    \setbeamertemplate{navigation symbols}{}
    \begin{frame}[plain]{Diagramma di \hr{}}
        \begin{tikzpicture}[remember picture,overlay]
            \node[at=(current page.center)] {
                %\includegraphics[width=\paperwidth]{yourimage}
            };
        \end{tikzpicture}
     \end{frame}
}
\againframe<2>{noinside}
\begin{frame}{Pulsazioni stellari}{Modi Normali}
\begin{columns}
\begin{column}{0.5\textwidth}  %%<--- here
    \begin{center}
     %\includegraphics[width=0.5\textwidth]{image1}
     \end{center}
\end{column}
\begin{column}{0.5\textwidth}
\onslide<1-> \begin{block}{Stelle pulsanti}
Onde stazionarie: Pulsazione radiale/non radiale: .
\onslide<2-> meccanismo di eccitazione: solar-like pulsator, Cefeidi.
\onslide<3-> Modo fondamentale $\Pi\approx\tau_{dyn}=\sqrt{\frac{R^3}{GM}}\propto\overline{\rho}\expy{-\frac{1}{2}}$.
\onslide<4-> Modi di oscillazione\onslide<5-> - informazioni sull'interno stellare
\onslide<5-> Elio-sismologia: Modi $\Leftrightarrow$ Modelli solari
\onslide<5-> Astero-sismologia: Modi $\Leftrightarrow$ Spazio parametri modello stellare
\end{block}
\end{column}
\end{columns}
\end{frame}

\end{comment}

\section{Osservabili solari}

\begin{frame}{Dati osservativi}

\begin{block}{Et\'a, luminosit\'a, raggio solari}
\begin{tabular}{l|c}
\hline
$\agesun{}$&\SI[separate-uncertainty=true]{4.57\pm0.02e9}{\year}\\
\hline
$\rsun{}$&\SI{695658+-140}{\kilo\meter}\\
\hline
$G\msun$&\num{132712440018+-8}\SI{e9}{\cubic\meter\per\square\second}\\
\hline
$\lsun{}$&\SI{3.8275+-0.0014e33}{\erg\per\second}\\
\hline
\end{tabular}
%\caption[Osservabili solari principali.]{Osservabili solari principali. \cite{haberreiter2008solving}.}
\label{tab:sunO}
\end{block}

\begin{block}{Simmetria sferica}
Deviazioni da forma sferica trascurabili (campi magnetici, rotazione)
\end{block}

\end{frame}

\begin{frame}{Dati osservativi}

\begin{block}{Composizione chimica}
\begin{itemize}
\item Righe di assorbimento: attuale (non $Y_{ph}$)
\item Meteoriti CI: primordiale (refrattari)
\end{itemize}

\begin{table}[]

\pgfplotstabletypeset[
every head row/.style={
 before row={\toprule &\multicolumn{4}{c|}{Attuale}
 %&\multicolumn{4}{c|}{Primordiale}
 \\\midrule},
 every last row/.style={after row=\bottomrule},
 after row={\midrule}
},
every nth row={2}{before row=\midrule},every last row/.style={after row=\bottomrule},
every first column/.style={column type/.add={|}{}},
every last column/.style={column type/.add={}{|}},
columns/x/.style = {column type/.add={|}{}},
columns/xi/.style = {column type/.add={|}{}},
display columns/0/.style={column name={}},
display columns/1/.style={column name={$X$}},
display columns/2/.style={column name={$Y$}},
display columns/3/.style={column name={$Z$}},
display columns/4/.style={column name={$\frac{Z}{X}$}},
%display columns/5/.style={column name={$X$}},
%display columns/6/.style={column name={$Y$}},
%display columns/7/.style={column name={$Z$}},
%display columns/8/.style={column name={$\frac{Z}{X}$}},
create on use/authors/.style={create col/set list={
%Anders \& Grevesse (1989),Grevesse \& Noels (1993),
Grevesse et al. (1998),Lodders (2003),Asplund et al. (2005),Lodders et al. (2009),\cite{asplund2009chemical},\cite{caffau2011solar}}},
columns/authors/.style={string type},
columns={authors,x, y, z, zx
%,xi,yi,zi, zxi
},
/pgf/number format/precision=4
     ]{asplund.txt} %%%
\captionof{table}{Metallicit\'a attuale determinata da varii autori.}\label{tab:Zhistory}
\end{table}

\end{block}

\end{frame}


\section{Strutture autogravitanti in equilibrio}

\begin{frame}{Distribuzione di massa - Conservazione di massa e momento - tempo scala dinamico}

\begin{block}{Massa}

%\begin{align}
%&dm=4\pi r^2\rho \,dr-4\pi r^2\rho v\,dt\label{eq:massvar}\\
%\end{align}

\begin{equation}
\PDy{t}{\rho}+\nabla\cdot(\rho\vec{v})=0\label{eq:continuityeq}
\end{equation}

\begin{equation}
dm=4\pi r^2\rho \,dr\label{eq:massaguscio}
\end{equation}

\end{block}

\begin{block}{Momento}
\begin{align}
&\rho\TDy{t}{\vec{v}}=-\nabla P+\rho\vec{f}\label{eq:motion}\\
&\vec{g}=-\PDy{r}{\Phi}=-\frac{Gm(r)}{r^2}\hat{r}
\end{align}
\end{block}

\end{frame}

\begin{frame}{Equilibrio idrostatico: $\ddvec{r}=0$.}


\begin{align}
\nabla P=\rho \vec{f}\Label{eq:idrosta} \TDy{r}{P}=-\frac{Gm(r)\rho(r)}{r^2}\Label{eq:fidroequilibrio}
\end{align}


Per giustificare l'ipotesi di equilibrio idrostatico stimo i tempi caratteristici di evoluzione della struttura solare nel caso la forza dovuta alla pressione o la forza di gravit\'a non fossero bilanciate, approssimando il valore caratteristico della derivata di due variabili con il rapporto del loro valore caratteristico:
\begin{align}
&\tau_{ff}\approx\sqrt{\frac{\rsun{}}{g}}\\
&\tau_{esp}\approx \rsun{}\sqrt{\frac{\rho}{P}}
\end{align}

Per i valori solari \ref{wrap-tab:sunO} $\tau_{ff}\approx\tau_{esp}\approx\SI{27}{\minute}$.

\begin{equation}
\tau_{idro}^{\odot}= \sqrt{\frac{R^3}{GM}}\approx\frac{1}{2}(G\overline{\rho})\expy{-\frac{1}{2}}
\end{equation}

\end{frame}


\subsection{Equazione di stato $P(\rho,T)$}

deviazioni dalla legge dei gas perfetti per tenere conto dei fenomeni di ionizzazione parziale e stati atomici eccitati, della radiazione, della statistica di Fermi-Dirac per gli elettroni, \'e necessario considerare l'interazione Coulombiana.

\begin{frame}{Gas perfetto ioni-elettroni}


\begin{equation}
P_G=P_I+P_e=\frac{\rho}{\mu}\gasconstant{}T
\end{equation}

\begin{block}{Peso molecolare medio}
massa media in amu per particella libera
\begin{align}
&\mu=\frac{1}{\bar{n}_HX+\bar{n}_{He}Y+\bar{n}_{Z}Z}\label{eq:meanmw}\\
&\bar{n}_i=\frac{1+f_i}{A_i}
\end{align}

\end{block}


\end{frame}

\subsection{Energia interna per unit\'a di massa}

\begin{frame}{Energia interna: traslazioni}

\begin{align}
&u=\frac{1}{\rho}\sum_i\int f^{(0)}(\vec{p}_i)\frac{p^2_i}{2m_i}=\frac{3}{2}\frac{P}{\rho}=\frac{3}{2}\frac{\gasconstant T}{\mu}\\
&E_i=\int_0^Mu\,dm=\frac{3}{2}\int_M\frac{P}{\rho}\,dm\label{eq:traslintenergy}
\end{align}

 $f^{(0)}(\vec{p}_i)$ \'e il numero di particelle della specie i per unit\'a di volume con impulso in $[\vec{p},\vec{p}+d\vec{p}]$

\end{frame}


\subsection{Correzioni alla legge dei gas perfetti}

\begin{frame}{Correzioni alla legge dei gas perfetti}

\begin{itemize}
\item Degenerazione elettronica: $\Delta P\leq2\%$.

\item Pressione di radiazione: $P_r=\frac{1}{3}aT^4$.

\item Ionizzazione.

\item Interazioni coulombiane.

\begin{align}
&\frac{1}{r_D^2}=\frac{4\pi e^2}{kT}\sum Z^2\overline{n}_Z=\frac{4\pi e^2}{kT}N_A\zeta\label{eq:debyeradius}\\
&\zeta=\sum_{i}(Z_i^2+Z_i)\frac{\rho X_i}{A_i}
\end{align}

\begin{equation}
u_c=\frac{1}{2}\int\phi(\vec{r})\rho(\vec{r})\,d^3r,\ P_c=\frac{1}{3}u_c
\end{equation}

Regioni di ionizzazione parziale di idrogeno ed elio

\end{itemize}

\end{frame}

\begin{frame}{EOS}


Due approcci usati per determinare l'equazione di stato e quindi le grandezze termodinamiche del plasma solare sono lo schema chimico e lo schema fisico: il primo considera atomi e molecole, la cui popolazione per stati eccitati e diversi gradi di ionizzazione \'e ottenuto minimizzando l'energia libera da cui sono ricavate le altre grandezze termodinamiche; utilizzando questo approccio \'e stata ricavata l'equazione di stato MHD. Il secondo considera nuclei ed elettroni come costituenti fondamentali interagenti tramite potenziale Coulombiano e trova le soluzione dell'equazione di Schr\"oedinger per un problema a molti corpi, questo approccio, usato per ricavare l'equazione di stato OPAL, \'e pi\'u adatto per trattare le regioni interne del Sole.


\begin{figure}[!ht]
        \includegraphics[height=0.4\textwidth,keepaspectratio]{ionfraction}\label{fig:ionfraction}
        \caption{Profilo radiale della popolazione dei diversi gradi di ionizzazione per $\cel{He}{4}{}{}$, CNO, $\cel{Ne}{20}{}{}$, $\cel{Fe}{56}{}{}$. Stati di ionizzazione maggiore sono pi\'u interni. Da \cite{basu2008helioseismology}.}
\end{figure}

\begin{figure}[!ht]
        \includegraphics[height=0.4\textwidth,keepaspectratio]{gamma1eos}\label{fig:gamma1eos}
        \subcaption{Andamento di $\Gamma_1$ calcolato tramite equazione di stato MHD/OPAL. Da \cite{trampedach2006synoptic}.}
\end{figure}


\end{frame}


\section{Trasporto dell'energia}

\section{Produzione di energia - reazioni di fusione}

\begin{block}{Schermaggio debole: $e\phi\ll KT$.}




Formula di Boltzmann per la densit\'a delle particelle con carica Z:
\begin{align}
&n_Z=\overline{n}_Z\exp{-\frac{Ze\phi_i}{kT}}\\
&\nabla^2\phi=-4\pi e\sum Zn_Z-4\pi\sum Z_i\delta(\vec{r}-\vec{r}_i)\label{eq:poissonscreened}
\end{align}
In \eqref{eq:poissonscreened} rimane il termine lineare in $\phi$.





\begin{align}
&\frac{r_D^2}{r_i}\TtwoDy{r}{(r_i\phi_i)}\tag{\ref{eq:poissonscreened} $Z_i$}\\
&\phi=\sum_i\phi_i
\end{align}

La soluzione di \eqref{eq:poissonscreened} \'e
\begin{equation}\label{eq:screenedpotential}
\phi_i=\frac{Z_ie}{r_i}\exp{-\midfrac{r}{r_D}}
\end{equation}


\end{block}

\section{Modello solare standard e osservabili sismologiche}




%%% SOS
%\part{OSCILLAZIONI SOS}
%\begin{comment}
\section{Osservabili stellari/demo beamer}
\begin{frame}<1>[label=noinside]{Modello stellare}{Come indagare la fisica interna a una stella?}
\onslide<1->\begin{block}{Osservabili stellari:}
$L$, $M$, $R$, $T_e$, $(\frac{Z}{X})_{ph}$, $g_{ph}$.
\end{block}
\onslide<1->\begin{block}{Informazioni sulla struttura interna?} Condizione di equilibrio idrostatico
\end{block}
%Teorema Vogt-Russel: $X_i(r)$, $M$ \pause equilibrio (idrostatico/termico) determinano struttura stellare .
%\pause
\onslide<1->\begin{block}{Modello stellare: diagramma di \hr{}.}
\end{block}
\onslide<2->\begin{block}{Descrizione fisica interno stellare: parametri aggiuntivi}
Convezione, diffusione e sedimentazione elementi pesanti, equazione di stato, opacit\'a
\end{block}
\onslide<2->\begin{block}{Astrosismologia}
Restringo spazio parametri sistemi stellari lontani
\end{block}
\end{frame}
{ % all template changes are local to this group.
\setbeamertemplate{navigation symbols}{}
    \begin{frame}[plain]{Diagramma di \hr{}}
        \begin{tikzpicture}[remember picture,overlay]
            \node[at=(current page.center)] {
                %\includegraphics[width=\paperwidth]{yourimage}
            };
        \end{tikzpicture}
     \end{frame}
}
\againframe<2>{noinside}
\section{Osservazioni}
\subsection{Fitting polinomiale: inversione ''1.5-D''.}
\begin{frame}<1>[label=noinside]{Modello stellare}{Come indagare la fisica interna a una stella?}
\onslide<1->\begin{block}{Rotazione superficiale}
\begin{equation*}
\frac{\Omega(\theta)}{2\pi}=\SI{451.5}{\nano\hertz}-\SI{65.3}{\nano\hertz}\cos^2{\theta}-\SI{66.7}{\nano\hertz}\cos^4{\theta}
\end{equation*}
\end{block}
\onslide<1->\begin{block}{Informazioni sulla struttura interna?} Condizione di equilibrio idrostatico
\end{block}
%Teorema Vogt-Russel: $X_i(r)$, $M$ \pause equilibrio (idrostatico/termico) determinano struttura stellare .
%\pause
\onslide<1->\begin{block}{Modello stellare: diagramma di \hr{}.}
\end{block}
\onslide<2->\begin{block}{Descrizione fisica interno stellare: parametri aggiuntivi}
Convezione, diffusione e sedimentazione elementi pesanti, equazione di stato, opacit\'a
\end{block}
\onslide<2->\begin{block}{Astrosismologia}
Restringo spazio parametri sistemi stellari lontani
\end{block}
\end{frame}
\subsection{Osservazione dello splitting in m: inversione ''2D''.}
\begin{figure}[!ht]
\centering
\includegraphics[keepaspectratio,width=0.8\textwidth]{invertedrotation}
\caption{Inversione della velocit\'a di rotazione a diverse latitudini. La linea verticale tratteggiata indica la base della zona convettiva. Da \cite{chr02helioseismology}.}
\end{figure}
Considero la correzione al primo ordine in $\Omega$. Il campo di velocit\'a rotazionale in coordinate sferiche \'e 
\begin{align}
&\vec{v_0}=(0,0,r\Omega\sin{\theta})=\vecp{\Omega}{r}\\
&\vec{\Omega(r,\theta)}=(\Omega(r,\theta)\cos{\theta},-\Omega(r,\theta)\sin{\theta},0)
\end{align}
In assenza di moti macroscopici il termine d'inerzia \'e $\rho_0\TDy{t}{\vec{v}}=\rho_0\PtwoDy{t}{\vec{\xi}}$, mentre in caso di rotazione si ha
\begin{equation}
\rho_0(\PDof{t}+\scap{v_0}{\nabla})^2\vec{\xi}
\end{equation}
Considero il termine dovuto alla rotazione come una piccola correzione alle frequenze dei modi
\begin{align}
&\omega_{(l,m)}+\Delta\omega_{(l,m)}&\intertext{quindi l'equazione del moto al primo ordine nella perturbazione, con $\alpha=(l,m)$, \'e}\nonumber\\
&\rho_0(\omega_{\alpha}^2+2\omega_{\alpha}\Delta\omega_{\alpha})\vec{\xi}=\nabla P_1-\frac{\rho_1}{\rho_0}\nabla P_0+\rho_0\nabla\Phi_1+2i\omega_{\alpha}\rho_0(\scap{v_0}{\nabla})\vec{\xi}\\
&\intertext{da cui si deduce}\nonumber\\
&\Delta\omega_{\alpha}=\frac{i\int\rho_0\xi_{\alpha}^*(\scap{v_0}{\nabla})\xi_{\alpha}}{\int\rho_0\xi_{\alpha}^*\xi_{\alpha}}=\frac{-m\int\rho_0\Omega\xi_{\alpha}^*\xi_{\alpha}\,dV+i\int\rho_0\xi_{\alpha}^*(\vecp{\Omega}{\xi_{\alpha}})\,dV}{\int\rho_0\xi_{\alpha}^*\xi_{\alpha}}
\end{align}
Il problema di trovare $\Omega(r,\theta)$ dalla differenza $\Delta\omega_{\alpha}$ \'e lineare in $\Omega$ quindi $\Delta\omega_{\alpha}\propto\Omega$. Per determinare quindi la rotazione dobbiamo conoscere l'autovalore $\xi_{\alpha}$ dello stato imperturbato.
%Per rotazione puramente radiale $\Omega(r)$ la relazione tra lo splitting delle frequenze e la rotazione \'e
%\begin{equation}
%\Delta\omega_{\alpha}=-m\frac{\int_0^{\rsun{}}\rho_0\Omega\{|\xi_r-\xi_h|^2+[l(l+1)-2]|\xi_h|^2\}r^2\,dr}{\int_0^{\rsun{}}\rho_0\{|\xi_r|^2+l(l+1)|\xi_h|^2\}r^2\,dr}=\int_0^{\rsun{}}K_{\alpha}(r)\Omega(r)\,dr
%\end{equation}
%Any given $\Delta\omega_{\alpha}$ samples angular velocity in the depth range corresponding to $\xi_{\alpha}$.
La velocit\'a angolare contribuisce a $\Delta\omega_{\alpha}$ negli strati in cui $\xi_{\alpha}$ \'e apprezzabile. Nel caso di rotazione dipendente solo da r si ha che $\Delta\omega_{\alpha}$ \'e lineare in m: ho $2l+1$ frequenze equispaziate.

\end{comment}

\section{Modi normali della struttura solare}

\begin{frame}[label=noinside]{Modi di oscillazione adiabatici}{Perturbazione dello stato di equilibrio.}

\begin{block}{campi di velocit\'a/effetti non lineari}
Descrivo le oscillazioni come piccole perturbazioni attorno allo stato di equilibrio stazionario (gli effetti non lineari, fra cui lo scambio di energia tra i modi, sono dell'ordine di $\frac{v}{c_s}$ dove v \'e l'ampiezza della velocit\'a dell'oscillazione). 
In generale pu\'o essere presente un campo di velocit\'a $\vec{v}_0$:
\begin{align}
&\vec{v}=\vec{v}_0+\vec{v}'\\
&\TDof{t}=\PDof{t}+(\vec{v}_0\cdot\nabla)
\end{align}
in prima approssimazione prendo $\vec{v}_0=0$ per poi considerare come perturbazioni gli effetti dovuti a campi di velocit\'a in specie rotazione.

\end{block}

\begin{block}{Perturbazione pressione densit\'a}

Indico con $P'(\vec{r},t)$ e $\delta P$ la perturbazione euleriana e lagrangiana della pressione e con $\rho'$, $\Phi'$ e $\vec{g}'$ la perturbazione euleriana della densit\'a , e le perturbazioni euleriane del potenziale gravitazionale e dell'accelerazione di gravit\'a conseguenti  con $\delta\vec{r}=\vec{\xi}$ il vettore spostamento perturbato:
\begin{align}
&P(\vec{r},t)=P_0(\vec{r})+P'(\vec{r},t)\label{eq:pressureperturbation}\\
&\Lvar{P(\vec{r})}=P(\vec{r}+\Lvar{\vec{r}})-P_0(\vec{r})=P'(\vec{r})+\Lvar{\vec{r}}\cdot\nabla P_0\\
&\vec{g}'=-\nabla\Phi',\ \nabla^2\Phi'=4\pi G\rho'\label{eq:gapert}
\end{align}

\end{block}


\end{frame}

\begin{frame}[label=noinside]{Modi di oscillazione adiabatici}{Modi di oscillazione lineari adiabatici.}

\begin{block}{Equazione del moto perturbata}

l'equazione del moto perturbato sostituendo \eqref{eq:pressureperturbation} nell'equazione del moto \eqref{eq:motion} considerando solo i termini lineari nella perturbazione:
\begin{equation}
\rho_0\TDof{t}\vec{v}=\rho_0\PtwoDy{t}{\Lvar{\vec{r}}}=-\nabla P'+\rho_0\vec{g}'+\rho'\vec{g}_0\label{eq:emper}
\end{equation}

\end{block}

\end{frame}

\begin{frame}[label=noinside]{Modi di oscillazione adiabatici}{Equazione di continuit\'a perturbata}

\begin{block}{Equazione di continuit\'a perturbata}

Analogamente per l'equazione di continuit\'a ottengo
\begin{equation}
\rho'+\div{(\rho_0\Lvar{\vec{r}})}=0\label{eq:contper}
\end{equation}

\end{block}

\end{frame}

\begin{frame}[label=noinside]{Modi di oscillazione adiabatici}{Condizione di adiabaticit\'a}


  \begin{overlayarea}{\textwidth}{1cm}
   \only<1>{
   energia interna per unit\'a di massa
\begin{equation}
\TDy{t}{q}=\TDy{t}{u}+P\TDof{t}(\frac{1}{\rho})\label{eq:prima}
\end{equation}

\begin{equation}
\TDy{t}{T}-\frac{\Gamma_2-1}{\Gamma_2}\frac{T}{P}\TDy{t}{P}=\frac{1}{c_P}(\epsilon-\frac{1}{\rho}\scap{\nabla}{F})
\end{equation}
il termine a destra \'e trascurabile:
\begin{equation}
\TDy{t}{q}=0
\end{equation}
   }
   \only<2>{
   Il moto di una elemento di fluido \'e descritto dalla relazione adiabatica
\begin{equation}
\TDy{t}{P}=\frac{\Gamma_1P}{\rho}\TDy{t}{\rho}
\end{equation}
}
   \only<3>{
  
  La condizione di perturbazione adiabatica linearizzata \'e
\begin{align}
&\PDy{t}{\Lvar{P}}-\frac{\Gamma_{1,0}P_0}{\rho_0}\PDy{t}{\Lvar{\rho}}=0\\
&P'+\Lvar{\vec{\xi}}\cdot\nabla P_0=\frac{\Gamma_{1,0}P_0}{\rho}(\rho'+\Lvar{\vec{\xi}}\cdot\nabla\rho_0)\label{eq:adper}
\end{align}

   }
  \end{overlayarea}


\end{frame}


\begin{figure}[!ht]

\subfigure[Distribuzione dei modi con $l\leq300$ nel diagramma $\nu-l$ determinata usando i primi 144 giorni di osservazione di MDI. Da \cite{chr02helioseismology}.]{
\includegraphics[keepaspectratio,width=0.45\textwidth]{midlmodes}}
\label{fig:midlmodes}
~
\subfigure[Modi adiabatici calcolati sulla base di un modello solare. Da \cite{chr02helioseismology}.]{
\includegraphics[keepaspectratio,width=0.6\textwidth]{nrmodesLAWE}\label{fig:nrmodesLAWE}}

\end{figure}

\section{Campo di velocit\'a solare}

\section{Caratteristiche asintotiche delle oscillazioni adiabatiche}



%%% SOS
%\part{INVERSIONE SOS}
%\begin{comment}
\section{Osservabili stellari/demo beamer}

\begin{frame}<1>[label=noinside]{Modello stellare}{Come indagare la fisica interna a una stella?}

\onslide<1->\begin{block}{Osservabili stellari:}
$L$, $M$, $R$, $T_e$, $(\frac{Z}{X})_{ph}$, $g_{ph}$.
\end{block}

\onslide<1->\begin{block}{Informazioni sulla struttura interna?} Condizione di equilibrio idrostatico
\end{block}

%Teorema Vogt-Russel: $X_i(r)$, $M$ \pause equilibrio (idrostatico/termico) determinano struttura stellare .
%\pause

\onslide<1->\begin{block}{Modello stellare: diagramma di \hr{}.}
\end{block}

\onslide<2->\begin{block}{Descrizione fisica interno stellare: parametri aggiuntivi}
Convezione, diffusione e sedimentazione elementi pesanti, equazione di stato, opacit\'a
\end{block}

\onslide<2->\begin{block}{Astrosismologia}
Restringo spazio parametri sistemi stellari lontani
\end{block}

\end{frame}

{ % all template changes are local to this group.
    \setbeamertemplate{navigation symbols}{}
    \begin{frame}[plain]{Diagramma di \hr{}}
        \begin{tikzpicture}[remember picture,overlay]
            \node[at=(current page.center)] {
                %\includegraphics[width=\paperwidth]{yourimage}
            };
        \end{tikzpicture}
     \end{frame}
}
\againframe<2>{noinside}

\section{Inversione della legge di Duvall}

\section{Inversione non asintotica}

\section{Vincoli al modello solare dalle osservazioni sismologiche}

\end{comment}


In questa parte considero come estrarre informazioni sulla struttura di equilibrio dalle frequenze dei modi osservati: modi distinti sono confinati in cavit\'a di profondit\'a diversa e le ampiezze di oscillazione hanno differente comportamento spaziale, \'e quindi possibile invertire il problema date le frequenze osservate per ricavare il profilo radiale di $(P,\rho,\Gamma_1)$.

Un'inversione indipendente dal modello \'e possibile utilizzando l'approssimazione asintotica, valida nelle regioni in cui le autofunzioni variano molto pi\'u rapidamente delle grandezze di equilibrio e che trascura la perturbazione del potenziale gravitazionale.
 
L'inversione del sistema completo di equazioni dei modi si effettua considerando le perturbazioni al MSS che danno un miglior accordo tra frequenze osservate e misurate: considero solo i termini lineari nelle perturbazioni e quindi le correzioni agli autovettori sono trascurate.

I risultati dell'inversione sismologica permettono di valutare l'accuratezza della struttura del \mss{} , in particolare del profilo radiale della velocit\'a del suono, della densit\'a e di $\Gamma_1$; inoltre, usando l'equazione di stato per esplicitare la dipendenza di $\Gamma_1$ da $Y$, \'e possibile ricavare l'abbondanza di elio nella zona convettiva $Y_{CZ}$.

\'E possibile ricavare le caratteristiche della base della zona convettiva, profondit\'a della zona convettiva $d_{CZ}=\rsun{}-R_{CZ}$, $\rho_b=\rho(R_{CZ})$, $c_s=c_s(R_{CZ})$, oltre a $Y_{CZ}$, con grande accuratezza.

\section{Inversione della legge di Duvall. % Inizio chapter "Inversione asintotica." senza nuava pagina

%Si ricava, usando le espressioni asintotiche \eqref{cowosc:main}, il profilo radiale della velocit\'a del suono indipendente dal modello solare e si determina l'effetto dell'evoluzione stellare sui modi p di basso ordine radiale.
%Considerando la differenza tra risultati relativi a diversi set di frequenze \'e possibile attenuare gli effetti degli errori sistematici dovuti alla descrizione asintotica.


L'inversione della legge di Duvall \eqref{eq:duvallexpli} permette di ricavare il profilo di $c_s(r)$ sulla base dei modi osservati, tuttavia le approssimazioni fatte introducono errori sistematici: per i modi pi\'u penetranti nell'interno solare la perturbazione del potenziale gravitazionale influenza sensibilmente $F(\frac{\omega}{L})$ mentre per modi confinati vicino alla superficie $\alpha$ dipende da l.


\subsection{Inversione analitica della velocit\'a del suono}

\begin{figure}[!ht]
        \includegraphics[width=0.44\textwidth,keepaspectratio]{soundspeed}
        \caption{Profilo radiale di $c_s^2$ determinato invertendo \eqref{eq:analinversionc} dalle frequenze dei modi osservate. Da \cite{christensen1985speed}.}\label{fig:soundspeedccm}
\end{figure}

L'equazione \eqref{eq:duvallf} pu\'o essere invertita analiticamente:
\begin{equation}
r=R\Exp{-\frac{2}{\pi}\int_{a_s}^a(w\expy{-2}-a\expy{-2})\expy{-\frac{1}{2}}\TDy{w}{F}\,dw}\label{eq:analinversionc}
\end{equation}

dove $a=\frac{c_s}{r}$.

Dall'equazione precedente, nota la funzione $F(w)$ dalle osservazioni, di ricavare $c_s(r)$ (vedi figura \ref{fig:soundspeedccm}). Il confronto tra $c_{sm}(r)$ calcolato tramite un modello solare e $c_{s0}(r)$ invertito usando l'equazione precedente per lo stesso modello mostra che l'errore sistematico dovuto alla tecnica di inversione nel range $0.4\leq x \leq 0.9$ \'e minore del $2.5\%$.


\subsection{Struttura dei modi penetranti nel core stellare}

La deviazione dalla \eqref{eq:freqequi} fornisce informazioni sull'evoluzione chimica del core di fusione: infatti estendendo ancora l'espansione di \eqref{eq:duvallf} si ha una misura della variazione di $c_s$ nel core della stella
\begin{equation}\label{eq:tassoul}
    d_{nl}=\nu_{nl}-\nu_{n-1,l+2}\approx-(4l+6)\frac{\Delta\nu}{4\pi^2\nu_{nl}}\int_0^R\frac{dc_s}{dr}\frac{dr}{r}
\end{equation}
La velocit\'a del suono \'e ridotta a causa dell'aumentare di $\mu$ durante la fusione di H in He durante l'evoluzione stellare: il centro solare \'e un minimo locale per la velocit\'a del suono e quindi, essendo il gradiente della velocit\'a del suono positivo, la parte centrale da un contributo sempre pi\'u negativo in \eqref{eq:tassoul} con l'evolversi della stella.

\subsection{Forma differenziale della legge di Duvall}

Considero l'effetto di perturbazioni del modello sulle frequenze dei modi introducendo nella legge di Duvall \eqref{eq:duvallexpli} perturbazioni nel profilo della velocit\'a del suono e in $\alpha$:
\begin{equation}
S_{nl}\frac{\delta\omega_{nl}}{\omega_{nl}}\approx H_1(\frac{\omega_{nl}}{L})+H_2(\omega_{nl})\label{eq:Dlinear}
\end{equation}

\begin{align}
&S_{nl}=\int_{r_t}^R(1-\frac{L^2c^2}{r^2\omega_{nl}^2})\expy{-\frac{1}{2}}\frac{dr}{c}-\pi\TDy{\omega}{\alpha}\\
&H_1(w)=\int_{r_t}^R(1-\frac{c^2}{r^2w^2})\expy{-\frac{1}{2}}\frac{\delta_rc}{c}\frac{dr}{c},\ H_2(\omega)=\frac{\pi}{\omega}\delta\alpha(\omega)
\end{align}

La funzione $S_{nl}$ \'e approssimabile con un temine proporzionale a $Q_{nl}$ (\cite{christensen1991solar}):
\begin{align}
&\frac{S_{nl}}{\tau_0}\approx Q_{nl}\intxt{con}
&\tau_0=\int_{0}^R\frac{dr}{c_s}
\end{align}

Le funzioni $H_1(\frac{\omega_{nl}}{L})$ e $H_2(\omega_{nl})$ possono essere ottenute separatamente attraverso fitting dei dati sperimentali: la prima caratterizza il contributo alle differenze nelle frequenze dei modi dovuto alle differenze del profilo radiale della velocit\'a del suono, la seconda alle diffenze nella regione vicino alla superficie.

\begin{figure}[!ht]
        \includegraphics[width=0.44\textwidth,keepaspectratio]{H2dnd}
        \caption{a) Residuo della differenza di frequenze fra il sole e un modello senza diffusione a cui \'e stato sottratto $H_1$. b) Fit di $H_2$ linea continua e per contrasto fit di $H_2$ per differenze di frequenze tra Sole e modello con diffusione. Da \cite{dal03notes}.}\label{fig:H2dnd}
\end{figure}

La relazione \eqref{eq:Dlinear}, considerando che $1-\midfrac{L^2c^2}{r^2\omega^2}\approx1$ ad eccezione delle regioni vicino al punto d'inversione $r_t$, pu\'o essere approssimata da
\begin{equation}
\frac{\delta\omega}{\omega}\approx\frac{\int_{r_t}^{R}\frac{\delta_rc_s}{c_s}\frac{dr}{c_s}}{\int_{r_t}^R\frac{dr}{c_s}}
\end{equation}
Le differenze nella velocit\'a del suono nelle varie regioni influiscono sulle differenze nelle frequenze con un peso dato dal tempo impiegato da un'onda sonora ad attraversare la regione: le differenze nella regione vicino alla superficie dove $c_s$ \'e minore hanno un effetto relativamente grande sulle differenze di frequenza.

Una volta determinato $H_1$ le differenze nel profilo radiale di $c_s$ sono determinate tramite
\begin{equation}
\frac{\delta_rc_s}{c_s}=-\frac{2a}{\pi}\TDof{\ln{r}}\int_{a_s}^a(a^2-w^2)\expy{-\frac{1}{2}}H_1(w)\,dw
\end{equation}


La funzione $H_2$ \'e determinata dalla regione sotto la fotosfera. \cite{chr92phase}, analizzando la relazione tra $H_2(\omega)$ e le differenze in $c_s(r)$ e $\Gamma_1$ nelle regioni esterne, hanno visto che discrepanze pi\'u vicino alla superficie generano una componente lentamente oscillante in $H_2(\omega)$ e la ''frequenza'' aumenta con l'aumentare della profondit\'a. \'E inoltre possibile indagare l'andamento di $\Gamma_1$ nella regione di seconda ionizzazione di He e pi\'u in generale il comportamento di $H_2(\omega)$ nelle zone di ionizzazione di H e He consente un'analisi dell'equazione di stato e determinazione dell'abbondanza di elio nella zona convettiva. In figura \ref{fig:H2dnd} si mostra le differenze nelle frequenze tra il Sole ed un modello che non considera la diffusione degli elementi: l'andamento oscillatorio della linea continua nel pannello b \'e dovuto alla differenza nell'abbondanza di idrogeno negli strati superficiali.


\section{Inversione non asintotica.} % Inizio chapter "Inversione non asintotica." senza nuava pagina

La soluzione del problema inverso per il sistema completo di equazioni si basa sulla linearizzazione delle variazioni attorno ad un modello di cui siano calcolabili le autofunzioni dell'operatore $L$ definito in \eqref{eq:variational}.

Utilizzo la formula \eqref{eq:variational} specializzata al problema dell'inversione delle differenze $\delta\omega_{nl}=\omega_{\odot}-\omega_{Mod}$ fra frequenze osservate e predette da un modello. Per l'inversione della struttura idrostatica si ha:
\begin{align}
&\frac{\delta\omega_{nl}}{\omega_{nl}}=\int_0^R[K^{nl}_{c^2,\rho}(r)\frac{\delta_rc^2}{c^2}(r)+K^{nl}_{\rho,c^2}(r)\frac{\delta_r\rho}{\rho}(r)]\,dr+I_{nl}\expy{-1}F_{Surf}(\omega_{nl})+\sigma_i\label{eq:invstructure}\intxt{dove $\sigma_i$ \'e l'incertezza sulle frequenze osservate e}
&\frac{\delta_rc^2}{c^2}(r)=\frac{[c_{\odot}^2(r)-c_{mod}^2(r)]}{c^2(r)},\ \frac{\delta_r\rho}{\rho}(r)=\frac{[\rho_{\odot}(r)-\rho_{mod}(r)]}{\rho(r)}
\end{align}

E quindi si determinano attraverso procedure numeriche le correzioni alla struttura del modello sulla base delle differenze tra frequenze dei modi. Il peso che una perturbazione ha sulla differenze in frequenza \'e determinato dalle autofunzioni dei modi calcolate tramite un modello solare.
%Asymptotic approximation for radial eigenfunction (integral equation connectin sound speed $c(r)$ to $\Omega_{nl}$) is inadequate (especially in deep interior)

In \ref{fig:deltacwu} la banda chiara attorno allo zero, che indica il modello solare in perfetto accordo con le osservazioni sismologiche, mostra l'incertezza nell'inversione della velocit\'a del suono; essa \'e dovuta a

\begin{itemize}

\item Incertezze nelle frequenze dei modi osservate. Oltre all'incertezza statistica propria del determinato strumento \'e possibile valutare gli effette di eventuali errori sistematici confrontando i risultati dell'inversione per set di frequenze ottenute con strumenti diversi: la differenza nella velocit\'a del suono \'e minore di $0.02\%$.

\item Incertezze inerenti la tecnica di inversione.

\item Incertezze legate alla dipendenza da un modello solare per il calcolo dei kernel in \eqref{eq:invstructure}, \eqref{eq:invdGammadrho}, \eqref{eq:splitfreqrotation}.

\end{itemize}


\subsection{Tecniche di inversione numeriche.}
%vedi JCD 2002 pg 25-32

Elenco alcune tecniche numeriche usate. Considero per sempicit\'a una sola funzione da invertire $\frac{\delta f}{f}$, legata a $\delta\omega$ da \eqref{eq:variational}.


\subsubsection{RLS}

Usando la tecnica del minimo $\chi^2$ regolarizzato si parametrizza la funzione incognita $\frac{\delta f}{f}$ tramite funzioni di base opportune.

La funzione da minimizzare \'e
\begin{equation}
Y=(N-N_p)\chi^2+\alpha N\int_0^1(x\TDof{x}\frac{\Delta f}{f})^2\,dx\label{eq:minimizerls}
\end{equation}
con
\begin{equation}
\chi^2=\frac{1}{N-N_p}\sum_{\alpha=1}^N(\frac{\delta\omega_{obs}-\delta\omega_{fit}}{\sigma})^2_{\alpha}
\end{equation}
N indica il numero totale di modi $\alpha$, $N_p$ il numero di parametri da determinare, $\Delta\nu_{fit}$, ricavato tramite \eqref{eq:variational}, contiene la funzione incognita opportunamente parametrizzata; il secondo addendo del lato destro di \eqref{eq:minimizerls} \'e introdotto per ridurre oscillazioni indesiderate nel risultato dell'inverisone con $\alpha$, parametro di regolarizzazione, scelto opportunamente.

\subsubsection{Subtractive Optimally Localized Averaging}

Scelgo dei coefficienti $c_i(r_0)$ tali che $\sum c_i(r_0)\frac{\delta\omega_i}{\omega_i}$ fornisca una media del valore di $\frac{\delta f(r)}{f(r)}$ in $r=r_0$:

\begin{equation}\label{eq:SOLAfmean}
\sum_ic_i(r_0)\frac{\delta\omega_i}{\omega_i}=\int_0^R\sum_ic_i(r_0)K^i(r)\frac{\delta f(r)}{f(r)}\,dr=\exv{\frac{\delta f(r_0)}{f(r_0)}}
\end{equation}

e i coefficienti $c_i(r_0)$ sono determina minimizzando la funzione

\begin{equation}\label{eq:SOLAcir0min}
\int_0^R[\mathcal{K}(r_0,r)-\mathcal{T}(r_0,r)]^2\,dr+\mu\sum_i\sigma_ic_i(r_0)c_j(r_0)\\
\end{equation}
con $\mathcal{K}(r_0,r)=\sum_ic_i(r_0)K^i(r)$ e la funzione target $\mathcal{T}(r_0,r)$, la cui larghezza \'e anch'essa parametro del fit, determina la natura precisa della localizzazione.

La larghezza finita di $\mathcal{K}(r_0,r)$ determina il valor medio di $\frac{\delta f}{f}$ in intorno di $r_0$ e ci\'o causa una differenza sistematica  dal valore effettivo in $r_0$: per $c_s$ l'errore \'e minore del $0.03\%$ e le regioni pi\'u afflitte sono la base della zona convettiva, per la rapida variazione di $\frac{\delta c_s}{c_s}$ e la regione centrale, per il ridotto numero di modi che penetrano in questar zona.


Illustro la tecnica SOLA per determinare $\frac{\delta_rc^2}{c^2}$: si formano delle combinazioni lineari di $\frac{\Lvar{\omega_i}}{\omega_i}$ pesate da coefficienti $c_i(r_0)$ tali che $\frac{\Lvar{c^2}}{c^2}$ sia centrato attorno $r_0$ e che gli altri termini in \eqref{eq:invstructure} siano soppressi, queste compongo un averaging kernel $\mathcal{K}_{c^2,\rho}(r_0,r)=\sum_ic_i(r_0)K_{c^2,\rho}^i(r)$, con $\int_0^R\mathcal{K}(r_0,r)\,dr=1$.

Determino i coefficienti minimizzando l'espressione
\begin{equation}
\int_0^R[\mathcal{K}_{c^2,\rho}(r_0,r)-\mathcal{T}(r_0,r)]^2\,dr+\beta\int_0^R\mathcal{L}_{\rho,c^2}(r_0,r)\,dr+\mu\sum_i\sigma_ic_i(r_0)c_j(r_0)
\end{equation}
dove il kernel cross-term
\begin{equation}
\mathcal{L}_{\rho,c^2}(r_0,r)=\sum_ic_i(r_0)K_{\rho,c^2}^i(r)
\end{equation}
contralla i contributi indesiderati di $\frac{\delta_r\rho}{\rho}$.


Il primo termine approssima il valore di $\frac{\delta c^2}{c^2}$ pesato dal kernel $\mathcal{K}(r,r_0)=\sum_ic_i(r_0)K_{c^2,\rho}^i(r)$, il secondo tiene conto dell'influenza che hanno le discrepanze della seconda funzione su quelle della funzione che abbiamo scelto di invertire pesate da $\mathcal{L}_{\rho,c^2}(r_0,r)=\sum_ic_i(r_0)K_{\rho,c^2}^i(r)$, il terzo \'e il termine di superficie: i coefficienti $c_i(r_0)$ sono scelti in maniera da riprodurre la funzione target, minimizzare la contaminazione delle $\frac{\delta \rho}{\rho}$ via $\mathcal{L}_{\rho,c^2}$ e il rumore.


\subsection{Inversione della rotazione.}

\begin{figure}[!ht]
\centering
\includegraphics[keepaspectratio,width=0.8\textwidth]{invertedrotation}
\caption{Inversione della velocit\'a di rotazione a diverse latitudini. La linea verticale tratteggiata indica la base della zona convettiva. Da \cite{chr02helioseismology}.}
\end{figure}

Per inversione 2D, cio\'e che considera la dipendenza generica $\Omega(r,\theta)$, si esprimono direttamente le differenze in frequenze:
\begin{equation}
\omega_{nlm}-\omega_{nl0}=m\int_0^R\int_0^{\pi}K_{nlm}(r,\theta)\Omega(r,\theta)r\,dr\,d\theta\label{eq:invrot2D}
\end{equation}

mentre nel caso si abbiano i coefficienti $a_{2s+1}$, scrivo la velocit\'a angolare nella forma
\begin{equation}
\Omega(r,\theta)=\sum_{s=0}^{s_m}\Omega_{s}(r)\psi_{2s}(\cos{\theta})\label{eq:angularv15}
\end{equation}
dove $\psi_{2s}$ sono polinomi opportuni.

Esiste una funzione opportuna $K_{nls}^{s}(r)$ tale che
\begin{equation}
2\pi a_{2j+1}(n,l)=\int_0^R\int_0^{\pi}K_{nls}^{s}(r)\Omega_s(r)\,dr
\end{equation}
e quindi \'e possibile determinare $\Omega_s(r)$.

\section{Vincoli al modello solare dalle osservazioni sismologiche.} %%chapter: vincoli al modello solare: HCSM.


\begin{figure}[!ht]%{r}{0.5\textwidth}
        \includegraphics[width=0.9\textwidth,keepaspectratio]{deltacwu}
        \caption{Differenza relativa nel profilo di $c_s$ risultanei dall'inversione delle differenze in frequenza tra Sole e modello: la linea chiara si riferisce alle frequenze di un modello con composzione GS98, la linea scura a composizione AGSS09. La banda chiara mostra l'errore inerente l'inversione eliosismologica, la banda scura l'incertezza a $1\sigma$ sul profilo di $c_s$ predetto dal modello. Da \cite{villante2014chemical}.}\label{fig:deltacwu}
\end{figure}

\begin{table}[!ht]%{r}{0.7\textwidth}

\pgfplotstabletypeset[
math/.style={%
        preproc cell content/.append style={/pgfplots/table/@cell content/.add={$}{$}},
    },
every head row/.style={
 before row={\toprule
 %&\multicolumn{4}{c|}{Primordiale}
 },
 every last row/.style={after row=\bottomrule},
 after row={\midrule}
},
every last row/.style={after row=\bottomrule},
every first column/.style={column type/.add={|}{}},
every last column/.style={column type/.add={}{|}},
%columns/0/.style = {column type/.add={|}{}},
display columns/0/.style={column name={Composizione}},
display columns/1/.style={column name={$Z/X$}},
display columns/2/.style={column name={$R_{CZ}$}},
display columns/3/.style={column name={$Y_{CZ}$}},
display columns/4/.style={column name={$Y_0$}},
create on use/comp/.style={create col/set list={
inversione,GS98,AGS05,AGSS09,C+11}},
columns/comp/.style = {column type/.add={|}{}},
columns/comp/.style={string type},
columns/ZX/.style={string type},
columns/ZX/.append style={math},
columns/RCZ/.style={string type},
columns/RCZ/.append style={math},
columns/YCZ/.style={string type},
columns/YCZ/.append style={math},
columns/Y0/.style={string type},
columns/Y0/.append style={math},
columns={comp,ZX,RCZ,YCZ,Y0},
%/pgf/number format/precision=4
     ]{CZvsZ.txt} %%%
     \caption{Caratteristiche della zona convettiva: confronto tra valore eliosismologico e valore ricavato da modello solare con diverse metallicit\'a del raggio della base della zona convettiva $R_{CZ}$, dell'abbondanza di elio superficiale $Y_{CZ}$ e dell'abbondanza di elio primordiale. Da \cite{basu2016global}.}
\label{tab:CZZvar}
\end{table}

L'inversione di $c_s$ o $\rho$ mostra se un modello solare riproduce accuratamente la posizione della base della zona convettiva in quanto nella zona convettiva si ha gradiente adiabtico maggiore del gradiente radiativo; diminuzione di opacit\'a, nel caso determinata da una minore metallicit\'a, sposta la base della zona convettiva pi\'u in alto come da tabella \ref{tab:CZZvar}.

La figura \ref{fig:deltacwu} mostra che un modello solare con composizione GS98, meno accurata di AGSS09, riproduce il profilo di $c_s$ in maniera pi\'u accurata: ci\'o pu\'o indicare un'opacit\'a da incrementare nel modello. Analogamente la diminuzione dell'opacit\'a diminuisce il gradiente termico nella regione radiativa quindi a pari luminosit\'a si ha un contenuto di idrogeno maggiore.




\end{document}