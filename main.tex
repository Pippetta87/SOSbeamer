\documentclass[11pt,xcolor={usenames},fleqn,mathserif,serif]{beamer}

%%%Usefull link
%tikz-equations:
%http://www.wekaleamstudios.co.uk/posts/creating-a-presentation-with-latex-beamer-equations-and-tikz/

% There are many different themes available for Beamer. A comprehensive
% list with examples is given here:
% http://deic.uab.es/~iblanes/beamer_gallery/index_by_theme.html
% You can uncomment the themes below if you would like to use a different
% one:
%\usetheme{AnnArbor}
%\usetheme{Antibes}
%\usetheme{Bergen}
%\usetheme{Berkeley}
%\usetheme{Berlin}
%\usetheme{Boadilla}
%\usetheme{boxes}
%\usetheme{CambridgeUS}
%\usetheme{Copenhagen}
%\usetheme{Darmstadt}
\usetheme{default}
%\usetheme{Frankfurt}
%\usetheme{Goettingen}
%\usetheme{Hannover}
%\usetheme{Ilmenau}
%\usetheme{JuanLesPins}
%\usetheme{Luebeck}
%\usetheme{Madrid}
%\usetheme{Malmoe}
%\usetheme{Marburg}
%\usetheme{Montpellier}
%\usetheme{PaloAlto}
%\usetheme{Pittsburgh}
%\usetheme{Rochester}
%\usetheme{Singapore}
%\usetheme{Szeged}
%\usetheme{Warsaw}

\setbeamertemplate{caption}{%\insertcaption}
\setdefaultlanguage{french}

%% colors
\definecolor{bittersweet}{rgb}{1.0, 0.44, 0.37}
\definecolor{brilliantlavender}{rgb}{0.96, 0.73, 1.0}
\definecolor{antiquefuchsia}{rgb}{0.57, 0.36, 0.51}
\definecolor{violetw}{rgb}{0.93, 0.51, 0.93}
\definecolor{Veronica}{rgb}{0.63, 0.36, 0.94}
\definecolor{atomictangerine}{rgb}{1.0, 0.6, 0.4}
\definecolor{darkgray}{rgb}{0.66, 0.66, 0.66}
\definecolor{brightcerulean}{rgb}{0.11, 0.67, 0.84}
\definecolor{cadmiumorange}{rgb}{0.93, 0.53, 0.18}
\definecolor{ochre}{rgb}{0.8, 0.47, 0.13}
\definecolor{midnightblue}{rgb}{0.1, 0.1, 0.44}
\definecolor{lemon}{rgb}{1.0, 0.97, 0.0}
\definecolor{grey}{rgb}{0.7, 0.75, 0.71}
\definecolor{amber}{rgb}{1.0, 0.75, 0.0}
\definecolor{almond}{rgb}{0.94, 0.87, 0.8}
\definecolor{bf}{RGB}{88, 86, 88}
\definecolor{bb}{RGB}{177, 177, 177}


%%%%%%%%%%%%%%%%%%%%%%%%%%%%%%%%%%% importa pacchetti
\usepackage{usepkg}
%%%%%%%%%%%%%%%%%%%%%%%%%%%%%%%%%%% Funzioni generali
\usepackage{functions}
%http://tex.stackexchange.com/questions/246/when-should-i-use-input-vs-include
\newcommand{\setmuskip}[2]{#1=#2\relax} %%problem usinig mu with calc (req by mathtools) loaded
\usepackage{sources}
%\usepackage{length}
%%%%%%%%%%%%%%%%%%%%%%%%%%%%%%%%%%% Funzioni per questo file main
\usepackage{mathOp}
\usepackage{LocalF}
%%%%%%%%%%%%%%%%%%%%%%%%%%%%%%%%%

\title{Modi normali di oscillazione del Sole (Presentazione)}

% A subtitle is optional and this may be deleted
\subtitle{Struttura interna e modi di oscillazione}

%\author{F.~Author\inst{1} \and S.~Another\inst{2}}
% - Give the names in the same order as the appear in the paper.
% - Use the \inst{?} command only if the authors have different
%   affiliation.

%\institute[Universities of Somewhere and Elsewhere] % (optional, but mostly needed)
%{
% \inst{1}
% Department of Computer Science\\
%  University of Somewhere
%  \and
%  \inst{2}%
%  Department of Theoretical Philosophy\\
%  University of Elsewhere}
% - Use the \inst command only if there are several affiliations.
% - Keep it simple, no one is interested in your street address.

\date{Appello Giugno, \today}
% - Either use conference name or its abbreviation.
% - Not really informative to the audience, more for people (including
%   yourself) who are reading the slides online

\subject{Eliosismologia}
% This is only inserted into the PDF information catalog. Can be left
% out. 

% If you have a file called "university-logo-filename.xxx", where xxx
% is a graphic format that can be processed by latex or pdflatex,
% resp., then you can add a logo as follows:

% \pgfdeclareimage[height=0.5cm]{university-logo}{university-logo-filename}
% \logo{\pgfuseimage{university-logo}}

% Delete this, if you do not want the table of contents to pop up at
% the beginning of each subsection:
%\AtBeginPart[]
%{
%  \begin{frame}<beamer>{Outline}    %\tableofcontents[currentsection]
%  \end{frame}
%}

\makeatletter
\AtBeginPart{%
  \addtocontents{toc}{\protect\beamer@partintoc{\the\c@part}{\beamer@partnameshort}{\the\c@page}}%
}
%% number, shortname, page.
\providecommand\beamer@partintoc[3]{%
  \ifnum\c@tocdepth=-1\relax
    % requesting onlyparts.
    \makebox[6em]{PART #1:} #2
    \par
  \fi
}
\define@key{beamertoc}{onlyparts}[]{%
  \c@tocdepth=-1\relax
}
\makeatother%

\setbeamertemplate{navigation symbols}{}

\makeatletter
\setbeamertemplate{footline}
{
    \leavevmode%
    \hbox{%Refintro
        \begin{beamercolorbox}[wd=.1\paperwidth,ht=2.25ex,dp=1ex,center]{author in head/foot}%
            \hyperlink{intro}{Intro}
        \end{beamercolorbox}%

 \begin{beamercolorbox}[wd=.1\paperwidth,ht=2.25ex,dp=1ex,center]{author in head/foot}%refs Part 1
            \hyperlink{part:MSS}{MSS}
        \end{beamercolorbox}%

 \begin{beamercolorbox}[wd=.2\paperwidth,ht=2.25ex,dp=1ex,center]{author in head/foot}%refs Part 2
            \hyperlink{part:oscillations}{Oscillazioni lineari adiabatiche}
        \end{beamercolorbox}%
        
         \begin{beamercolorbox}[wd=.2\paperwidth,ht=2.25ex,dp=1ex,center]{author in head/foot}%refs Part 3
            \hyperlink{part:inverseproblem}{Problema inverso}
        \end{beamercolorbox}%inverseproblem
        
        \begin{beamercolorbox}[wd=.35\paperwidth,ht=2.25ex,dp=1ex,right]{date in head/foot}%
            %   \usebeamerfont{date in head/foot}\insertshortdate{}\hspace*{2em}
            \insertframenumber{} \hspace*{2ex}  / \hspace*{2ex} \inserttotalframenumber
            \hspace*{2ex} 
        \end{beamercolorbox}}%
        \vskip0pt%
    }
    \makeatother

\AtBeginSection{\frame{\sectionpage}}

% Let's get started
\begin{document}

\begin{frame}
  \titlepage
\end{frame}


% Section and subsections will appear in the presentation overview
% and table of contents.
%\frame{\tableofcontents[onlyparts]}

\begin{frame}[label={intro}]{Studio delle oscillazioni solari}{Argomenti trattai nella tesina}
\begin{block}{Equazioni fondamentali e input fisici del MSS}

MSS riproduce $\lsun{}$, $\rsun{}$.

Grandezze sismologiche.

\end{block}


\begin{block}{Oscillazoni lineari adiabatiche}
Perturbazioni adiabatiche dello stato di equilibrio: modi normali.

Eccitazione dei modi: fenomeno stazionario stacastico.

Campi di velocit\'a solari: rotazione.

Analisi asintotica: cavit\'a risonanti.

\end{block}

\begin{block}{Tecniche di inversione}
Soluzione del problema inverso: vincoli su $Y_{ph}$, e struttura interna.
\end{block}
\end{frame}



\part{Modello solare e osservabili sismologiche}\label{part:MSS}
\frame{\partpage}

\begin{frame}{Argomenti}
  \tableofcontents[part=1%,pausesections
  ]
  % You might wish to add the option [pausesections]
\end{frame}

\begin{comment}
\section{Osservabili stellari/demo beamer}
\begin{frame}<1>[label=noinside]{Modello stellare}{Come indagare la fisica interna a una stella?}
\onslide<1->\begin{block}{Osservabili stellari:}
$L$, $M$, $R$, $T_e$, $(\frac{Z}{X})_{ph}$, $g_{ph}$.
\end{block}
\onslide<1->\begin{block}{Informazioni sulla struttura interna?} Condizione di equilibrio idrostatico
\end{block}
%Teorema Vogt-Russel: $X_i(r)$, $M$ \pause equilibrio (idrostatico/termico) determinano struttura stellare .
%\pause
\onslide<1->\begin{block}{Modello stellare: diagramma di \hr{}.}
\end{block}
\onslide<2->\begin{block}{Descrizione fisica interno stellare: parametri aggiuntivi}
Convezione, diffusione e sedimentazione elementi pesanti, equazione di stato, opacit\'a
\end{block}
\onslide<2->\begin{block}{Astrosismologia}
Restringo spazio parametri sistemi stellari lontani
\end{block}
\end{frame}
{ % all template changes are local to this group.
    \setbeamertemplate{navigation symbols}{}
    \begin{frame}[plain]{Diagramma di \hr{}}
        \begin{tikzpicture}[remember picture,overlay]
            \node[at=(current page.center)] {
                %\includegraphics[width=\paperwidth]{yourimage}
            };
        \end{tikzpicture}
     \end{frame}
}
\againframe<2>{noinside}
\begin{frame}{Pulsazioni stellari}{Modi Normali}
\begin{columns}
\begin{column}{0.5\textwidth}  %%<--- here
    \begin{center}
     %\includegraphics[width=0.5\textwidth]{image1}
     \end{center}
\end{column}
\begin{column}{0.5\textwidth}
\onslide<1-> \begin{block}{Stelle pulsanti}
Onde stazionarie: Pulsazione radiale/non radiale: .
\onslide<2-> meccanismo di eccitazione: solar-like pulsator, Cefeidi.
\onslide<3-> Modo fondamentale $\Pi\approx\tau_{dyn}=\sqrt{\frac{R^3}{GM}}\propto\overline{\rho}\expy{-\frac{1}{2}}$.
\onslide<4-> Modi di oscillazione\onslide<5-> - informazioni sull'interno stellare
\onslide<5-> Elio-sismologia: Modi $\Leftrightarrow$ Modelli solari
\onslide<5-> Astero-sismologia: Modi $\Leftrightarrow$ Spazio parametri modello stellare
\end{block}
\end{column}
\end{columns}
\end{frame}

\end{comment}

\section{Osservabili solari}

\begin{frame}{Dati osservativi}

\begin{block}{Et\'a, luminosit\'a, raggio solari}
\begin{tabular}{l|c}
\hline
$\agesun{}$&\SI[separate-uncertainty=true]{4.57\pm0.02e9}{\year}\\
\hline
$\rsun{}$&\SI{695658+-140}{\kilo\meter}\\
\hline
$G\msun$&\num{132712440018+-8}\SI{e9}{\cubic\meter\per\square\second}\\
\hline
$\lsun{}$&\SI{3.8275+-0.0014e33}{\erg\per\second}\\
\hline
\end{tabular}
%\caption[Osservabili solari principali.]{Osservabili solari principali. \cite{haberreiter2008solving}.}
\label{tab:sunO}
\end{block}

\begin{block}{Simmetria sferica}
Deviazioni da forma sferica trascurabili (campi magnetici, rotazione)
\end{block}

\end{frame}

\begin{frame}{Dati osservativi}

\begin{block}{Composizione chimica}
\begin{itemize}
\item Righe di assorbimento: attuale (non $Y_{ph}$)
\item Meteoriti CI: primordiale (refrattari)
\end{itemize}

\begin{table}[]

\pgfplotstabletypeset[
every head row/.style={
 before row={\toprule &\multicolumn{4}{c|}{Attuale}
 %&\multicolumn{4}{c|}{Primordiale}
 \\\midrule},
 every last row/.style={after row=\bottomrule},
 after row={\midrule}
},
every nth row={2}{before row=\midrule},every last row/.style={after row=\bottomrule},
every first column/.style={column type/.add={|}{}},
every last column/.style={column type/.add={}{|}},
columns/x/.style = {column type/.add={|}{}},
columns/xi/.style = {column type/.add={|}{}},
display columns/0/.style={column name={}},
display columns/1/.style={column name={$X$}},
display columns/2/.style={column name={$Y$}},
display columns/3/.style={column name={$Z$}},
display columns/4/.style={column name={$\frac{Z}{X}$}},
%display columns/5/.style={column name={$X$}},
%display columns/6/.style={column name={$Y$}},
%display columns/7/.style={column name={$Z$}},
%display columns/8/.style={column name={$\frac{Z}{X}$}},
create on use/authors/.style={create col/set list={
%Anders \& Grevesse (1989),Grevesse \& Noels (1993),
Grevesse et al. (1998),Lodders (2003),Asplund et al. (2005),Lodders et al. (2009),\cite{asplund2009chemical},\cite{caffau2011solar}}},
columns/authors/.style={string type},
columns={authors,x, y, z, zx
%,xi,yi,zi, zxi
},
/pgf/number format/precision=4
     ]{asplund.txt} %%%
\captionof{table}{Metallicit\'a attuale determinata da varii autori.}\label{tab:Zhistory}
\end{table}

\end{block}

\end{frame}


\section{Strutture autogravitanti in equilibrio}

\begin{frame}{Distribuzione di massa - Conservazione di massa e momento - tempo scala dinamico}

\begin{block}{Massa}

%\begin{align}
%&dm=4\pi r^2\rho \,dr-4\pi r^2\rho v\,dt\label{eq:massvar}\\
%\end{align}

\begin{equation}
\PDy{t}{\rho}+\nabla\cdot(\rho\vec{v})=0\label{eq:continuityeq}
\end{equation}

\begin{equation}
dm=4\pi r^2\rho \,dr\label{eq:massaguscio}
\end{equation}

\end{block}

\begin{block}{Momento}
\begin{align}
&\rho\TDy{t}{\vec{v}}=-\nabla P+\rho\vec{f}\label{eq:motion}\\
&\vec{g}=-\PDy{r}{\Phi}=-\frac{Gm(r)}{r^2}\hat{r}
\end{align}
\end{block}

\end{frame}

\begin{frame}{Equilibrio idrostatico: $\ddvec{r}=0$.}


\begin{align}
\nabla P=\rho \vec{f}\Label{eq:idrosta} \TDy{r}{P}=-\frac{Gm(r)\rho(r)}{r^2}\Label{eq:fidroequilibrio}
\end{align}


Per giustificare l'ipotesi di equilibrio idrostatico stimo i tempi caratteristici di evoluzione della struttura solare nel caso la forza dovuta alla pressione o la forza di gravit\'a non fossero bilanciate, approssimando il valore caratteristico della derivata di due variabili con il rapporto del loro valore caratteristico:
\begin{align}
&\tau_{ff}\approx\sqrt{\frac{\rsun{}}{g}}\\
&\tau_{esp}\approx \rsun{}\sqrt{\frac{\rho}{P}}
\end{align}

Per i valori solari \ref{wrap-tab:sunO} $\tau_{ff}\approx\tau_{esp}\approx\SI{27}{\minute}$.

\begin{equation}
\tau_{idro}^{\odot}= \sqrt{\frac{R^3}{GM}}\approx\frac{1}{2}(G\overline{\rho})\expy{-\frac{1}{2}}
\end{equation}

\end{frame}


\subsection{Equazione di stato $P(\rho,T)$}

deviazioni dalla legge dei gas perfetti per tenere conto dei fenomeni di ionizzazione parziale e stati atomici eccitati, della radiazione, della statistica di Fermi-Dirac per gli elettroni, \'e necessario considerare l'interazione Coulombiana.

\begin{frame}{Gas perfetto ioni-elettroni}


\begin{equation}
P_G=P_I+P_e=\frac{\rho}{\mu}\gasconstant{}T
\end{equation}

\begin{block}{Peso molecolare medio}
massa media in amu per particella libera
\begin{align}
&\mu=\frac{1}{\bar{n}_HX+\bar{n}_{He}Y+\bar{n}_{Z}Z}\label{eq:meanmw}\\
&\bar{n}_i=\frac{1+f_i}{A_i}
\end{align}

\end{block}


\end{frame}

\subsection{Energia interna per unit\'a di massa}

\begin{frame}{Energia interna: traslazioni}

\begin{align}
&u=\frac{1}{\rho}\sum_i\int f^{(0)}(\vec{p}_i)\frac{p^2_i}{2m_i}=\frac{3}{2}\frac{P}{\rho}=\frac{3}{2}\frac{\gasconstant T}{\mu}\\
&E_i=\int_0^Mu\,dm=\frac{3}{2}\int_M\frac{P}{\rho}\,dm\label{eq:traslintenergy}
\end{align}

 $f^{(0)}(\vec{p}_i)$ \'e il numero di particelle della specie i per unit\'a di volume con impulso in $[\vec{p},\vec{p}+d\vec{p}]$

\end{frame}


\subsection{Correzioni alla legge dei gas perfetti}

\begin{frame}{Correzioni alla legge dei gas perfetti}

\begin{itemize}
\item Degenerazione elettronica: $\Delta P\leq2\%$.

\item Pressione di radiazione: $P_r=\frac{1}{3}aT^4$.

\item Ionizzazione.

\item Interazioni coulombiane.

\begin{align}
&\frac{1}{r_D^2}=\frac{4\pi e^2}{kT}\sum Z^2\overline{n}_Z=\frac{4\pi e^2}{kT}N_A\zeta\label{eq:debyeradius}\\
&\zeta=\sum_{i}(Z_i^2+Z_i)\frac{\rho X_i}{A_i}
\end{align}

\begin{equation}
u_c=\frac{1}{2}\int\phi(\vec{r})\rho(\vec{r})\,d^3r,\ P_c=\frac{1}{3}u_c
\end{equation}

Regioni di ionizzazione parziale di idrogeno ed elio

\end{itemize}

\end{frame}

\begin{frame}{EOS}


Due approcci usati per determinare l'equazione di stato e quindi le grandezze termodinamiche del plasma solare sono lo schema chimico e lo schema fisico: il primo considera atomi e molecole, la cui popolazione per stati eccitati e diversi gradi di ionizzazione \'e ottenuto minimizzando l'energia libera da cui sono ricavate le altre grandezze termodinamiche; utilizzando questo approccio \'e stata ricavata l'equazione di stato MHD. Il secondo considera nuclei ed elettroni come costituenti fondamentali interagenti tramite potenziale Coulombiano e trova le soluzione dell'equazione di Schr\"oedinger per un problema a molti corpi, questo approccio, usato per ricavare l'equazione di stato OPAL, \'e pi\'u adatto per trattare le regioni interne del Sole.


\begin{figure}[!ht]
        \includegraphics[height=0.4\textwidth,keepaspectratio]{ionfraction}\label{fig:ionfraction}
        \caption{Profilo radiale della popolazione dei diversi gradi di ionizzazione per $\cel{He}{4}{}{}$, CNO, $\cel{Ne}{20}{}{}$, $\cel{Fe}{56}{}{}$. Stati di ionizzazione maggiore sono pi\'u interni. Da \cite{basu2008helioseismology}.}
\end{figure}

\begin{figure}[!ht]
        \includegraphics[height=0.4\textwidth,keepaspectratio]{gamma1eos}\label{fig:gamma1eos}
        \subcaption{Andamento di $\Gamma_1$ calcolato tramite equazione di stato MHD/OPAL. Da \cite{trampedach2006synoptic}.}
\end{figure}


\end{frame}


\section{Trasporto dell'energia}

\section{Produzione di energia - reazioni di fusione}

\begin{block}{Schermaggio debole: $e\phi\ll KT$.}




Formula di Boltzmann per la densit\'a delle particelle con carica Z:
\begin{align}
&n_Z=\overline{n}_Z\exp{-\frac{Ze\phi_i}{kT}}\\
&\nabla^2\phi=-4\pi e\sum Zn_Z-4\pi\sum Z_i\delta(\vec{r}-\vec{r}_i)\label{eq:poissonscreened}
\end{align}
In \eqref{eq:poissonscreened} rimane il termine lineare in $\phi$.





\begin{align}
&\frac{r_D^2}{r_i}\TtwoDy{r}{(r_i\phi_i)}\tag{\ref{eq:poissonscreened} $Z_i$}\\
&\phi=\sum_i\phi_i
\end{align}

La soluzione di \eqref{eq:poissonscreened} \'e
\begin{equation}\label{eq:screenedpotential}
\phi_i=\frac{Z_ie}{r_i}\exp{-\midfrac{r}{r_D}}
\end{equation}


\end{block}

\section{Modello solare standard e osservabili sismologiche}




\part{Oscillazioni della fotosfera con grande coerenza spaziale e temporale - Modi normali di cavit\'a risonanti dell'interno solare}\label{part:oscillations}

\frame{\partpage}

\begin{frame}{Argomenti}
  \tableofcontents[part=2%,pausesections
  ]
  % You might wish to add the option [pausesections]
\end{frame}

\begin{comment}
\section{Osservabili stellari/demo beamer}
\begin{frame}<1>[label=noinside]{Modello stellare}{Come indagare la fisica interna a una stella?}
\onslide<1->\begin{block}{Osservabili stellari:}
$L$, $M$, $R$, $T_e$, $(\frac{Z}{X})_{ph}$, $g_{ph}$.
\end{block}
\onslide<1->\begin{block}{Informazioni sulla struttura interna?} Condizione di equilibrio idrostatico
\end{block}
%Teorema Vogt-Russel: $X_i(r)$, $M$ \pause equilibrio (idrostatico/termico) determinano struttura stellare .
%\pause
\onslide<1->\begin{block}{Modello stellare: diagramma di \hr{}.}
\end{block}
\onslide<2->\begin{block}{Descrizione fisica interno stellare: parametri aggiuntivi}
Convezione, diffusione e sedimentazione elementi pesanti, equazione di stato, opacit\'a
\end{block}
\onslide<2->\begin{block}{Astrosismologia}
Restringo spazio parametri sistemi stellari lontani
\end{block}
\end{frame}
{ % all template changes are local to this group.
\setbeamertemplate{navigation symbols}{}
    \begin{frame}[plain]{Diagramma di \hr{}}
        \begin{tikzpicture}[remember picture,overlay]
            \node[at=(current page.center)] {
                %\includegraphics[width=\paperwidth]{yourimage}
            };
        \end{tikzpicture}
     \end{frame}
}
\againframe<2>{noinside}
\section{Osservazioni}
\subsection{Fitting polinomiale: inversione ''1.5-D''.}
\begin{frame}<1>[label=noinside]{Modello stellare}{Come indagare la fisica interna a una stella?}
\onslide<1->\begin{block}{Rotazione superficiale}
\begin{equation*}
\frac{\Omega(\theta)}{2\pi}=\SI{451.5}{\nano\hertz}-\SI{65.3}{\nano\hertz}\cos^2{\theta}-\SI{66.7}{\nano\hertz}\cos^4{\theta}
\end{equation*}
\end{block}
\onslide<1->\begin{block}{Informazioni sulla struttura interna?} Condizione di equilibrio idrostatico
\end{block}
%Teorema Vogt-Russel: $X_i(r)$, $M$ \pause equilibrio (idrostatico/termico) determinano struttura stellare .
%\pause
\onslide<1->\begin{block}{Modello stellare: diagramma di \hr{}.}
\end{block}
\onslide<2->\begin{block}{Descrizione fisica interno stellare: parametri aggiuntivi}
Convezione, diffusione e sedimentazione elementi pesanti, equazione di stato, opacit\'a
\end{block}
\onslide<2->\begin{block}{Astrosismologia}
Restringo spazio parametri sistemi stellari lontani
\end{block}
\end{frame}
\subsection{Osservazione dello splitting in m: inversione ''2D''.}
\begin{figure}[!ht]
\centering
\includegraphics[keepaspectratio,width=0.8\textwidth]{invertedrotation}
\caption{Inversione della velocit\'a di rotazione a diverse latitudini. La linea verticale tratteggiata indica la base della zona convettiva. Da \cite{chr02helioseismology}.}
\end{figure}
Considero la correzione al primo ordine in $\Omega$. Il campo di velocit\'a rotazionale in coordinate sferiche \'e 
\begin{align}
&\vec{v_0}=(0,0,r\Omega\sin{\theta})=\vecp{\Omega}{r}\\
&\vec{\Omega(r,\theta)}=(\Omega(r,\theta)\cos{\theta},-\Omega(r,\theta)\sin{\theta},0)
\end{align}
In assenza di moti macroscopici il termine d'inerzia \'e $\rho_0\TDy{t}{\vec{v}}=\rho_0\PtwoDy{t}{\vec{\xi}}$, mentre in caso di rotazione si ha
\begin{equation}
\rho_0(\PDof{t}+\scap{v_0}{\nabla})^2\vec{\xi}
\end{equation}
Considero il termine dovuto alla rotazione come una piccola correzione alle frequenze dei modi
\begin{align}
&\omega_{(l,m)}+\Delta\omega_{(l,m)}&\intertext{quindi l'equazione del moto al primo ordine nella perturbazione, con $\alpha=(l,m)$, \'e}\nonumber\\
&\rho_0(\omega_{\alpha}^2+2\omega_{\alpha}\Delta\omega_{\alpha})\vec{\xi}=\nabla P_1-\frac{\rho_1}{\rho_0}\nabla P_0+\rho_0\nabla\Phi_1+2i\omega_{\alpha}\rho_0(\scap{v_0}{\nabla})\vec{\xi}\\
&\intertext{da cui si deduce}\nonumber\\
&\Delta\omega_{\alpha}=\frac{i\int\rho_0\xi_{\alpha}^*(\scap{v_0}{\nabla})\xi_{\alpha}}{\int\rho_0\xi_{\alpha}^*\xi_{\alpha}}=\frac{-m\int\rho_0\Omega\xi_{\alpha}^*\xi_{\alpha}\,dV+i\int\rho_0\xi_{\alpha}^*(\vecp{\Omega}{\xi_{\alpha}})\,dV}{\int\rho_0\xi_{\alpha}^*\xi_{\alpha}}
\end{align}
Il problema di trovare $\Omega(r,\theta)$ dalla differenza $\Delta\omega_{\alpha}$ \'e lineare in $\Omega$ quindi $\Delta\omega_{\alpha}\propto\Omega$. Per determinare quindi la rotazione dobbiamo conoscere l'autovalore $\xi_{\alpha}$ dello stato imperturbato.
%Per rotazione puramente radiale $\Omega(r)$ la relazione tra lo splitting delle frequenze e la rotazione \'e
%\begin{equation}
%\Delta\omega_{\alpha}=-m\frac{\int_0^{\rsun{}}\rho_0\Omega\{|\xi_r-\xi_h|^2+[l(l+1)-2]|\xi_h|^2\}r^2\,dr}{\int_0^{\rsun{}}\rho_0\{|\xi_r|^2+l(l+1)|\xi_h|^2\}r^2\,dr}=\int_0^{\rsun{}}K_{\alpha}(r)\Omega(r)\,dr
%\end{equation}
%Any given $\Delta\omega_{\alpha}$ samples angular velocity in the depth range corresponding to $\xi_{\alpha}$.
La velocit\'a angolare contribuisce a $\Delta\omega_{\alpha}$ negli strati in cui $\xi_{\alpha}$ \'e apprezzabile. Nel caso di rotazione dipendente solo da r si ha che $\Delta\omega_{\alpha}$ \'e lineare in m: ho $2l+1$ frequenze equispaziate.

\end{comment}

\section{Modi normali della struttura solare}

\begin{frame}[label=noinside]{Modi di oscillazione adiabatici}{Perturbazione dello stato di equilibrio.}

\begin{block}{campi di velocit\'a/effetti non lineari}
Descrivo le oscillazioni come piccole perturbazioni attorno allo stato di equilibrio stazionario (gli effetti non lineari, fra cui lo scambio di energia tra i modi, sono dell'ordine di $\frac{v}{c_s}$ dove v \'e l'ampiezza della velocit\'a dell'oscillazione). 
In generale pu\'o essere presente un campo di velocit\'a $\vec{v}_0$:
\begin{align}
&\vec{v}=\vec{v}_0+\vec{v}'\\
&\TDof{t}=\PDof{t}+(\vec{v}_0\cdot\nabla)
\end{align}
in prima approssimazione prendo $\vec{v}_0=0$ per poi considerare come perturbazioni gli effetti dovuti a campi di velocit\'a in specie rotazione.

\end{block}

\begin{block}{Perturbazione pressione densit\'a}

Indico con $P'(\vec{r},t)$ e $\delta P$ la perturbazione euleriana e lagrangiana della pressione e con $\rho'$, $\Phi'$ e $\vec{g}'$ la perturbazione euleriana della densit\'a , e le perturbazioni euleriane del potenziale gravitazionale e dell'accelerazione di gravit\'a conseguenti  con $\delta\vec{r}=\vec{\xi}$ il vettore spostamento perturbato:
\begin{align}
&P(\vec{r},t)=P_0(\vec{r})+P'(\vec{r},t)\label{eq:pressureperturbation}\\
&\Lvar{P(\vec{r})}=P(\vec{r}+\Lvar{\vec{r}})-P_0(\vec{r})=P'(\vec{r})+\Lvar{\vec{r}}\cdot\nabla P_0\\
&\vec{g}'=-\nabla\Phi',\ \nabla^2\Phi'=4\pi G\rho'\label{eq:gapert}
\end{align}

\end{block}


\end{frame}

\begin{frame}[label=noinside]{Modi di oscillazione adiabatici}{Modi di oscillazione lineari adiabatici.}

\begin{block}{Equazione del moto perturbata}

l'equazione del moto perturbato sostituendo \eqref{eq:pressureperturbation} nell'equazione del moto \eqref{eq:motion} considerando solo i termini lineari nella perturbazione:
\begin{equation}
\rho_0\TDof{t}\vec{v}=\rho_0\PtwoDy{t}{\Lvar{\vec{r}}}=-\nabla P'+\rho_0\vec{g}'+\rho'\vec{g}_0\label{eq:emper}
\end{equation}

\end{block}

\end{frame}

\begin{frame}[label=noinside]{Modi di oscillazione adiabatici}{Equazione di continuit\'a perturbata}

\begin{block}{Equazione di continuit\'a perturbata}

Analogamente per l'equazione di continuit\'a ottengo
\begin{equation}
\rho'+\div{(\rho_0\Lvar{\vec{r}})}=0\label{eq:contper}
\end{equation}

\end{block}

\end{frame}

\begin{frame}[label=noinside]{Modi di oscillazione adiabatici}{Condizione di adiabaticit\'a}


  \begin{overlayarea}{\textwidth}{1cm}
   \only<1>{
   energia interna per unit\'a di massa
\begin{equation}
\TDy{t}{q}=\TDy{t}{u}+P\TDof{t}(\frac{1}{\rho})\label{eq:prima}
\end{equation}

\begin{equation}
\TDy{t}{T}-\frac{\Gamma_2-1}{\Gamma_2}\frac{T}{P}\TDy{t}{P}=\frac{1}{c_P}(\epsilon-\frac{1}{\rho}\scap{\nabla}{F})
\end{equation}
il termine a destra \'e trascurabile:
\begin{equation}
\TDy{t}{q}=0
\end{equation}
   }
   \only<2>{
   Il moto di una elemento di fluido \'e descritto dalla relazione adiabatica
\begin{equation}
\TDy{t}{P}=\frac{\Gamma_1P}{\rho}\TDy{t}{\rho}
\end{equation}
}
   \only<3>{
  
  La condizione di perturbazione adiabatica linearizzata \'e
\begin{align}
&\PDy{t}{\Lvar{P}}-\frac{\Gamma_{1,0}P_0}{\rho_0}\PDy{t}{\Lvar{\rho}}=0\\
&P'+\Lvar{\vec{\xi}}\cdot\nabla P_0=\frac{\Gamma_{1,0}P_0}{\rho}(\rho'+\Lvar{\vec{\xi}}\cdot\nabla\rho_0)\label{eq:adper}
\end{align}

   }
  \end{overlayarea}


\end{frame}


\begin{figure}[!ht]

\subfigure[Distribuzione dei modi con $l\leq300$ nel diagramma $\nu-l$ determinata usando i primi 144 giorni di osservazione di MDI. Da \cite{chr02helioseismology}.]{
\includegraphics[keepaspectratio,width=0.45\textwidth]{midlmodes}}
\label{fig:midlmodes}
~
\subfigure[Modi adiabatici calcolati sulla base di un modello solare. Da \cite{chr02helioseismology}.]{
\includegraphics[keepaspectratio,width=0.6\textwidth]{nrmodesLAWE}\label{fig:nrmodesLAWE}}

\end{figure}

\section{Campo di velocit\'a solare}

\section{Caratteristiche asintotiche delle oscillazioni adiabatiche}



\part{Tecniche e risultati di inversione}\label{part:inverseproblem}

\frame{\partpage}

\begin{frame}{Argomenti}
  \tableofcontents[part=3%,pausesections
  ]
  % You might wish to add the option [pausesections]
\end{frame}

\begin{comment}
\section{Osservabili stellari/demo beamer}

\begin{frame}<1>[label=noinside]{Modello stellare}{Come indagare la fisica interna a una stella?}

\onslide<1->\begin{block}{Osservabili stellari:}
$L$, $M$, $R$, $T_e$, $(\frac{Z}{X})_{ph}$, $g_{ph}$.
\end{block}

\onslide<1->\begin{block}{Informazioni sulla struttura interna?} Condizione di equilibrio idrostatico
\end{block}

%Teorema Vogt-Russel: $X_i(r)$, $M$ \pause equilibrio (idrostatico/termico) determinano struttura stellare .
%\pause

\onslide<1->\begin{block}{Modello stellare: diagramma di \hr{}.}
\end{block}

\onslide<2->\begin{block}{Descrizione fisica interno stellare: parametri aggiuntivi}
Convezione, diffusione e sedimentazione elementi pesanti, equazione di stato, opacit\'a
\end{block}

\onslide<2->\begin{block}{Astrosismologia}
Restringo spazio parametri sistemi stellari lontani
\end{block}

\end{frame}

{ % all template changes are local to this group.
    \setbeamertemplate{navigation symbols}{}
    \begin{frame}[plain]{Diagramma di \hr{}}
        \begin{tikzpicture}[remember picture,overlay]
            \node[at=(current page.center)] {
                %\includegraphics[width=\paperwidth]{yourimage}
            };
        \end{tikzpicture}
     \end{frame}
}
\againframe<2>{noinside}

\section{Inversione della legge di Duvall}

\section{Inversione non asintotica}

\section{Vincoli al modello solare dalle osservazioni sismologiche}

\end{comment}


In questa parte considero come estrarre informazioni sulla struttura di equilibrio dalle frequenze dei modi osservati: modi distinti sono confinati in cavit\'a di profondit\'a diversa e le ampiezze di oscillazione hanno differente comportamento spaziale, \'e quindi possibile invertire il problema date le frequenze osservate per ricavare il profilo radiale di $(P,\rho,\Gamma_1)$.

Un'inversione indipendente dal modello \'e possibile utilizzando l'approssimazione asintotica, valida nelle regioni in cui le autofunzioni variano molto pi\'u rapidamente delle grandezze di equilibrio e che trascura la perturbazione del potenziale gravitazionale.
 
L'inversione del sistema completo di equazioni dei modi si effettua considerando le perturbazioni al MSS che danno un miglior accordo tra frequenze osservate e misurate: considero solo i termini lineari nelle perturbazioni e quindi le correzioni agli autovettori sono trascurate.

I risultati dell'inversione sismologica permettono di valutare l'accuratezza della struttura del \mss{} , in particolare del profilo radiale della velocit\'a del suono, della densit\'a e di $\Gamma_1$; inoltre, usando l'equazione di stato per esplicitare la dipendenza di $\Gamma_1$ da $Y$, \'e possibile ricavare l'abbondanza di elio nella zona convettiva $Y_{CZ}$.

\'E possibile ricavare le caratteristiche della base della zona convettiva, profondit\'a della zona convettiva $d_{CZ}=\rsun{}-R_{CZ}$, $\rho_b=\rho(R_{CZ})$, $c_s=c_s(R_{CZ})$, oltre a $Y_{CZ}$, con grande accuratezza.

\section{Inversione della legge di Duvall. % Inizio chapter "Inversione asintotica." senza nuava pagina

%Si ricava, usando le espressioni asintotiche \eqref{cowosc:main}, il profilo radiale della velocit\'a del suono indipendente dal modello solare e si determina l'effetto dell'evoluzione stellare sui modi p di basso ordine radiale.
%Considerando la differenza tra risultati relativi a diversi set di frequenze \'e possibile attenuare gli effetti degli errori sistematici dovuti alla descrizione asintotica.


L'inversione della legge di Duvall \eqref{eq:duvallexpli} permette di ricavare il profilo di $c_s(r)$ sulla base dei modi osservati, tuttavia le approssimazioni fatte introducono errori sistematici: per i modi pi\'u penetranti nell'interno solare la perturbazione del potenziale gravitazionale influenza sensibilmente $F(\frac{\omega}{L})$ mentre per modi confinati vicino alla superficie $\alpha$ dipende da l.


\subsection{Inversione analitica della velocit\'a del suono}

\begin{figure}[!ht]
        \includegraphics[width=0.44\textwidth,keepaspectratio]{soundspeed}
        \caption{Profilo radiale di $c_s^2$ determinato invertendo \eqref{eq:analinversionc} dalle frequenze dei modi osservate. Da \cite{christensen1985speed}.}\label{fig:soundspeedccm}
\end{figure}

L'equazione \eqref{eq:duvallf} pu\'o essere invertita analiticamente:
\begin{equation}
r=R\Exp{-\frac{2}{\pi}\int_{a_s}^a(w\expy{-2}-a\expy{-2})\expy{-\frac{1}{2}}\TDy{w}{F}\,dw}\label{eq:analinversionc}
\end{equation}

dove $a=\frac{c_s}{r}$.

Dall'equazione precedente, nota la funzione $F(w)$ dalle osservazioni, di ricavare $c_s(r)$ (vedi figura \ref{fig:soundspeedccm}). Il confronto tra $c_{sm}(r)$ calcolato tramite un modello solare e $c_{s0}(r)$ invertito usando l'equazione precedente per lo stesso modello mostra che l'errore sistematico dovuto alla tecnica di inversione nel range $0.4\leq x \leq 0.9$ \'e minore del $2.5\%$.


\subsection{Struttura dei modi penetranti nel core stellare}

La deviazione dalla \eqref{eq:freqequi} fornisce informazioni sull'evoluzione chimica del core di fusione: infatti estendendo ancora l'espansione di \eqref{eq:duvallf} si ha una misura della variazione di $c_s$ nel core della stella
\begin{equation}\label{eq:tassoul}
    d_{nl}=\nu_{nl}-\nu_{n-1,l+2}\approx-(4l+6)\frac{\Delta\nu}{4\pi^2\nu_{nl}}\int_0^R\frac{dc_s}{dr}\frac{dr}{r}
\end{equation}
La velocit\'a del suono \'e ridotta a causa dell'aumentare di $\mu$ durante la fusione di H in He durante l'evoluzione stellare: il centro solare \'e un minimo locale per la velocit\'a del suono e quindi, essendo il gradiente della velocit\'a del suono positivo, la parte centrale da un contributo sempre pi\'u negativo in \eqref{eq:tassoul} con l'evolversi della stella.

\subsection{Forma differenziale della legge di Duvall}

Considero l'effetto di perturbazioni del modello sulle frequenze dei modi introducendo nella legge di Duvall \eqref{eq:duvallexpli} perturbazioni nel profilo della velocit\'a del suono e in $\alpha$:
\begin{equation}
S_{nl}\frac{\delta\omega_{nl}}{\omega_{nl}}\approx H_1(\frac{\omega_{nl}}{L})+H_2(\omega_{nl})\label{eq:Dlinear}
\end{equation}

\begin{align}
&S_{nl}=\int_{r_t}^R(1-\frac{L^2c^2}{r^2\omega_{nl}^2})\expy{-\frac{1}{2}}\frac{dr}{c}-\pi\TDy{\omega}{\alpha}\\
&H_1(w)=\int_{r_t}^R(1-\frac{c^2}{r^2w^2})\expy{-\frac{1}{2}}\frac{\delta_rc}{c}\frac{dr}{c},\ H_2(\omega)=\frac{\pi}{\omega}\delta\alpha(\omega)
\end{align}

La funzione $S_{nl}$ \'e approssimabile con un temine proporzionale a $Q_{nl}$ (\cite{christensen1991solar}):
\begin{align}
&\frac{S_{nl}}{\tau_0}\approx Q_{nl}\intxt{con}
&\tau_0=\int_{0}^R\frac{dr}{c_s}
\end{align}

Le funzioni $H_1(\frac{\omega_{nl}}{L})$ e $H_2(\omega_{nl})$ possono essere ottenute separatamente attraverso fitting dei dati sperimentali: la prima caratterizza il contributo alle differenze nelle frequenze dei modi dovuto alle differenze del profilo radiale della velocit\'a del suono, la seconda alle diffenze nella regione vicino alla superficie.

\begin{figure}[!ht]
        \includegraphics[width=0.44\textwidth,keepaspectratio]{H2dnd}
        \caption{a) Residuo della differenza di frequenze fra il sole e un modello senza diffusione a cui \'e stato sottratto $H_1$. b) Fit di $H_2$ linea continua e per contrasto fit di $H_2$ per differenze di frequenze tra Sole e modello con diffusione. Da \cite{dal03notes}.}\label{fig:H2dnd}
\end{figure}

La relazione \eqref{eq:Dlinear}, considerando che $1-\midfrac{L^2c^2}{r^2\omega^2}\approx1$ ad eccezione delle regioni vicino al punto d'inversione $r_t$, pu\'o essere approssimata da
\begin{equation}
\frac{\delta\omega}{\omega}\approx\frac{\int_{r_t}^{R}\frac{\delta_rc_s}{c_s}\frac{dr}{c_s}}{\int_{r_t}^R\frac{dr}{c_s}}
\end{equation}
Le differenze nella velocit\'a del suono nelle varie regioni influiscono sulle differenze nelle frequenze con un peso dato dal tempo impiegato da un'onda sonora ad attraversare la regione: le differenze nella regione vicino alla superficie dove $c_s$ \'e minore hanno un effetto relativamente grande sulle differenze di frequenza.

Una volta determinato $H_1$ le differenze nel profilo radiale di $c_s$ sono determinate tramite
\begin{equation}
\frac{\delta_rc_s}{c_s}=-\frac{2a}{\pi}\TDof{\ln{r}}\int_{a_s}^a(a^2-w^2)\expy{-\frac{1}{2}}H_1(w)\,dw
\end{equation}


La funzione $H_2$ \'e determinata dalla regione sotto la fotosfera. \cite{chr92phase}, analizzando la relazione tra $H_2(\omega)$ e le differenze in $c_s(r)$ e $\Gamma_1$ nelle regioni esterne, hanno visto che discrepanze pi\'u vicino alla superficie generano una componente lentamente oscillante in $H_2(\omega)$ e la ''frequenza'' aumenta con l'aumentare della profondit\'a. \'E inoltre possibile indagare l'andamento di $\Gamma_1$ nella regione di seconda ionizzazione di He e pi\'u in generale il comportamento di $H_2(\omega)$ nelle zone di ionizzazione di H e He consente un'analisi dell'equazione di stato e determinazione dell'abbondanza di elio nella zona convettiva. In figura \ref{fig:H2dnd} si mostra le differenze nelle frequenze tra il Sole ed un modello che non considera la diffusione degli elementi: l'andamento oscillatorio della linea continua nel pannello b \'e dovuto alla differenza nell'abbondanza di idrogeno negli strati superficiali.


\section{Inversione non asintotica.} % Inizio chapter "Inversione non asintotica." senza nuava pagina

La soluzione del problema inverso per il sistema completo di equazioni si basa sulla linearizzazione delle variazioni attorno ad un modello di cui siano calcolabili le autofunzioni dell'operatore $L$ definito in \eqref{eq:variational}.

Utilizzo la formula \eqref{eq:variational} specializzata al problema dell'inversione delle differenze $\delta\omega_{nl}=\omega_{\odot}-\omega_{Mod}$ fra frequenze osservate e predette da un modello. Per l'inversione della struttura idrostatica si ha:
\begin{align}
&\frac{\delta\omega_{nl}}{\omega_{nl}}=\int_0^R[K^{nl}_{c^2,\rho}(r)\frac{\delta_rc^2}{c^2}(r)+K^{nl}_{\rho,c^2}(r)\frac{\delta_r\rho}{\rho}(r)]\,dr+I_{nl}\expy{-1}F_{Surf}(\omega_{nl})+\sigma_i\label{eq:invstructure}\intxt{dove $\sigma_i$ \'e l'incertezza sulle frequenze osservate e}
&\frac{\delta_rc^2}{c^2}(r)=\frac{[c_{\odot}^2(r)-c_{mod}^2(r)]}{c^2(r)},\ \frac{\delta_r\rho}{\rho}(r)=\frac{[\rho_{\odot}(r)-\rho_{mod}(r)]}{\rho(r)}
\end{align}

E quindi si determinano attraverso procedure numeriche le correzioni alla struttura del modello sulla base delle differenze tra frequenze dei modi. Il peso che una perturbazione ha sulla differenze in frequenza \'e determinato dalle autofunzioni dei modi calcolate tramite un modello solare.
%Asymptotic approximation for radial eigenfunction (integral equation connectin sound speed $c(r)$ to $\Omega_{nl}$) is inadequate (especially in deep interior)

In \ref{fig:deltacwu} la banda chiara attorno allo zero, che indica il modello solare in perfetto accordo con le osservazioni sismologiche, mostra l'incertezza nell'inversione della velocit\'a del suono; essa \'e dovuta a

\begin{itemize}

\item Incertezze nelle frequenze dei modi osservate. Oltre all'incertezza statistica propria del determinato strumento \'e possibile valutare gli effette di eventuali errori sistematici confrontando i risultati dell'inversione per set di frequenze ottenute con strumenti diversi: la differenza nella velocit\'a del suono \'e minore di $0.02\%$.

\item Incertezze inerenti la tecnica di inversione.

\item Incertezze legate alla dipendenza da un modello solare per il calcolo dei kernel in \eqref{eq:invstructure}, \eqref{eq:invdGammadrho}, \eqref{eq:splitfreqrotation}.

\end{itemize}


\subsection{Tecniche di inversione numeriche.}
%vedi JCD 2002 pg 25-32

Elenco alcune tecniche numeriche usate. Considero per sempicit\'a una sola funzione da invertire $\frac{\delta f}{f}$, legata a $\delta\omega$ da \eqref{eq:variational}.


\subsubsection{RLS}

Usando la tecnica del minimo $\chi^2$ regolarizzato si parametrizza la funzione incognita $\frac{\delta f}{f}$ tramite funzioni di base opportune.

La funzione da minimizzare \'e
\begin{equation}
Y=(N-N_p)\chi^2+\alpha N\int_0^1(x\TDof{x}\frac{\Delta f}{f})^2\,dx\label{eq:minimizerls}
\end{equation}
con
\begin{equation}
\chi^2=\frac{1}{N-N_p}\sum_{\alpha=1}^N(\frac{\delta\omega_{obs}-\delta\omega_{fit}}{\sigma})^2_{\alpha}
\end{equation}
N indica il numero totale di modi $\alpha$, $N_p$ il numero di parametri da determinare, $\Delta\nu_{fit}$, ricavato tramite \eqref{eq:variational}, contiene la funzione incognita opportunamente parametrizzata; il secondo addendo del lato destro di \eqref{eq:minimizerls} \'e introdotto per ridurre oscillazioni indesiderate nel risultato dell'inverisone con $\alpha$, parametro di regolarizzazione, scelto opportunamente.

\subsubsection{Subtractive Optimally Localized Averaging}

Scelgo dei coefficienti $c_i(r_0)$ tali che $\sum c_i(r_0)\frac{\delta\omega_i}{\omega_i}$ fornisca una media del valore di $\frac{\delta f(r)}{f(r)}$ in $r=r_0$:

\begin{equation}\label{eq:SOLAfmean}
\sum_ic_i(r_0)\frac{\delta\omega_i}{\omega_i}=\int_0^R\sum_ic_i(r_0)K^i(r)\frac{\delta f(r)}{f(r)}\,dr=\exv{\frac{\delta f(r_0)}{f(r_0)}}
\end{equation}

e i coefficienti $c_i(r_0)$ sono determina minimizzando la funzione

\begin{equation}\label{eq:SOLAcir0min}
\int_0^R[\mathcal{K}(r_0,r)-\mathcal{T}(r_0,r)]^2\,dr+\mu\sum_i\sigma_ic_i(r_0)c_j(r_0)\\
\end{equation}
con $\mathcal{K}(r_0,r)=\sum_ic_i(r_0)K^i(r)$ e la funzione target $\mathcal{T}(r_0,r)$, la cui larghezza \'e anch'essa parametro del fit, determina la natura precisa della localizzazione.

La larghezza finita di $\mathcal{K}(r_0,r)$ determina il valor medio di $\frac{\delta f}{f}$ in intorno di $r_0$ e ci\'o causa una differenza sistematica  dal valore effettivo in $r_0$: per $c_s$ l'errore \'e minore del $0.03\%$ e le regioni pi\'u afflitte sono la base della zona convettiva, per la rapida variazione di $\frac{\delta c_s}{c_s}$ e la regione centrale, per il ridotto numero di modi che penetrano in questar zona.


Illustro la tecnica SOLA per determinare $\frac{\delta_rc^2}{c^2}$: si formano delle combinazioni lineari di $\frac{\Lvar{\omega_i}}{\omega_i}$ pesate da coefficienti $c_i(r_0)$ tali che $\frac{\Lvar{c^2}}{c^2}$ sia centrato attorno $r_0$ e che gli altri termini in \eqref{eq:invstructure} siano soppressi, queste compongo un averaging kernel $\mathcal{K}_{c^2,\rho}(r_0,r)=\sum_ic_i(r_0)K_{c^2,\rho}^i(r)$, con $\int_0^R\mathcal{K}(r_0,r)\,dr=1$.

Determino i coefficienti minimizzando l'espressione
\begin{equation}
\int_0^R[\mathcal{K}_{c^2,\rho}(r_0,r)-\mathcal{T}(r_0,r)]^2\,dr+\beta\int_0^R\mathcal{L}_{\rho,c^2}(r_0,r)\,dr+\mu\sum_i\sigma_ic_i(r_0)c_j(r_0)
\end{equation}
dove il kernel cross-term
\begin{equation}
\mathcal{L}_{\rho,c^2}(r_0,r)=\sum_ic_i(r_0)K_{\rho,c^2}^i(r)
\end{equation}
contralla i contributi indesiderati di $\frac{\delta_r\rho}{\rho}$.


Il primo termine approssima il valore di $\frac{\delta c^2}{c^2}$ pesato dal kernel $\mathcal{K}(r,r_0)=\sum_ic_i(r_0)K_{c^2,\rho}^i(r)$, il secondo tiene conto dell'influenza che hanno le discrepanze della seconda funzione su quelle della funzione che abbiamo scelto di invertire pesate da $\mathcal{L}_{\rho,c^2}(r_0,r)=\sum_ic_i(r_0)K_{\rho,c^2}^i(r)$, il terzo \'e il termine di superficie: i coefficienti $c_i(r_0)$ sono scelti in maniera da riprodurre la funzione target, minimizzare la contaminazione delle $\frac{\delta \rho}{\rho}$ via $\mathcal{L}_{\rho,c^2}$ e il rumore.


\subsection{Inversione della rotazione.}

\begin{figure}[!ht]
\centering
\includegraphics[keepaspectratio,width=0.8\textwidth]{invertedrotation}
\caption{Inversione della velocit\'a di rotazione a diverse latitudini. La linea verticale tratteggiata indica la base della zona convettiva. Da \cite{chr02helioseismology}.}
\end{figure}

Per inversione 2D, cio\'e che considera la dipendenza generica $\Omega(r,\theta)$, si esprimono direttamente le differenze in frequenze:
\begin{equation}
\omega_{nlm}-\omega_{nl0}=m\int_0^R\int_0^{\pi}K_{nlm}(r,\theta)\Omega(r,\theta)r\,dr\,d\theta\label{eq:invrot2D}
\end{equation}

mentre nel caso si abbiano i coefficienti $a_{2s+1}$, scrivo la velocit\'a angolare nella forma
\begin{equation}
\Omega(r,\theta)=\sum_{s=0}^{s_m}\Omega_{s}(r)\psi_{2s}(\cos{\theta})\label{eq:angularv15}
\end{equation}
dove $\psi_{2s}$ sono polinomi opportuni.

Esiste una funzione opportuna $K_{nls}^{s}(r)$ tale che
\begin{equation}
2\pi a_{2j+1}(n,l)=\int_0^R\int_0^{\pi}K_{nls}^{s}(r)\Omega_s(r)\,dr
\end{equation}
e quindi \'e possibile determinare $\Omega_s(r)$.

\section{Vincoli al modello solare dalle osservazioni sismologiche.} %%chapter: vincoli al modello solare: HCSM.


\begin{figure}[!ht]%{r}{0.5\textwidth}
        \includegraphics[width=0.9\textwidth,keepaspectratio]{deltacwu}
        \caption{Differenza relativa nel profilo di $c_s$ risultanei dall'inversione delle differenze in frequenza tra Sole e modello: la linea chiara si riferisce alle frequenze di un modello con composzione GS98, la linea scura a composizione AGSS09. La banda chiara mostra l'errore inerente l'inversione eliosismologica, la banda scura l'incertezza a $1\sigma$ sul profilo di $c_s$ predetto dal modello. Da \cite{villante2014chemical}.}\label{fig:deltacwu}
\end{figure}

\begin{table}[!ht]%{r}{0.7\textwidth}

\pgfplotstabletypeset[
math/.style={%
        preproc cell content/.append style={/pgfplots/table/@cell content/.add={$}{$}},
    },
every head row/.style={
 before row={\toprule
 %&\multicolumn{4}{c|}{Primordiale}
 },
 every last row/.style={after row=\bottomrule},
 after row={\midrule}
},
every last row/.style={after row=\bottomrule},
every first column/.style={column type/.add={|}{}},
every last column/.style={column type/.add={}{|}},
%columns/0/.style = {column type/.add={|}{}},
display columns/0/.style={column name={Composizione}},
display columns/1/.style={column name={$Z/X$}},
display columns/2/.style={column name={$R_{CZ}$}},
display columns/3/.style={column name={$Y_{CZ}$}},
display columns/4/.style={column name={$Y_0$}},
create on use/comp/.style={create col/set list={
inversione,GS98,AGS05,AGSS09,C+11}},
columns/comp/.style = {column type/.add={|}{}},
columns/comp/.style={string type},
columns/ZX/.style={string type},
columns/ZX/.append style={math},
columns/RCZ/.style={string type},
columns/RCZ/.append style={math},
columns/YCZ/.style={string type},
columns/YCZ/.append style={math},
columns/Y0/.style={string type},
columns/Y0/.append style={math},
columns={comp,ZX,RCZ,YCZ,Y0},
%/pgf/number format/precision=4
     ]{CZvsZ.txt} %%%
     \caption{Caratteristiche della zona convettiva: confronto tra valore eliosismologico e valore ricavato da modello solare con diverse metallicit\'a del raggio della base della zona convettiva $R_{CZ}$, dell'abbondanza di elio superficiale $Y_{CZ}$ e dell'abbondanza di elio primordiale. Da \cite{basu2016global}.}
\label{tab:CZZvar}
\end{table}

L'inversione di $c_s$ o $\rho$ mostra se un modello solare riproduce accuratamente la posizione della base della zona convettiva in quanto nella zona convettiva si ha gradiente adiabtico maggiore del gradiente radiativo; diminuzione di opacit\'a, nel caso determinata da una minore metallicit\'a, sposta la base della zona convettiva pi\'u in alto come da tabella \ref{tab:CZZvar}.

La figura \ref{fig:deltacwu} mostra che un modello solare con composizione GS98, meno accurata di AGSS09, riproduce il profilo di $c_s$ in maniera pi\'u accurata: ci\'o pu\'o indicare un'opacit\'a da incrementare nel modello. Analogamente la diminuzione dell'opacit\'a diminuisce il gradiente termico nella regione radiativa quindi a pari luminosit\'a si ha un contenuto di idrogeno maggiore.



\part{Come fare una presentazione}

\section{Importanza studio oscillazioni stellari}

\subsection{Osservabili stellari}

\begin{frame}<1>[label=noinside]{Modello stellare}{Come indagare la fisica interna a una stella?}

\onslide<1->\begin{block}{Osservabili stellari:}
$L$, $M$, $R$, $T_e$, $(\frac{Z}{X})_{ph}$, $g_{ph}$.
\end{block}

\onslide<1->\begin{block}{Informazioni sulla struttura interna?} Condizione di equilibrio idrostatico
\end{block}

%Teorema Vogt-Russel: $X_i(r)$, $M$ \pause equilibrio (idrostatico/termico) determinano struttura stellare .
%\pause

\onslide<1->\begin{block}{Modello stellare: diagramma di \hr{}.}
\end{block}

\onslide<2->\begin{block}{Descrizione fisica interno stellare: parametri aggiuntivi}
Convezione, diffusione e sedimentazione elementi pesanti, equazione di stato, opacit\'a
\end{block}

\onslide<2->\begin{block}{Astrosismologia}
Restringo spazio parametri sistemi stellari lontani
\end{block}

\end{frame}


{ % all template changes are local to this group.
    \setbeamertemplate{navigation symbols}{}
    \begin{frame}[plain]{Diagramma di \hr{}}
        \begin{tikzpicture}[remember picture,overlay]
            \node[at=(current page.center)] {
                %\includegraphics[width=\paperwidth]{yourimage}
            };
        \end{tikzpicture}
     \end{frame}
}



\againframe<2>{noinside}



\subsection{Elisismologia}

\begin{frame}{Pulsazioni stellari}{Modi Normali}
\begin{columns}

\begin{column}{0.5\textwidth}  %%<--- here
    \begin{center}
     %\includegraphics[width=0.5\textwidth]{image1}
     \end{center}
\end{column}

\begin{column}{0.5\textwidth}
\onslide<1-> \begin{block}{Stelle pulsanti}
Onde stazionarie: Pulsazione radiale/non radiale: .

\onslide<2-> meccanismo di eccitazione: solar-like pulsator, Cefeidi.

\onslide<3-> Modo fondamentale $\Pi\approx\tau_{dyn}=\sqrt{\frac{R^3}{GM}}\propto\overline{\rho}\expy{-\frac{1}{2}}$.

\onslide<4-> Modi di oscillazione\onslide<5-> - informazioni sull'interno stellare

\onslide<5-> Elio-sismologia: Modi $\Leftrightarrow$ Modelli solari

\onslide<5-> Astero-sismologia: Modi $\Leftrightarrow$ Spazio parametri modello stellare


\end{block}

\end{column}

\end{columns}
\end{frame}



\section{Oscillazioni solari}

\subsection{Problema osservativo}

\begin{frame}{Fenomeni periodici sulla superficie solare}{Oscillazioni dei 5 minuti}
Oscillazioni nella fotosfera:

Effetto doppler, Variazioni intensit\'a
\pause
\cite{lei62velocity}: righe di ??, 296 s, dimensioni fisiche osservazion ?? 
\pause
caratteristiche oscillazione: ampiezza, frequenze

\pause
Intensit\'a


\end{frame}



\subsection{Oscillazioni lineari adiabatiche}


\begin{frame}{Onde in un gas}

\begin{block}{Oscillazioni adiabatiche}

\end{block}

\end{frame}

\begin{frame}{Modi normali}{Modello di Ulrich}
Frequenze discrete \pause Distribuzione diagramma $k_h-\omega$.

\pause
Spettro acustic: a basse frequenze non c'\'e propagazione di fase.
(propagazione riflessione densit\'a)
\pause

Interferenza costruttiva

\end{frame}

\begin{frame}{Oscillazioni lineari adiabatiche}{Equazione del moto perturbato}
\begin{equation}
\rho\TDof{t}\vec{v}=\rho(\PDof{t}+\scap{v}{\nabla})\vec{v}=-\nabla P+\rho\vec{g}
\end{equation}

\end{frame}

% Placing a * after \section means it will not show in the
% outline or table of contents.
\section*{Summary}

\begin{frame}{Summary}
  \begin{itemize}
  \item
    The \alert{first main message} of your talk in one or two lines.
  \item
    The \alert{second main message} of your talk in one or two lines.
  \item
    Perhaps a \alert{third message}, but not more than that.
  \end{itemize}
  
  \begin{itemize}
  \item
    Outlook
    \begin{itemize}
    \item
      Something you haven't solved.
    \item
      Something else you haven't solved.
    \end{itemize}
  \end{itemize}
\end{frame}



% All of the following is optional and typically not needed. 
\appendix
\section<presentation>*{\appendixname}
\subsection<presentation>*{For Further Reading}

\begin{frame}[allowframebreaks]
  \frametitle<presentation>{For Further Reading}
    
  \begin{thebibliography}{10}
    
  \beamertemplatebookbibitems
  % Start with overview books.

  \bibitem{Author1990}
    A.~Author.
    \newblock {\em Handbook of Everything}.
    \newblock Some Press, 1990.
 
    
  \beamertemplatearticlebibitems
  % Followed by interesting articles. Keep the list short. 

  \bibitem{Someone2000}
    S.~Someone.
    \newblock On this and that.
    \newblock {\em Journal of This and That}, 2(1):50--100,
    2000.
  \end{thebibliography}
\end{frame}

\end{document}