\section{Importanza studio oscillazioni stellari}

\subsection{Osservabili stellari}

\begin{frame}<1>[label=noinside]{Modello stellare}{Come indagare la fisica interna a una stella?}

\onslide<1->\begin{block}{Osservabili stellari:}
$L$, $M$, $R$, $T_e$, $(\frac{Z}{X})_{ph}$, $g_{ph}$.
\end{block}

\onslide<1->\begin{block}{Informazioni sulla struttura interna?} Condizione di equilibrio idrostatico
\end{block}

%Teorema Vogt-Russel: $X_i(r)$, $M$ \pause equilibrio (idrostatico/termico) determinano struttura stellare .
%\pause

\onslide<1->\begin{block}{Modello stellare: diagramma di \hr{}.}
\end{block}

\onslide<2->\begin{block}{Descrizione fisica interno stellare: parametri aggiuntivi}
Convezione, diffusione e sedimentazione elementi pesanti, equazione di stato, opacit\'a
\end{block}

\onslide<2->\begin{block}{Astrosismologia}
Restringo spazio parametri sistemi stellari lontani
\end{block}

\end{frame}


{ % all template changes are local to this group.
    \setbeamertemplate{navigation symbols}{}
    \begin{frame}[plain]{Diagramma di \hr{}}
        \begin{tikzpicture}[remember picture,overlay]
            \node[at=(current page.center)] {
                %\includegraphics[width=\paperwidth]{yourimage}
            };
        \end{tikzpicture}
     \end{frame}
}



\againframe<2>{noinside}



\subsection{Elisismologia}

\begin{frame}{Pulsazioni stellari}{Modi Normali}
\begin{columns}

\begin{column}{0.5\textwidth}  %%<--- here
    \begin{center}
     %\includegraphics[width=0.5\textwidth]{image1}
     \end{center}
\end{column}

\begin{column}{0.5\textwidth}
\onslide<1-> \begin{block}{Stelle pulsanti}
Onde stazionarie: Pulsazione radiale/non radiale: .

\onslide<2-> meccanismo di eccitazione: solar-like pulsator, Cefeidi.

\onslide<3-> Modo fondamentale $\Pi\approx\tau_{dyn}=\sqrt{\frac{R^3}{GM}}\propto\overline{\rho}\expy{-\frac{1}{2}}$.

\onslide<4-> Modi di oscillazione\onslide<5-> - informazioni sull'interno stellare

\onslide<5-> Elio-sismologia: Modi $\Leftrightarrow$ Modelli solari

\onslide<5-> Astero-sismologia: Modi $\Leftrightarrow$ Spazio parametri modello stellare


\end{block}

\end{column}

\end{columns}
\end{frame}



\section{Oscillazioni solari}

\subsection{Problema osservativo}

\begin{frame}{Fenomeni periodici sulla superficie solare}{Oscillazioni dei 5 minuti}
Oscillazioni nella fotosfera:

Effetto doppler, Variazioni intensit\'a
\pause
\cite{lei62velocity}: righe di ??, 296 s, dimensioni fisiche osservazion ?? 
\pause
caratteristiche oscillazione: ampiezza, frequenze

\pause
Intensit\'a


\end{frame}



\subsection{Oscillazioni lineari adiabatiche}


\begin{frame}{Onde in un gas}

\begin{block}{Oscillazioni adiabatiche}

\end{block}

\end{frame}

\begin{frame}{Modi normali}{Modello di Ulrich}
Frequenze discrete \pause Distribuzione diagramma $k_h-\omega$.

\pause
Spettro acustic: a basse frequenze non c'\'e propagazione di fase.
(propagazione riflessione densit\'a)
\pause

Interferenza costruttiva

\end{frame}

\begin{frame}{Oscillazioni lineari adiabatiche}{Equazione del moto perturbato}
\begin{equation}
\rho\TDof{t}\vec{v}=\rho(\PDof{t}+\scap{v}{\nabla})\vec{v}=-\nabla P+\rho\vec{g}
\end{equation}

\end{frame}

% Placing a * after \section means it will not show in the
% outline or table of contents.
\section*{Summary}

\begin{frame}{Summary}
  \begin{itemize}
  \item
    The \alert{first main message} of your talk in one or two lines.
  \item
    The \alert{second main message} of your talk in one or two lines.
  \item
    Perhaps a \alert{third message}, but not more than that.
  \end{itemize}
  
  \begin{itemize}
  \item
    Outlook
    \begin{itemize}
    \item
      Something you haven't solved.
    \item
      Something else you haven't solved.
    \end{itemize}
  \end{itemize}
\end{frame}



% All of the following is optional and typically not needed. 
\appendix
\section<presentation>*{\appendixname}
\subsection<presentation>*{For Further Reading}

\begin{frame}[allowframebreaks]
  \frametitle<presentation>{For Further Reading}
    
  \begin{thebibliography}{10}
    
  \beamertemplatebookbibitems
  % Start with overview books.

  \bibitem{Author1990}
    A.~Author.
    \newblock {\em Handbook of Everything}.
    \newblock Some Press, 1990.
 
    
  \beamertemplatearticlebibitems
  % Followed by interesting articles. Keep the list short. 

  \bibitem{Someone2000}
    S.~Someone.
    \newblock On this and that.
    \newblock {\em Journal of This and That}, 2(1):50--100,
    2000.
  \end{thebibliography}
\end{frame}