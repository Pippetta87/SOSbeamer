\subsection{Osservabili stellari}

\begin{frame}<1>[label=noinside]{Modello stellare}{Come indagare la fisica interna a una stella?}

\onslide<1->\begin{block}{Osservabili stellari:}
$L$, $M$, $R$, $T_e$, $(\frac{Z}{X})_{ph}$, $g_{ph}$.
\end{block}

\onslide<1->\begin{block}{Informazioni sulla struttura interna?} Condizione di equilibrio idrostatico
\end{block}

%Teorema Vogt-Russel: $X_i(r)$, $M$ \pause equilibrio (idrostatico/termico) determinano struttura stellare .
%\pause

\onslide<1->\begin{block}{Modello stellare: diagramma di \hr{}.}
\end{block}

\onslide<2->\begin{block}{Descrizione fisica interno stellare: parametri aggiuntivi}
Convezione, diffusione e sedimentazione elementi pesanti, equazione di stato, opacit\'a
\end{block}

\onslide<2->\begin{block}{Astrosismologia}
Restringo spazio parametri sistemi stellari lontani
\end{block}

\end{frame}


{ % all template changes are local to this group.
    \setbeamertemplate{navigation symbols}{}
    \begin{frame}[plain]{Diagramma di \hr{}}
        \begin{tikzpicture}[remember picture,overlay]
            \node[at=(current page.center)] {
                %\includegraphics[width=\paperwidth]{yourimage}
            };
        \end{tikzpicture}
     \end{frame}
}



\againframe<2>{noinside}